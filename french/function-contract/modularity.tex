Pour terminer cette partie nous allons parler de la composition des appels de
fonctions et commencer à entrer dans les détails de fonctionnement de WP. Nous
allons en profiter pour regarder comment se traduit le découpage de nos
programmes en fichiers lorsque nous voulons les spécifier et les prouver avec WP.



Notre but sera de prouver la fonction \CodeInline{max\_abs} qui renvoie les maximums
entre les valeurs absolues de deux valeurs :



\CodeBlockInput[6][11]{c}{max_abs.c}



Commençons par (sur-)découper les déclarations et définitions des différentes
fonctions dont nous avons besoin (et que nous avons déjà prouvé) en couples
headers/source, à savoir \CodeInline{abs} et \CodeInline{max}. Cela donne pour
\CodeInline{abs} :



Fichier abs.h :

\CodeBlockInput{c}{abs.h}


Fichier abs.c :

\CodeBlockInput{c}{abs.c}



Nous découpons en mettant le contrat de la fonction dans le header. Le but de
ceci est de pouvoir, lorsque nous aurons besoin de la fonction dans un autre
fichier, importer la spécification en même temps que la déclaration de
celle-ci. En effet, WP en aura besoin pour montrer que les appels à cette
fonction sont valides. D'abord pour prouver que la précondition est respectée
(et donc que l'appel est légal) et ensuite pour savoir ce qu'il peut apprendre
en retour (à savoir la postcondition) afin de pouvoir l'utiliser pour prouver
la fonction appelante.



Nous pouvons créer un fichier sous le même formatage pour la fonction \CodeInline{max}.
Dans les deux cas, nous pouvons ré-ouvrir le fichier source (pas besoin de
spécifier les fichiers headers dans la ligne de commande) avec Frama-C et
remarquer que la spécification est bien associée à la fonction et que nous
pouvons la prouver.



Maintenant, nous pouvons préparer le terrain pour la fonction \CodeInline{max\_abs}.
Dans notre header :



\CodeBlockInput{c}{max_abs_uns.h}



Et dans le source :



\CodeBlockInput{c}{max_abs.c}



Et ouvrir ce dernier fichier dans Frama-C. Si nous regardons le panneau latéral,
nous pouvons voir que les fichiers header que nous avons inclus dans le fichier
\CodeInline{abs\_max.c} y apparaissent et que les contrats de fonction sont décorés avec des
pastilles particulières (vertes et bleues) :



\image{max_abs}


Ces pastilles nous disent qu'en l'absence d'implémentation, les propriétés sont
supposées vraies. Et c'est une des forces de la preuve déductive de programmes
par rapport à certaines autres méthodes formelles, les fonctions sont vérifiées
en isolation les unes des autres.



En dehors de la fonction, sa spécification est considérée comme étant
vérifiée : nous ne cherchons pas à reprouver que la fonction fait bien son travail
à chaque appel, nous nous contenterons de vérifier que les préconditions sont
réunies au moment de l'appel. Cela donne donc des preuves très modulaires et donc
des spécifications plus facilement réutilisables. Évidemment, si notre preuve
repose sur la spécification d'une autre fonction, cette fonction doit-elle même
être vérifiable pour que la preuve soit formellement complète. Mais nous pouvons
également vouloir simplement faire confiance à une bibliothèque externe sans la
prouver.



Finalement, le lecteur pourra essayer de spécifier la fonction \CodeInline{max\_abs}.



La spécification peut ressembler à ceci:



\CodeBlockInput[4][14]{c}{max_abs.h}



\levelThreeTitle{Exercices}



\levelFourTitle{Jours du mois}
\label{l4:contract-modularity-ex-days-of-month}


Spécifier la fonction année bissextile qui retourne vrai si l'année reçue
en entrée est bissextile. Utiliser cette fonction pour compéter la fonction
jours du mois de façon à retourner le nombre de jour du mois reçu en entrée,
incluant le bon comportement lorsque le mois en question est février et que
l'année est bissextile.


\CodeBlockInput{c}{ex-1-days-month.c}


\levelFourTitle{Caractères alpha-numériques}
\label{l4:contract-modularity-ex-alpha-num}


Écrire et spécifier les différentes fonctions utilisées par
\CodeInline{is\_alpha\_num}. Fournir un contrat pour chacune d'elles et
fournir le contrat de \CodeInline{is\_alpha\_num}.


\begin{CodeBlock}{c}
int is_alpha_num(char c){
  return
    is_lower_alpha(c) ||
    is_upper_alpha(c) ||
    is_digit(c) ;
}
\end{CodeBlock}

Déclarer une énumération avec les valeurs \CodeInline{LOWER}, \CodeInline{UPPER},
\CodeInline{DIGIT} et \CodeInline{OTHER}, et une fonction \CodeInline{character\_kind}
qui retourne, en utilisant les différentes fonctions \CodeInline{is\_lower},
\CodeInline{is\_upper} et \CodeInline{is\_digit}, la sorte de caractère reçue
en entrée. Utiliser les comportements pour spécifier le contrat de cette fonction
en s'assurant qu'ils sont complets et disjoints.



\levelFourTitle{Ordonner trois valeurs}


Reprendre la fonction \CodeInline{max\_ptr} dans sa version qui « ordonne »
les deux valeurs. Écrire une fonction \CodeInline{min\_ptr} qui utilise la
fonction précédente pour effectuer l'opération inverse. Utiliser ces fonctions
pour compléter les quatre fonctions qui ordonnent trois valeurs. Pour chaque
variante (ordre croissant et décroissant), l'écrire une première fois en
utilisant uniquement \CodeInline{max\_ptr} et une seconde en utilisant
\CodeInline{min\_ptr}. Écrire un contrat précis pour chacune de ces fonctions
et les prouver.


\CodeBlockInput{c}{ex-3-order-3.c}
