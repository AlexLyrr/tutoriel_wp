\levelThreeTitle{Bien traduire ce qui est attendu}


C'est certainement notre tâche la plus difficile. En soi, la programmation est
déjà un effort consistant à écrire des algorithmes qui répondent à notre 
besoin. La spécification nous demande également de faire ce travail, la 
différence est que nous ne nous occupons plus de préciser la manière de répondre
au besoin mais le besoin lui-même. Pour prouver que la réalisation implémente 
bien ce que nous attendons, il faut donc être capable de décrire précisément le
besoin.



Changeons d'exemple et spécifions la fonction suivante :



\CodeBlockInput{c}{max-0.c}



Le lecteur pourra écrire et prouver sa spécification. Pour la suite, nous 
travaillerons avec celle-ci :



\CodeBlockInput[1][6]{c}{max-1.c}



Si nous donnons ce code à WP, il accepte sans problème de prouver la fonction. 
Pour autant cette spécification est-elle suffisante ? Nous pouvons par exemple 
essayer de voir si ce code est validé :



\CodeBlockInput[8][14]{c}{max-1.c}



La réponse est non. En fait, nous pouvons aller plus loin en modifiant le corps 
de la fonction \CodeInline{max} et remarquer que le code suivant est également valide 
quant à la spécification :



\CodeBlockInput[1][8]{c}{max-2.c}




Même si elle est correcte, notre spécification est trop permissive. Il faut que nous
soyons plus précis.
Nous attendons du résultat non seulement qu'il soit supérieur ou égal à nos 
deux paramètres mais également qu'il soit exactement l'un des deux :



\CodeBlockInput[1][7]{c}{max-3.c}


Nous pouvons également prouver que cette spécification est vérifiée par notre
fonction. Mais nous pouvons maintenant prouver en plus l'assertion présente dans
notre fonction \CodeInline{foo}, et nous ne pouvons plus prouver que
l'implémentation qui retourne \CodeInline{INT\_MAX} vérifie la spécification.



\levelThreeTitle{Pointeurs}


S'il y a une notion à laquelle nous sommes confrontés en permanence en 
langage C, c'est bien la notion de pointeur. C'est une notion complexe et 
l'une des principales cause de bugs critiques dans les programmes, ils ont 
donc droit à un traitement de faveur dans ACSL.



Prenons par exemple une fonction \CodeInline{swap} pour les entiers :



\CodeBlockInput{c}{swap-0.c}



\levelFourTitle{Historique des valeurs}


Ici, nous introduisons une première fonction logique fournie de base par 
ACSL : \CodeInline{\textbackslash{}old}, qui permet de parler de l'ancienne valeur d'un élément. 
Ce que nous dit donc la spécification c'est « la fonction doit assurer que
\CodeInline{*a} soit égal à l'ancienne valeur (au sens : la valeur avant l'appel) de \CodeInline{*b}
et inversement ».



La fonction \CodeInline{\textbackslash{}old} ne peut être utilisée que dans la postcondition d'une
fonction. Si nous avons besoin de ce type d'information ailleurs, nous 
utilisons \CodeInline{\textbackslash{}at} qui nous permet d'exprimer des propriétés à propos de la 
valeur d'une variable à un point donné. Elle reçoit deux paramètres. Le premier 
est la variable (ou position mémoire) dont nous voulons obtenir la valeur et le 
second la position (sous la forme d'un label C) à laquelle nous voulons 
contrôler la valeur en question.



Par exemple, nous pourrions écrire :



\CodeBlockInput[2][6]{c}{at.c}



En plus des labels que nous pouvons nous-mêmes créer. Il existe 6 labels 
qu'ACSL nous propose par défaut:



\begin{itemize}
\item \CodeInline{Pre}/\CodeInline{Old} : valeur avant l'appel de la fonction,
\item \CodeInline{Post} : valeur après l'appel de la fonction,
\item \CodeInline{LoopEntry} : valeur en début de boucle,
\item \CodeInline{LoopCurrent} : valeur en début du pas actuel de la boucle,
\item \CodeInline{Here} : valeur au point d'appel.
\end{itemize}


\begin{Information}
Le comportement de \CodeInline{Here} est en fait le comportement par défaut lorsque
nous parlons de la valeur d'une variable. Son utilisation avec \CodeInline{\textbackslash{}at} nous 
servira généralement à s'assurer de l'absence d’ambiguïté lorsque nous parlons
de divers points de programme dans la même expression.
\end{Information}


À la différence de \CodeInline{\textbackslash{}old}, qui ne peut être utilisée que dans les 
postconditions de contrats de fonction, \CodeInline{\textbackslash{}at} peut être utilisée partout.
En revanche, tous les points de programme ne sont pas accessibles selon le type
d'annotation que nous sommes en train d'écrire. \CodeInline{Old} et \CodeInline{Post} ne sont 
disponibles que dans les postconditions d'un contrat, \CodeInline{Pre} et \CodeInline{Here}
sont disponibles partout. \CodeInline{LoopEntry} et \CodeInline{LoopCurrent} ne sont 
disponibles que dans le contexte de boucles (dont nous parlerons plus loin dans
le tutoriel).


Notons qu'il est important de s'assurer que l'on utilise \CodeInline{\textbackslash{}old} at
\CodeInline{\textbackslash{}at} pour des valeurs qui ont du sens. C'est pourquoi par
exemple dans un contrat, toutes les valeurs reçues en entrée sont placées dans un
appel à \CodeInline{\textbackslash{}old} par Frama-C lorsqu'elles sont utilisées dans
les postconditions, la nouvelle valeur d'une variable fournie en entrée d'une
fonction n'a aucun sens pour l'appelant puisque cette valeur est inaccessible par
lui: elles sont locales à la fonction appelée. Par exemple, si nous regardons le
contrat de la fonction \CodeInline{swap} dans Frama-C, nous pouvons voir que dans
la postcondition, chaque pointeur se trouve dans un appel à \CodeInline{\textbackslash{}old} :


\image{2-old-swap}


Pour la fonction built-in \CodeInline{\textbackslash{}at}, nous devons plus
explicitement faire attention à cela. En particulier, le label transmis en entrée
doit avoir un sens par rapport à la portée de la variable que l'on lui transmet.
Par exemple, dans le programme suivant, Frama-C détecte que nous demandons la valeur
de la variable \CodeInline{x} à un point du programme où elle n'existe pas:


\CodeBlockInput[1][5]{c}{at-2.c}


\image{2-at-1.png}


Cependant, dans certains cas, tout ce que nous pouvons obtenir est un échec de
la preuve, parce que déterminer si la valeur existe ou non à un label particulier
ne peut être fait par une analyse purement syntaxique. Par exemple, si la variable
est déclarée mais pas définie, ou si nous demandons la valeur d'une zone mémoire
pointée:


\CodeBlockInput[7][19]{c}{at-2.c}


Ici, il est facile de remarquer le problème. Cependant, le label que nous
transmettons à la fonction \CodeInline{\textbackslash{}at} est propagé également
aux sous-expressions. Dans certains cas, des termes qui paraissent tout à fait
innocents peuvent en réalité nous donner des comportements surprenant si nous
ne gardons pas cette idée en tête. Par exemple dans le programme suivant:


\CodeBlockInput[21][26]{c}{at-2.c}


La première assertion est prouvée, et tandis que la seconde assertion a l'air
d'exprimer la même propriété, elle ne peut pas être prouvée. Et la raison est
simplement qu'elle n'exprime pas la même propriété. L'expression
\CodeInline{\textbackslash{}at(x[*p], Pre)} doit être lue comme
\CodeInline{\textbackslash{}at(x[\textbackslash{}at(*p)], Pre)} parce que le
label est propagé à la sous-expression \CodeInline{*p}, pour laquelle nous ne
connaissons pas la valeur au label \CodeInline{Pre} (qui n'est pas spécifié).


Pour le moment, nous n'utiliserons pas \CodeInline{\textbackslash{}at}, mais elle peut rapidement se
montrer indispensable pour écrire des spécifications précises.



\levelFourTitle{Validité de pointeurs}


Si nous essayons de prouver le fonctionnement de \CodeInline{swap} (en activant
la vérification des RTE), notre postcondition est bien vérifiée mais WP nous 
indique qu'il y a un certain nombre de possibilités de \textit{runtime-error}. Ce qui 
est normal, car nous n'avons pas précisé à WP que les pointeurs que nous
recevons en entrée de fonction sont valides.



Pour ajouter cette précision, nous allons utiliser le prédicat \CodeInline{\textbackslash{}valid} qui
reçoit un pointeur en entrée :



\CodeBlockInput[3][11]{c}{swap-1.c}



À partir de là, les déréférencements qui sont effectués par la suite sont 
acceptés car la fonction demande à ce que les pointeurs d'entrée soient 
valides.



Comme nous le verrons plus tard, \CodeInline{\textbackslash{}valid} peut recevoir plus qu'un 
pointeur en entrée. Par exemple, il est possible de lui transmettre une 
expression de cette forme : \CodeInline{\textbackslash{}valid(p + (s .. e))} qui voudra dire « pour
tout \CodeInline{i} entre \CodeInline{s} et \CodeInline{e} (inclus), \CodeInline{p+i} est un pointeur valide », ce sera important 
notamment pour la gestion des tableaux dans les spécifications.



Si nous nous intéressons aux assertions ajoutées par WP dans la fonction \CodeInline{swap}
avec la validation des RTEs, nous pouvons constater qu'il existe une variante
de \CodeInline{\textbackslash{}valid} sous le nom \CodeInline{\textbackslash{}valid\_read}. Contrairement au premier, 
celui-ci assure qu'il est uniquement nécessaire que le pointeur puisse
être déréférencé en lecture et pas forcément en écriture, pour pouvoir
réaliser l'opération de lecture. Cette subtilité est due au fait qu'en C, le
\textit{downcast} de pointeur vers un élément \CodeInline{const} est très facile à faire mais
n'est pas forcément légal.



Typiquement, dans le code suivant :



\CodeBlockInput{c}{unref.c}



Le déréférencement de \CodeInline{p} est valide, pourtant la précondition de \CodeInline{unref}
ne sera pas validée par WP car le déréférencement de l'adresse de \CodeInline{value} 
n'est légal qu'en lecture. Un accès en écriture sera un comportement 
indéterminé. Dans un tel cas, nous pouvons préciser que dans \CodeInline{unref}, le 
pointeur \CodeInline{p} doit être nécessairement \CodeInline{\textbackslash{}valid\_read} et pas \CodeInline{\textbackslash{}valid}.



\levelFourTitle{Effets de bord}


Notre fonction \CodeInline{swap} est bien prouvable au regard de sa spécification et
de ses potentielles erreurs à l'exécution, mais est-elle pour autant 
suffisamment spécifiée ? Pour voir cela, nous pouvons modifier légèrement le code
de cette façon (nous utilisons \CodeInline{assert} pour analyser des propriétés 
ponctuelles) :



\CodeBlockInput{c}{swap-1.c}



Le résultat n'est pas vraiment celui escompté :



\image{2-swap-1.png}


En effet, nous n'avons pas spécifié les effets de bords autorisés pour notre
fonction. Pour spécifier les effets de bords, nous utilisons la clause \CodeInline{assigns}
qui fait partie des postconditions de la fonction. Elle nous permet de spécifier 
quels éléments \textbf{non locaux} (on vérifie bien des effets de bord), sont 
susceptibles d'être modifiés par la fonction.



Par défaut, WP considère qu'une fonction a le droit de modifier n'importe quel
élément en mémoire. Nous devons donc préciser ce qu'une fonction est en droit 
de modifier. Par exemple pour notre fonction \CodeInline{swap}, nous pouvons
spécifier que seules les valeurs pointées par les pointeurs reçus peuvent être
modifiées :



\CodeBlockInput[3][14]{c}{swap-2.c}



Si nous rejouons la preuve avec cette spécification, la fonction et les 
assertions que nous avions demandées dans le \CodeInline{main} seront validées par WP.



Finalement, il peut arriver que nous voulions spécifier qu'une fonction ne 
provoque pas d'effets de bords. Ce cas est précisé en donnant \CodeInline{\textbackslash{}nothing}
à \CodeInline{assigns} :



\CodeBlockInput{c}{assigns-nothing.c}



Le lecteur pourra maintenant reprendre les exemples précédents pour y intégrer 
la bonne clause \CodeInline{assigns}.



\levelFourTitle{Séparation des zones de la mémoire}


Les pointeurs apportent le risque d'\textit{aliasing} (plusieurs pointeurs ayant accès à
la même zone de mémoire). Si dans certaines fonctions, cela ne pose pas de 
problème (par exemple si nous passons deux pointeurs égaux
à notre fonction \CodeInline{swap}, la spécification est toujours vérifiée par le 
code source), dans d'autre cas, ce n'est pas si simple :



\CodeBlockInput{c}{incr_a_by_b-0.c}



Si nous demandons à WP de prouver cette fonction, nous obtenons le 
résultat suivant :



\image{2-incr_a_by_b-1}


La raison est simplement que rien ne garantit que le pointeur \CodeInline{a} est bien
différent du pointeur \CodeInline{b}. Or, si les pointeurs sont égaux,



\begin{itemize}
\item la propriété \CodeInline{*a == \textbackslash{}old(*a) + *b} signifie en fait 
\CodeInline{*a == \textbackslash{}old(*a) + *a}, ce qui ne peut être vrai que si l'ancienne valeur 
pointée par \CodeInline{a} était 0, ce qu'on ne sait pas,
\item la propriété \CodeInline{*b == \textbackslash{}old(*b)} n'est pas validée car potentiellement,
nous la modifions.
\end{itemize}


\begin{Question}
Pourquoi la clause \CodeInline{assigns} est-elle validée ?

C'est simplement dû au fait, qu'il n'y a bien que la zone mémoire pointée par
\CodeInline{a} qui est modifiée étant donné que si \CodeInline{a != b} nous ne modifions bien 
que cette zone et que si \CodeInline{a == b}, il n'y a toujours que cette zone, et 
pas une autre.
\end{Question}


Pour assurer que les pointeurs sont bien sur des zones séparées de mémoire, 
ACSL nous offre le prédicat \CodeInline{\textbackslash{}separated(p1, ..., pn)} qui reçoit en entrée 
un certain nombre de pointeurs et qui va nous assurer qu'ils sont deux à deux 
disjoints. Ici, nous spécifierions :



\CodeBlockInput{c}{incr_a_by_b-1.c}



Et cette fois, la preuve est effectuée :



\image{2-incr_a_by_b-2.png}


Nous pouvons noter que nous ne nous intéressons pas ici à la preuve de 
l'absence d'erreur à l'exécution car ce n'est pas l'objet de cette section.
Cependant, si cette fonction faisait partie d'un programme complet à vérifier,
il faudrait définir le contexte dans lequel on souhaite l'utiliser et définir
les préconditions qui nous garantissent l'absence de débordement en conséquence.


\levelFourTitle{Écrire le bon contrat}


Trouver les bonnes préconditions à une fonction est parfois difficile. Il est
intéressant de noter qu'une bonne manière de vérifier qu'une spécification est
suffisamment précise est d'écrire des tests pour voir si le contrat nous permet,
depuis un code appelant, de déduire des propriétés intéressantes. Et en fait,
c'est exactement ce que nous avons fait pour nos exemples \CodeInline{max} et
\CodeInline{swap}. Nous avons écrit une première version de notre spécification
et du code appelant qui nous a servi à déterminer si nous pouvions prouver des
propriétés que nous estimions devoir être capables de prouver à l'aide du
contrat.



Le plus important est avant tout de déterminer le contrat sans prendre en compte
le contenu de la fonction (au moins dans un premier temps). En effet, nous
essayons de prouver une fonction, mais elle pourrait contenir un bug, donc si
nous suivons de trop près le code de la fonction, nous risquons d'introduire
dans la spécification le même bug présent dans le code, par exemple en prenant
en compte une condition erronée. C'est pour cela que
l'on souhaitera généralement que la personne qui développe le programme et la 
personne qui le spécifie formellement soient différentes (même si elles ont pu
préalablement s'accorder sur une spécification textuelle par exemple).



Une fois que le contrat est posé, alors seulement, nous nous intéressons aux
spécifications dues au fait que nous sommes soumis aux contraintes de notre langage
et notre matériel. Cela concerne principalement nos préconditions. Par exemple,
la fonction valeur absolue n'a, au fond, pas  vraiment de précondition à respecter,
c'est la machine cible qui détermine qu'une condition supplémentaire doit être
respectée en raison du complément à deux. Comme nous le verrons dans le chapitre
\ref{l1:proof-methodologies}, vérifier l'absence de runtime errors peut aussi
impacter nos postconditions, pour l'instant laissons cela de côté.


\levelThreeTitle{Exercices}


\levelFourTitle{Division et reste}


Spécifier la postcondition de la fonction suivante, qui calcule le résultat
de la division de \CodeInline{a} par \CodeInline{b} et le reste de cette
division et écrit ces deux valeurs à deux positions mémoire \CodeInline{p}
et \CodeInline{q} :


\CodeBlockInput{c}{ex-1-div-rem.c}


Lancer la commande :


\begin{CodeBlock}{bash}
frama-c-gui your-file.c -wp 
\end{CodeBlock}


Une fois que la fonction est prouvée, lancer :

\begin{CodeBlock}{bash}
frama-c-gui your-file.c -wp -wp-rte
\end{CodeBlock}


Si cela échoue, compléter le contrat en ajoutant la bonne précondition.


\levelFourTitle{Remettre à zéro selon une condition}


Donner un contrat à la fonction suivante qui remet à zéro la valeur
pointée par le première paramètre si et seulement si celle pointée par le
second est vraie. Ne pas oublier d'exprimer que la valeur pointée par le second
paramètre doit rester la même :


\CodeBlockInput{c}{ex-2-reset-on-cond.c}


Lancer la commande :


\begin{CodeBlock}{bash}
frama-c-gui your-file.c -wp -wp-rte
\end{CodeBlock}


\levelFourTitle{Addition de valeurs pointées}


La fonction suivante reçoit deux pointeurs en entrée et retourne la
somme des valeurs pointées. Écrire le contrat de cette fonction :


\CodeBlockInput{c}{ex-3-add-ptr.c}


Lancer la commande :


\begin{CodeBlock}{bash}
frama-c-gui your-file.c -wp -wp-rte
\end{CodeBlock}


Une fois que la fonction et son code appelant sont prouvées, modifier
la signature de la fonction comme suit :


\begin{CodeBlock}{c}
void add(int* a, int* b, int* r);
\end{CodeBlock}


Le résultat doit maintenant être stocké à la position mémoire \CodeInline{r}.
Modifier l'appel dans la fonction main et le code de la fonction de façon à
implémenter ce comportement. Modifier le contrat de la fonction
\CodeInline{add} et recommencer la preuve.


\levelFourTitle{Maximum de valeurs pointées}


Le code suivant calcule le maximum des valeurs pointées par \CodeInline{a}
et \CodeInline{b}. Écrire le contrat de cette fonction :


\CodeBlockInput{c}{ex-4-max-ptr.c}


Lancer la commande :


\begin{CodeBlock}{bash}
frama-c-gui your-file.c -wp -wp-rte
\end{CodeBlock}


Une fois que la fonction est prouvée, modifier la signature de la
fonction comme suit :


\begin{CodeBlock}{c}
void max_ptr(int* a, int* b);
\end{CodeBlock}


La fonction doit maintenant s'assurer qu'après l'exécution, \CodeInline{*a}
contient le maximum des valeurs pointées et \CodeInline{*b} contient l'autre
valeur. Modifier le code de façon à assurer cela ainsi que le contrat.
Notons que la variable  \CodeInline{x} n'est plus nécessaire dans la fonction
\CodeInline{main} et que nous pouvons changer l'assertion en ligne 15 pour
mettre en lumière le nouveau comportement de la fonction.


\levelFourTitle{Ordonner trois valeurs}


La fonction suivante doit ordonner trois valeurs reçues en entrée dans
l'ordre coirssant. Écrire le code correspondant et la spécification de la
fonction :

\CodeBlockInput{c}{ex-5-order-3.c}


Et lancer la commande :


\begin{CodeBlock}{bash}
frama-c-gui your-file.c -wp -wp-rte
\end{CodeBlock}


Il faut bien garder en tête qu'ordonner des valeurs ne consiste pas seulement
à s'assurer qu'elles sont dans l'ordre croissant et que chaque valeur doit
être l'une de celles d'origine. Toutes les valeurs d'origine doivent être
présente et en même quantité. Pour exprimer cette idée, nous pouvons nous
reposer à nouveau sur les ensembles. La propriété suivante est vraie par
exemple :


\begin{CodeBlock}{c}
//@ assert { 1, 2, 3 } == { 2, 3, 1 };
\end{CodeBlock}


Nous pouvons l'utiliser pour exprimer que l'ensemble des valeurs d'entrée et
de sortie est le même. Cependant, ce n'est pas la seule chose à prendre en
compte car un ensemble ne contient qu'une occurrence de chaque valeur. Donc,
si \CodeInline{*a == *b == 1}, alors \CodeInline{\{ *a, *b \} == \{ 1 \}}.
Par conséquent nous devons considérer trois autres cas particuliers:


\begin{itemize}
\item toutes les valeurs d'origine sont les mêmes ;
\item deux valeurs d'origine sont les mêmes, la dernière est plus grande ;
\item deux valeurs d'origine sont les mêmes, la dernière est plus petite.
\end{itemize}


Qui nous permet d'ajouter la bonne contrainte aux valeurs de sorties.


Pour la réalisation de la spécification, le programme de test suivant peut
nous aider :


\CodeBlockInput[27][43]{c}{ex-5-order-3-answer.c}


Si la spécification est suffisamment précise, chaque assertion devrait être
prouvée. Cependant, cela ne signifie pas que tous les cas ont été considérés,
il ne faut pas hésiter à ajouter d'autres tests.
