Il peut arriver qu'une fonction ait divers comportements potentiellement très
différents en fonction de l'entrée. Un cas typique est la réception d'un
pointeur vers une ressource optionnelle : si le pointeur est \CodeInline{NULL}, nous
aurons un certain comportement et un comportement complètement différent s'il ne
l'est pas.



Nous avons déjà vu une fonction qui avait des comportements différents, la
fonction \CodeInline{abs}. Nous allons la reprendre comme exemple. Les deux
comportements que nous pouvons isoler sont le cas où la valeur est positive et
le cas où la valeur est négative.



Les comportements nous servent à spécifier les différents cas pour les
postconditions. Nous les introduisons avec le mot-clé \CodeInline{behavior}.
Chaque comportement a un nom. Pour un comportement donné, nous trouvons
différentes suppositions à propos de l'entrée de la fonction, elles sont
introduites à l'aide du mot clé \CodeInline{assumes} (notons que, puisqu'elles
caractérisent les entrées, le mot clé \CodeInline{\textbackslash{}old} ne peut
pas être utilisé ici). Cependant, chaque propriété exprimée par ces clauses
n'a pas \textbf{besoin} d'être vérifiée avant à l'appel, elle \textbf{peut}
être vérifiée et dans ce cas, les postconditions associées à ce comportement
s'appliquent. Ces postconditions sont à nouveau introduites à l'aide du mot
clé \texttt{ensures}. Finalement, nous pouvons également demander à WP
de vérifier le fait que les comportements sont disjoints (pour garantir
le déterminisme) et complets (pour garantir que nous couvrons toutes les
entrées possibles).



Les comportements sont disjoints si pour toute entrée de la fonction, elle ne
correspond aux suppositions (\textit{assumes}) que d'un seul comportement. Les
comportements sont complets si les suppositions recouvrent bien tout le domaine
des entrées.



Par exemple pour \CodeInline{abs} :



\CodeBlockInput{c}{abs.c}


Notons qu'introduire des comportements ne nous interdit pas de spécifier une
postcondition globale. Par exemple ici, nous avons spécifié que quel que soit
le comportement, la fonction doit retourner une valeur positive.


Pour comprendre ce que font précisément \CodeInline{complete} et \CodeInline{disjoint}, il est utile
d'expérimenter deux possibilités :



\begin{itemize}
\item remplacer la supposition de « pos » par \CodeInline{val > 0} auquel cas les
comportements seront disjoints mais incomplets (il nous manquera le cas
\CodeInline{val == 0}) ;
\item remplacer la supposition de « neg » par \CodeInline{val <= 0} auquel cas les
comportements seront complets mais non disjoints (le cas \CodeInline{val == 0}) sera
présent dans les deux comportements.
\end{itemize}


\begin{Warning}
Même si \CodeInline{assigns} est une postcondition, à ma connaissance, il n'est pas
possible de mettre des \CodeInline{assigns} pour chaque \textit{behavior}. Si nous avons
besoin d'un tel cas, nous spécifions :

\begin{itemize}
\item \CodeInline{assigns} avant les \textit{behavior} (comme dans notre exemple) avec tout
élément non-local susceptible d'être modifié,
\item en postcondition de chaque \textit{behavior} les éléments qui ne sont finalement
pas modifiés en les indiquant égaux à leur ancienne (\CodeInline{\textbackslash{}old}) valeur.
\end{itemize}
\end{Warning}


Les comportements sont très utiles pour simplifier l'écriture de spécifications
quand les fonctions ont des effets très différents en fonction de leurs
entrées. Sans eux, les spécifications passent systématiquement par des
implications traduisant la même idée mais dont l'écriture et la lecture sont
plus difficiles (nous sommes susceptibles d'introduire des erreurs).
D'autre part, la traduction de la complétude et de la disjonction devraient
être écrites manuellement, ce qui serait fastidieux et une nouvelle fois source
d'erreurs.



\levelThreeTitle{Exercices}


\levelFourTitle{Distance}


Reprendre l'exemple à propos du calcul de la distance entre deux entiers.
En considérant que le contrat était :


\CodeBlockInput{c}{ex-1-distance.c}


Le ré-écrire en utilisant des comportements.


\levelFourTitle{Remettre à zéro selon une condition}


Reprendre l'exemple « Remettre à zéro selon une condition » de la section
précédente. En considérant que le contrat était :



\CodeBlockInput[1][13]{c}{ex-2-reset-on-cond.c}


Le ré-écrire en utilisant des comportements.

\levelFourTitle{Jours du mois}


Reprendre l'exemple « Jours du mois » de la première section.
En supposant que le contrat écrit était :



\CodeBlockInput{c}{ex-3-day-month.c}


Le ré-écrire en utilisant des comportements.


\levelFourTitle{Maximum des valeurs pointées I}


Reprendre l'exemple « Maximum des valeurs pointées » de la section
précédente et plus précisément la version qui « ordonne » les valeurs
pointées. En considérant que le contrat était :



\CodeBlockInput[1][13]{c}{ex-4-max-ptr.c}


Le ré-écrire en utilisant des comportements.



\levelFourTitle{Maximum des valeurs pointées II}


Reprendre l'exemple « Maximum des valeurs pointées » de la section
précédente et plus précisément la version qui retourne la plus grande
valeur. En considérant que le contrat était :


\CodeBlockInput[1][10]{c}{ex-5-max-ptr.c}


\begin{enumerate}
\item Le ré-écrire en utilisant des comportements
\item Modifier le contrat de 1. de sorte à ce que les comportements ne
  soient pas disjoint. Excepté cette propriété, tout le reste devrait
  être correctement prouvé
\item Modifier le contrat de 1. de sorte à ce que les comportements ne
  soient pas complets, puis ajouter un nouveau comportement pour le
  rendre de nouveau complet
\item Modifier la fonction de 1. de façon à accepter la valeur
  \CodeInline{NULL} pour les pointeurs d'entrées, si les deux pointeurs
  sont nuls, retourner  \CodeInline{INT\_MIN}, si l'un seulement est
  nul, retourner l'autre valeur, sinon retourner le maximum des deux
  valeurs. Modifier le contrat de façon à prendre en compte tout cela
  par de nouveaux comportements. Attention à assurer que les comportement
  sont complets et disjoints.
\end{enumerate}



\levelFourTitle{Ordonner trois valeurs}


Reprendre l'exemple « Ordonner trois valeurs » de la section
précédente, en considérant que le contrat était :


\CodeBlockInput{c}{ex-6-order-3.c}


Le ré-récrire en utilisant des comportements. Notons que le contrat
devrait être composé d'un comportement général et de trois
comportements spécifiques. Est ce que ces comportements sont complets ?
Sont-ils disjoints ?
