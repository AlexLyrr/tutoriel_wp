Les fonctions logiques nous permettent de décrire des fonctions mathématiques
qui contrairement aux prédicats nous permettent de renvoyer différents types.
Elles ne seront utilisables que dans les spécifications. Cela nous permet d'une
part, de les factoriser, et d'autre part de définir des opérations sur les
types \CodeInline{integer} et \CodeInline{real} qui ne peuvent pas déborder
contrairement aux types machines.



Comme les prédicats, elles peuvent recevoir divers \textit{labels} et valeurs en 
paramètre.



\levelThreeTitle{Syntaxe}


Pour déclarer une fonction logique, l'écriture est la suivante :



\begin{CodeBlock}{c}
/*@
  logic type_retour ma_fonction{ Label0, ..., LabelN }( type0 arg0, ..., typeN argN ) =
    formule mettant en jeu les arguments ;
*/
\end{CodeBlock}



Nous pouvons par exemple décrire une \externalLink{fonction affine}{https://fr.wikipedia.org/wiki/Fonction\_affine} générale du côté de la logique :



\CodeBlockInput[1][4]{c}{linear-0.c}



Elle peut nous servir à prouver le code de la fonction suivante :



\CodeBlockInput[6][12]{c}{linear-0.c}



\image{linear-1}


Le code est bien prouvé mais les contrôles d'\textit{overflow}, eux, ne le sont pas.
Nous pouvons ajouter la contrainte en précondition que le calcul doit entrer dans
les bornes d'un entier:


\CodeBlockInput[8][15]{c}{linear-1.c}


Certains contrôles de débordement ne sont pas encore prouvé. En effet, tandis que
la borne fournie pour \CodeInline{x} par notre fonction logique est définie pour
le calcul complet, elle ne nous dit rien à propos des calculs intermédiaires. Par
exemple ici, le fait \CodeInline{3 * x + 4} ne soit pas inférieur à
\CodeInline{INT\_MIN} ne nous garantit pas que c'est aussi le cas pour
\CodeInline{3 * x}. Nous pouvons imaginer deux manières différentes de résoudre
le problème, ce choix doit être guidé par les besoins du projets.


Soit nous pouvons augmenter les restrictions sur l'entrée :


\CodeBlockInput[17][25]{c}{linear-1.c}


Soit nous pouvons modifier le code de manière à corriger le risque
de débordement :


\CodeBlockInput[27][37]{c}{linear-1.c}


Notons que comme dans la spécification, les calculs sont effectués à l'aide 
d'entiers mathématiques, nous n'avons pas à nous préoccuper d'un quelconque
risque de débordement lorsque nous utilisons la fonction logique \CodeInline{ax\_b}:


\CodeBlockInput[39][41]{c}{linear-1.c}


est correctement déchargé par WP, qui ne génère aucune alarme liée aux
débordements :


\image{linear-math}


\levelThreeTitle{Récursivité et limites}


Les fonctions logiques peuvent être définie récursivement. Cependant, une telle
définition montrera très rapidement ses limites pour la preuve. En effet, 
pendant les manipulations des prouveurs automatiques sur les propriétés 
logiques, si l'usage d'une telle fonction est présente, elle devra être évaluée.
Or, les prouveurs ne sont pas conçus pour faire ce genre d'évaluation, qui se 
révélera donc généralement très coûteuse et produisant alors des temps de preuve
trop longs menant à des \textit{timeouts}.



Exemple concret, nous pouvons définir la fonction factorielle, dans la logique
et en C :



\CodeBlockInput{c}{facto-0.c}



Sans contrôle de borne, cette fonction se prouve rapidement. Si nous ajoutons
le contrôle des RTE, nous voyons qu'il y a un risque de débordement
arithmétique sur la multiplication.



Sur le type \CodeInline{int}, le maximum que nous pouvons calculer est la factorielle de 
12. Au-delà, cela produit un dépassement. Nous pouvons donc ajouter cette 
précondition :



\CodeBlockInput[5][12]{c}{facto-1.c}



Si nous demandons la preuve avec cette entrée, Alt-ergo échouera pratiquement à 
coup sûr. En revanche, le prouveur Z3 produit la preuve en moins d'une seconde.
Parce que dans ce cas précis, les heuristiques de Z3 considèrent que c'est une
bonne idée de passer un peu plus de temps sur l'évaluation de la fonction. Nous
pouvons par exemple changer la valeur maximale de \CodeInline{n} pour voir comment se 
comporte les différents prouveurs. Avec un \CodeInline{n} maximal fixé à 9, Alt-ergo produit
la preuve en moins de 10 secondes, tandis que pour une valeur à 10, même une 
minute ne suffit pas.



Les fonctions logiques peuvent donc être définies récursivement mais sans astuces
supplémentaires, nous venons vite nous heurter au fait que les prouveurs vont au 
choix devoir faire de l'évaluation, ou encore « raisonner » par induction, deux 
tâches pour lesquelles ils ne sont pas du tout fait, ce qui limite nos 
possibilités de preuve. Nous verrons plus tard dans ce tutoriel comment éviter
cette limitation.


\levelThreeTitle{Exercices}


\levelFourTitle{Distance}


Spécifier et prouver le programme suivant:


\CodeBlockInput{c}{ex-1-distance.c}


Pour cela, définir deux fonctions logiques \CodeInline{abs} and \CodeInline{distance}.
Utiliser ces fonctions pour écrire la spécification de la fonction.


\levelFourTitle{Carré}


Écrire le corps de la fonction \CodeInline{square}. Spécifier et prouver le
programme. Utiliser une fonction logique \CodeInline{square}.


\CodeBlockInput{c}{ex-2-square.c}


Attention aux types des différentes variables, de telle manière à ne pas
sur-contraindre les entrées de la fonction. De plus, pour vérifier l'absence
d'erreurs à l'exécution, utiliser l'option \CodeInline{-warn-unsigned-overflow}.


\levelFourTitle{Iota}


Voici une implémentation possible de la fonction \CodeInline{iota} :


\CodeBlockInput{c}{ex-3-iota.c}


Écrire une fonction logique qui retourne sa valeur d'entrée incrémentée de 1. 
Prouver qu'après l'exécution de \CodeInline{iota}, la première valeur du tableau
est celle reçue en entrée et que chaque valeur du tableau correspond à la valeur
précédente plus 1 (en utilisant la fonction logique définie).



\levelFourTitle{Addition sur un vecteur}



Dans le programme suivante, la fonction \CodeInline{vec\_add} ajoute le second
vecteur reçu en entrée dans le premier. Écrire un contrat pour la fonction
\CodeInline{show\_the\_difference} qui exprime pour chaque valeur du vecteur
\CodeInline{v1} la différence entre la pré et la postcondition. Pour cela,
définir une fonction logique \CodeInline{diff} qui retourne la différence de
valeur à une position mémoire entre un label \CodeInline{L1} et la valeur au
label \CodeInline{L2}.



\CodeBlockInput{c}{ex-4-vec-inc.c}


Ré-exprimer la prédicat \CodeInline{unchanged} en utilisant la fonction logique.



\levelFourTitle{La somme des N premiers entiers}
\label{l4:acsl-properties-functions-n-first-ints}


La fonction suivante calcule la somme des N premiers entiers. Écrire une fonction
logique récursive qui retourne la somme des N premiers entiers et écrire
la spécification de la fonction C effectuant ce calcul en spécifiant qu'elle
retourne la même valeur que celle fournie par la fonction logique.


\CodeBlockInput{c}{ex-5-n-first-ints.c}


Essayer de vérifier l'absence d'erreurs à l'exécution. Le débordement entier
n'est pas si simple à régler. Cependant, écrire une précondition qui devrait
être suffisante pour cela (rappel: la somme des N premiers entiers peut être
exprimée avec une formule très simple ...). Cela ne sera sûrement pas suffisant
pour arriver au bout de la preuve, mais nous réglerons cela dans la prochaine
section.
