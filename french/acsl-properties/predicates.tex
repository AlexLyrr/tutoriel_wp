Un prédicat est une propriété portant sur des objets et pouvant être vraie ou
fausse. En résumé, des prédicats, c'est ce que nous écrivons depuis le début de
ce tutoriel dans les clauses de nos contrats et de nos invariants de boucle.
ACSL nous permet de créer des versions nommées de ces prédicats, à la manière
d'une fonction booléenne en C par exemple. À la différence près tout de même que
les prédicats (ainsi que les fonctions logiques que nous verrons par la suite)
doivent être pures, c'est-à-dire qu'elles ne peuvent pas produire d'effets de
bords en modifiant des valeurs pointées par exemple.



Ces prédicats peuvent prendre un certain nombre de paramètres. En plus de cela,
ils peuvent également recevoir un certain nombre de \textit{labels} (au sens C du terme)
qui vont permettre d'établir des relations entre divers points du code.



\levelThreeTitle{Syntaxe}


Les prédicats sont, comme les spécifications, introduits au travers
d'annotations. La syntaxe est la suivante :



\begin{CodeBlock}{c}
  /*@
  predicate nom_du_predicat { Lbl0, Lbl1, ..., LblN }(type0 arg0, type1 arg1, ..., typeN argN) =
  //une relation logique entre toutes ces choses.
  */
\end{CodeBlock}



Nous pouvons par exemple définir le prédicat nous disant qu'un entier en mémoire n'a
pas changé entre deux points particuliers du programme :


\CodeBlockInput[1][4]{c}{unchanged-loc.c}


\begin{Warning}
  Gardez bien en mémoire que le passage se fait, comme en C, par valeur. Nous ne
  pouvons pas écrire ce prédicat en passant directement \CodeInline{i} :

  \begin{CodeBlock}{c}
    /*@
    predicate unchanged{L0, L1}(int i) =
    \at(i, L0) == \at(i, L1);
    */
  \end{CodeBlock}

  Car \CodeInline{i} est juste une copie de la variable reçue en paramètre.
\end{Warning}


Nous pouvons par exemple vérifier ce petit code :


\CodeBlockInput[6][15]{c}{unchanged-loc.c}



Nous pouvons également regarder les buts générés par WP et constater que,
même s'il subit une petite transformation syntaxique, le prédicat n'est pas
déroulé par WP. Ce sera au prouveur de déterminer s'il veut raisonner avec.


\image{unchanged-loc}


Comme nous l'avons dit plus tôt, une des utilités des prédicats et fonctions (que
nous verrons un peu plus tard) est de rendre plus lisible nos spécifications et
de les factoriser. Un exemple est d'écrire un prédicat pour la validité en
lecture/écriture d'un tableau sur une plage particulière. Cela nous évite d'avoir
à réécrire l'expression en question qui est moins compréhensible au premier
coup d’œil.



\CodeBlockInput[3]{c}{search.c}



Dans cette portion de spécification, le \textit{label} pour les prédicats n'est pas
précisé, ni pour leur création, ni pour leur utilisation. Pour la création,
Frama-C va automatiquement en ajouter un dans la définition du prédicat.
Pour l'appel, le \textit{label} passé sera implicitement \CodeInline{Here}. La non-déclaration
du \textit{label} dans la définition n'interdit pour autant pas de passer explicitement
un \textit{label} lors de l'appel.



Bien entendu, les prédicats peuvent être déclarés dans des fichiers \textit{headers} afin
de produire une bibliothèque d'utilitaires de spécifications par exemple.


\levelFourTitle{Surcharger des prédicats}

Il est possible de surcharger les prédicats tant que le types des paramètres
est différents ou que le nombre de paramètres change. Par exemple, nous
pouvons redéfinir \CodeInline{valid\_range\_r} comme un prédicat qui prend
en paramètre à la fois le début et la fin de la plage considérée. Ensuite,
nous pouvons écrire une surcharge qui utilise le prédicat précédent pour le
cas particulier des plages qui commencent à 0 :


\CodeBlockInput[3]{c}{search-overload.c}



\levelThreeTitle{Abstraction}


Une autre utilité importante des prédicats est de définir l'état logique de nos
structures quand les programmes se complexifient. Nos structures doivent
généralement respecter un invariant (encore) que chaque fonction de manipulation
devra maintenir pour assurer que la structure sera toujours utilisable et
qu'aucune fonction ne commettra de bavure.



Cela permet notamment de faciliter la lecture de spécifications. Par exemple, nous
pourrions poser les spécifications nécessaires à la sûreté d'une pile de taille
limitée. Et cela donnerait quelque chose comme :


\CodeBlockInput{c}{stack.c}


(Notons qu'ici, nous ne fournissons pas la définition des prédicats car ce n'est
pas l'objet de cet exemple. Le lecteur pourra considérer ceci comme un exercice)

Ici la spécification n'exprime pas de propriétés fonctionnelles. Par exemple,
rien ne nous spécifie que lorsque nous faisons un \textit{push} d'une valeur puis que nous
demandons \textit{top}, nous auront effectivement cette valeur. Mais elle nous donne
déjà tout ce dont nous avons besoin pour produire un code où, à défaut d'avoir
exactement les résultats que nous attendons (des comportements tels que « si
j'empile une valeur $v$, l'appel à \CodeInline{top} renvoie la valeur $v$ », par exemple), nous
pouvons au moins garantir que nous n'avons pas d'erreur d'exécution (à condition
de poser une spécification correcte pour nos prédicats et de prouver les fonctions
d'utilisation de la structure).


\levelThreeTitle{Exercices}


\levelFourTitle{Les jours du mois}


Reprendre la solution de l'exercice~\ref{l4:contract-modularity-ex-days-of-month},
écrire un prédicat pour exprimer qu'une année est bissextile et modifier le contrat
de façon à l'utiliser.


\levelFourTitle{Caractères alpha-numériques}


Reprendre la solution de l'exercice~\ref{l4:contract-modularity-ex-alpha-num} à
propos des caractères alpha-numériques. Écrire des prédicats pour exprimer qu'un
caractères est une majuscule, une autre pour les minuscules, et un dernier pour les
chiffres. Adapter les contrats des différentes fonctions en les utilisant.


\levelFourTitle{Maximum de 3 valeurs}


La fonction suivante retourne le maximum de 3 valeurs d'entrée :


\CodeBlockInput{c}{ex-3-max-3.c}


Écrire un prédicat qui exprime qu'une valeur est l'une de trois valeurs pointées
à un état mémoire donné :


\begin{CodeBlock}{c}
  /*@
  predicate one_of{L}(int value, int *a, int *b, int *c) =
  // ...
  */
\end{CodeBlock}


Utiliser la notation ensembliste. Écrire un contrat pour la fonction et prouver
qu'elle le vérifie.



\levelFourTitle{Recherche dichotomique}
\label{l4:acsl-properties-predicates-ex-bsearch}


Reprendre la solution de l'exercice~\ref{l4:statements-loops-ex-bsearch} à propos
de la recherche dichotomique utilisant des indices non-signés. Écrire un prédicat
qui exprime qu'une plage de valeur est triée entre \CodeInline{begin} et
\CodeInline{end} (exclu). Écrire une surcharge de ce prédicat pour rendre
\CodeInline{begin} optionnel avec une valeur par défaut à $0$. Écrire un prédicat
qui vérifie si un élément est dans un tableau pour des indices compris entre
\CodeInline{begin} et \CodeInline{end} (exclu), à nouveau, écrire une surcharge
qui rend la première borne optionnelle.


Utiliser ces prédicats pour simplifier le contrat de la fonction. Notons que les
clauses \CodeInline{assumes} des deux comportements devraient être modifiées.



\levelFourTitle{Chercher et remplacer}


Reprendre l'exemple~\ref{l4:statements-loops-ex-search-and-replace}, à propos de
la fonction « chercher et remplacer ». Écrire des prédicats qui exprime qu'une plage
de valeurs dans un tableau pour des indices compris entre \CodeInline{begin} et
\CodeInline{end} (exclu), les valeurs :

\begin{itemize}
  \item restent inchangées entre deux états mémoire,
  \item sont remplacées par une valeur si elles sont égales à une valeur donnée,
        sinon laissée inchangée.
\end{itemize}


Surcharger les deux prédicats de manière à rendre la première borne optionnelle.
Utiliser les prédicats obtenus pour simplifier le contrat et l'invariant de
boucle de la fonction.
