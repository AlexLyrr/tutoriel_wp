
Les lemmes sont des propriétés générales à propos des prédicats ou encore des 
fonctions. Une fois ces propriétés exprimées, la preuve peut être réalisée en
isolation du reste de la preuve du programme, en utilisant des prouveurs 
automatiques ou (plus souvent) des prouveurs interactifs. Une fois la preuve
réalisée, la propriété énoncée peut être utilisée directement par les prouveurs
automatiques sans que cela ne nécessite d'en réaliser la preuve à nouveau. Par
exemple, si nous énonçons un lemme $L$ qui dit que $P \Rightarrow Q$, et dans
une autre preuve nous avons besoin de prouver $Q$ alors que nous savons déjà
que $P$ est vérifiée, nous pouvons utiliser directement le lemme $L$ pour
conclure sans avoir besoin de faire à nouveau le raisonnement complet qui
amène de $P$ à $Q$.



Dans la section précédent, nous avons dit que les fonctions récursives logiques
peuvent rendre les preuves plus difficile pour les solveurs SMT. Dans un tel cas,
les lemmes peuvent nous aider. Nous pouvons écrire nous même les preuves qui 
nécessitent de raisonner par induction pour certaines propriétés qui nous
énonçons comme des lemmes, et ces lemmes peuvent ensuite être utilisés 
efficacement par les prouveurs pour effectuer les autres preuves à propos du
programme.


\levelThreeTitle{Syntaxe}


Une nouvelle fois, nous les introduisons à l'aide d'annotations ACSL. La syntaxe
utilisée est la suivante :



\begin{CodeBlock}{c}
/*@
  lemma name_of_the_lemma { Label0, ..., LabelN }:
    property ;
*/
\end{CodeBlock}



Cette fois les propriétés que nous voulons exprimer ne dépendent pas de 
paramètres reçus (hors de nos \textit{labels} bien sûr). Ces propriétés seront donc 
exprimées sur des variables quantifiées. Par exemple, nous pouvons poser ce 
lemme qui est vrai, même s'il est trivial :



\begin{CodeBlock}{c}
/*@
  lemma lt_plus_lt:
    \forall integer i, j ; i < j ==> i+1 < j+1;
*/
\end{CodeBlock}



Cette preuve peut être effectuée en utilisant WP. La propriété est bien sûr 
trivialement prouvée par Qed.



\levelThreeTitle{Exemple : propriété fonction affine}


Nous pouvons par exemple reprendre nos fonctions affines et exprimer quelques 
propriétés intéressantes à leur sujet :



\CodeBlockInput[20][30]{c}{linear-0.c}



Pour ces preuves, il est fort possible qu'Alt-ergo ne parvienne pas à les 
décharger. Dans ce cas, le prouveur Z3 devrait, lui, y arriver. Nous pouvons 
ensuite construire cet exemple de code :



\CodeBlockInput[32][57]{c}{linear-0.c}



Si nous ne renseignons pas les lemmes mentionnés plus tôt, il y a peu de chances 
qu'Alt-ergo réussisse à produire la preuve que \CodeInline{fmin} est inférieur à \CodeInline{fmax}.
Avec ces lemmes présents en revanche, il y parvient sans problème car cette 
propriété est une simple instance du lemme \CodeInline{ax\_b\_monotonic\_pos}, la preuve 
étant ainsi triviale car notre lemme nous énonce cette propriété comme étant vraie.
Notons que sur cette version généralisée, Z3 sera probablement plus efficace pour
prouver l'absence d'erreurs à l'exécution.



\levelThreeTitle{Exemple: tableaux et labels}


Plus tard dans ce tutoriel, nous allons voir certains types de définitions à propos
desquels il est parfois difficile de raisonner pour les solveurs SMT quand des
modifications ont lieu en mémoire. Par conséquent, nous aurons souvent besoin de
lemmes pour indiquer les relations qui existent à propos du contenu de la mémoire
entre deux labels.


Pour le moment, illustrons cela avec un exemple simple. Considérons les deux
prédicats suivant :


\CodeBlockInput[1][8]{c}{unchanged-sorted.c}


Nous pourrions par exemple vouloir énoncer que lorsqu'un tableau est trié, et que
la mémoire est modifiée (créant donc un nouvel état mémoire), mais que le contenu du
tableau reste inchangé, alors le tableau est toujours trié. Cela peut être réalisé
avec le lemme suivant :


\CodeBlockInput[10][16]{c}{unchanged-sorted.c}


Nous énonçons ce lemme pour deux labels \CodeInline{L1} et \CodeInline{L2}, et
exprime que pour toute plage de valeurs dans un tableau, si elle est triée au label
\CodeInline{L1}, et reste inchangée depuis \CodeInline{L1} vers \CodeInline{L2},
alors elle reste triée au label \CodeInline{L2}.


Notons qu'ici, cette propriété est facilement prouvée par les prouveurs SMT. Nous
verrons plus tard des exemples pour lesquels il n'est pas si simple d'obtenir une
preuve.



\levelThreeTitle{Exercices}


\levelFourTitle{Propriété de la multiplication}


Écrire un lemme qui énonce que pour trois entiers $x$, $y$ et $z$, si $x$ est
plus grand ou égal à $0$, si $z$ est plus grand ou égal à $y$, alors $x * z$ 
est  plus grand ou égal à $x * y$.


Ce lemme ne sera probablement pas prouvé par les solveurs SMT. En revanche, en
demandant une preuve avec Coq, la tactique par défaut devrait la décharger
automatiquement.


\levelFourTitle{Localement trié vers globalement trié}
\label{l4:acsl-properties-lemmas-lsorted-gsorted}


Le programme suivant contient une fonction qui demande à ce qu'un tableau soit
trié au sens que chaque élément doit plus petit ou égal à l'élément qui le suit
puis appelle la fonction de recherche dichotomique.


\CodeBlockInput[59][81]{c}{ex-2-sorted-link.c}


Pour cet exercice, reprendre la solution de 
l'exercice~\ref{l4:acsl-properties-predicates-ex-bsearch} à propos de la recherche
dichotomique. La précondition de cette recherche peut sembler plus forte que celle
reçue par la précondition de de \CodeInline{bsearch\_callee}. La première demande
chaque paire d'éléments d'être ordonnée, la seconde simplement que chaque élément
soit inférieur à celui qui le suit. Cependant, la seconde implique la première.
Écrire un un lemme qui énonce que si \CodeInline{element\_level\_sorted} est vraie
pour un tableau, \CodeInline{sorted} est vraie aussi. Ce lemme ne sera probablement
pas prouvé par un solveur SMT, toutes les autres propriétés devraient être prouvées
automatiquement.


Une solution et la preuve Coq du lemme sont disponibles sur le GitHub de ce tutoriel.


\levelFourTitle{Somme des N premiers entiers}
\label{l4:acsl-properties-lemmas-n-first-ints}


Reprendre la solution de l'exercice~\ref{l4:acsl-properties-functions-n-first-ints}
à propos de la somme des N premiers entiers. Écrire un lemme qui énonce que la valeur
calculée par la fonction logique récursive qui permet la spécifications de la somme des
N premiers entiers est $n*(n+1)/2$. Ce lemme ne sera pas prouvé par un solveur SMT.


Une solution et la preuve Coq du lemme sont disponibles sur le GitHub de ce tutoriel.


\levelFourTitle{Transitivité d'un glissement d'éléments}
\label{l4:acsl-properties-lemmas-shift-trans}


Le programme suivant est composé de deux fonctions. La première est
\CodeInline{shift\_array} et permet de faire glisser des éléments dans un tableau
d'un certain nombre de cellule (nommé \CodeInline{shift}). La seconde effectue deux
glissements successifs des éléments d'un même tableau.



\CodeBlockInput{c}{ex-4-shift-transitivity.c}


Compléter les prédicats \CodeInline{shifted} et \CodeInline{unchanged} (le second
devrait utiliser le premier). Compléter le contrat de la fonction
\CodeInline{shift\_array} et prouver sa correction avec WP.


Exprimer deux lemmes à propos de la propriété \CodeInline{shifted}.


Le premier, nommé \CodeInline{shift\_ptr}, doit énoncer que déplacer une plage de
valeur \CodeInline{fst+s1} à \CodeInline{last+s1} dans un tableau \CodeInline{array}
d'un décalage \CodeInline{s2} est équivalent à déplacer une plage de valeurs
\CodeInline{fst} à \CodeInline{last} pour la position mémoire \CodeInline{array+s1}
avec un décalage \CodeInline{s2}.


Le second doit énoncer que quand les éléments d'un tableau sont déplacés une première
fois avec un décalage \CodeInline{s1} puis une seconde fois avec un décalage
\CodeInline{s2}, alors le déplacement final correspond à un décalage avec un
déplacement \CodeInline{s1+s2}.


Le lemme \CodeInline{shift\_ptr} ne sera probablement pas prouvé par un SMT solveur.
Nous fournissons une solution et la preuve Coq de ce lemme sur le GitHub de ce livre.
Les propriétés restantes doivent être prouvées automatiquement.


\levelFourTitle{Déplacement d'une plage triée}


Le programme suivant est composé de deux fonctions. La fonction
\CodeInline{shift\_and\_search} déplace les éléments d'un tableau puis effectue
une recherche dichotomique.


\CodeBlockInput{c}{ex-5-shift-sorted.c}


Reprendre la solution de la recherche dichotomique de 
l'exercice~\ref{l4:acsl-properties-predicates-ex-bsearch}. Modifier cette
recherche et sa spécification de façon à ce que la fonction permette de
chercher dans toute plage triée de valeurs. La preuve doit toujours fonctionner.


Reprendre également la fonction prouvée \CodeInline{shift\_array} de l'exercice
précédent.


Compléter le contrat de la fonction \CodeInline{shift\_and\_search}. La 
précondition qui demande à ce que le tableau soit trié avant la recherche ne 
sera pas validée. Compléter le lemme \CodeInline{shifted\_still\_sorted} qui
doit énoncer que si une plage de valeur est triée à un label, puis déplacée,
alors elle reste triée.


Ce lemme ne sera probablement pas prouvé par un solveur SMT. Une solution et
sa preuve Coq sont disponibles sur le GitHub de ce livre.
