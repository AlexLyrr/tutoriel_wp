
Il est plus que temps d'entamer les hostilités. Contrairement aux tutoriels 
sur divers langages, nous commencerons par les fonctions. D'abord parce 
qu'il faut savoir en écrire avant d'entamer un tel tutoriel et surtout 
parce que cela permettra rapidement de produire des exemples simples que
nous pouvons vérifier à l'aide de nos outils.



Au contraire, après le travail sur les fonctions, nous entamerons les notions 
les plus simples comme les affectations ou les structures conditionnelles pour 
comprendre comment fonctionne l'outil sous le capot.



Pour prouver qu'un code est valide, il faut d'abord pouvoir spécifier
ce que nous attendons de lui. La preuve de programme consiste ensuite à
s'assurer que le code que nous avons écrit effectue bien une action conforme à
la spécification. Comme mentionné plus tôt dans le tutoriel, la 
spécification de code pour Frama-C est faite avec le langage ACSL, celui-ci 
nous permet (mais pas seulement, comme nous le verrons dans la suite) de poser
un contrat pour chaque fonction.



\begin{levelTwo}
  {Définition d'un contrat}
  {contract}
\end{levelTwo}

\begin{levelTwo}
  {De l'importance d'une bonne spécification}
  {well-specified}
\end{levelTwo}

\begin{levelTwo}
  {Comportements}
  {behaviors}
\end{levelTwo}

\begin{levelTwo}
  {Modularité du WP}
  {modularity}
\end{levelTwo}

\horizontalLine
\newpage


Pendant cette partie, nous avons vu comment spécifier les fonctions par 
l'intermédiaire de leurs contrats, à savoir leurs pré et postconditions, ainsi
que quelques fonctionnalités offertes par ACSL pour exprimer ces propriétés. 
Nous avons également vu pourquoi il est important d'être précis dans la 
spécification et comment l'introduction des comportements nous permet d'avoir
des spécifications plus compréhensibles et moins sujettes aux erreurs.



En revanche, nous n'avons pas encore vu un point important : la spécification 
des boucles. Avant d'entamer cette partie, nous devrions regarder plus 
précisément comment fonctionne l'outil WP.
