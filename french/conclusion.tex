\begin{Quotation}[Jean-Louis Aubert, \textit{Bleu Blanc Vert}, 1989]
Voilà, c'est fini ...
\end{Quotation}



... pour cette introduction à la preuve de programmes avec Frama-C et WP.



Tout au long de ce tutoriel, nous avons pu voir comment utiliser ces outils
pour spécifier ce que nous attendons de nos programmes et vérifier que le code que
nous avons produit correspond effectivement à la spécification que nous en 
avons faite. Cette spécification passe par l'annotation de nos fonctions avec 
le contrat qu'elle doit respecter, c'est-à-dire les propriétés que doivent
respecter les entrées de la fonction pour garantir son bon fonctionnement et 
les propriétés que celle-ci s'engage à assurer sur les sorties en retour.



À partir de nos programmes spécifiés, WP est capable de produire un 
raisonnement nous disant si oui, ou non, notre programme répond au besoin 
exprimé. La forme de raisonnement utilisée étant complètement modulaire, elle 
nous permet de prouver les fonctions une à une et de les composer. WP ne 
comprend pas, à l'heure actuelle, l'allocation dynamique. Une fonction qui en 
utiliserait ne pourrait donc pas être prouvée.



Mais même sans cela, une large variété de fonctions n'ont pas besoin 
d'effectuer d'allocation dynamique, travaillant donc sur des données déjà 
allouées. Et ces fonctions peuvent ensuite être appelées avec l'assurance 
qu'elles font correctement leur travail. Si nous ne voulons pas prouver le 
code appelant, nous pouvons toujours écrire quelque chose comme cela :



\begin{CodeBlock}{c}
/*@
  requires some_properties_on(a);
  requires some_other_on(b);

  assigns ...
  ensures ...
*/
void ma_fonction(int* a, int b){
  //je parle bien du "assert" de "assert.h"
  assert(/*properties on a*/ && "must respect properties on a");  
  assert(/*properties on b*/ && "must respect properties on b");
}
\end{CodeBlock}



Ce qui nous permet ainsi de bénéficier de la robustesse de notre fonction tout en
ayant la possibilité de débugger un appel incorrect dans un code que nous ne 
voulons ou pouvons pas prouver.



Écrire les spécifications est parfois long voire fastidieux. Les constructions 
de plus haut niveau d'ACSL (prédicats, fonctions logiques, axiomatisations) 
permettent d'alléger un peu ce travail, de la même manière que nos langages de
programmation nous permettent de définir des types englobant d'autres types et
des fonctions appelant des fonctions, nous menant vers le programme final. Mais
malgré cela, l'écriture de spécification dans un langage formel quel qu'il soit
représente une tâche ardue.



Cependant, cette étape de \textbf{formalisation} du besoin est \textbf{très importante}. 
Concrètement, une telle formalisation est, à bien y réfléchir, un travail que 
tout développeur devrait s'efforcer de faire. Et pas seulement quand il cherche 
à prouver son programme. Même la production de tests pour une fonction 
nécessite de bien comprendre son but si nous voulons tester ce qui est nécessaire 
et uniquement ce qui est nécessaire. Et écrire les spécifications dans un 
langage formel est une aide formidable (même si elle peut être frustrante, 
reconnaissons le) pour avoir des spécifications claires.



Les langages formels, proches des mathématiques, sont précis. Les mathématiques
ont cela pour elles. Qu'y a-t-il de plus terrible que lire une spécification 
écrite en langue naturelle pure beurre, avec toute sa panoplie de phrase à 
rallonge, de verbes conjugués à la forme conditionnelle, de termes imprécis, 
d'ambiguïtés, compilée dans des documents administratifs de centaines de pages,
et dans laquelle nous devons chercher pour déterminer « bon alors cette fonction, 
elle doit faire quoi ? Et qu'est ce qu'il faut valider à son sujet ? ».



Les méthodes formelles ne sont, à l'heure actuelle, probablement pas assez 
utilisées, parfois par méfiance, parfois par ignorance, parfois à cause de 
préjugés datant des balbutiements des outils, il y a 20 ans. Nos outils
évoluent, nos pratiques dans le développement changent, probablement plus
vite que dans de nombreux autres domaines techniques. Ce serait probablement
faire un raccourci bien trop rapide que dire que ces outils ne pourront 
jamais être mis en œuvre quotidiennement. Après tout nous voyons chaque jour
à quel point le développement est différent aujourd'hui par rapport à il y a
10 ans, 20 ans, 40 ans. Et pouvons à peine imaginer à quel point il sera 
différent dans 10 ans, 20 ans, 40 ans.



Ces dernières années, les questions de sûreté et de sécurité sont devenues de
plus en plus présentes et cruciales. Les méthodes formelles connaissent également
de fortes évolutions et leurs apports pour ces questions sont très appréciés. 
Par exemple, \externalLink{ce lien}{http://sfm.seas.harvard.edu/report.html} mène vers
un rapport d'une conférence sur la sécurité ayant rassemblé des acteurs du monde
académique et industriel, dans lequel nous pouvons lire :



\begin{Quotation}[Formal Methods for Security, 2016]
Adoption of formal methods in various areas (including verification of hardware
and embedded systems, and analysis and testing of software) has dramatically 
improved the quality of computer systems.  We anticipate that formal methods 
can provide similar improvement in the security of computer systems.

...

\textbf{Without broad use of formal methods, security will always remain fragile.}
\end{Quotation}



\begin{Spoiler}
Traduction :

L'adoption des méthodes formelles dans différents domaines (notamment la 
vérification du matériel et des systèmes embarqués, et l'analyse et le test
de logiciel) a fortement amélioré la qualité des systèmes informatiques. 
Nous nous attendons à ce que les méthodes formelles puissent fournir des 
améliorations similaires pour la sécurité des systèmes informatiques.

...

\textbf{Sans une utilisation plus large des méthodes formelles, la sécurité sera
toujours fragile.}
\end{Spoiler}


\levelTwoTitle{Pour aller plus loin}


\levelThreeTitle{Avec Frama-C}


Frama-C propose divers moyen d'analyser les programmes. Dans les outils les
plus courants qui sont intéressants à connaître d'un point de vue vérification
statique et dynamique pour l'évaluation du bon fonctionnement d'un programme :



\begin{itemize}
\item l'analyse par interprétation abstraite avec 
\externalLink{Value}{http://frama-c.com/value.html},
\item la transformation d'annotation en vérification à l'exécution avec 
\externalLink{E-ACSL}{http://frama-c.com/eacsl.html}.
\end{itemize}


Le but de la première est de calculer les domaines des différentes variables à
tous les points de programme. Quand nous connaissons précisément ces domaines,
nous sommes capables de déterminer si ces variables peuvent provoquer des erreurs.
Par contre cette analyse est effectuée sur la totalité du programme et pas 
modulairement, elle est par ailleurs fortement dépendante du type de domaine 
utilisé (nous n'entrerons pas plus dans les détails ici) et elle conserve moins
bien les relations entre les variables. En compensation, elle est vraiment 
complètement automatique (modulo les entrées du programme), il n'y a même pas
besoin de poser des invariants de boucle ! La partie plus « manuelle » sera de
déterminer si oui ou non les alarmes lèvent des vrais erreurs ou si ce sont de
fausses alarmes.



La seconde analyse permet de générer depuis un programme d'origine, un nouveau
programme où les assertions sont transformées en vérification à l'exécution. Si
les assertions échouent, le programme échoue. Si elles sont valides, le programme
a le même comportement que s'il n'y avait pas d'assertions. Un exemple 
d'utilisation est d'utiliser l'option \CodeInline{-rte} de Frama-C pour générer les 
vérifications d'erreur d'exécution comme assertion et de générer le programme de 
sortie qui va contenir les vérifications en question.



Il existe divers autres greffons pour des tâches très différentes :



\begin{itemize}
\item analyse d'impact de modifications,
\item analyse du flot de données pour le parcourir efficacement,
\item ...
\end{itemize}


Et finalement, la dernière possibilité qui va motiver l'utilisation de Frama-C,
c'est la possibilité de développer ses propres greffons d'analyse, à partir de
l'API fournie au niveau du noyau. Beaucoup de tâches peuvent être réalisées par
l'analyse du code source et Frama-C permet de forger différentes analyses.



\levelThreeTitle{Avec la preuve déductive}


Tout au long du tutoriel nous avons utilisé WP pour générer des obligations de 
preuve à partir de programmes et de leurs spécifications. Par la suite nous avons
utilisé des solveurs automatiques pour assurer que les propriétés en question sont
bien vérifiées.



Lorsque nous passons par d'autres solveurs que Alt-Ergo, le dialogue avec ceux-ci
est assuré par une traduction vers le langage de Why3 qui va ensuite faire le pont
avec les prouveurs automatiques. Mais ce n'est pas la seule manière d'utiliser 
Why3. Celui-ci peut tout à fait être utilisé seul pour écrire et prouver des
algorithmes. Il embarque notamment un ensemble de théories déjà présentes pour un
certain nombre de structures de données.



Il y a un certain nombre de preuves qui ne peuvent être déchargées par les 
prouveurs automatiques. Dans ce genre de cas, nous devons nous rabattre sur de la 
preuve interactive. WP comme Why3 peuvent extraire vers Coq, et il est très
intéressant d'étudier ce langage, il peut par exemple servir à constituer des 
bibliothèques de lemmes génériques et prouvés. À noter que Why3 peut également
extraire ses obligations vers Isabelle ou PVS qui sont d'autres assistants de
preuve.



Finalement, il existe d'autres logiques de programmes, comme la logique de 
séparation ou les logiques pour les programmes concurrents. Encore une fois ce
sont des notions intéressantes à connaître, elles peuvent inspirer la manière dont
nous spécifions nos programmes pour la preuve avec WP, elles pourraient également
donner lieu à de nouveaux greffons pour Frama-C. Bref, tout un monde de méthodes à
explorer.
