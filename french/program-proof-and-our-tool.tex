
Le but de cette première partie est, dans une première section d'introduire
rapidement en quoi consiste la preuve de programmes sans entrer dans les 
détails. Puis dans une seconde section de donner les quelques instructions 
nécessaires pour mettre en place Frama-C et les quelques prouveurs 
automatiques dont nous auront besoin pendant le tutoriel.


\begin{levelTwo}
  {Preuve de programmes}
  {program-proof}
\end{levelTwo}


\begin{levelTwo}
  {Frama-C}
  {frama-c}
\end{levelTwo}


\horizontalLine

\newpage

Voilà. Nos outils sont installés et prêts à fonctionner.



Le but de cette partie, en plus de l'installation de nos outils de travail
pour la suite, est de faire ressortir deux informations claires :



\begin{itemize}
\item la preuve est un moyen d'assurer que nos programmes n'ont que des 
comportements conformes à notre spécification sans les exécuter ;
\item il est toujours de notre devoir d'assurer que cette spécification est
correcte.
\end{itemize}
