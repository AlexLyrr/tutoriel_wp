
Le but de cette première partie est, dans une première section d'introduire
rapidement en quoi consiste la preuve de programmes sans entrer dans les 
détails. Puis dans une seconde section de donner les quelques instructions 
nécessaires pour mettre en place Frama-C et les quelques prouveurs 
automatiques dont nous auront besoin pendant le tutoriel.



\levelTwoTitle{Preuve de programmes}

\levelThreeTitle{Ensure program reliability}


It is often difficult to ensure that our programs have only correct
behaviors. Moreover, it is already complex to establish a good criteria
that makes us confident enough to say that a program correctly works:



\begin{itemize}
\item
  beginners simply ``try'' to use their programs and considers that
  these programs work if they do not crash (which is not a really good
  indicator in C language),
\item
  more advanced developers establish some test cases for which they know
  the expected result and compare the output they obtain,
\item
  most of companies establish complete test bases, that covers as much
  code as they can ; which are systematically executed on their code.
  Some of them apply test driven development,
\item
  in critical domains, such as aerospace, railway or armament, every
  code needs to be certified using standardized processes with very
  strict criteria about coding rules and code covering by the test.
\end{itemize}

In all these ways to ensure that a program produces only what we expect
it to produce, a word appears to be common: \emph{test}. We \emph{try}
different inputs in order to isolate cases that are problematic. We
provide inputs that we \emph{estimate to be representative} of the
actual use of the program (note that unexpected use case are often not
considered whereas there are generally the most dangerous ones) and we
verify that the results we get are correct. But we cannot test
\emph{everything}. We cannot try \emph{every} combination of
\emph{every} possible input of a program. It is then quite hard to
choose good tests.


The goal of program proof is to ensure that, for any input provided to a
given program, if it respects the specification, then the program will
only well-behave. However, since we cannot test everything, we will
formally, mathematically, establish a proof that our software can only
exhibit specified behaviors, and that runtime-errors are not part of
these behaviors.



A very well-known quote from Dijkstra precisely express the difference
between test and proof :



\begin{Quotation}[Dijkstra]
Program testing can be used to show the presence of bugs, but never to show 
their absence!
\end{Quotation}


\levelFourTitle{The developer's Holy Grail: the bug-free software}


Every time we can read news about attacks on computer systems, or
viruses, or bugs leading to crashes in well known apps, there is always
the one same comment ``the bug-free/perfectly secure program does not
exist''. And, if this sentence is quite true, it is a bit misunderstood.



Apart from the difference between safety and security (that we can very
vaguely define by the existence of a malicious entity), we do not really
define what we mean by ``bug-free''. Creating software always rely at
least on two steps: we establish a specification of what we expect from
the program and then we produce the source code of the program that must
respect this specification. Both of these steps can lead to the
introduction of errors.



In this tutorial, we will show how we can prove that an implementation
verify a given specification. But what are the arguments of program
proof, compared to program testing? First, the proof is complete, it
cannot forget some corner case if the behavior is specified (program
test cannot be complete, being exhaustive would be far too costly).
Then, the obligation to formally specify with a logic formulation
requires to exactly understand what we have to prove about our program.



One could cynically say that program proofs shows that ``the
implementation does not contain bugs that do not exist in the
specification''. But, well, it is already a big step compared to ``the
implementation does not contain too many bugs that do not exist in the
specification''. Moreover, there also exist approaches that allow to
analyze specifications to find errors or under-specified behaviors. For
example, with model checking techniques, we can create an abstract model
from the specification and produce the set of states that can be reach
according to this model. By characterizing what is an error state, we
can determine if reachable states are error states.



\levelThreeTitle{A bit of context}


Formal methods, as we name them, allow in computer science to
rigorously, mathematically, reason about programs. There exist a lot of
formal methods that can take place at different levels from program
design to implementation, analysis and validation, and for all system
that allow to manipulate information.



Here, we will focus on a method that allows to formally verify that our
programs have only correct behaviors. We will use tools that are able to
analyze a source code and to determine whether a program correctly
implements what we want to express. The analyzer we will use provides a
static analysis, that we can oppose to dynamic analysis.



In static analysis, the analyzed program is not executed, we reason on a
mathematical model of the states it can reach during its execution. On
the opposite, dynamic analyses such as program testing, require to
execute the analyzed source code. Note that there exist formal dynamic
analysis methods, for example automatic test generation, or code
monitoring techniques that allows to instrument a source code to verify
some properties about it during execution (correct memory use, for
example).



Talking about static analyses, the model we use can be more or less
abstract depending on the techniques, it is always an approximation of
possible states of the program. The more the approximation is precise,
the more the model is concrete, the more the approximation is vague, the
more it is abstract.



To illustrate the difference between concrete and abstract model, we can
have a look to the model of simple chronometer. A very abstract model of
a chronometer could be the following:



\image{model-en.svg}[A very abstract model of a chronometer]


We have a model of the behavior of our chronometer with the different
states it can reach according to the different actions we can perform.
However, we do not have modeled how these states are depicted inside the
program (is this a C enumeration? a particular program point in the
source code?), nor how is modeled the time computation (a single
variable? multiple ones?). It would then be difficult to specify
properties about our program. We could add some information:


\begin{itemize}
\item State stopped at 0 : time = 0s
\item State running : time > 0s
\item State stopped : time > 0s
\end{itemize}



Which gives us a more concrete model but that is still not precise
enough to ask interesting questions like: ``is it possible to be in the
state stopped and that time is still updated?'', as we do not model how
the time measurement is updated by the chronometer.



On the opposite, with the source code of the program, we have a concrete
model of the chronometer. The source code expresses the behavior of the
chronometer since it will allow us to produce the executable. But this
is still not the more concrete model! For example, the executable in
machine code format, that we obtain after compilation, is far more
concrete than our program.



The more a model is concrete, the more it precisely describes the
behavior of our program. The source code more precisely describes the
behavior than our diagram, but it is less precise than the machine code.
However, the more the model is precise, the more it is difficult to have
a global view of the defined behavior. Our diagram is understandable in
a blink of an eye, the source code requires more time, and for the
executable \ldots{} Every single person that has already opened an
executable with a text editor by error knows that it is not really
pleasant to read\textsuperscript{\ref{footnote:1}}



When we create an abstraction of a system, we approximate it, in order
to limit the knowledge we have about it and ease our reasoning. A
constraint we must respect, if we want our analysis to be correct, is to
never under-approximate behaviors: we would risk to remove a behavior
that contains an error. However, when we over-approximate it, we can add
behaviors that cannot happen, and if we add to many of them, we could
not be able to prove our program is correct, since some of them could be
faulty behaviors.



In our case, the model is quite concrete. Every type of instruction, of
control structure, is associated to a precise semantics, a model of its
behavior in a pure logic, mathematical, world. The logic we use here is
a variant of the Hoare logic, adapted to the C language and all its
complex subtleties (which makes this model concrete).

\footnotetext[1]{\label{footnote:1}
  There also exists formal methods which are
  interested in understanding how executable machine code work, for
  example in order to understand what malwares do or to detect security
  breaches introduced during compilation.}



\levelThreeTitle{Hoare triples}


Hoare logic is a program formalization method proposed by Tony Hoare in
1969 in a paper entitled \emph{An Axiomatic Basis for Computer
Programming}. This method defines:



\begin{itemize}
\item   axioms, that are properties we admit, such as ``the skip action does
  not change the program state'',
\item   rules to reason about the different allowed combinations of actions,
  for example ``the skip action followed by the action A'' is equivalent
  to ``the action A''.
\end{itemize}


The behavior of the program is defined by what we call ``Hoare
triples'':




\begin{center}
$\{P\} C \{Q\}$


\end{center}


Where $P$ and $Q$ are predicates, logic formulas that express
properties about the memory at particular program points. $C$ is a
list of instructions that defines the program. This syntax expresses the
following idea: ``if we are in a state where $P$ is verified, after
executing $C$ and if $C$ terminates, then $Q$ is verified for the
new state of the execution''. Put in another way, $P$ is a sufficient
precondition to ensure that $C$ will bring us to the postcondition
$Q$. For example, the Hoare triples that corresponds to the skip
action is the following one:




\begin{center}
$\{P\}$ \textbf{skip} $\{P\}$


\end{center}


When we do nothing, the postcondition is the precondition.



Along this tutorial, we will present the semantics of different program
constructs (conditional blocks, loops, etc) using Hoare logic. So, we
will not enter into details now since we will work on it later. It is
not necessary to memorize these notions nor to understand all the
theoretical background, but it is still useful to have some ideas about
the way our tool works.



All of this gives us the basics that allows us to say ``here is what
this action does'' but it does not give us anything to mechanize a
proof. The tool we will use rely on a technique called weakest
precondition calculus.



\levelThreeTitle{Weakest precondition calculus}


The weakest precondition calculus is a form of predicate transform
semantics proposed by Dijkstra in 1975 in \emph{Guarded commands,
non-determinacy and formal derivation of programs}.

This sentence can appear complex but the actual meaning is in fact quite
simple. We have seen before that Hoare logic gives us rules that explain
the behavior of the different actions of a program, but it does not say
how to apply these rules to establish a complete proof of the program.

Dijkstra reformulate the Hoare logic by explaining, in the triple
$\{P\}C\{Q\}$, how the instruction, or the block of instructions,
$C$ transforms the predicate $P$ in $Q$. This kind of reasoning is
called \emph{forward-reasoning}. We calculate from the precondition and
from one or multiple instructions, the strongest postcondition we can
reach. Informally, considering what we have in input, we calculate what
we will get in output. If the postcondition we want is as strong or
weaker, then we prove that there is not any unwanted behaviors.



For example:



\begin{CodeBlock}{c}
int a = 2;
a = 4;
//calculated postcondition: a == 4
//wanted postconditione   : 0 <= a <= 30
\end{CodeBlock}



Ok, 4 is an allowed value for \texttt{a}.



The form of predicate transformer semantics which we are interested in
works the opposite way, we speak about \emph{backward-reasoning}. From
the wanted postcondition and the instructions we are reasoning about, we
find the weakest precondition that ensures this behavior. If our actual
precondition is at least as strong, that is to say, if it implies the
computed precondition, then our program is correct.



For example, if we have the instruction:



$\{P\}$ $x$ $:=$ a $\{x = 42\}$



What is the weakest precondition to validate the postcondition
$\{x = 42\}$ ? The rule will define that $P$ is $\{$a$=42\}$.



For now, let us forget about it, we will come back to these notions as
we use them in this tutorial to understand how our tools work. So now,
we can have a look to these tools.




\levelTwoTitle{Frama-C}


\image{frama-c}


\levelThreeTitle{Frama-C? WP?}



Frama-C (FRAmework for Modular Analysis of C code) is a platform
dedicated to the analysis of C programs created by the CEA LIST and
Inria. It is based on a modular architecture allowing to use different
(collaborating or not) plugins. The default plugins comprises different
static analyses (that do not execute source code), dynamic analyses
(that requires code execution), or combining both.


\image{gallery}



Frama-C provides a specification language called ACSL (``Axel'') for
ANSI C Specification Language and that allows us to express the
properties we want to verify about our programs. These properties are
written using code annotations in comment sections. If one has
already used Doxygen, it is quite similar, except that we write logic
formulas and not text. During this tutorial, we will extensively write
ASCL code, so let us just skip this for now.



The analysis we will use in this tutorial is provided by the WP plugin
(for Weakest Precondition), a deductive verification plugin. It implements
the technique we mentioned earlier: from
ACSL annotations and the source code, the plugin generates what we call
verification conditions, that are logic formulas that must be verified to be
satisfiable or not. This verification can be performed manually or
automatically, here we use automatic tools.



We will use a SMT solver
(\externalLink{statisfiability modulo theory}{https://fr.wikipedia.org/wiki/Satisfiability\_modulo\_theories}, we do not detail how it works). This solver is
\externalLink{Alt-Ergo}{http://alt-ergo.lri.fr/}, that was initially developed
by the Laboratoire de Recherche en Informatique d'Orsay, and is today
maintained by OCamlPro.



\levelThreeTitle{Installation}



Frama-C is a software developed on Linux and OSX. Its support is thus better
on those operating system. Nevertheless, it is possible to install Frama-C
on Windows and in theory, its use will be identical to its use on Linux.
However:



\begin{Warning}
\begin{itemize}
  \item the tutorial presents the use of Frama-C on Linux (or OSX) and
    the author did not experiment the differences that could exists with Windows,
  \item in recent versions of Windows 10, a possibility is to use Windows
    Subsystem for Linux, in combination with a Xserver installed on Windows
    to get GUI,
  \item the ``Bonus'' section of this part could not be accessible on Windows.
  \end{itemize}
\end{Warning}


\levelFourTitle{Linux}


\levelFiveTitle{Using package managers}


On Debian, Ubuntu and Fedora, there exist packages for Frama-C. In
such a case, it is enough to type a command like:



\begin{CodeBlock}{bash}
apt-get/yum install frama-c
\end{CodeBlock}



However, these repositories are not necessarily up to date with the latest
version of Frama-C. Most of the tutorial should be usable for the version
available on your Linux distribution if it is recent enough. However, some
specific features could be missing.



Go to the section ``Verify installation'' to perform some tests about
the installation.



\levelFiveTitle{Via opam}



A second option is to use Opam, a package manager for Ocaml libraries
and applications.



First of all, Opam must be installed (see its documentation). Then, some
packages from the Linux distribution must be installed before
installing Frama-C. On most systems, it is possible to ask Opam to install
dependencies of the packages we want to install. For this, first install
the depext tool of Opam:


\begin{CodeBlock}{bash}
opam install depext
\end{CodeBlock}


And then use the depext tool to install the dependencies of Frama-C:


\begin{CodeBlock}{bash}
opam depext frama-c
\end{CodeBlock}


If the depext tool does not support your distribution, the following
libraries and software should be installed:

\begin{itemize}
\item GTK2 (development library)
\item GTKSourceview 2 (development library)
\item GnomeCanvas 2 (development library)
\item autoconf
\end{itemize}

On recent versions of some distributions, GTK2 might not be available, in
this case, or if you want GTK3 and not GTK2, the packages GTK2, GTKSourceview2
and GnomeCanvas2 must be replaced with GTK3 and GTKSourceview 3.


Once these packages are installed, we can install Frama-C and Alt-Ergo.


\begin{CodeBlock}{bash}
opam install frama-c
opam install alt-ergo-free
\end{CodeBlock}


Note that for versions older than Potassium (Argon and before), if you
want to use the Why3 platform as a backend (introduced later), the version
0.88.3 should be used for Why3.


Go to the section ``Verify installation'' to perform some tests about
the installation.



\levelFiveTitle{Via ``manual'' compilation}


The packages we have listed in the Opam section are required (of course,
Opam itself is not). It requires a recent version of Ocaml and its
compiler (including a compiler to native code). Is is also necessary to
install Why3 (at least version 1.2.0), which is available either on Opam
or on their website (\externalLink{Why3}{http://why3.lri.fr/}).



After having extracted the folder available here :
\externalLink{http://frama-c.com/download.html}{http://frama-c.com/download.html} (Source distribution).
Navigate to the folder and then execute the command line:



\begin{CodeBlock}{bash}
autoconf && ./configure && make && sudo make install
\end{CodeBlock}



Go to the section ``Verify installation'' to perform some tests about
the installation.



\levelFourTitle{OSX}


On OSX, the use of Homebrew and Opam is recommended to install Frama-C.
The author does not use OSX, so here is a shameful copy and paste of the
installation guide of Frama-C for OSX.




General Mac OS tools for OCaml:



\begin{CodeBlock}{bash}
> xcode-select --install
> open http://brew.sh
> brew install autoconf opam
\end{CodeBlock}



Graphical User Interface:



\begin{CodeBlock}{bash}
> brew install gtk+ --with-jasper
> brew install gtksourceview libgnomecanvas graphviz
> opam install lablgtk ocamlgraph
\end{CodeBlock}



Recommended for Frama-C:



\begin{CodeBlock}{bash}
> brew install gmp
> opam install zarith
\end{CodeBlock}



Necessary for Frama-C/WP:



\begin{CodeBlock}{bash}
> opam install alt-ergo-free
> opam install frama-c
\end{CodeBlock}



\levelFourTitle{Windows}


Currently, the best way to install Frama-C on Windows is to use the
Windows Subsystem for Linux. Basically, once the Linux system is installed,
one should install Opam and follow the instructions provided in the Linux
section. To get a usable graphical user interface, a X server should be
installed on Windows.


\levelThreeTitle{Verify installation}


In order to verify that the installation has been correctly performed,
we will use the following code:



\CodeBlockInput{c}{verify.c}


Copy and paste this code in a file named \CodeInline{main.c}.
Then, from a terminal, in the folder where the file has been created, we
start Frama-C with the following command line:



\begin{CodeBlock}{bash}
frama-c-gui -wp -rte main.c
\end{CodeBlock}



The following window should appear:



\image{verif_install-1}


Clicking \CodeInline{main.c} in the left side panel to select it, we can see
its content (slightly) modified, and some green bullets on different
lines as illustrated here:



\image{verif_install-2}




\begin{Warning}
  The graphical user interface of Frama-C does not allow source code edition.
\end{Warning}


\begin{Information}
  For color blinds, it is possible to start Frama-C with another theme where
  color bullets are replaced:

\begin{CodeBlock}{bash}
frama-c-gui -gui-theme colorblind
\end{CodeBlock}
\end{Information}


\levelThreeTitle{(Bonus) Some more provers}


This part is optional, nothing in this section should be particularly
useful \emph{in the tutorial}. However, when we start to be interested
in proving more complex programs, it is often possible to reach the
limits of Alt-Ergo, which is available by default, and we would thus need
some other provers. For basic properties, almost all solvers have the
same capabilities, for more complex ones, each solvers has its predilection
domains.


\levelFourTitle{Why3}


\externalLink{Why3}{http://why3.lri.fr/} is a deductive proof platform
developed by the LRI in Orsay. It
provides a programming language and a specification language, as well as
a module that allows to interact with a wide variety of automatic and
interactive provers. This point is the one that interests us here. WP uses
Why3 as a backend to talk with external provers.


On their website, we can find the list of
\externalLink{supported provers}{http://why3.lri.fr/\#provers}.
We recommend to install
\externalLink{Z3}{https://github.com/Z3Prover/z3/wiki} which is
developed by Microsoft Research, and
\externalLink{CVC4}{http://cvc4.cs.stanford.edu/web/} which is developed by many
research teams (New York University, University of Iowa, Google, CEA
List). Those two provers are very efficient and somewhat complementary.


New provers can be installed any time after the installation of Frama-C.
However, Why3 provers list must be updated:

\begin{CodeBlock}{bash}
why3 config --detect
\end{CodeBlock}

And then activated in Frama-C. In the left side panel, in the WP part,
clic "Provers..." :


\image{provers}


And then "Detect" in the window that appears. Once it is done, provers
can be activated thanks to the button next to their name.


\image{detect-and-select}


\levelFourTitle{Coq}


Coq, which is developed by Inria, is a proof assistant. Basically, we
write the proofs ourselves in a dedicated language and the assistant
verifies (using typing) that the proof is actually a valid proof.



Why would we need such a tool? Sometimes, the properties we want to
prove can be too complex to be solved automatically by SMT solvers,
typically when they requires careful inductive reasoning with precise
choices at each step. In this situation, WP allows us to generate
verification conditions translated in Coq language, and to write the
proof ourselves.




To learn Coq, we would recommend
\externalLink{this tutorial}{http://www.cis.upenn.edu/~bcpierce/sf/current/index.html}.



\begin{Information}
  If Frama-C has been installed using the package manager of a Linux
  distribution, Coq could be automatically installed.
\end{Information}


If one needs more information about Coq and its installation, this page
can help: \externalLink{The Coq Proof Assistant}{https://coq.inria.fr/}.



The current support of Coq is deprecated in Frama-C but is still available.
As we provide Coq scripts that can be used, we provide these instruction
so that this way of proving programs can be tested by users. To run Frama-C
with Coq support, use:



\begin{CodeBlock}{bash}
  frama-c -wp-provers=native:coq
\end{CodeBlock}


\horizontalLine



Voilà. Nos outils sont installés et prêts à fonctionner.



Le but de cette partie, en plus de l'installation de nos outils de travail
pour la suite, est de faire ressortir deux informations claires :



\begin{itemize}
\item la preuve est un moyen d'assurer que nos programmes n'ont que des 
comportements conformes à notre spécification ;
\item il est toujours de notre devoir d'assurer que cette spécification est
correcte.
\end{itemize}
