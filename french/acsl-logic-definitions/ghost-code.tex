Les techniques que nous avons vu précédemment dans ce chapitre ont pour but de
rendre la spécification plus abstraite. Le rôle du code fantôme est inverse. Ici,
nous enrichirons nos spécifications à l'aide d'information exprimées
en langage C. L'idée est d'ajouter des variables et du code source qui
n'intervient pas dans le programme réel mais permettant de créer des états
logiques qui ne seront visibles que depuis la preuve. Par cet intermédiaire,
nous pouvons rendre explicites des propriétés logiques qui étaient auparavant
implicites.


\levelThreeTitle{Syntaxe}


Le code fantôme est ajouté par l'intermédiaire d'annotations qui contiennent
du code C ainsi que la mention \CodeInline{ghost}.



\begin{CodeBlock}{c}
/*@
  ghost
  // code en langage C
*/
\end{CodeBlock}


Dans un code fantôme, nous écrivons du C normal. Nous expliquerons certaines
petites subtilités plus tard. Pour l'instant, intéressons nous aux éléments
basiques que nous pouvons écrire avec du code fantôme.


Nous pouvons déclarer des variables :


\begin{CodeBlock}{c}
//@ ghost int ghost_glob_var = 0;

void foo(int a){
  //@ ghost int ghost_loc_var = a;
}
\end{CodeBlock}


Ces variables peuvent être modifiées via des opérations et structures
conditionnelles :


\begin{CodeBlock}{c}
//@ ghost int ghost_glob_var = 0;

void foo(int a){
  //@ ghost int ghost_loc_var = a;
  /*@ ghost
    for(int i = 0 ; i < 10 ; i++){
      ghost_glob_var += i ;
      if(i < 5) ghost_local_var += 2 ;
    }
  */
}
\end{CodeBlock}


Nous pouvons déclarer des \textit{labels} fantôme, que l'on peut utiliser dans
nos annotations (ou même pour effectuer un \CodeInline{goto} depuis le code
fantôme lui même, sous certaines conditions que nous expliquerons plus tard) :


\begin{CodeBlock}{c}
void foo(int a){
  //@ ghost Ghost_label: ;
  a = 28 ;
  //@ assert ghost_loc_var == \at(a, Ghost_label) == \at(a, Pre);
}
\end{CodeBlock}


Une construction conditionnelle \CodeInline{if} peut être étendue avec un
\CodeInline{else} fantôme s'il n'a pas de \CodeInline{else} à la base. Par
exemple :


\begin{CodeBlock}{c}
void foo(int a){
  //@ ghost int a_was_ok = 0 ;
  if(a < 5){
    a = 5 ;
  } /*@ ghost else {
    a_was_ok = 1 ;
  } */
}
\end{CodeBlock}


Une fonction peut avoir des paramètres fantômes, cela permet de transmettre des
informations supplémentaires pour la vérification de la fonction. Par exemple,
si l'on imagine la vérification d'une fonction qui reçoit une liste chaînée,
nous pourrions transmettre un paramètre fantôme qui représenterait la
longueur de la liste :


\begin{CodeBlock}{c}
void function(struct list* l) /*@ ghost (int length) */ {
  // visit the list
  /*@ variant length ; */
  while(l){
    l = l->next ;
    //@ ghost length--;
  }
}
void another_function(struct list* l){
  //@ ghost int length ;

  // ... do something to compute the length

  function(l) /*@ ghost(n) */ ; // we call 'function' with the ghost argument
}
\end{CodeBlock}


Notons que si une fonction prend des paramètres fantômes, tous les appels doivent
fournir les arguments fantômes correspondant.


Une fonction toute entière peut être fantôme. Par exemple, nous pourrions avoir
une fonction fantôme qui calcule la longueur d'une liste que nous aurions utilisée
au sein du code précédent :

\begin{CodeBlock}{c}
/*@ ghost
  /@ ensures \result == logic_length_of_list(l) ; @/
  int compute_length(struct list* l){
    // does the right computation
  }
*/

void another_function(struct list* l){
  //@ ghost int length ;

  //@ ghost length = compute_length(l) ;
  function(l) /*@ ghost(n) */ ; // we call 'function' with the ghost argument
}
\end{CodeBlock}


Ici, nous pouvons voir une syntaxe spécifique pour le contrat de la fonction
fantôme. En effet, il est souvent utile d'écrire des contrats ou des assertions
dans du code fantôme. Comme nous devons écrire ces spécifications dans du code
qui est déjà englobé dans des commentaires C, nous avons accès à une syntaxe
spécifique pour fournir des contrats ou des assertions fantômes. Nous ouvrons
une annotation fantôme avec la syntaxe \CodeInline{/@} et nous les fermons avec
\CodeInline{@/}. Cela s'applique aussi aux boucles dans le code fantôme par
exemple :


\begin{CodeBlock}{c}
void foo(unsigned n){
 /*@ ghost
   unsigned i ;

   /@
     loop invariant 0 <= i <= n ;
     loop assigns i ;
     loop variant n - i ;
   @/
   for(i = 0 ; i < n ; ++i){

   }
   /@ assert i == n ; @/
 */
}
\end{CodeBlock}


\levelThreeTitle{Validité du code fantôme, ce que Frama-C vérifie}


Frama-C vérifie plusieurs propriétés à propos du code fantôme que nous écrivons :


\begin{itemize}
\item le code fantôme ne peut pas modifier le graphe de flot de contrôle
      du programme ;
\item le code normal ne peut pas accéder à la mémoire fantôme ;
\item le code fantôme ne peut modifier qu'une zone de mémoire fantôme.
\end{itemize}


Très simplement, ces propriétés cherchent à garantir que pour n'importe
quel programme son comportement observable pour toute entrée est le même
avec ou sans le code fantôme.


\begin{Warning}
  Avant Frama-C 21 Scandium, la plupart de ces propriétés n'étaient pas vérifiées
  par le noyau de Frama-C. Par conséquent, si l'on travaille avec une version
  antérieure, il faut s'assurer soit même que ces propriétés sont vérifiées.
\end{Warning}


Si certaines de ces propriétés ne sont pas vérifiées, cela voudrait dire que le
code fantôme peut changer le comportement du programme vérifié. Analysons de plus
prêt chacune de ces contraintes.


\levelFourTitle{Maintien du flot de contrôle}


Le flot de contrôle d'un programme est l'ordre dans lequel les instructions sont
exécutées par le programme. Si le code fantôme change cet ordre, ou permet de ne
plus exécuter certaines instructions du programme d'origine, alors le comportement
du programme n'est plus le même, et nous ne vérifions donc plus le même programme.


Par exemple, dans la fonction suivante calcule la somme des $n$ premiers entiers :


\CodeBlockInput{c}{control-flow-1.c}


Par l'introduction, dans du code fantôme, de l'instruction \CodeInline{break} dans
le corps de la boucle, le programme n'a plus le même comportement : au lieu de
parcourir l'ensemble des $i$ de $0$ à $n+1$, la boucle s'arrête dès le premier
tour de la boucle. En conséquence, ce programme sera rejeté par Frama-C :


\begin{CodeBlock}{text}
[kernel:ghost:bad-use] file.c:4: Warning:
  Ghost code breaks CFG starting at:
  /*@ ghost break; */
  x += i;
\end{CodeBlock}


Il est important de noter que lorsqu'un code fantôme altère le flot de contrôle
c'est le point de départ du code fantôme qui est pointé par l'erreur, par exemple
si nous introduisons une conditionnelle autour de notre \CodeInline{break} :


\CodeBlockInput{c}{control-flow-2.c}


Le problème est indiqué pour le \CodeInline{if} englobant :


\begin{CodeBlock}{text}
[kernel:ghost:bad-use] file.c:4: Warning:
  Ghost code breaks CFG starting at:
  /*@ ghost if (i < 3) break; */
  x += i;
\end{CodeBlock}


Notons que la vérification que le flot de contrôle n'est pas altéré est purement
syntaxique. Par exemple, si le \CodeInline{break} est inatteignable parce que la
condition est toujours fausse, une erreur sera quand même levée :


\CodeBlockInput{c}{control-flow-3.c}


\begin{CodeBlock}{text}
[kernel:ghost:bad-use] file.c:4: Warning:
  Ghost code breaks CFG starting at:
  /*@ ghost if (i > n) break; */
  x += i;
\end{CodeBlock}


Finalement, remarquons qu'il existe deux manières générales d'altérer le flot de
contrôle. La première est d'utiliser un saut (donc \CodeInline{break},
\CodeInline{continue}, ou \CodeInline{goto}), la seconde est d'introduire un code
non terminant. Pour ce dernier, à moins que le code soit trivialement non terminant
le noyau ne peut pas vérifier la non-altération du flot de contrôle, et ne le fait
donc jamais. Nous traiterons cette question dans la
section~\ref{l3:acsl-logic-definitions-what-remains}.


\levelFourTitle{Accès à la mémoire}


Le code fantôme est un observateur du code normal. En conséquence, le code normal
n'est pas autorisé à accéder au code fantôme, que ce soit sa mémoire ou ses
fonctions. Le code fantôme lui, peut lire la mémoire du code normal, mais ne peut
pas la modifier. Actuellement, le code fantôme ne peut pas non plus appeler de
fonctions du code normal, nous parlerons de cette restriction à la fin de cette
section.


Refuser que le code normal voit le code fantôme a une raison toute simple, si le
code normal tentait d'accéder à des variables fantômes, il ne pourrait même pas
être compilé : le compilateur ne voit pas les variables déclarées dans les
annotations. Par exemple :


\CodeBlockInput{c}{non-ghost-access-ghost.c}


ne peut pas être compilé :


\begin{CodeBlock}{text}
# gcc -c file.c
file.c: In function ‘sum_42’:
file.c:5:10: error: ‘r’ undeclared (first use in this function)
    5 |     x += r;
      |          ^
\end{CodeBlock}


et n'est donc pas accepté par Frama-C non plus :


\begin{CodeBlock}{text}
[kernel] file.c:5: User Error:
  Variable r is a ghost symbol. It cannot be used in non-ghost context. Did you forget a /*@ ghost ... /?
\end{CodeBlock}


Dans le code fantôme, les variables normales ne doivent pas être modifiées. En effet,
cela impliquerait de pouvoir par exemple modifier le résultat d'un programme en
ajoutant du code fantôme. Par exemple dans le code suivant :


\CodeBlockInput{c}{ghost-access-non-ghost-1.c}


Le résultat du programme ne serait pas le même avec ou sans le code fantôme.
Frama-C interdit donc un tel code :


\begin{CodeBlock}{text}
[kernel:ghost:bad-use] file.c:5: Warning:
  'x' is a non-ghost lvalue, it cannot be assigned in ghost code
\end{CodeBlock}


Notons que cette vérification est faite grâce au type des différentes variables.
Une variable déclarée dans du code normal a un statut de variable normale, tandis
qu'une variable déclarée dans du code fantôme a un statut de variable fantôme.
Par conséquent, une nouvelle fois, même si le code fantôme, dans les faits,
n'altère pas le comportement du programme, toute écriture d'une variable normale
dans le code fantôme est interdite :


\CodeBlockInput[3]{c}{ghost-access-non-ghost-2.c}


\begin{CodeBlock}{text}
[kernel:ghost:bad-use] file.c:9: Warning:
  'x' is a non-ghost lvalue, it cannot be assigned in ghost code
[kernel:ghost:bad-use] file.c:10: Warning:
  'x' is a non-ghost lvalue, it cannot be assigned in ghost code
\end{CodeBlock}


Cela s'applique également à la clause \CodeInline{assigns} lorsqu'elle est dans
du code fantôme :


\CodeBlockInput{c}{assigns-clause-1.c}


\begin{CodeBlock}{text}
[kernel:ghost:bad-use] file.c:4: Warning:
  'x' is a non-ghost lvalue, it cannot be assigned in ghost code
[kernel:ghost:bad-use] file.c:11: Warning:
  'x' is a non-ghost lvalue, it cannot be assigned in ghost code
\end{CodeBlock}


En revanche, les contrats des fonctions et boucles normales peuvent (et doivent)
permettre de spécifier les zones de mémoire fantôme modifiées. Par exemple, si
nous corrigeons le petit programme précédent en rendant \CodeInline{x} fantôme,
d'une part nos clauses \CodeInline{assigns} précédentes sont bien acceptées,
mais en plus, nous pouvons spécifier que la fonction \CodeInline{foo} modifie
la variable globale fantôme \CodeInline{x} :


\CodeBlockInput{c}{assigns-clause-2.c}


\levelFourTitle{Typage des éléments fantômes}


Il convient de donner quelques précisions au sujet des types des variables
créées dans du code fantôme. Par exemple, parfois il est intéressant de pouvoir
créer un tableau fantôme pour stocker des informations :


\CodeBlockInput{c}{ghost-array-1.c}


Ici, nous utilisons des indices pour accéder à nos tableaux, mais nous pourrions
par exemple vouloir y accéder en utilisant un pointeur :


\CodeBlockInput{c}{ghost-array-2.c}


Mais nous voyons immédiatement que Frama-C n'est pas d'accord avec notre manière
de faire :


\begin{CodeBlock}{text}
[kernel:ghost:bad-use] file.c:3: Warning:
  Invalid cast of 'even' from 'int \ghost *' to 'int *'
[kernel:ghost:bad-use] file.c:6: Warning:
  '*pe' is a non-ghost lvalue, it cannot be assigned in ghost code
\end{CodeBlock}


En particulier, le premier message nous indique que nous essayons de transformer
un pointeur sur \CodeInline{int \textbackslash{}ghost} en pointeur sur
\CodeInline{int}. En effet, lorsqu'une variable est déclarée dans du code fantôme,
seule la variable est considérée fantôme. Donc dans le cas d'un pointeur, la mémoire
pointée par ce pointeur, elle, n'est pas considérée comme fantôme (et donc ici, bien
que \CodeInline{pe} soit fantôme, la mémoire pointée par \CodeInline{pe} ne l'est
pas). Pour résoudre ce problème, Frama-C nous offre le qualifieur
\CodeInline{\textbackslash{}ghost}, qui nous permet d'ajouter un caractère fantôme
à un type :


\CodeBlockInput{c}{ghost-array-3.c}


Sur certains aspects, le qualifieur \CodeInline{\textbackslash{}ghost} ressemble
au mot clé \CodeInline{const} du C. Cependant, son comportement n'est pas
exactement le même pour deux raisons.


Tout d'abord, alors que la définition \CodeInline{const} suivante
est autorisée,  il n'est pas possible d'avoir une déclaration de
forme similaire avec le qualifieur \CodeInline{\textbackslash{}ghost} :


\begin{CodeBlock}{c}
int const * * const p ;
//@ ghost int \ghost * * p ;
\end{CodeBlock}


\begin{CodeBlock}{text}
[kernel:ghost:bad-use] file.c:2: Warning:
  Invalid type for 'p': indirection from non-ghost to ghost
\end{CodeBlock}



Déclarer un pointeur constant sur une zone que l'on peut modifier et qui contient
des pointeurs vers de la mémoire constante ne pose pas de problème. En revanche,
il est impossible de faire de même avec le qualifieur
\CodeInline{\textbackslash{}ghost}, cela signifierait que la mémoire normale
contient des pointeurs vers la mémoire fantôme, ce qui n'est pas possible.


D'autre part, il est possible d'assigner un pointeur
vers des données non-constantes à un pointeur vers des données constantes :


\begin{CodeBlock}{c}
int a[10] ;
int const * p = a ;
\end{CodeBlock}


Ce code ne pose pas de problème car l'on ne fait que restreindre notre capacité
à modifier les données lorsque l'on initialise (ou affecte) \CodeInline{p} à
\CodeInline{\&a[0]}. En revanche, les deux initialisations (ou affectations
équivalentes) des pointeurs suivants sont refusées avec le qualifieur
\CodeInline{\textbackslash{}ghost} :


\begin{CodeBlock}{c}
int non_ghost_int ;
//@ ghost int ghost_int ;

//@ ghost int \ghost * p = & non_ghost_int ;
//@ ghost int * q = & ghost_int ;
\end{CodeBlock}


Si la raison du refus de la première initialisation est tout à fait directe : elle
permettrait de modifier le contenu de la mémoire normale depuis du code fantôme,
refuser la seconde peut être un peu moins intuitif. Et en effet, nous devons passer
par des moyens détournés pour provoquer un problème avec cette conversion :


\CodeBlockInput{c}{why-not-const.c}


Ici, nous faisons une conversion qui pourrait sembler autorisée. En effet, nous
passons l'adresse d'un pointeur sur une zone de mémoire ghost à une fonction qui
attend un pointeur sur une zone de mémoire non-ghost, cela ne fait que restreindre
l'accès à la mémoire pointée. Cependant, par cet appel de fonction, la fonction
\CodeInline{assign} assigne la valeur actuelle de \CodeInline{q} (qui est
\CodeInline{\&x}) à \CodeInline{p} et nous permet donc, par la dernière opération
de modifier \CodeInline{x} dans du code fantôme. En conséquence, une telle
conversion n'est jamais autorisée.


Finalement, le code fantôme ne peut actuellement pas appeler de fonction non
fantôme, pour des raisons semblables à celle évoquée pour l'interdiction de toutes
les conversions. Certains cas particuliers pourraient être traités de manière à
accepter plus de code, mais ce n'est actuellement pas supporté par Frama-C.


\levelThreeTitle{Validité du code fantôme, ce qu'il reste à vérifier}
\label{l3:acsl-logic-definitions-what-remains}


Mis à part les restrictions que nous avons mentionné dans la section précédente,
le code fantôme est juste du code C normal. Cela veut dire que si nous voulons
faire la vérification de notre programme d'origine, nous devons faire attention,
nous mêmes, à au moins deux aspects supplémentaires :


\begin{itemize}
\item l'absence d'erreurs à l'exécution,
\item la terminaison du code fantôme.
\end{itemize}


Le premier cas ne nécessite pas plus d'attention que le reste de notre code.
En effet, la vérification d'absence d'erreurs à l'exécution sera traitée par
le plugin RTE comme pour le reste de notre programme.


Comme nous l'avons dit dans la section~\ref{l3:statements-loops-variant}, il y
a deux sortes de correction : la correction partielle et la correction totale,
la seconde permettant de prouver qu'un programme termine. Dans le cas du code
normal, montrer la terminaison n'est pas toujours souhaitable pour l'ensemble
du programme. En revanche, si nous utilisons du code fantôme pour aider la
vérification, montrer que la correction est totale est absolument nécessaire
car une boucle infinie dans le code fantôme peut nous permettre de prouver
n'importe quoi à propos du programme.


\begin{CodeBlock}{c}
/*@ ensures \false ; */
void foo(void){
  /*@ ghost
    while(1){}
  */
}
\end{CodeBlock}


\levelThreeTitle{Expliciter un état logique}


Le but du code fantôme est de rendre explicite des informations généralement
implicites. Par exemple, dans la vérification de l'algorithme de tri, nous nous en
sommes servi pour ajouter un \textit{label} dans le programme qui n'est pas
visible par le compilateur mais que nous avons pu utiliser pour la vérification.
Le fait que les valeurs ont été échangées entre les deux points de programme était
implicitement garanti par le contrat de la fonction d'échange, ajouter ce
\textit{label} fantôme nous a donné la possibilité de rendre cette propriété
explicite par une assertion.



Prenons maintenant un exemple plus poussé. Nous voulons par exemple prouver que
la fonction suivante nous retourne la valeur maximale des sommes de sous-tableaux
possibles d'un tableau donné. Un sous-tableau d'un tableau \CodeInline{a} est un
sous-ensemble contigu de valeur de \CodeInline{a}. Par exemple, pour un tableau
\CodeInline{\{ 0 , 3 , -1 , 4 \}}, des exemples de sous tableaux peuvent être
\CodeInline{\{\}}, \CodeInline{\{ 0 \}}, \CodeInline{\{ 3 , -1 \}},
\CodeInline{\{ 0 , 3 , -1 , 4 \}}, ... Notons que comme nous autorisons le
tableau vide, la somme est toujours au moins égale à 0. Dans le tableau précédent,
le sous-tableau de valeur maximale est \CodeInline{\{ 3 , -1 , 4 \}}, la fonction
renverra donc 6.



\CodeBlockInput[3]{c}{max_subarray-0.c}



Pour spécifier la fonction précédente, nous aurons besoin d'exprimer
axiomatiquement la somme. Ce n'est pas très complexe, et le lecteur pourra
s'exercer en exprimant les axiomes nécessaires au bon fonctionnement de cette
axiomatique :



\CodeBlockInput[3][5]{c}{max_subarray-1.c}



La correction est disponible à la section~\ref{l2:acsl-logic-definitions-answers}.


La spécification de notre fonction est la suivante :



\CodeBlockInput[14][19]{c}{max_subarray-1.c}



Pour toute paire de bornes, la valeur retournée par la fonction doit être
supérieure ou égale à la somme des éléments entre les bornes, et il doit exister
une paire de bornes telle que la somme des éléments entre ces bornes est
exactement la valeur retournée par la fonction. Par rapport à cette spécification,
si nous devons ajouter les invariants de boucles, nous nous apercevons rapidement
qu'il nous manquera des informations. Nous avons besoin d'exprimer ce que sont
les valeurs \CodeInline{max} et \CodeInline{cur} et quelles relations existent entre elles,
mais rien ne nous le permet !



En substance, notre postcondition a besoin de savoir qu'il existe des
bornes \CodeInline{low} et \CodeInline{high} telles que la somme calculée correspond à ces bornes.
Or, notre code n'exprime rien de tel d'un point de vue logique et rien ne nous
permet \textit{a priori} de faire cette liaison en utilisant des formulations logiques.
Nous utiliserons du code \textit{ghost} pour conserver ces bornes et exprimer
l'invariant de notre boucle.



Nous aurons d'abord besoin de deux variables qui nous permettront de stocker
les valeurs des bornes de la plage maximum, nous les appellerons \CodeInline{low}
et \CodeInline{high}. Chaque fois que nous trouverons une plage où la somme est plus
élevée nous les mettrons à jour. Ces bornes correspondront donc à la somme indiquée
par \CodeInline{max}. Cela induit que nous avons encore besoin d'une autre paire de
bornes : celle correspondant à la variable de somme \CodeInline{cur} à partir de laquelle
nous pourrons construire les bornes de \CodeInline{max}. Pour celle-ci, nous n'avons
besoin que d'ajouter une variable \textit{ghost} : le minimum actuel \CodeInline{cur\_low}, la
borne supérieure de la somme actuelle étant indiquée par la variable \CodeInline{i} de la
boucle.



\CodeBlockInput[14][50]{c}{max_subarray-1.c}



L'invariant \CodeInline{BOUNDS} exprime comment sont ordonnées les différentes bornes
pendant le calcul. L'invariant \CodeInline{REL} exprime ce que signifient les
valeurs \CodeInline{cur} et \CodeInline{max} par rapport à ces bornes. Finalement,
l'invariant \CodeInline{POST} permet de faire le lien entre les invariants précédents
et la postcondition de la fonction.



Le lecteur pourra vérifier que cette fonction est effectivement prouvée sans la
vérification des RTE. Si nous ajoutons également le contrôle des RTE, nous pouvons
voir que le calcul de la somme indique un dépassement possible sur les entiers.



Ici, nous ne chercherons pas à le corriger, car ce n'est pas l'objet de l'exemple.
Le moyen de prouver cela dépend en fait fortement du contexte dans lequel on
utilise la fonction. Une possibilité est de restreindre fortement le contrat en
imposant des propriétés à propos des valeurs et de la taille du tableau. Par
exemple, nous pourrions imposer une taille maximale et des bornes fortes pour
chacune des cellules. Une autre possibilité est d'ajouter une valeur d'erreur
en cas de dépassement (par exemple $-1$), et de spécifier qu'en cas de
dépassement, c'est cette valeur qui est renvoyée.



\levelThreeTitle{Exercices}


\levelFourTitle{Validité du code ghost}


Dans ces fonctions, et sans exécuter Frama-C, expliquer en quoi le code fantôme
pose problème. Lorsque Frama-C devrait rejeter le code, expliquer pourquoi.
Notons qu'il est possible d'exécuter Frama-C sans contrôle du code fantôme en
utilisant l'option \CodeInline{-kernel-warn-key ghost=inactive}.


\CodeBlockInput{c}{ex-1-ghost-reject.c}


\levelFourTitle{Multiplication par 2}


Le programme suivant calcule \CodeInline{2 * x} en utilisant une boucle.
Utiliser une variable fantôme \CodeInline{i} pour exprimer comme invariant que
la valeur de \CodeInline{r} est \CodeInline{i * 2} et compléter la preuve.


\CodeBlockInput{c}{ex-2-times-2.c}


\levelFourTitle{Tableaux}


Cette fonction reçoit un tableau et effectue une boucle dans laquelle nous ne
faisons rien, sauf que nous avons indiqué que le contenu du tableau est modifié.
Cependant, nous voudrions pouvoir prouver qu'en postcondition, le tableau n'a
pas été modifié.


\CodeBlockInput{c}{ex-3-array.c}


Sans modifier la clause \CodeInline{assigns} de la boucle et sans utiliser le mot
clé \CodeInline{\textbackslash{}at}, prouver que la fonction ne modifie pas le
contenu du tableau. Pour cela, compléter le code fantôme et l'invariant de boucle
en assurant que le contenu du tableau \CodeInline{g} représente l'ancien contenu
de \CodeInline{a}.


Lorsque c'est fait, créer une fonction fantôme qui effectue cette même copie, et
l'utiliser dans la fonction \CodeInline{foo} pour effectuer la même preuve.


\levelFourTitle{Chercher et remplacer}


Le programme suivant effectue une opération de recherche et remplacement :


\CodeBlockInput{c}{ex-4-replace.c}


En supposant que la fonction \CodeInline{replace} demande à ce que
\CodeInline{old} et \CodeInline{new} soient différents, écrire un contrat pour
\CodeInline{replace} et prouver que la fonction le satisfait.


Maintenant, nous voudrions savoir quelles cellules du tableau ont été mises à
jour par la fonction. Ajouter un paramètre fantôme à la fonction
\CodeInline{replace} de manière à pouvoir recevoir un second tableau qui servira
à enregistrer les cellules mise à jour (ou non) par la fonction. En ajoutant
également le code suivant après l'appel à \CodeInline{replace} :


\begin{CodeBlock}{c}
  /*@ ghost
    /@ loop invariant 0 <= i <= 40 ;
       loop assigns i;
       loop variant 40 - i ;
    @/
    for(size_t i = 0 ; i < 40 ; ++i){
      if(updated[i]){
        /@ assert a[i] != \at(a[\at(i, Here)], L); @/
      } else {
        /@ assert a[i] == \at(a[\at(i, Here)], L); @/
      }
    }
  */
  \end{CodeBlock}


  Tout devrait être prouvé.
