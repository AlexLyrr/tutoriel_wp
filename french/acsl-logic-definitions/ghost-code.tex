Les techniques que nous avons vu précédemment dans ce chapitre ont pour but de
rendre la spécification plus abstraite. Le rôle du code fantôme est inverse. Ici,
nous enrichirons nos spécifications à l'aide d'information exprimées
en langage C. L'idée est d'ajouter des variables et du code source qui
n'intervient pas dans le programme réel mais permettant de créer des états
logiques qui ne seront visibles que depuis la preuve. Par cet intermédiaire,
nous pouvons rendre explicites des propriétés logiques qui étaient auparavant
implicites.


\levelThreeTitle{Syntaxe}


Le code fantôme est ajouté par l'intermédiaire d'annotations qui contiennent
du code C ainsi que la mention \CodeInline{ghost}.



\begin{CodeBlock}{c}
/*@
  ghost
  // code en langage C
*/
\end{CodeBlock}



Les seules règles que nous devons respecter dans un tel code est que nous ne
devons jamais écrire une portion de mémoire qui n'est pas elle-même définie dans
du code fantôme et que le code en question doit terminer tout bloc qu'il ouvrirait.
Mis à par cela, tout calcul peut être inséré tant qu’il ne modifie \textit{que} les variables
du code fantôme.



Voici quelques exemples pour la syntaxe de code fantôme :



\begin{CodeBlock}{c}
//@ ghost int ghost_glob_var = 0;

void foo(int a){
  //@ ghost int ghost_loc_var = a;

  //@ ghost Ghost_label: ;
  a = 28 ;

  //@ ghost if(a < 0){ ghost_loc_var = 0; }

  //@ assert ghost_loc_var == \at(a, Ghost_label) == \at(a, Pre);
}
\end{CodeBlock}


Bien que dans ce chapitre, cela ne sera pas nécessaire, nous verrons plus tard
qu'il est parfois utile d'écrire des contrats ou des assertions dans du code
fantôme. Comme nous devons écrire ces spécifications dans du code qui est
déjà englobé dans des commentaires C, nous avons accès à une syntaxe spécifique
pour fournir des contrats ou des assertions fantômes. Nous ouvrons une annotation
fantôme avec la syntaxe \CodeInline{/@} et nous les fermons avec \CodeInline{@/}.


Par exemple :

\begin{CodeBlock}{c}
void foo(unsigned n){
 /*@ ghost
   unsigned i ;

   /@
     loop invariant 0 <= i <= n ;
     loop assigns i ;
     loop variant n - i ;
   @/
   for(i = 0 ; i < n ; ++i){

   }
   /@ assert i == n ; @/
 */
}
\end{CodeBlock}



\levelThreeTitle{Validité du code fantôme}


Il faut être très prudent lorsque nous produisons ce genre de code. En effet,
aucune vérification n'est effectuée pour s'assurer que nous n'écrivons pas dans
la mémoire globale par erreur. Ce problème étant comme la vérification elle-même,
un problème indécidable, une telle analyse serait un travail de preuve à part
entière. Par exemple, ce code est accepté en entrée de Frama-C, alors qu'il
modifie manifestement l'état de la mémoire du programme :



\begin{CodeBlock}{c}
int a;

void foo(){
  //@ ghost a = 42;
}
\end{CodeBlock}



Il est donc très facile de modifier le comportement d'un programme en ajoutant
du code fantôme et dans ce cas, si nous prouvons des propriétés à propos du
programme, nous sommes en train de le faire à propos d'un programme modifié et
pas du programme réel. Il faut donc faire très attention à ne modifier que des
positions mémoire fantôme dans du code fantôme.


De plus, comme nous n'avons pas de restriction sur le type de code C que nous
pouvons écrire, nous pouvons écrire des boucles. Comme nous l'avons dit dans la
section~\ref{l3:statements-loops-variant}, il y a deux sortes de correction : la
correction partielle et la correction totale, la seconde permettant de prouver
qu'un programme termine. Dans le cas du code fantôme, montrer que la correction
est totale est absolument nécessaire car une boucle infinie dans le code fantôme
peut nous permettre de prouver n'importe quoi à propos du programme.


Par exemple, le programme suivant est prouvé parce que le code fantôme ne termine
pas :

\begin{CodeBlock}{c}
/*@ ensures \false ; */
void foo(void){
  /*@ ghost
    while(1){}
  */
}
\end{CodeBlock}

\levelThreeTitle{Expliciter un état logique}


Le but du code \textit{ghost} est de rendre explicite des informations généralement
implicites. Par exemple, dans la section précédente, nous nous en sommes servi
pour récupérer explicitement un état logique connu à un point de programme
donné.



Prenons maintenant un exemple plus poussé. Nous voulons par exemple prouver que
la fonction suivante nous retourne la valeur maximale des sommes de sous-tableaux
possibles d'un tableau donné. Un sous-tableau d'un tableau \CodeInline{a} est un
sous-ensemble contigu de valeur de \CodeInline{a}. Par exemple, pour un tableau
\CodeInline{\{ 0 , 3 , -1 , 4 \}}, des exemples de sous tableaux peuvent être
\CodeInline{\{\}}, \CodeInline{\{ 0 \}}, \CodeInline{\{ 3 , -1 \}},
\CodeInline{\{ 0 , 3 , -1 , 4 \}}, ... Notons que comme nous autorisons le
tableau vide, la somme est toujours au moins égale à 0. Dans le tableau précédent,
le sous-tableau de valeur maximale est \CodeInline{\{ 3 , -1 , 4 \}}, la fonction
renverra donc 6.



\CodeBlockInput[3]{c}{max_subarray-0.c}



Pour spécifier la fonction précédente, nous aurons besoin d'exprimer
axiomatiquement la somme. Ce n'est pas très complexe, et le lecteur pourra
s'exercer en exprimant les axiomes nécessaires au bon fonctionnement de cette
axiomatique :



\CodeBlockInput[3][5]{c}{max_subarray-1.c}



La correction est disponible à la section~\ref{l2:acsl-logic-definitions-answers}.


La spécification de notre fonction est la suivante :



\CodeBlockInput[14][19]{c}{max_subarray-1.c}



Pour toute paire de bornes, la valeur retournée par la fonction doit être
supérieure ou égale à la somme des éléments entre les bornes, et il doit exister
une paire de bornes telle que la somme des éléments entre ces bornes est
exactement la valeur retournée par la fonction. Par rapport à cette spécification,
si nous devons ajouter les invariants de boucles, nous nous apercevons rapidement
qu'il nous manquera des informations. Nous avons besoin d'exprimer ce que sont
les valeurs \CodeInline{max} et \CodeInline{cur} et quelles relations existent entre elles,
mais rien ne nous le permet !



En substance, notre postcondition a besoin de savoir qu'il existe des
bornes \CodeInline{low} et \CodeInline{high} telles que la somme calculée correspond à ces bornes.
Or, notre code n'exprime rien de tel d'un point de vue logique et rien ne nous
permet \textit{a priori} de faire cette liaison en utilisant des formulations logiques.
Nous utiliserons du code \textit{ghost} pour conserver ces bornes et exprimer
l'invariant de notre boucle.



Nous aurons d'abord besoin de deux variables qui nous permettront de stocker
les valeurs des bornes de la plage maximum, nous les appellerons \CodeInline{low}
et \CodeInline{high}. Chaque fois que nous trouverons une plage où la somme est plus
élevée nous les mettrons à jour. Ces bornes correspondront donc à la somme indiquée
par \CodeInline{max}. Cela induit que nous avons encore besoin d'une autre paire de
bornes : celle correspondant à la variable de somme \CodeInline{cur} à partir de laquelle
nous pourrons construire les bornes de \CodeInline{max}. Pour celle-ci, nous n'avons
besoin que d'ajouter une variable \textit{ghost} : le minimum actuel \CodeInline{cur\_low}, la
borne supérieure de la somme actuelle étant indiquée par la variable \CodeInline{i} de la
boucle.



\CodeBlockInput[14][50]{c}{max_subarray-1.c}



L'invariant \CodeInline{BOUNDS} exprime comment sont ordonnées les différentes bornes
pendant le calcul. L'invariant \CodeInline{REL} exprime ce que signifient les
valeurs \CodeInline{cur} et \CodeInline{max} par rapport à ces bornes. Finalement,
l'invariant \CodeInline{POST} permet de faire le lien entre les invariants précédents
et la postcondition de la fonction.



Le lecteur pourra vérifier que cette fonction est effectivement prouvée sans la
vérification des RTE. Si nous ajoutons également le contrôle des RTE, nous pouvons
voir que le calcul de la somme indique un dépassement possible sur les entiers.



Ici, nous ne chercherons pas à le corriger, car ce n'est pas l'objet de l'exemple.
Le moyen de prouver cela dépend en fait fortement du contexte dans lequel on
utilise la fonction. Une possibilité est de restreindre fortement le contrat en
imposant des propriétés à propos des valeurs et de la taille du tableau. Par
exemple, nous pourrions imposer une taille maximale et des bornes fortes pour
chacune des cellules. Une autre possibilité est d'ajouter une valeur d'erreur
en cas de dépassement (par exemple $-1$), et de spécifier qu'en cas de
dépassement, c'est cette valeur qui est renvoyée.



\levelThreeTitle{Exercices}


\levelFourTitle{Multiplication par 2}


Le programme suivant calcule \CodeInline{2 * x} en utilisant une boucle.
Utiliser une variable fantôme \CodeInline{i} pour exprimer comme invariant que
la valeur de \CodeInline{r} est \CodeInline{i * 2} et compléter la preuve.


\CodeBlockInput{c}{ex-1-times-2.c}


\levelFourTitle{Tableaux}


Cette fonction reçoit un tableau et effectue une boucle dans laquelle nous ne
faisons rien, sauf que nous avons indiqué que le contenu du tableau est modifié.
Cependant, nous voudrions pouvoir prouver qu'en postcondition, le tableau n'a
pas été modifié.


\CodeBlockInput{c}{ex-2-array.c}


Sans modifier la clause \CodeInline{assigns} de la boucle et sans utiliser le mot
clé \CodeInline{\textbackslash{}at}, prouver que la fonction ne modifie pas le
contenu du tableau. Pour cela, compléter le code fantôme et l'invariant de boucle
en assurant que le contenu du tableau \CodeInline{g} représente l'ancien contenu
de \CodeInline{a}.
