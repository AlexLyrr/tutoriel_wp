Les prédicats inductifs nous donne une manière d'énoncer des propriétés dont la
vérification nécessite de raisonner par induction, c'est à dire que pour une propriété
$p(input)$, on peut assurer qu'elle est vraie, soit parce qu'elle correspond à un
certain cas de base (par exemple, $0$ est un entier naturel pair parce que l'on
définit le cas 0 comme un cas de base), ou alors parce que sachant que $p$ est vraie
pour un certain $smaller input$ (qui est "plus proche" du cas de base) et sachant
que $smaller input$ et $input$ sont reliés par une propriété donnée (qui dépend
de notre définition) nous pouvons conclure (par exemple nous pouvons définir que
si un naturel $n$ est pair, le naturel $n+2$ est pair, donc pour vérifier qu'un
naturel supérieur à 0 est pair, on peut regarder si ce naturel moins 2 est pair).



\levelThreeTitle{Syntaxe}


Pour le moment, introduions la syntaxe des prédicats inductifs :

\begin{CodeBlock}{c}
/*@
  inductive property{ Label0, ..., LabelN }(type0 a0, ..., typeN aN) {
  case c_1{Lq_0, ..., Lq_X}: p_1 ;
  ...
  case c_m{Lr_0, ..., Lr_Y}: p_km ;
  }
*/
\end{CodeBlock}


Où tous les \CodeInline{c\_i} sont des noms et tous les \CodeInline{p\_i} sont
des fomules logiques où \CodeInline{property} apparaît en conclusion. Pour résumer,
\CodeInline{property} est vraie pour certains paramètres et labels mémoire, s'ils
valident l'un des cas du prédicat inductif.


Jetons un oeil à la petite propriété dont nous parlions plus tôt: comment définir
qu'un entier naturel est pair par induction ? Nous pouvons traduire la phrase :
"0 est un naturel pair, et si $n$ est un naturel pair, alors $n+2$ est aussi un
naturel pair".


\CodeBlockInput{c}{even-0.c}


Ce prédicat nous définit bien les deux cas :


\begin{itemize}
\item $0$ est un naturel pair (case de base),
\item si un naturel $n$ est pair, $n+2$ est pair aussi (induction)
\end{itemize}



Cependant, cette définition n'est pas complètement satisfaisante. Par exemple,
nous ne pouvons pas déduire qu'un naturel impair n'est pas pair. Si nous essayons
de prouver que $1$ est pair, nous devons vérifier que si $-1$ est pair, puis $-3$,
$-5$, etc. De plus, nous préférons généralement indiquer la condition selon
laquelle une conclusion donnée est vraie en utilisant les variables quantifiées
dans la conclusion. Par exemple, ici, pour montrer qu'un entier $n$ est naturel,
comment faire ? D'abord vérifier s'il est égal à 0, et si ce n'est pas le cas,
vérifier qu'il est plus grand que 0, et dans ce cas, vérifier que $n-2$ est pair :


\CodeBlockInput[1][9]{c}{even-1.c}


Ici nous distinguons à nouveau deux cas :

\begin{itemize}
\item 0 est pair,
\item un naturel $n$ est pair s'il est plus grand que $0$ et $n-2$ est un
      naturel pair.
\end{itemize}


Pour un naturel donné, s'il est plus grand que 0, nous allons décroître
récursivement réduire sa valeur jusqu'à atteindre 0, et dans ce cas, il est pair,
ou -1 et ici nous n'aurons aucun cas du prédicat inductif qui puisse correspondre
à cette valeur et nous pourrons conclure que la propriété est fausse (même si
nous verrons plus tard que nous avons besoin d'un assistant de preuve pour
arriver cette conclusion dans le cas de WP).



\CodeBlockInput[11][14]{c}{even-1.c}


Bien sûr, définir la notion d'entier naturel pair par induction n'est pas une
bonne idée, un modulo serait plus simple. Nous utilisons généralement les
propriété inductives pour définir des propriétés récursives complexes.


\levelFourTitle{Consistance}


Les définitions inductives apportent le risque d'introduire des inconsistances.
En effet, les propriétés spécifiées dans les différents cas sont considérées
comme étant toujours vraies, donc si nous introduisons des propriétés permettant
de prouver \CodeInline{false}, nous sommes en mesure de prouver n'importe quoi.
Même si nous parlerons plus longuement des axiomes dans la
Section~\ref{l2:acsl-logic-definitions-axiomatic}, nous pouvons donner quelques
conseils pour ne pas construire une mauvaise définition.


D'abord, nous pouvons nous assurer que les prédicats inductives sont bien fondés.
Cela peut être fait en restreignant syntaxiquement que nous acceptons dans une
définition inductive en nous assurans que chaque cas est de la forme :

\begin{CodeBlock}{c}
/*@
  \forall y1,...,ym ; h1 ==> ··· ==> hl ==> P(t1,...,tn) ;
*/
\end{CodeBlock}


où le prédicat \CodeInline{P} ne peut apparaître que positivement (donc sans la
négation \CodeInline{!} - $\neg$) dans les différentes hypothèses \CodeInline{hx}.
Intuitivement, cela assure que nous ne pouvons pas constuire des occurrences à la
fois positives et négatives de \CodeInline{P} pour un ensemble de paramètres donnés
(ce qui serait incohérent).



Cette propriété est par exemple vérifiée pour notre définition précédente du
prédicat \CodeInline{even\_natural}. Tandis qu'une définition comme :


\CodeBlockInput[1][11]{c}{even-bad.c}


ne respecte pas cette contrainte car la propriété \CodeInline{even\_natural}
apparaît négativément à la ligne~8.


Ensuite, il est possible de demander la génération de la version Coq de la
preuve d'une propriété qui nécessite le prédicat \CodeInline{P}. Par exemple,
nous pouvons écrire une fonction :



\begin{CodeBlock}{c}
/*@
  requires P( params ... ) ;
  ensures  BAD: \false ;
*/ void function(params){

}
\end{CodeBlock}


Pour notre définition de \CodeInline{even\_natural}, cela donnerait :


\CodeBlockInput[13][18]{c}{even-bad.c}


Comme Coq est plus strict que Frama-C et WP pour ce type de définitions, si
le prédicat inductif est trop étrange (s'il n'est pas bien fondé), il sera
refusé par Coq avec une erreur. Et, en effet, pour la propriété
\CodeInline{even\_natural} que nous venons de définir, Coq la refuse quand
nous tentons de prouver \CodeInline{ensures \textbackslash{}false}, parce
qu'il existe une occurrences non positive de \CodeInline{P\_even\_natural}
qui encode le \CodeInline{even\_natural} que nous avons écrit en ACSL.

\begin{CodeBlock}{bash}
frama-c-gui -wp -wp-prop=BAD -wp-prover=native:coq file.c
\end{CodeBlock}

\image{coq-failed}

\image{coq-failed-error}


Cependant, cela ne signifie pas que nous ne pouvon pas écrire une définition
inductive inconsistante. Par exemple, la définition inductive suivante est bien
fondée mais nous permet de prouver faux :


\begin{CodeBlock}{c}
/*@
  inductive bad {
    case always: \false ;
  }
*/
\end{CodeBlock}


Donc nous devons être prudent pendant le design de définition inductive afin
de ne pas introduire une définition qui nous permette de produire une preuve de
faux.


\levelThreeTitle{Définitions de prédicats récursifs}


Les prédicats inductifs sont souvent utiles pour exprimer des propriéts récursivement
car ils permettent souvent d'empêcher les solveurs SMT de dérouler la récursion quand
c'est possible.


Par exemple, nous pouvons définir qu'un tableau ne contient que des zéros de cette
façon:


\CodeBlockInput[3][11]{c}{reset-0.c}


Et nous pouvons à nouveau prouver notre fonction de remise à zéro avec cette
nouvelle définition :



\CodeBlockInput[14][28]{c}{reset-0.c}



Selon votre version de Frama-C et de vos prouveurs automatiques, la preuve de
préservation de l'invariant peut échouer. Une raison à cela est que le prouveur ne
parvient pas à garder l'information que ce qui précède la cellule en cours de
traitement par la boucle est toujours à 0. Nous pouvons ajouter un lemme dans
notre base de connaissance, expliquant que si l'ensemble des valeurs d'un tableau
n'a pas changé, alors la propriété est toujours vérifiée :



\CodeBlockInput[13][20]{c}{reset-1.c}



Et d'énoncer une assertion pour spécifier ce qui n'a pas changé entre le début
du bloc de la boucle (marqué par le \textit{label} \CodeInline{L} dans le code) et la fin (qui se
trouve être \CodeInline{Here} puisque nous posons notre assertion à la fin) :



\CodeBlockInput[35][39]{c}{reset-1.c}



À noter que dans cette nouvelle version du code, la propriété énoncée par notre
lemme n'est pas prouvée par les solveurs automatiques, qui ne savent pas raisonner
pas induction. Pour les curieux, la (très simple) preuve en Coq est disponible
ici:~\ref{l2:acsl-logic-definitions-answers}.



Dans le cas présent, utiliser une axiomatique est contre-productif : notre
propriété est très facilement exprimable en logique du premier ordre comme
nous l'avons déjà fait précédemment. Les axiomatiques sont faites pour écrire
des définitions qui ne sont pas simples à exprimer dans le formalisme de base
d'ACSL. Mais il est mieux de commencer avec un exemple facile à lire.




\levelThreeTitle{Exemple : le tri}
\label{l3:acsl-logic-definitions-inductive-sort}




Nous allons prouver un simple tri par sélection :



\CodeBlockInput[3]{c}{sort-0.c}


Le lecteur pourra s'exercer en spécifiant et en prouvant les fonctions de
recherche de minimum et d'échange de valeur. Nous cachons la correction
(Réponses:~\ref{l2:acsl-logic-definitions-answers}) et allons nous concentrer
plutôt sur la spécification et la preuve de la fonction de tri qui sont une
illustration intéressant de l'usage des axiomatiques.


En effet, une erreur commune que nous pourrions faire dans le cas de la preuve
du tri est de poser cette spécification (qui est vraie !) :



\CodeBlockInput[6]{c}{sort-incomplete.c}



\textbf{Cette spécification est correcte}. Mais si nous nous rappelons la
partie concernant les spécifications, nous nous devons d'exprimer précisément ce
que nous attendons. Avec la spécification actuelle, nous ne prouvons pas toutes
les propriétés nécessaires d'un tri ! Par exemple, cette fonction remplit
pleinement la spécification :



\CodeBlockInput[8]{c}{sort-false.c}



En fait, notre spécification oublie que tous les éléments qui étaient
originellement présents dans le tableau à l'appel de la fonction doivent
toujours être présents après l'exécution de notre fonction de tri. Dit
autrement, notre fonction doit en fait produire la permutation triée des
valeurs du tableau.



Une propriété comme la définition de ce qu'est une permutation s'exprime
extrêmement bien par l'utilisation d'une défintiion inductive. En effet, pour
déterminer qu'un tableau est la permutation d'un autre, les cas sont très limités.
Premièrement, le tableau est une permutation de lui-même, puis l'échange de
deux valeurs sans changer les autres est également une permutation, et
finalement si nous créons la permutation $p_2$ d'une permutation $p_1$, puis que
nous créons la permutation $p_3$ de $p_2$, alors par transitivité $p_3$ est une
permutation de $p_1$.



La définition inductive correspondante est la suivante :



\CodeBlockInput[37][56]{c}{sort-1.c}



Nous spécifions alors que notre tri nous crée la permutation triée du tableau
d'origine et nous pouvons prouver l'ensemble en complétant l'invariant de la
fonction :



\CodeBlockInput[64][84]{c}{sort-1.c}


Cette fois, notre propriété est précisément définie, la preuve reste assez
simple à faire passer, ne nécessitant que l'ajout d'une assertion que le bloc
de la fonction n'effectue qu'un échange des valeurs dans le tableau (et donnant
ainsi la transition vers la permutation suivante du tableau). Pour définir cette
notion d'échange, nous utilisons une annotation particulière (à la ligne 16),
introduite par le mot-clé \CodeInline{ghost}. Ici, le but est d'introduire un \textit{label}
fictif dans le code qui est uniquement visible d'un point de vue spécification.
C'est l'objet de la section finale de ce chapitre, parlons d'abord des définitions
axiomatiques.



\levelThreeTitle{Exercices}



\levelFourTitle{La somme des N premiers entiers}


Reprendre la solution de l'exerice~\ref{l4:acsl-properties-lemmas-n-first-ints} à
propos de la somme des N premiers entiers. Réécrire la fonction logique en utilisant
plutôt un prédicat inductif qui exprimer l'égalité entre un entier et la somme des
N premiers entiers.



\CodeBlockInput{c}{ex-1-sum-N-first-ints.c}


Adapter le contrat de la fonction et le(s) lemme(s). Notons que les lemmes ne
seront certainement pas prouvés par les solveurs SMT. Nous fournissons une solution
et les preuves Coq sur le répertoire GitHub de ce livre.



\levelFourTitle{Plus grand diviseur commun}


Écrire un prédicat inductif qui exprime qu'un entier est le plus grand diviseur
commun de deux autres. Le but de cet exercice est de prouver que la fonction
\CodeInline{gcd} calcule le plus grand diviseur commun. Nous n'avons donc pas
à spécifier tous les cas du prédicat. En effey, si nous regardons de près la boucle,
nous pouvons voir qu'après la première itération, \CodeInline{a} est supérieur ou
égal à \CodeInline{b}, et que cette propriété est maintenue par la boucle. Donc,
considérons deux cas pour le prédicat inductif :

\begin{itemize}
\item \CodeInline{b} est 0,
\item si une valeur \CodeInline{d} est le PGCD de \CodeInline{b} et \CodeInline{a \% b},
      alors c'est le PGCD de \CodeInline{a} et \CodeInline{b}.
\end{itemize}


\CodeBlockInput{c}{ex-2-gcd.c}


Exprimer la postcondition de la fonction et compléter l'invariant pour prouver
qu'elle est vérifiée. Notons que l'invariant devrait utiliser le cas inductive
\CodeInline{gcd\_succ}.



\levelFourTitle{Puissance}


Dans cet exercice, nous ne considèrerons pas les RTEs.


Écrire un prédicat inductif qui exprimer qu'un entier \CodeInline{r} est égal
à $x^n$. Considérer deux cas: soit $n$ est 0, soit il est plus grand et à ce moment
là, la valeur de \CodeInline{r} est reliée à la valeur $x^{n-1}$.


\CodeBlockInput[1][6]{c}{ex-3-fpower.c}


Prouver d'abord la version naïve de la fonction puissance:


\CodeBlockInput[13][28]{c}{ex-3-fpower.c}


Maintenant tentons de prouver une version plus rapide de la fonction puissance:


\CodeBlockInput[30][49]{c}{ex-3-fpower.c}


Dans cette version, nous exploitons deux propriétés de l'opérateur puissance:


\begin{itemize}
\item $(x^2)^n = x^{2n}$
\item $x * (x^2)^n = x^{2n+1}$
\end{itemize}


Qui permet de diviser $n$ par 2 à chaque tour de boucle au lieu de le décrémenter
de un (ce qui permet à l'algorithme d'être en $0(log n)$ et pas $O(n)$). Exprimer
les deux propriétés précédentes sous forme de lemmes:


\CodeBlockInput[8][11]{c}{ex-3-fpower.c}


Exprimer d'abord le lemme \CodeInline{power\_even}, les solveurs SMT pourraient
être capables de combiner ce lemme avec la définition inductive pour déduire
\CodeInline{power\_odd}. La preuve Coq de \CodeInline{power\_even} est fournie
sur le répertoire GitHub de ce livre, ainsi que la preuve de \CodeInline{power\_odd}
si les solveurs SMT échouent.


Finalement, compléter le contrat et l'invariant de la fonction \CodeInline{fast\_power}.
Pour cela, notons qu'au début de la boucle $x^{old(n)} = p^n$, et que chaque itération
utilise les propriétés précédentes pour mettre à jour $r$, au sens que nous avons
$x^{old(n)} = r * p^n$ pendant toute la boucle, jusqu'à obtenir $n = 0$ et donc
$p ^n = 1$.



\levelFourTitle{Permutation}



Reprendre la définition des prédicats \CodeInline{shifted} et \CodeInline{unchanged}
de l'exercice~\ref{l4:acsl-properties-lemmas-shift-trans}. Utiliser le prédicat
\CodeInline{shifted} pour exprimer le prédicat \CodeInline{rotate} qui exprime que
les éléments d'un tableau sont "tournés" vers la gauche, dans le sens où tous les
éléments sont glissés vers la gauche, sauf le dernier qui est mis à la première
cellule de la plage de valeur. Utiliser ce prédicat pour prouver la fonction
\CodeInline{rotate}.



\CodeBlockInput[12][27]{c}{ex-4-rotate-permutation.c}


Exprimer une nouvelle version de la notion de permutation avec un prédicat inductif
qui considère les cas suivants :

\begin{itemize}
\item la permutation est réflexive,
\item si la partie gauche d'une plage de valeur (jusqu'à un certain indice) est
      "tournée" entre deux labels, nous avons toujours un permutation,
\item si la partie droite d'une plage de valeur (à partir d'un certain indice) est
      "tournée" entre deux labels, nous avons toujours un permutation,
\item la permutation est transitive.
\end{itemize}


\CodeBlockInput[30][37]{c}{ex-4-rotate-permutation.c}


Compléter le contrat de la fonction \CodeInline{two\_rotates} qui fait de rotation
successives, de la première et la seconde moitié de la plage considérée, et prouver
qu'elle maintient la permutation.


\CodeBlockInput[43][47]{c}{ex-4-rotate-permutation.c}
