\levelThreeTitle{Appel de fonction}

\levelFourTitle{Formel - Calcul de plus faible précondition}

Lorsqu'une fonction est appelée, le contrat de cette fonction est utilisé pour
déterminer la précondition de l'appel. Mais il est important de garder en tête deux
aspects important pour exprimer le calcul de plus faible précondition.


Premièrement, la postcondition d'une fonction $f$ qui serait appelée dans un 
programme n'est pas nécessairement directement la précondition calculée pour le code
qui suit l'appel à $f$. Par exemple, si nous avons un programme:
\CodeInline{x = f() ; c }, et si $wp(\texttt{c}, Q) = 0 \leq x \leq 10$ alors que
la postcondition de la fonction \CodeInline{f} est $1 \leq x \leq 9$, nous avons
besoin d'exprimer l'affaiblissement de la précondition réelle de \CodeInline{c}
vers celle que l'on a calculé. Pour cela cela, nous renvoyons à la
section~\ref{l3:statements-basic-consequence}, l'idée est simplement de vérifier 
que la postcondition de la fonction implique la précondition calculée.



Deuxièmement, en C, une fonction peut avoir des effets de bord. Par conséquent, les
valeurs référencées en entrée de fonction ne restent pas nécessairement les mêmes
après l'appel à la fonction, et le contrat devrait exprimer certaines propriétés à
propos des valeurs avant et après l'appel. Donc, si nous avons des labels (C) dans
la postcondition, nous devons faire les remplacements qui s'imposent par rapport au
lieu d'appel.



Pour définir le calcul de plus faible précondition associé aux fonctions, introduisons
d'abord quelques notations pour rendre les explications plus claires. Pour cela,
considérons l'exemple suivant :

\begin{CodeBlock}{c}
/*@ requires \valid(x) && *x >= 0 ;
    assigns *x ;
    ensures *x == \old(*x)+1 ; */
void inc(int* x);

void foo(int* a){
  L1:
  inc(a) ;
  L2:
}
\end{CodeBlock}


Le calcul de plus faible précondition de l'appel de fonction nous demande de
considérer le contrat de la fonction appelée (ici, dans \CodeInline{foo}, quand
nous appelons la fonction \CodeInline{inc}). Bien sûr, avant l'appel de la fonction
nous devons vérifier sa précondition, qui fait donc partie de la plus faible
précondition. Mais, nous devons aussi considérer la postcondition de la fonction,
sinon cela voudrait dire que nous ne prenons pas en compte son effet sur l'état du
programme.


Par conséquent, il est important de noter que dans la précondition, l'état mémoire
considéré est bien celui pour lequel la plus faible précondition doit être vraie,
tandis que pour la postcondition, ce n'est pas le cas: l'état considéré est celui
qui suit l'appel, alors que dans la postcondition, lorsque nous parlons des valeurs
avant l'appel nous devons explicitement ajouter le mot clé 
\CodeInline{\textbackslash{}old}. Par exemple, pour le contrat de \CodeInline{inc}
lorsque nous l'appelons depuis \CodeInline{foo}, \CodeInline{*x} dans la
précondition est \CodeInline{*a} au label \CodeInline{L1}, alors que \CodeInline{*x}
dans la postcondition est \CodeInline{*a} au label \CodeInline{L2}. Par conséquent,
la pré et la postcondition doivent être considérées de manière légèrement différentes
lorsque nous devons parler des positions mémoire mutables. Notez que pour la valeur
du paramètre \CodeInline{x} lui même, ce n'est pas le cas : cette valeur ne peut pas
être modifiée par l'appel (du point de vue de l'appelant).
 


Maintenant, définissons le calcul de plus faible précondition d'un appel de fonction.
Pour cela, notons:

\begin{itemize}
\item $\vec{v}$ un vecteur de valeurs $v_1, ..., v_n$ et $v_i$ la $i^{ème}$ valeur,
\item $\vec{t}$ les arguments fournies à la fonction lors de l'appel,
\item $\vec{x}$ les paramètres dans la définition de la fonction,
\item $\vec{a}$ les valeurs modifiées (vues de l'extérieur, une fois instanciées),
\item $here(x)$ une valeur en postcondition,
\item $old(x)$ une valeur en précondition. 
\end{itemize}

Nous nommons $\texttt{f:Pre}$ la précondition de la fonction, et $\texttt{f:Post}$,
la postcondition.

\begin{center}
\begin{tabular}{rl}
  $wp( f(\vec{t}), Q ) :=$ & $\texttt{f:Pre}[x_i \leftarrow t_i]$ \\
  $\wedge$ & $\forall \vec{v}, \quad (
              \texttt{f:Post}[x_i \leftarrow t_i,
                              here(a_j) \leftarrow v_j,
                              old(a_j) \leftarrow a_j] \Rightarrow
              Q[here(a_j) \leftarrow v_j])$
\end{tabular}
\end{center}

Nous pouvons détailler les étapes du raisonnement dans les différentes parties de
cette formule.


Premièrement, notons que dans les pré et postconditions, chaque paramètre $x_i$
est remplacé par l'argument correspondant ($[x_i \leftarrow t_i]$), comme nous
l'avons dit juste avant, nous n'avons pas de question d'état mémoire à considérer
ici puisque ces valeurs ne peuvent pas être changées par l'appel de fonction. Par
exemple dans le contrat de \CodeInline{inc}, chaque occurrence de \CodeInline{x}
serait remplacée par l'argument \CodeInline{a}.


Ensuite, dans la partie de notre formule qui correspond à la postcondition, nous
pouvons voir que nous introduisons un $\forall \vec{v}$. Le but ici est de modéliser
le fait que la fonction peut changer la valeur des positions mémoire spécifiées par
la clause \CodeInline{assigns} du contrat. Donc, pour chaque position potentiellement
modifiée $a_j$ (qui est, pour notre exemple d'appel à \CodeInline{inc}, 
\CodeInline{*(\&a)}), nous générons une valeur $v_j$ qui représente sa valeur après
l'appel. Mais si nous voulons vérifier que la postcondition nous donne le bon 
résultat, nous ne pouvons pas accepter \emph{toute valeur} pour les positions mémoire
modifiées, nous voulons celles \emph{qui permettent de satisfaire la postcondition}.



Nous utilisons donc ces valeur pour transformer la postcondition de la fonction et
pour vérifier qu'elle implique la postcondition reçue en entrée du calcul de plus
faible précondition. Nous faisons cela en remplaçant, pour chaque position mémoire
modifiée $a_j$, sa valeur $here$ avec la valeur $v_j$ qu'elle obtient après l'appel
($here(a_j) \leftarrow v_j$). Finalement, nous devons remplacer chaque $old$ valeur
par sa valeur avant l'appel, et pour chaque $old(a_j)$, cette valeur est
simplement $a_j$ ($old(a_j) \leftarrow a_j$).



\levelFourTitle{Formel - Exemple}


Illustrons tout cela sur un exemple en appliquant le calcul de plus faible
précondition sur ce petit code, en supposant le contrat précédemment proposé pour
la fonction \CodeInline{swap}.

\begin{CodeBlock}{c}
  int a = 4 ;
  int b = 2 ;

  swap(&a, &b) ;

  //@ assert a == 2 && b == 4 ;
\end{CodeBlock}


Nous pouvons appliquer le calcul de plus faible précondition :

\begin{tabular}{l}
  $wp(a = 4; b = 2; swap(\&a, \&b), a = 2 \wedge b = 4) = $\\
  $\quad wp(a = 4, wp(b = 2; swap(\&a, \&b), a = 2 \wedge b = 4)) = $\\
  $\quad wp(a = 4, wp(b = 2, wp(swap(\&a, \&b), a = 2 \wedge b = 4)))$
\end{tabular}


Et considérons séparément :
$$wp(swap(\&a, \&b), a = 2 \wedge b = 4)$$


Par la clause \CodeInline{assigns}, nous savons que les valeurs modifiées par la
fonction sont $*(\&a) = a$ et $*(\&b) = b$. (nous raccourcissons $here$ avec $H$ 
et $old$ avec $O$).

\begin{tabular}{rl}
  $\quad \quad \texttt{swap:Pre}[x \leftarrow \&a,\ y \leftarrow \&b]$ & \\
  $\quad \wedge \forall v_a, v_b,(\texttt{swap:Post}$ & $ [ x \leftarrow \&a,\ y \leftarrow \&b, $ \\
                               & $H(*(\&a)) \leftarrow v_a,\ H(*(\&b)) \leftarrow v_b,$ \\
                               & $O(*(\&a)) \leftarrow *(\&a),\ O(*(\&b)) \leftarrow *(\&b)])$\\
  \multicolumn{2}{r}{$\quad \quad \Rightarrow (H(a) = 2 \wedge H(b) = 4)[H(a)) \leftarrow v_a, H(b)) \leftarrow v_b])$}
\end{tabular}


Pour la précondition, nous obtenons :
$$valid(\&a) \wedge valid(\&b)$$
Pour la postcondition, commençons par écrire l'expression depuis laquelle nous
allons travailler avant de faire les remplacements (et sans la syntaxe de remplacement
pour rester concis) :
$$H(*x) = O(*y) \wedge H(*y) = O(*x) \Rightarrow H(a) = 2 \wedge H(b) = 4$$
Remplaçons d'abord les pointeurs ($x \leftarrow \&a,\ y \leftarrow \&b$) :
$$H(*(\&a)) = O(*(\&b)) \wedge H(*(\&b)) = O(*(\&a)) \Rightarrow H(a) = 2 \wedge H(b) = 4$$
Puis les valeurs $here$, avec les valeurs quantifiées $v_i$ ($H(a)) \leftarrow v_a, H(b)) \leftarrow v_b$) :
$$v_a = O(*(\&b)) \wedge v_b = O(*(\&a)) \Rightarrow v_a = 2 \wedge v_b = 4$$
Et les valeurs $old$, avec les valeurs avant l'appel ($O(*(\&a)) \leftarrow *(\&a),\ O(*(\&b)) \leftarrow *(\&b)$) :
$$v_a = *(\&b) \wedge v_b = *(\&a) \Rightarrow v_a = 2 \wedge v_b = 4$$
Nous pouvons maintenant simplifier cette formule en :
$$v_a = b \wedge v_b = a \Rightarrow v_a = 2 \wedge v_b = 4$$


Donc, $wp(swap(\&a, \&b), a = 2 \wedge b = 4)$ est :
$$P: valid(\&a) \wedge valid(\&b) \wedge \forall v_a, v_b, \quad v_a = b \wedge v_b = a \Rightarrow v_a = 2 \wedge v_b = 4$$
Nous pouvons immédiatement simplifier cette formule en constatant que les propriétés
de validité sont trivialement vraies (puisque les variables sont allouées sur la pile
juste avant) :
$$P: \forall v_a, v_b, \quad v_a = b \wedge v_b = a \Rightarrow v_a = 2 \wedge v_b = 4$$


Maintenant, calculons $wp(a = 4, wp(b = 2, P)))$, en remplaçant d'abord $b$
par $2$ par la règle d'affectation :
$$\forall v_a, v_b, \quad v_a = 2 \wedge v_b = a \Rightarrow v_a = 2 \wedge v_b = 4$$
et ensuite $a$ par $4$ par la même règle :
$$\forall v_a, v_b, \quad v_a = 2 \wedge v_b = 4 \Rightarrow v_a = 2 \wedge v_b = 4$$


Cette dernière propriété est trivialement vraie, le programme est vérifié.


\levelFourTitle{Que devrions nous garder en tête ?}


Les fonctions sont absolument nécessaire pour programmer modulairement, et le calcul
de plus faible précondition est pleinement compatible avec cette idée, permettant de
raisonner localement à propos de chaque fonction et donc de composer les preuves juste
de la même manière que nous composons les appels de fonction.


Comme pense-bête, nous devrions toujours garder en tête le schéma suivant:


\begin{CodeBlock}{c}
/*@
  requires foo_R ;
  assigns ... ;
  ensures foo_E ;
*/
type foo(parameters...){
  // Ici, nous supposons que foo_R est vérifiée


  // Ici, nous devons prouver que bar_R est vérifiée
  bar(some parameters ...) ;
  // Ici, nous supposons que bar_E est vérifiée


  // Ici nous devons prouver que foo_E est vérifiée
  return ... ;
}
\end{CodeBlock}


Notons qu'à propos du dernier commentaire, en calcul de plus faible précondition,
l'idée est plutôt de montrer que notre précondition est assez forte pour assurer
que le code nous amène à la postcondition. Cependant, premièrement, cette vision
est plus facile à comprendre et deuxièmement, un greffon comme WP (et comme n'importe
quel outil réaliste pour la preuve de programme) ne suit pas strictement un calcul
de plus faible précondition mais une manière fortement optimisée de calculer les
conditions de vérification qui ne suit pas exactement les mêmes règles.


\levelThreeTitle{Fonctions récursives}


Actuellement, WP ne vérifie pas encore la terminaison des fonctions. Bien sûr, si
une fonction est seulement composée de boucles qui terminent (dont le variant 
indiqué est vérifié) et des appels de fonctions qui elles-mêmes terminent, elle
termine. En revanche, parfois ce raisonnement est insuffisant : dans le cas des
fonctions récursives et mutuellement récursives. C'est la terminaison de ces 
fonctions qui n'est pas encore vérifiée par WP.


Cela signifie que nous pouvons, par mégarde, prouver n'importe quoi, si l'on 
défini et prouve une fonction qui ne termine pas.


\CodeBlockInput{c}{trick.c}


\image{recursive-trick}


Nous pouvons voir que la fonction et l'assertion sont prouvées. Et, effectivement,
la preuve est correcte : nous considérons une correction partielle (puisque nous ne
pouvons pas prouver la terminaison), et cette fonction ne termine pas. Toute
assertion suivant cette fonction est vraie: elle est inatteignable.


Une question que l'on peut alors se poser est donc: que peut on faire dans un tel 
cas ? Nous pouvons à nouveau utiliser une notion de variant pour borner le nombre
d'appels récursifs. En ACSL, c'est le rôle de la clause \CodeInline{decreases} :


\CodeBlockInput{c}{decreases.c}


Cette clause exprime essentiellement la même idée que le \CodeInline{loop variant}.
L'expression spécifiée par la clause \CodeInline{decreases} est une valeur positive
qui décroît strictement à chaque nouvel appel récursif de la fonction. Cependant,
cette clause n'est pas encore supporté par WP, donc nous ne pouvons pas encore 
faire la preuve totale d'une fonction récursive avec le greffon WP.
