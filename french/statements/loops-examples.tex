\levelThreeTitle{Exemple avec un tableau read-only}


S'il y a une structure de données que nous traitons avec les boucles c'est bien
le tableau. C'est une bonne base d'exemples pour les boucles car ils permettent
rapidement de présenter des invariants intéressants et surtout, ils vont nous 
permettre d'introduire des constructions très importantes d'ACSL.



Prenons par exemple la fonction qui cherche une valeur dans un tableau :



\CodeBlockInput{c}{search.c}



Cet exemple est suffisamment fourni pour introduire des notations importantes.



D'abord, comme nous l'avons déjà mentionné, le prédicat \CodeInline{\textbackslash{}valid\_read} (de 
même que \CodeInline{\textbackslash{}valid}) nous permet de spécifier non seulement la validité d'une 
adresse en lecture mais également celle de tout un ensemble d'adresses 
contiguës. C'est la notation que nous avons utilisée dans cette expression :



\begin{CodeBlock}{c}
//@ requires \valid_read(a + (0 .. length-1));
\end{CodeBlock}



Cette pré-condition nous atteste que les adresses a+0, a+1 ..., a+length-1 sont
valides en lecture.



Nous avons également introduit deux notations qui vont nous être très utiles, à 
savoir \CodeInline{\textbackslash{}forall} ($\forall$) et \CodeInline{\textbackslash{}exists} ($\exists$), les 
quantificateurs de la logique. Le premier nous servant à annoncer que pour tout
élément, la propriété suivante est vraie. Le second pour annoncer qu'il existe
un élément tel que la propriété est vraie. Si nous commentons les deux lignes en 
questions, nous pouvons les lire de cette façon :



\begin{CodeBlock}{c}
/*@
//pour tout "off" de type "size_t", tel que SI "off" est compris entre 0 et "length"
//                                 ALORS la case "off" de "a" est différente de "element"
\forall size_t off ; 0 <= off < length ==> a[off] != element;

//il existe "off" de type "size_t", tel que "off" soit compris entre 0 et "length"
//                                 ET que la case "off" de "a" vaille "element"
\exists size_t off ; 0 <= off < length && a[off] == element;
*/
\end{CodeBlock}



Si nous devions résumer leur utilisation, nous pourrions dire que sur un certain
ensemble d'éléments, une propriété est vraie, soit à propos d'au moins l'un
d'eux, soit à propos de la totalité d'entre eux. Un schéma qui reviendra 
typiquement dans ce cas est que nous restreindrons cet ensemble à travers une
première propriété (ici : \CodeInline{0 <= off < length}) puis nous voudrons prouver la
propriété réelle qui nous intéresse à propos d'eux. \textbf{Mais il y a une 
différence fondamentale entre l'usage de \CodeInline{exists} et celui de \CodeInline{forall}}.



Avec \CodeInline{\textbackslash{}forall type a ; p(a) ==> q(a)}, la restriction (\CodeInline{p}) est suivie
par une implication. Pour tout élément, s'il respecte une première propriété 
(\CodeInline{p}), alors vérifier la seconde propriété \CodeInline{q}. Si nous mettions un ET
comme pour le « il existe » (que nous expliquerons ensuite), cela voudrait dire que 
nous voulons que tout élément respecte à la fois les deux propriétés. Parfois, 
cela peut être ce que nous voulons exprimer, mais cela ne correspond alors plus 
à l'idée de restreindre un ensemble dont nous voulons montrer une propriété 
particulière.



Avec \CodeInline{\textbackslash{}exists type a ; p(a) \&\& q(a)}, la restriction (\CodeInline{p}) est suivie
par une conjonction, nous voulons qu'il existe un élément tel que cet élément 
est dans un certain état (défini par \CodeInline{p}), tout en respectant l'autre 
propriété \CodeInline{q}. Si nous mettions une implication comme pour le « pour tout », 
alors une telle expression devient toujours vraie à moins que \CodeInline{p} soit une 
tautologie ! Pourquoi ? Existe-t-il « a » tel que p(a) implique q(a) ? Prenons 
n'importe quel « a » tel que p(a) est faux, l'implication devient vraie.



Cette partie de l'invariant mérite une attention particulière :



\begin{CodeBlock}{c}
//@ loop invariant \forall size_t j; 0 <= j < i ==> array[j] != element;
\end{CodeBlock}



En effet, c'est la partie qui définit l'action de notre boucle, elle indique à
WP ce que la boucle va faire (ou apprendre dans le cas présent) tout au long de
son exécution. Ici en l'occurrence, cette formule nous dit qu'à chaque tour, nous 
savons que pour toute case entre 0 et la prochaine que nous allons visiter (\CodeInline{i} exclue), elle stocke une valeur différente de l'élément recherché.



Le but de WP associé à la préservation de cet invariant est un peu compliqué, il
n'est pour nous pas très intéressant de se pencher dessus. En revanche, la 
preuve de l'établissement de cet invariant est intéressante :



\image{trivial.png}[But trivial]


Nous pouvons constater que cette propriété, pourtant complexe, est prouvée par 
Qed sans aucun problème. Si nous regardons sur quelles parties du programme la 
preuve se base, nous pouvons voir l'instruction \CodeInline{i = 0} surlignée, et c'est 
bien la dernière instruction que nous effectuons sur \CodeInline{i} avant de commencer
la boucle. Et donc effectivement, si nous faisons le remplacement dans la formule 
de l'invariant :



\begin{CodeBlock}{c}
//@ loop invariant \forall size_t j; 0 <= j < 0 ==> array[j] != element;
\end{CodeBlock}



« Pour tout j, supérieur ou égal à 0 et inférieur strict à 0 », cette partie est
nécessairement fausse. Notre implication est donc nécessairement vraie.



\levelThreeTitle{Exemples avec tableaux mutables}


Nous allons voir deux exemples avec la manipulation de tableaux en mutation. 
L'un avec une modification totale, l'autre en modification sélective.



\levelFourTitle{Remise à zéro}


Regardons la fonction effectuant la remise à zéro d'un tableau.



\CodeBlockInput{c}{reset.c}



Les seules parties sur lesquelles nous pouvons nous attacher ici sont 
les \CodeInline{assigns} de la fonction et de la boucle. À nouveau, nous pouvons
utiliser la notation \CodeInline{n .. m} pour indiquer les parties du tableau 
qui sont modifiées.



\levelFourTitle{Chercher et remplacer}


Le dernier exemple qui nous intéresse est l'algorithme « chercher et remplacer ». 
C'est donc un algorithme qui va sélectivement modifier des valeurs dans une 
certaine plage d'adresses. Il est toujours un peu difficile de guider l'outil 
dans ce genre de cas car, d'une part, nous devons garder « en mémoire » ce qui est modifié 
et ce qui ne l'est pas et, d'autre part, parce que l'induction repose sur ce fait.



À titre d'exemple, la première spécification que nous pouvons réaliser pour 
cette fonction ressemblerait à ceci :



\CodeBlockInput{c}{search_and_replace-0.c}



Nous utilisons la fonction logique \CodeInline{\textbackslash{}at(v, Label)} qui nous donne la valeur de
la variable \CodeInline{v} au point de programme \CodeInline{Label}. Si nous regardons l'utilisation qui
en est faite ici, nous voyons que dans l'invariant de boucle, nous cherchons à 
établir une relation entre les anciennes valeurs du tableau et leurs potentielles 
nouvelles valeurs :



\begin{CodeBlock}{c}
/*@
  loop invariant \forall size_t j; 0 <= j < i && \at(array[j], Pre) == old 
                   ==> array[j] == new;
  loop invariant \forall size_t j; 0 <= j < i && \at(array[j], Pre) != old 
                   ==> array[j] == \at(array[j], Pre);
*/
\end{CodeBlock}



Pour tout élément que nous avons visité, s'il valait la valeur qui doit être
remplacée, alors il vaut la nouvelle valeur, sinon il n'a pas changé. En fait, si nous essayons de prouver l'invariant, WP n'y parvient pas. Dans ce genre de 
cas, le plus simple est encore d'ajouter diverses assertions exprimant les 
propriétés intermédiaires que nous nous attendons à voir facilement prouvées 
et impliquant l'invariant. En fait, nous nous apercevons rapidement que WP 
n'arrive pas à maintenir le fait que nous n'avons pas encore modifié la fin du 
tableau :



\begin{CodeBlock}{c}
for(size_t i = 0; i < length; ++i){
    //@assert array[i] == \at(array[i], Pre); // échec de preuve
    if(array[i] == old) array[i] = new;
}
\end{CodeBlock}



Nous pouvons donc ajouter cette information comme invariant :



\CodeBlockInput[13][26]{c}{search_and_replace-1.c}



Et cette fois, la preuve passera. À noter que si nous tentons la preuve 
directement avec la vérification des RTE, il est possible qu'Alt-Ergo n'y
parvienne pas (CVC4 décharge l'ensemble sans problème). Dans ce cas, nous
pouvons faire séparément les deux preuves (sans, puis avec RTE) ou encore 
ajouter des assertions permettant de guider la preuve dans la boucle :



\CodeBlockInput[24][33]{c}{search_and_replace-2.c}



À mesure que nous cherchons à prouver des propriétés plus compliquées et 
notamment dépendantes de boucles, il va y avoir une part de tâtonnement pour
comprendre ce qui manque au prouveur pour réussir la preuve.



Ce qui peut lui manquer, ce sont des hypothèses. Dans ce type de cas, nous
pouvons tenter d'ajouter des assertions au code pour guider le prouveur. Avec
de l'expérience, nous pouvons regarder le contenu des obligations de preuve ou 
tenter de commencer la preuve avec Coq pour voir si la preuve semble réalisable. 
Parfois le prouveur manque juste de temps, auquel cas, il suffit d'augmenter 
(parfois de beaucoup) la durée du \textit{timeout}. Finalement, la propriété peut 
également être hors de portée du prouveur. Auquel cas, il faudra écrire une
preuve à la main avec un prouveur interactif.



Enfin, il reste le cas où l'implémentation est effectivement fausse, et dans ce
cas, il faut la corriger. Et c'est là que nous utiliserons plutôt le test que la
preuve, car le test permet de prouver la présence d'un bug.
