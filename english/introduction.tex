\begin{Warning}
  The source code of
  the tutorial is available on GitHub, as well as the solutions to the
  different exercises (including the Coq proofs of some properties).

  If you find some errors, please do not hesitate to post an issue or
  a pull request at:

  \externalLink{https://github.com/AllanBlanchard/tutoriel\_wp}{https://github.com/AllanBlanchard/tutoriel_wp}
\end{Warning}


\begin{Information}
  In this tutorial, some examples and some elements of organization are
  similar to the ones used in the
  \externalLink
      {TAP 2013 tutorial}
      {https://frama-c.com/download/publications/tutorial_tap2013_slides.pdf}
  by Nikolai Kosmatov, Virgile Prevosto and Julien
  Signoles of the CEA LIST, since it is quite didactic. It also
  contains examples taken from
  \textit{\externalLink{ACSL By Example}{https://github.com/fraunhoferfokus/acsl-by-example}}
  by Jochen Burghardt, Jens Gerlach, Kerstin Hartig, Hans Pohl and Juan
  Soto from the Fraunhofer. For formal aspects, I verified my statements and
  explanations using the course on Why3 given by Andrei Paskevich
  \textit{\externalLink{at EJCP 2018}{https://ejcp2018.sciencesconf.org/file/441131}}.
  The remaining ideas come from my personal experience with Frama-C and WP.

  \horizontalLine

  The versions of the tools considered in this tutorial are:
  \begin{itemize}
  \item Frama-C 21.1 Scandium
  \item Why3 1.3.1
  \item Alt-Ergo 2.3.2
  \item Coq 8.11.2 (for provided scripts, not used in the tutorial)
  \item Z3 4.8.4 (in one example, it is not absolutely necessary to follow the course)
  \end{itemize}
  Depending on the versions used by the reader some differences could appear in
  what is proved and what is not. Some presented features are only available
  in recent versions of Frama-C.

  Coq proof scripts should work at least from version 8.7.2 to 8.11.2.

  \horizontalLine

  The only requirement to this tutorial is to have a basic knowledge of the C
  language, and at least to be familiar with the notion of pointer.
\end{Information}

\newpage

Despite its old age, C is still a widely used programming language.
Indeed, no other language can pretend to be available on so many
different (hardware and software) platforms, its low-level orientation
and the amount of time invested in the optimization of its compilers
allows to generate very light and efficient machine code (if the code
allows it of course), and that there are a lot of experts in C language,
which is an important knowledge base.


Furthermore, a lot of systems rely on a huge amount of code historically
written in C, that needs to be maintained and sometimes fixed, as it
would be far too costly to rewrite these systems.


But anyone who has already developed with C also knows that it is very
hard to perfectly master this language. There are numerous reasons, but
the complexity of the ISO C, and the fact that it is extremely permissive,
especially about memory management, make the development of robust C
program very hard, even for an experienced programmer.



However, the C language is often chosen for critical systems (avionics,
railway, armament, \ldots{}) where it is appreciated for its good
performances, its technological maturity and the predictability of its
compilation.



In such cases, the needs in terms of test coverage of the source code
become extremely high. Thus, the question ``is our software tested
enough?'' becomes a question to which it is very hard to answer. Program
proof can help us.
Rather than testing all possible and (un)imaginable inputs of the program,
we will \emph{statically} and \emph{mathematically} prove that there
cannot be any problem at runtime.



The goal of this tutorial is to use Frama-C, a tool developed at CEA
LIST, and WP, its deductive proof plugin, to learn the basics about C
program proof. More than the use of the tool itself, the goal of this
tutorial is to convince that it is more and more possible to write
programs without any programming error, but also to sensitize to simple
notions that allows to better understand and write programs.



\begin{Information}
  Many thanks to the different beta-testers for their constructive
  feedback:

\begin{itemize}
\item \externalLink{Taurre}{https://zestedesavoir.com/membres/voir/Taurre/}
\item \externalLink{barockobamo}{https://zestedesavoir.com/membres/voir/barockobamo/}
\item \externalLink{Vayel}{https://zestedesavoir.com/membres/voir/Vayel/}
\item \externalLink{Aabu}{https://zestedesavoir.com/membres/voir/Aabu/}
\end{itemize}
  I thank ZesteDeSavoir validators who helped me improve again the quality of
  this tutorial:

\begin{itemize}
\item \externalLink{Taurre}{https://zestedesavoir.com/membres/voir/Taurre/} (again)
\item \externalLink{Saroupille}{https://zestedesavoir.com/membres/voir/Saroupille/}
\item \externalLink{Aabu}{https://zestedesavoir.com/membres/voir/Aabu/} (again)
\end{itemize}
  Finally, many thanks to Jens Gerlach for his help during the translation of
  this tutorial from French to English, and to Rafael Bachmann for their reviews
  and remarks.
\end{Information}
