\begin{Warning}
  Please note that this document is the very first English version of the
  tutorial. If you find some errors, please do not hesitate to contribute at:

  \externalLink{https://github.com/AllanBlanchard/tutoriel\_wp}{https://github.com/AllanBlanchard/tutoriel_wp}

  Please use the markdown files, the LaTeX file being generated from them.
\end{Warning}


\begin{Information}
  In this tutorial, some examples and some elements of organization are
  similar to the ones used in the 
  \externalLink
      {TAP 2013 tutorial}
      {https://frama-c.com/download/publications/tutorial_tap2013_slides.pdf} 
  by Nikolai Kosmatov, Virgile Prevosto and Julien
  Signoles of the CEA LIST, since it is quite didactic. It also
  contains examples taken from
  \textit{\externalLink{ACSL By Example}{https://github.com/fraunhoferfokus/acsl-by-example}}
  by Jochen Burghardt, Jens Gerlach, Kerstin Hartig, Hans Pohl and Juan
  Soto from the Fraunhofer. The remaining ideas come from my personal
  experience with Frama-C and WP. The only requirement to this tutorial
  is to have a basic knowledge of the C language, and at least to
  be familiar with the notion of pointer.
\end{Information}


Despite its old age, C is still a widely used programming language.
Indeed, no other language can pretend to be available on so many
different (hardware and software) platforms, its low-level orientation
and the amount of time invested in the optimization of its compilers
allows to generate very light and efficient machine code (if the code
allows it of course), and that there are a lot of experts in C language,
which is an important knowledge base.


Furthermore, a lot of systems rely on a huge amount of code historically
written in C, that needs to be maintained and sometimes fixed, as it
would be far too costly to rewrite these systems.


But anyone who has already developed with C also know that it is very
hard to perfectly master this language. There are numerous reasons, but
ambiguities in the ISO C, and the fact that it is extremely permissive,
especially about memory management, make the development of robust C
program very hard, even for an experienced programmer.



However, the C language is often chosen for critical systems (avionics,
railway, armament, \ldots{}) where it is appreciated for its good
performances, its technological maturity and the predictability of its
compilation.



In such cases, the needs in term of code covering by tests become
important. The question ``is our software tested enough?'' becomes a
question to which it is very hard to answer. Program proof can help us.
Rather than test all possible and (un)imaginable inputs to the program,
we will \emph{mathematically} prove that there cannot be any problem at
runtime.



The goal of this tutorial is to use Frama-C, a tool developed at the CEA
LIST, and WP, its deductive proof plugin, to learn the basics about C
program proof. More than the use of the tool itself, the goal of this
tutorial is to convince that it is more and more possible to write
programs without any programming error, but also to sensitize to simple
notions that allows to better understand and write programs.



\begin{Information}
  Many thanks to the different beta-testers for their constructives
  feedback:

\begin{itemize}
\item \externalLink{Taurre}{https://zestedesavoir.com/membres/voir/Taurre/}
\item \externalLink{barockobamo}{https://zestedesavoir.com/membres/voir/barockobamo/}
\item \externalLink{Vayel}{https://zestedesavoir.com/membres/voir/Vayel/}
\end{itemize}
  I thank ZesteDeSavoir validators who helped me improve again the quality of
  this tutorial:

\begin{itemize}
\item \externalLink{Taurre}{https://zestedesavoir.com/membres/voir/Taurre/} (again)
\item \externalLink{Saroupille}{https://zestedesavoir.com/membres/voir/Saroupille/}
\end{itemize}
  Finally, many thanks to Jens Gerlach for his help during the translation of
  this tutorial from French to English, and to Rafael Bachmann for his review
  and remarks.
\end{Information}
