In this part of the tutorial, we will present two important notions of
ACSL:



\begin{itemize}
\item axiomatic definitions,
\item ghost code.
\end{itemize}


In some cases, this two notions are absolutely needed to ease the
process of specification and, more importantly, proof. On one hand by
forcing some properties to be more abstract when an explicit modeling
involve to much computation during proof, on the other hand by forcing
some properties to be explicitly modeled since they are harder to reason
with when they are implicit.




Using this two notions, we expose ourselves to the possibility to make
our proof irrelevant if we make mistakes writing specification with it.
The first one allows us to introduce ``false'' in our assumptions, or to
write imprecise definitions. The second one opens the risk to silently
modify the verified program \ldots{} making us prove another program,
which is not the one we want to prove.


\begin{levelTwo}
  {Axiomatic definitions}
  {axiomatic}
\end{levelTwo}

\begin{levelTwo}
  {Ghost code}
  {ghost-code}
\end{levelTwo}


\horizontalLine



In this part, we have covered some advanced constructions of the ACSL
language that allows to express and prove more complex properties about
programs.

Badly used, this features can make our analyses incorrect, we then need
to be careful manipulating them and not hesitate to check them again and
again, or eventually express properties to verify about them to assure
that we are not introducing an incoherence in our program or our
assumptions.
