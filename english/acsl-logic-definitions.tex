In this part of the tutorial, we will present three important notions of ACSL:

\begin{itemize}
\item inductive definitions,
\item axiomatic definitions,
\item ghost code.
\end{itemize}


In some cases, these notions are absolutely needed to ease the process of
specification and, more importantly, proof. On one hand they force some
properties to be more abstract when an explicit modeling would involve too much
computation during proof. On the other hand, they force some properties to be
explicitly modeled when they are harder to reason about when they are implicit.

Using these notions, we expose ourselves to the possibility to make our
proof irrelevant if we make mistakes writing specification with it. Inductive
predicates and axiomatic definitions involve the risk to introduce ``false'' in
our assumptions, or to write imprecise definitions. Ghost code opens the risk
to silently modify the verified program \ldots{} making us prove another program,
which is not the one we want to prove.


\begin{levelTwo}
  {Inductive definitions}
  {inductive}
\end{levelTwo}

\begin{levelTwo}
  {Axiomatic definitions}
  {axiomatic}
\end{levelTwo}

\begin{levelTwo}
  {Ghost code}
  {ghost-code}
\end{levelTwo}

\begin{levelTwo}
  {Hidden content}
  {answers}
\end{levelTwo}


\horizontalLine
\newpage


In this part, we have covered some advanced constructions of the ACSL
language that allow to express and prove more complex properties about
programs.

Badly used, these features can make our analyses incorrect, we then need
to be careful manipulating them and not hesitate to check them again and
again, or eventually express properties to verify about them to assure
that we are not introducing an incoherence in our program or our
assumptions.
