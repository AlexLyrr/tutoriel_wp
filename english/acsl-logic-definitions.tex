In this part of the tutorial, we will present two important notions of
ACSL:



\begin{itemize}
\item axiomatic definitions,
\item ghost code.
\end{itemize}


In some cases, this two notions are absolutely needed to ease the
process of specification and, more importantly, proof. On one hand by
forcing some properties to be more abstract when an explicit modeling
involve to much computation during proof, on the other hand by forcing
some properties to be explicitly modeled since they are harder to reason
with when they are implicit.




Using this two notions, we expose ourselves to the possibility to make
our proof irrelevant if we make mistakes writing specification with it.
The first one allows us to introduce ``false'' in our assumptions, or to
write imprecise definitions. The second one opens the risk to silently
modify the verified program \ldots{} making us prove another program,
which is not the one we want to prove.



\levelTwoTitle{Axiomatic definitions}


Les axiomes sont des propriétés dont nous énonçons qu'elles sont vraies quelle 
que soit la situation. C'est un moyen très pratique d'énoncer des notions 
complexes qui vont pouvoir rendre le processus très efficace en abstrayant 
justement cette complexité. Évidemment, comme toute propriété exprimée comme un
axiome est supposée vraie, il faut également faire très attention à ce que nous
définissons car si nous introduisons une propriété fausse dans les notions que 
nous supposons vraies alors ... nous saurons tout prouver, même ce qui est faux.



\levelThreeTitle{Syntaxe}


Pour introduire une définition axiomatique, nous utilisons la syntaxe suivante :



\begin{CodeBlock}{c}
/*@
  axiomatic Name_of_the_axiomatic_definition {
    // ici nous pouvons définir ou déclarer des fonctions et prédicats

    axiom axiom_name { Label0, ..., LabelN }:
      // property ;

    axiom other_axiom_name { Label0, ..., LabelM }:
      // property ;

    // ... nous pouvons en mettre autant que nous voulons
  }
*/
\end{CodeBlock}



Nous pouvons par exemple définir cette axiomatique :



\begin{CodeBlock}{c}
/*@
  axiomatic lt_plus_lt{
    axiom always_true_lt_plus_lt:
      \forall integer i, j; i < j ==> i+1 < j+1 ;
  }
*/
\end{CodeBlock}



Et nous pouvons voir que dans Frama-C, la propriété est bien supposée vraie :



\image{5-1-1-premier-axiome.png}[Premier axiome, supposé vrai par Frama-C]


\begin{Spoiler}
Actuellement nos prouveurs automatiques n'ont pas la puissance nécessaire
pour calculer \textit{la réponse à la grande question sur la vie, l'univers et le 
reste}. Qu'à cela ne tienne nous pouvons l'énoncer comme axiome ! Reste à
comprendre la question pour savoir où ce résultat peut-être utile ...

\begin{CodeBlock}{c}
/*@
  axiomatic Ax_answer_to_the_ultimate_question_of_life_the_universe_and_everything {
    logic integer the_ultimate_question_of_life_the_universe_and_everything{L} ;

    axiom answer{L}:
      the_ultimate_question_of_life_the_universe_and_everything{L} = 42;
  }
*/
\end{CodeBlock}
\end{Spoiler}


\levelFourTitle{Lien avec la notion de lemme}


Les lemmes et les axiomes vont nous permettre d'exprimer les mêmes types de 
propriétés, à savoir des propriétés exprimées sur des variables quantifiées (et
éventuellement des variables globales, mais cela reste assez rare puisqu'il est
difficile de trouver une propriété qui soit globalement vraie à leur sujet tout
en étant intéressante). Outre ce point commun, il faut également savoir que 
comme les axiomes, en dehors de leur définition, les lemmes sont considérés 
vrais par WP.



La seule différence entre lemme et axiome du point de vue de la preuve est donc
que nous devrons fournir une preuve que le premier est valide alors que l'axiome
est toujours supposé vrai.



\levelThreeTitle{Définition de fonctions ou prédicats récursifs}


Les définitions axiomatiques de fonctions ou de prédicats récursifs sont 
particulièrement utiles car elles vont permettre d'empêcher les prouveurs de 
dérouler la récursion quand c'est possible.



L'idée est alors de ne pas définir directement la fonction ou le prédicat mais 
plutôt de la déclarer puis de définir des axiomes spécifiant son comportement.
Si nous reprenons par exemple la factorielle, nous pouvons la définir 
axiomatiquement de cette manière :



\begin{CodeBlock}{c}
/*@
  axiomatic Factorial{
    logic integer factorial(integer n);
    
    axiom factorial_0:
      \forall integer i; i <= 0 ==> 1 == factorial(i) ;

    axiom factorial_n:
      \forall integer i; i > 0 ==> i * factorial(i-1) == factorial(i) ;
  }
*/
\end{CodeBlock}



Dans cette définition axiomatique, notre fonction n'a pas de corps. Son 
comportement étant défini par les axiomes ensuite définis.



Une petite subtilité
est qu'il faut prendre garde au fait que si les axiomes énoncent des propriétés
à propos du contenu d'une ou plusieurs zones mémoires pointées, il faut 
spécifier ces zones mémoires en utilisant la notation \CodeInline{reads} au niveau de
la déclaration. Si nous oublions une telle spécification, le prédicat, ou la 
fonction, sera considéré comme énoncé à propos du pointeur et non à propos de la
zone mémoire pointée. Une modification de celle-ci n'entraînera donc pas 
l'invalidation d'une propriété connue axiomatiquement.



Si par exemple, nous voulons définir qu'un tableau ne contient que des 0, nous
pouvons le faire de cette façon :



\begin{CodeBlock}{c}
/*@
  axiomatic A_all_zeros{
    predicate zeroed{L}(int* a, integer b, integer e) reads a[b .. e-1];

    axiom zeroed_empty{L}:
      \forall int* a, integer b, e; b >= e ==> zeroed{L}(a,b,e);
      
    axiom zeroed_range{L}:
      \forall int* a, integer b, e; b < e ==>
        zeroed{L}(a,b,e-1) && a[e-1] == 0 <==> zeroed{L}(a,b,e);
  }
*/
\end{CodeBlock}



Et nous pouvons à nouveau prouver notre fonction de remise à zéro avec cette
nouvelle définition :



\begin{CodeBlock}{c}
###include <stddef.h>

/*@
  requires \valid(array + (0 .. length-1));
  assigns  array[0 .. length-1];
  ensures  zeroed(array,0,length);
*/
void raz(int* array, size_t length){
  /*@
    loop invariant 0 <= i <= length;
    loop invariant zeroed(array,0,i);
    loop assigns i, array[0 .. length-1];
    loop variant length-i;
  */
  for(size_t i = 0; i < length; ++i)
    array[i] = 0;
}
\end{CodeBlock}



Selon votre version de Frama-C et de vos prouveurs automatiques, la preuve de 
préservation de l'invariant peut échouer. Une raison à cela est que le prouveur ne
parvient pas à garder l'information que ce qui précède la cellule en cours de
traitement par la boucle est toujours à 0. Nous pouvons ajouter un lemme dans
notre base de connaissance, expliquant que si l'ensemble des valeurs d'un tableau
n'a pas changé, alors la propriété est toujours vérifiée :



\begin{CodeBlock}{c}
/*@
  predicate same_elems{L1,L2}(int* a, integer b, integer e) =
    \forall integer i; b <= i < e ==> \at(a[i],L1) == \at(a[i],L2);

  lemma no_changes{L1,L2}:
  \forall int* a, integer b, e;
  same_elems{L1,L2}(a,b,e) ==> zeroed{L1}(a,b,e) ==> zeroed{L2}(a,b,e);
*/
\end{CodeBlock}



Et d'énoncer une assertion pour spécifier ce qui n'a pas changé entre le début 
du bloc de la boucle (marqué par le \textit{label} \CodeInline{L} dans le code) et la fin (qui se
trouve être \CodeInline{Here} puisque nous posons notre assertion à la fin) :



\begin{CodeBlock}{c}
for(size_t i = 0; i < length; ++i){
  L:
  array[i] = 0;
  //@ assert same_elems{L,Here}(array,0,i);
}
\end{CodeBlock}



À noter que dans cette nouvelle version du code, la propriété énoncée par notre
lemme n'est pas prouvée par les solveurs automatiques, qui ne savent pas raisonner
pas induction. Pour les curieux, la (très simple) preuve en Coq est ci-dessous :



\begin{Spoiler}
\begin{CodeBlock}{coq}
(* Généré par WP *)
Definition P_same_elems (Mint_0 : farray addr Z) (Mint_1 : farray addr Z)
    (a : addr) (b : Z) (e : Z) : Prop :=
    forall (i : Z), let a_1 := (shift_sint32 a i%Z) in ((b <= i)%Z) ->
      ((i < e)%Z) -> (((Mint_0.[ a_1 ]) = (Mint_1.[ a_1 ]))%Z).
Goal
  forall (i_1 i : Z), forall (t_1 t : farray addr Z), forall (a : addr),
    ((P_zeroed t a i_1%Z i%Z)) -> ((P_same_elems t_1 t a i_1%Z i%Z)) -> ((P_zeroed t_1 a i_1%Z i%Z)).
(* Notre preuve *)
Proof.
  intros b e.
  (* par induction sur la distance entre b et e *)
  induction e using Z_induction with (m := b) ; intros mem_l2 mem_l1 a Hz_l1 Hsame.
  (* cas de base : Axiome "empty" *)
  + apply A_A_all_zeros.Q_zeroed_empty ; assumption.
  + replace (e + 1) with (1 + e) in * ; try omega.
    (* on utilise l'axiome range *)
    rewrite A_A_all_zeros.Q_zeroed_range in * ; intros Hinf.
    apply Hz_l1 in Hinf ; clear Hz_l1 ; inversion_clear Hinf as [Hlast Hothers].
    split.
    (* sous plage de Hsame *)
    - rewrite Hsame ; try assumption ; omega.
    (* hypothèse d'induction *)
    - apply IHe with (t := mem_l1) ; try assumption.
      * unfold P_same_elems ; intros ; apply Hsame ; omega.
Qed.
\end{CodeBlock}
\end{Spoiler}


Dans le cas présent, utiliser une axiomatique est contre-productif : notre
propriété est très facilement exprimable en logique du premier ordre comme
nous l'avons déjà fait précédemment. Les axiomatiques sont faites pour écrire
des définitions qui ne sont pas simples à exprimer dans le formalisme de base
d'ACSL. Mais il est mieux de commencer avec un exemple facile à lire.



\levelThreeTitle{Consistance}


En ajoutant des axiomes à notre base de connaissances, nous pouvons produire des
preuves plus complexes car certaines parties de cette preuve, mentionnées par 
les axiomes, ne nécessiteront plus de preuves qui allongeraient le processus 
complet. Seulement, en faisant cela \textbf{nous devons être extrêmement prudents}.
En effet, la moindre hypothèse fausse introduite dans la base pourraient rendre
tous nos raisonnements futiles. Notre raisonnement serait toujours correct, mais
basé sur des connaissances fausses, il ne nous apprendrait donc plus rien de correct.



L'exemple le plus simple à produire est le suivant:



\begin{CodeBlock}{c}
/*@
  axiomatic False{
    axiom false_is_true: \false;
  }
*/

int main(){
  // Exemples de propriétés prouvées

  //@ assert \false;
  //@ assert \forall integer x; x > x;
  //@ assert \forall integer x,y,z ; x == y == z == 42;
  return *(int*) 0;
}
\end{CodeBlock}



Et tout est prouvé, y compris que le déréférencement de l'adresse 0 est OK :



\image{false_axiom.png}[Preuve de tout un tas de choses fausses]


Évidemment cet exemple est extrême, nous n'écririons pas un tel axiome. Le
problème est qu'il est très facile d'écrire une axiomatique subtilement fausse
lorsque nous exprimons des propriétés plus complexes, ou que nous commençons à
poser des suppositions sur l'état global d'un système.



Quand nous commençons à créer de telles définitions, ajouter de temps en 
temps une preuve ponctuelle de « \textit{false} » dont nous voulons qu'elle échoue permet 
de s'assurer que notre définition n'est pas inconsistante. Mais cela ne fait pas 
tout ! Si la subtilité qui crée le comportement faux est suffisamment cachée, les
prouveurs peuvent avoir besoin de beaucoup d'informations autre que l'axiomatique
elle-même pour être menés jusqu'à l'inconsistance, donc il faut toujours être 
vigilant !



Notamment parce que par exemple, la mention des valeurs lues par une fonction
ou un prédicat défini axiomatiquement est également importante pour la 
consistance de l'axiomatique. En effet, comme mentionné précédemment, si nous
n'exprimons pas les valeurs lues dans le cas de l'usage d'un pointeur, la 
modification d'une valeur du tableau n'invalide pas une propriété que l'on 
connaitrait à propos du contenu du tableau par exemple. Dans un tel cas, la 
preuve passe mais l'axiome n'exprimant pas le contenu, nous ne prouvons rien.



Par exemple, si nous reprenons l'exemple de mise à zéro, nous pouvons modifier
la définition de notre axiomatique en retirant la mention des valeurs dont 
dépendent le prédicat : \CodeInline{reads a[b .. e-1]}. La preuve passera toujours
mais n'exprimera plus rien à propos du contenu des tableaux considérés.



\levelThreeTitle{Exemple : comptage de valeurs}


Dans cet exemple, nous cherchons à prouver qu'un algorithme compte bien les 
occurrences d'une valeur dans un tableau. Nous commençons par définir 
axiomatiquement la notion de comptage dans un tableau :



\begin{CodeBlock}{c}
/*@
  axiomatic Occurrences_Axiomatic{
    logic integer l_occurrences_of{L}(int value, int* in, integer from, integer to)
      reads in[from .. to-1];

    axiom occurrences_empty_range{L}:
      \forall int v, int* in, integer from, to;
        from >= to ==> l_occurrences_of{L}(v, in, from, to) == 0;

    axiom occurrences_positive_range_with_element{L}:
      \forall int v, int* in, integer from, to;
        (from < to && in[to-1] == v) ==>
      l_occurrences_of(v,in,from,to) == 1+l_occurrences_of(v,in,from,to-1);

    axiom occurrences_positive_range_without_element{L}:
      \forall int v, int* in, integer from, to;
        (from < to && in[to-1] != v) ==>
      l_occurrences_of(v,in,from,to) == l_occurrences_of(v,in,from,to-1);
  }
*/
\end{CodeBlock}



Nous avons trois cas à gérer :



\begin{itemize}
\item la plage de valeur concernée est vide : le nombre d'occurrences est 0 ;
\item la plage de valeur n'est pas vide et le dernier élément est celui recherché :
le nombre d'occurrences est : le nombre d'occurrences dans la plage actuelle que
l'on prive du dernier élément, plus 1 ;
\item la plage de valeur n'est pas vide et le dernier élément n'est pas celui 
recherché : le nombre d'occurrences est le nombre d'occurrences dans la plage
privée du dernier élément.
\end{itemize}


Par la suite, nous pouvons écrire la fonction C exprimant ce comportement et la
prouver :



\begin{CodeBlock}{c}
/*@
  requires \valid_read(in+(0 .. length));
  assigns  \nothing;
  ensures  \result == l_occurrences_of(value, in, 0, length);
*/
size_t occurrences_of(int value, int* in, size_t length){
  size_t result = 0;
  
  /*@
    loop invariant 0 <= result <= i <= length;
    loop invariant result == l_occurrences_of(value, in, 0, i);
    loop assigns i, result;
    loop variant length-i;
  */
  for(size_t i = 0; i < length; ++i)
    result += (in[i] == value)? 1 : 0;

  return result;
}
\end{CodeBlock}



Une alternative au fait de spécifier que dans ce code \CodeInline{result} est au 
maximum \CodeInline{i} est d'exprimer un lemme plus général à propos de la valeur
du nombre d'occurrences, dont nous savons qu'elle est comprise entre 0 et 
la taille maximale de la plage de valeurs considérée :



\begin{CodeBlock}{c}
/*@
lemma l_occurrences_of_range{L}:
  \forall int v, int* array, integer from, to:
    from <= to ==> 0 <= l_occurrences_of(v, a, from, to) <= to-from;
*/
\end{CodeBlock}



La preuve de ce lemme ne pourra pas être déchargée par un solveur automatique. Il
faudra faire cette preuve interactivement avec Coq par exemple. Exprimer des 
lemmes généraux prouvés manuellement est souvent une bonne manière d'ajouter des
outils aux prouveurs pour manipuler plus efficacement les axiomatiques, sans 
ajouter formellement d'axiomes qui augmenteraient nos chances d'introduire des
erreurs. Ici, nous devrons quand même réaliser les preuves des propriétés 
mentionnées.



\levelThreeTitle{Exemple : le tri}


Nous allons prouver un simple tri par sélection :



\begin{CodeBlock}{c}
size_t min_idx_in(int* a, size_t beg, size_t end){
  size_t min_i = beg;
  for(size_t i = beg+1; i < end; ++i)
    if(a[i] < a[min_i]) min_i = i;
  return min_i;
}

void swap(int* p, int* q){
  int tmp = *p; *p = *q; *q = tmp;
}

void sort(int* a, size_t beg, size_t end){
  for(size_t i = beg ; i < end ; ++i){
    size_t imin = min_idx_in(a, i, end);
    swap(&a[i], &a[imin]);
  }
}
\end{CodeBlock}



Le lecteur pourra s'exercer en spécifiant et en prouvant les fonctions de 
recherche de minimum et d'échange de valeur. Nous cachons la correction 
ci-dessous et allons nous concentrer plutôt sur la spécification et la preuve de
la fonction de tri qui sont une illustration intéressant de l'usage des
axiomatiques.



\begin{Spoiler}
\begin{CodeBlock}{c}
/*@
  requires \valid_read(a + (beg .. end-1));
  requires beg < end;

  assigns  \nothing;

  ensures  \forall integer i; beg <= i < end ==> a[\result] <= a[i];
  ensures  beg <= \result < end;
*/
size_t min_idx_in(int* a, size_t beg, size_t end){
  size_t min_i = beg;

  /*@
    loop invariant beg <= min_i < i <= end;
    loop invariant \forall integer j; beg <= j < i ==> a[min_i] <= a[j];
    loop assigns min_i, i;
    loop variant end-i;
  */
  for(size_t i = beg+1; i < end; ++i){
    if(a[i] < a[min_i]) min_i = i;
  }
  return min_i;
}

/*@
  requires \valid(p) && \valid(q);
  assigns  *p, *q;
  ensures  *p == \old(*q) && *q == \old(*p);
*/
void swap(int* p, int* q){
  int tmp = *p; *p = *q; *q = tmp;
}
\end{CodeBlock}
\end{Spoiler}


En effet, une erreur commune que nous pourrions faire dans le cas de la preuve 
du tri est de poser cette spécification (qui est vraie !) :



\begin{CodeBlock}{c}
/*@
  predicate sorted(int* a, integer b, integer e) =
    \forall integer i, j; b <= i <= j < e ==> a[i] <= a[j];
*/

/*@
  requires \valid(a + (beg .. end-1));
  requires beg < end;
  assigns  a[beg .. end-1];
  ensures sorted(a, beg, end);
*/
void sort(int* a, size_t beg, size_t end){
  /*@ //annotation de l'invariant */
  for(size_t i = beg ; i < end ; ++i){
    size_t imin = min_idx_in(a, i, end);
    swap(&a[i], &a[imin]);
  }
}
\end{CodeBlock}



\textbf{Cette spécification est vraie}. Mais si nous nous rappelons la 
partie concernant les spécifications, nous nous devons d'exprimer précisément ce
que nous attendons. Avec la spécification actuelle, nous ne prouvons pas toutes
les propriétés nécessaires d'un tri ! Par exemple, cette fonction remplit 
pleinement la spécification :



\begin{CodeBlock}{c}
/*@
  requires \valid(a + (beg .. end-1));
  requires beg < end;

  assigns  a[beg .. end-1];
  
  ensures sorted(a, beg, end);
*/
void fail_sort(int* a, size_t beg, size_t end){
  /*@
    loop invariant beg <= i <= end;
    loop invariant \forall integer j; beg <= j < i ==> a[j] == 0;
    loop assigns i, a[beg .. end-1];
    loop variant end-i;
  */
  for(size_t i = beg ; i < end ; ++i)
    a[i] = 0;
}
\end{CodeBlock}



En fait, notre spécification oublie que tous les éléments qui étaient 
originellement présents dans le tableau à l'appel de la fonction doivent
toujours être présents après l'exécution de notre fonction de tri. Dit
autrement, notre fonction doit en fait produire la permutation triée des
valeurs du tableau.



Une propriété comme la définition de ce qu'est une permutation s'exprime 
extrêmement bien par l'utilisation d'une axiomatique. En effet, pour déterminer
qu'un tableau est la permutation d'un autre, les cas sont très limités. 
Premièrement, le tableau est une permutation de lui-même, puis l'échange de
deux valeurs sans changer les autres est également une permutation, et 
finalement si nous créons la permutation $p_2$ d'une permutation $p_1$, puis que 
nous créons la permutation $p_3$ de $p_2$, alors par transitivité $p_3$ est une
permutation de $p_1$.



Ceci est exprimé par le code suivant :



\begin{CodeBlock}{c}
/*@
  predicate swap_in_array{L1,L2}(int* a, integer b, integer e, integer i, integer j) =
    b <= i < e && b <= j < e &&
    \at(a[i], L1) == \at(a[j], L2) && \at(a[j], L1) == \at(a[i], L2) &&
    \forall integer k; b <= k < e && k != j && k != i ==> \at(a[k], L1) == \at(a[k], L2);

  axiomatic Permutation{
    predicate permutation{L1,L2}(int* a, integer b, integer e)
     reads \at(*(a+(b .. e - 1)), L1), \at(*(a+(b .. e - 1)), L2);

    axiom reflexive{L1}: 
      \forall int* a, integer b,e ; permutation{L1,L1}(a, b, e);

    axiom swap{L1,L2}:
      \forall int* a, integer b,e,i,j ;
        swap_in_array{L1,L2}(a,b,e,i,j) ==> permutation{L1,L2}(a, b, e);
	
    axiom transitive{L1,L2,L3}:
      \forall int* a, integer b,e ; 
        permutation{L1,L2}(a, b, e) && permutation{L2,L3}(a, b, e) ==> permutation{L1,L3}(a, b, e);
  }
*/
\end{CodeBlock}



Nous spécifions alors que notre tri nous crée la permutation triée du tableau
d'origine et nous pouvons prouver l'ensemble en complétant l'invariant de la
fonction :



\begin{CodeBlock}{c}
/*@
  requires beg < end && \valid(a + (beg .. end-1));
  assigns  a[beg .. end-1];  
  ensures sorted(a, beg, end);
  ensures permutation{Pre, Post}(a,beg,end);
*/
void sort(int* a, size_t beg, size_t end){
  /*@
    loop invariant beg <= i <= end;
    loop invariant sorted(a, beg, i) && permutation{Begin, Here}(a, beg, end);
    loop invariant \forall integer j,k; beg <= j < i ==> i <= k < end ==> a[j] <= a[k];
    loop assigns i, a[beg .. end-1];
    loop variant end-i;
  */
  for(size_t i = beg ; i < end ; ++i){
    //@ ghost begin: ;
    size_t imin = min_idx_in(a, i, end);
    swap(&a[i], &a[imin]);
    //@ assert swap_in_array{begin,Here}(a,beg,end,i,imin);
  }
}
\end{CodeBlock}



Cette fois, notre propriété est précisément définie, la preuve reste assez
simple à faire passer, ne nécessitant que l'ajout d'une assertion que le bloc
de la fonction n'effectue qu'un échange des valeurs dans le tableau (et donnant
ainsi la transition vers la permutation suivante du tableau). Pour définir cette
notion d'échange, nous utilisons une annotation particulière (à la ligne 16),
introduite par le mot-clé \CodeInline{ghost}. Ici, le but est d'introduire un \textit{label} 
fictif dans le code qui est uniquement visible d'un point de vue spécification.
C'est l'objet de la prochaine section.



\levelTwoTitle{Ghost code}


Behind this title, that seems to be an action movie, we find in fact a
way to enrich our specification with information expressed as regular C
code. Here, the idea is to add variables and source code that will not
be part of the actual program but will model logic states that will only
be visible from a proof point of view. Using it, we can make explicit
some logic properties that were previously only known implicitly.


\levelThreeTitle{Syntax}



Ghost code is added using annotations that will contain C code
introduced using the \texttt{ghost} keyword:



\begin{CodeBlock}{c}
/*@
  ghost
  // code en langage C
*/
\end{CodeBlock}



The only rules we have to respect in such a code, is that we cannot
write a memory block that is not itself defined in ghost code, and that
the code must close any block it would open. Apart of this, any
computation can be inserted provided it \emph{only} modifies ghost
variables.




Here are some examples of ghost code:



\begin{CodeBlock}{c}
//@ ghost int ghost_glob_var = 0;

void foo(int a){
  //@ ghost int ghost_loc_var = a;

  //@ ghost Ghost_label: ;
  a = 28 ;

  //@ ghost if(a < 0){ ghost_loc_var = 0; }

  //@ assert ghost_loc_var == \at(a, Ghost_label) == \at(a, Pre);
}
\end{CodeBlock}



While for this chapter, it will not be necessary, as we will see later,
it is sometimes useful to write some contracts or assertions in ghost
code. As it must be specified in code that is already in C comments, we
can use a specific syntax to provide ghost contracts or assertions.
We open ghost annotations with the syntax \CodeInline{/@} and close
them with \CodeInline{@/}.



For example :



\begin{CodeBlock}{c}
void foo(unsigned n){
 /*@ ghost
   unsigned i ;

   /@
     loop invariant 0 <= i <= n ;
     loop assigns i ;
     loop variant n - i ;
   @/
   for(i = 0 ; i < n ; ++i){

   }
   /@ assert i == n ; @/
 */
}
\end{CodeBlock}



\levelThreeTitle{Ghost code validity}



We must again be careful using ghost code. Indeed, the tool will not
perform any verification to ensure that we do not write in the memory of
the program by mistake. This problem being, in fact, an undecidable
problem, this analysis would require a proof by itself. For example,
this code is allowed as input of Frama-C even if it explicitly modifies
the state of the program we want to verify:



\begin{CodeBlock}{c}
int a;

void foo(){
  //@ ghost a = 42;
}
\end{CodeBlock}



We then need to be really careful about what we are doing using ghost
code (by making it simple).



\levelThreeTitle{Make a logical state explicit}


The goal of ghost code is to make explicit some information that are
without them implicit. For example, in the previous section, we used it
to get an explicit logic state known at a particular point of the
program.

Let us take a more complex example. Here, we want to prove that the
following function returns the value of the maximal sum of subarrays of
a given array. A subarray of an array \texttt{a} is a contiguous subset
of values of \texttt{a}. For example, for an array \CodeInline{\{ 0 , 3 , -1 , 4 \}},
some subarrays can be
\CodeInline{\{\}}, \CodeInline{\{ 0 \}}, \CodeInline{\{ 3 , -1 \}}
, \CodeInline{\{ 0 , 3 , -1 , 4 \}}, ... Note that as we allow
empty arrays for subarrays, the sum is at least 0. In the previous
array, the subarray with the maximal sum is \CodeInline{\{ 3 , -1 , 4 \}},
the function would then return 6.



\CodeBlockInput[3]{c}{max_subarray-0.c}



In order to specify the previous function, we will need an axiomatic
definition about sum. This is not too complex, the careful reader
can express the needed axioms as an exercise:



\CodeBlockInput[3][5]{c}{max_subarray-1.c}



Some correct axioms are available at: \ref{l2:acsl-logic-definitions-answers}



The specification of the function is the following:



\CodeBlockInput[14][19]{c}{max_subarray-1.c}



For any bounds, the value returned by the function must be greater or
equal to the sum of the elements between these bounds, and there must
exist some bounds such that the returned value is exactly the sum of the
elements between these bounds. About this specification, when we want to
add the loop invariant, we will realize that we miss some information.
We want to express what are the values \CodeInline{max} and \CodeInline{cur} and
what are the relations between them, but we cannot do it!

Basically, our postcondition needs to know that there exists some bounds
\CodeInline{low} and \CodeInline{high} such that the computed sum corresponds to
these bounds. However, in our code, we do not have anything that express
it from a logic point of view, and we cannot \emph{a priori} make the
link between this logical formalization. We will then use ghost code to
record these bounds and express the loop invariant.

We will first need two variables that will allow us to record the bounds
of the maximum sum range, we will call them \CodeInline{low} and
\CodeInline{high}. Every time we will find a range where the sum is greater
than the current one, we will update our ghost variables. This bounds
will then corresponds to the sum currently stored by \CodeInline{max}. That
induces that we need other bounds: the ones that corresponds to the sum
store by the variable \CodeInline{cur} from which we will build the bounds
corresponding to \CodeInline{max}. For these bounds, we will only add a
single ghost variable: the current low bound \CodeInline{cur\_low}, the high
bound being the variable \CodeInline{i} of the loop.



\CodeBlockInput[14][50]{c}{max_subarray-1.c}



The invariant \CodeInline{BOUNDS} expresses how the different bounds are
ordered during the computation. The invariant \CodeInline{REL} expresses
what the variables \CodeInline{cur} and \CodeInline{max} mean depending on the
bounds. Finally, the invariant \CodeInline{POST} allows us to create a link
between the invariant and the postcondition of the function.



The reader can verify that this function is indeed correctly proved
without RTE verification. If we add RTE verification, the overflow on
the variable \CodeInline{cur}, that is the sum, seems to be possible (and it
is indeed the case).




Here, we will not try to fix this because it is not the topic of this
example. The way we can prove the absence of RTE here strongly depends
on the context where we use this function. A possibility is to strongly
restrict the contract, forcing some properties about values and the size
of the array. For example, we could strongly limit the maximal size of
the array and strong bounds on each value of the different cells. An
other possibility would be to add an error value in case of overflow
(\(-1\) for example), and to specify that when an overflow is produced,
this value is returned.



\horizontalLine



In this part, we have covered some advanced constructions of the ACSL
language that allows to express and prove more complex properties about
programs.

Badly used, this features can make our analyses incorrect, we then need
to be careful manipulating them and not hesitate to check them again and
again, or eventually express properties to verify about them to assure
that we are not introducing an incoherence in our program or our
assumptions.
