\begin{Warning}
  Please note that this document is the very first English version of the
  tutorial. If you find some errors, please do not hesitate to contribute at:

  \externalLink{https://github.com/AllanBlanchard/tutoriel\_wp}{https://github.com/AllanBlanchard/tutoriel_wp}

  Please use the markdown files, the LaTeX file being generated from them.
\end{Warning}


\begin{Information}
  In this tutorial, some examples and some elements of organization are
  similar to the ones used in the 
  \externalLink
      {TAP 2013 tutorial}
      {https://frama-c.com/download/publications/tutorial_tap2013_slides.pdf} 
  by Nikolai Kosmatov, Virgile Prevosto and Julien
  Signoles of the CEA LIST, since it is quite didactic. It also
  contains examples taken from
  \textit{\externalLink{ACSL By Example}{https://github.com/fraunhoferfokus/acsl-by-example}}
  by Jochen Burghardt, Jens Gerlach, Kerstin Hartig, Hans Pohl and Juan
  Soto from the Fraunhofer. The remaining ideas come from my personal
  experience with Frama-C and WP. The only requirement to this tutorial
  is to have a basic knowledge of the C language, and at least to
  be familiar with the notion of pointer.
\end{Information}


Despite its old age, C is still a widely used programming language.
Indeed, no other language can pretend to be available on so many
different (hardware and software) platforms, its low-level orientation
and the amount of time invested in the optimization of its compilers
allows to generate very light and efficient machine code (if the code
allows it of course), and that there are a lot of experts in C language,
which is an important knowledge base.


Furthermore, a lot of systems rely on a huge amount of code historically
written in C, that needs to be maintained and sometimes fixed, as it
would be far too costly to rewrite these systems.


But anyone who has already developed with C also know that it is very
hard to perfectly master this language. There are numerous reasons, but
ambiguities in the ISO C, and the fact that it is extremely permissive,
especially about memory management, make the development of robust C
program very hard, even for an experienced programmer.



However, the C language is often chosen for critical systems (avionics,
railway, armament, \ldots{}) where it is appreciated for its good
performances, its technological maturity and the predictability of its
compilation.



In such cases, the needs in term of code covering by tests become
important. The question ``is our software tested enough?'' becomes a
question to which it is very hard to answer. Program proof can help us.
Rather than test all possible and (un)imaginable inputs to the program,
we will \emph{mathematically} prove that there cannot be any problem at
runtime.



The goal of this tutorial is to use Frama-C, a tool developed at the CEA
LIST, and WP, its deductive proof plugin, to learn the basics about C
program proof. More than the use of the tool itself, the goal of this
tutorial is to convince that it is more and more possible to write
programs without any programming error, but also to sensitize to simple
notions that allows to better understand and write programs.



\begin{Information}
  Many thanks to the different beta-testers for their constructives
  feedback:

\begin{itemize}
\item \externalLink{Taurre}{https://zestedesavoir.com/membres/voir/Taurre/}
\item \externalLink{barockobamo}{https://zestedesavoir.com/membres/voir/barockobamo/}
\item \externalLink{Vayel}{https://zestedesavoir.com/membres/voir/Vayel/}
\end{itemize}
  I thank ZesteDeSavoir validators who helped me improve again the quality of
  this tutorial:

\begin{itemize}
\item \externalLink{Taurre}{https://zestedesavoir.com/membres/voir/Taurre/} (again)
\item \externalLink{Saroupille}{https://zestedesavoir.com/membres/voir/Saroupille/}
\end{itemize}
  Finally, many thanks to Jens Gerlach for his help during the translation of
  this tutorial from French to English, and to Rafael Bachmann for his review
  and remarks.
\end{Information}



\levelTwoTitle{Assignment, sequence and conditional}


\levelThreeTitle{Assignment}


Assignment is the most basic operation one can have in an imperative
language (leaving aside the ``do nothing'' operation that is not particularly
interesting). The weakest precondition calculus associates the following
$$wp(x = E , Post) := Post[x \leftarrow E]$$


Here the notation $P[x \leftarrow E]$ means ``the property $P$ where
$x$ is replaced by $E$''. In our case this corresponds to ``the
postcondition $Post$ where $x$ is replaced by $E$''. The idea is
that the postcondition of an assignment of $E$ to $x$ can only be
true if replacing all occurrences of $x$ in the formula by $E$ leads
to a property that is true. For example:



\begin{CodeBlock}{c}
// { P }
x = 43 * c ;
// { x = 258 }
\end{CodeBlock}


$$P = wp(x = 43*c , \{x = 258\}) = \{43*c = 258\}$$


The function $wp$ allows us to compute, as weakest precondition of 
the assignment provided our expected postcondition, the formula
$\{43*c = 258\}$, thus obtaining the following Hoare triple:


\begin{CodeBlock}{c}
// { 43*c = 258 }
x = 43 * c ;
// { x = 258 }
\end{CodeBlock}


In order to compute the precondition of the assignment we have replaced
each occurrence of $x$ in the postcondition by the assigned value
$E = 43*c$. If our program were of the form:



\begin{CodeBlock}{c}
int c = 6 ;
// { 43*c = 258 }
x = 43 * c ;
// { x = 258 }
\end{CodeBlock}



we could submit the formula " $43*6 = 258$ " to our automatic prover
in order to determine whether it is really valid. The answer would of
course be ``yes'' because the property is easy to verify. If we had,
however, given the value 7 to the variable \texttt{c} the prover's reply
would be ``no'' since the formula $43*7 = 258$ is not true.



Taking into account the weakest precondition calculus, we can now write
the inference rule for the Hoare triple of an assignment as
$$\dfrac{}{\{Q[x \leftarrow E] \}\quad x = E \quad\{ Q \}}$$


We note that there is no premise to verify. Does this mean that the
triple is necessarily true? Yes. However, it does not mean that the
precondition is respected by the program to which the assignment belongs
or that the precondition is at all possible. Here the automatic provers
come into play.



For example, we can ask Frama-C to verify the following line:



\begin{CodeBlock}{c}
int a = 42;
//@ assert a == 42;
\end{CodeBlock}



which is, of course, directly proven by Qed, since it is a simple
applications of the assignment rule.



\begin{Information}
  We remark that according to the C standard, an assignment is in fact an
  expression. This allows us, for example, to write
  \CodeInline{if( (a = foo()) == 42)}.
  In Frama-C, an assignment will always be treated as a statement. Indeed,
  if an assignment occurs within a larger expression, then the Frama-C
  preprocessor, while building the abstract syntax tree, systematically
  performs a \emph{normalization step} that produces a separate assignment
  statement.
\end{Information}



\levelFourTitle{Assignment of pointed value}



In C, thanks to (because of?) pointers, we can have programs with aliases,
meaning that two pointers can point to the same memory location. Our weakest
precondition calculus should consider these cases. For example, let us consider
this simple Hoare triple:


\begin{CodeBlock}{c}
//@ assert p = q ;
*p = 1 ;
//@ assert *p + *q == 2 ;
\end{CodeBlock}



This Hoare triple is correct, since \CodeInline{p} and \CodeInline{q} are in
alias, modifying \CodeInline{*p} also modifies \CodeInline{*q}, thus both these
expression evaluate to $1$ and the postcondition is true. However let us apply
the weakest precondition calculus from the postcondition:



\begin{tabular}{ll}
$wp(*p = 1, *p + *q = 2)$ & $= (*p + *q = 2)[*p \leftarrow 1]$\\
                          & $= (1 + *q = 2)$
\end{tabular}



We get the weakest precondition: \CodeInline{1 + *q == 2}, and thus we could
deduce that the weakest precondition is \CodeInline{*q == 1}, which is true, but
does not allow us to conclude that the program is correct, since in our formula
we do not have anything that models that \CodeInline{p == q ==> *q == 1}. In
fact, here, we would like to be able to compute a weakest precondition like:



\begin{tabular}{ll}
$wp(*p = 1, *p + *q = 2)$ & $= (1 + *q = 2 \vee q = p)$\\
                          & $= (*q = 1 \vee q = p)$
\end{tabular}



For this, we have to take care of aliasing. A common way to do this is to
consider that the memory is one particular variable (let us name this variable
$M$) on which we can perform two operations: get the element at a particular
location $l$ in memory (which returns an expression) and set the element at a
particular location $l$ to a new value $v$ (which returns the new memory).


We denote:


\begin{itemize}
\item $get(M,l)$ with the notation $M[l]$
\item $set(M,l,v)$ with the notation $M[l \mapsto v]$
\end{itemize}


And basically, the get operation can be seen as follows:


\begin{tabular}{ll}
  $M[l1 \mapsto v][l2] =$ & if $l1   =  l2$ then $v$ \\
                          & if $l1 \neq l2$ then $M[l2]$
\end{tabular}


If there is no value associated to the location we use for a get, the value is
undefined (thus, the memory is partial function). Of course, at the beginning of
a function, the memory context can be populated with the memory locations for
which a value is known to be defined.


Now, we can change a little bit the weakest precondition calculus for assignment
of pointed memory location. For this, we consider that we have an implicit
variable $M$ that models the memory, and we define the assignment of a memory
location as an update of the memory such that now the corresponding pointer points
to the written expression.
$$wp(*x = E, Q) := Q[M \leftarrow M[x \mapsto E]]$$


And evaluating a pointed value $*x$ in a formula now requires us to use the
get operator to ask the right value. Thus we can for example compute the weakest
precondition of our previous program:


\begin{tabular}{lll}
  $wp(*p = 1, *p + *q = 2)$
  & $= (*p + *q = 2)[M \leftarrow M[p \mapsto 1]]$ & (1)\\
  & $= (M[p] + M[q] = 2)[M \leftarrow M[p \mapsto 1]]$ & (2)\\
  & $= (M[p \mapsto 1][p] + M[p \mapsto 1][q] = 2)$ & (3)\\
  & $= (1 + M[p \mapsto 1][q] = 2)$ & (4)\\
  & $= (1 + (\texttt{if}\ q = p\ \texttt{then}\ 1\ \texttt{else}\ M[q]) = 2)$ & (5)\\
  & $= (\texttt{if}\ q = p\ \texttt{then}\ 1+1 = 2\ \texttt{else}\ 1+M[q] = 2)$ & (6)\\
  & $= (q = p \vee M[q] = 1)$ & (7)
\end{tabular}


\begin{enumerate}
\item we have to apply the rule of assignment for pointers, but for this we need
  to introduce $M$,
\item we replace pointer accesses in the formula by a call to $get$ on $M$,
\item we apply the replacement asked by the assignment rule,
\item we use the definition of the $get$ operator for the expression about $p$
  ($M[p \mapsto 1][p] = 1$)
\item we use the definition of the $get$ operator for the expression about $q$\\
  ($M[p \mapsto 1][q] = \texttt{if}\ q = p\ \texttt{then}\ 1\ \texttt{else}\ M[q]$)
\item we perform some simplification to the formula ...
\item ... and finally conclude that either $M[q] = 1$ or $p = q$.
\end{enumerate}


Then in our program, since we know that $p = q$, we can conclude that the program
is correct.


The WP plugin does not exactly work like this. In particular, it depends on the
memory model chosen for the proof that will make different assumption about the
memory is organized. For the memory model we use, the typed memory model, in
fact WP creates multiple variables for memory. However, let us have a look at
the verification condition generated for the postcondition of the swap function,
by preventing WP to simplify it using the option \CodeInline{-wp-no-simpl}:


\image{memory-model}


We can see, in the beginning of the verification condition, that a variable
\CodeInline{Mint\_0} representing a memory of values of integer types have been
created, and that this memory is updated and accessed using the operators we
previously introduced (see the definition of the variable \CodeInline{x\_2}).


\levelThreeTitle{Composition of statements}

For a statement to be valid, its precondition must allow us by means of
executing the said statement to reach the desired postcondition. Now we
would like to execute several statements one after another. Here the
idea is that the postcondition of the first statement is compatible with
the required precondition of the second statement and so on for the
third statement.



The inference rule that corresponds to this idea utilizes the following
Hoare triples:
$$\dfrac{\{P\}\quad S1 \quad \{R\} \ \ \ \{R\}\quad S2 \quad \{Q\}}{\{P\}\quad S1 ;\ S2 \quad \{Q\}}$$
In order to verify the composed statement $S1;\ S2$ we rely on an
intermediate property $R$ that is at the same time the postcondition
of $S1$ and the precondition of $S2$. (Please note that $S1$ and
$S2$ are not necessarily simple statements; they themselves can be
composed statements.) The problem is, however, that nothing indicates us
how to determine the properties $P$ and $R$.

The weakest-precondition calculus now says us that the intermediate
property $R$ can be computed as the weakest precondition of the second
statement. The property $P$, on the other hand, then is computed as
the weakest precondition of the first statement. In other words, the
weakest precondition of the composed statement $S1; S2$ is determined
as follows:
$$wp(S1;\ S2 , Post) := wp(S1, wp(S2, Post) )$$


The WP plugin of Frama-C performs all these computations for us. Thus,
we do not have to write the intermediate properties as ACSL assertions
between the lines of codes.



\begin{CodeBlock}{c}
int main(){
  int a = 42;
  int b = 37;

  int c = a+b; // i:1
  a -= c;      // i:2
  b += a;      // i:3

  //@assert b == 0 && c == 79;
}
\end{CodeBlock}



\levelFourTitle{Proof tree}


When we have more than two statements, we can consider the last
statement as second statement of our rule and all the preceding ones as
first statement. This way we traverse step by step backwards the
statements in our reasoning. With the previous program this looks like:


\begin{center}
\begin{tabular}{ccc}
  $\{P\}\quad i_1 ; \quad \{Q_{-2}\}$ & $\{Q_{-2}\}\quad i_2 ; \quad \{Q_{-1}\}$ & \\
  \cline{1-2}
  \multicolumn{2}{c}{$\{P\}\quad i\_1 ; \quad i\_2 ; \quad \{Q_{-1}\}$} & $\{Q_{-1}\} \quad i_3 ; \quad \{Q\}$\\
  \hline
  \multicolumn{3}{c}{$\{P\}\quad i\_1 ; \quad i\_2 ; \quad i\_3; \quad \{ Q \}$}
\end{tabular}
\end{center}

The weakest-precondition calculus allows us to construct the property
$Q_{-1}$ starting from the property $Q$ and statement $i_3$ which
in turn enables us to derive the property $Q_{-2}$ from the property
$Q_{-1}$ and statement $i_2$. Finally, $P$ can be determined from
$Q_{-2}$ and $i_1$.



Now that we can verify programs that consist of several statements it
is time to add some structure to them.



\levelThreeTitle{Conditional rule}


For a conditional statement to be true, one must be able to reach the
postcondition through both branches. Of course, for both branches, the
same precondition (of the conditional statement) must hold. In addition,
we have that in the if-branch the condition is true while in the
else-branch it is false.

We therefore have, as in the case of composed statements, two facts to
verify (in order to avoid confusion we are using here the syntax
$if\ B\ then\ S1\ else\ S2$):
$$\dfrac{\{P \wedge B\}\quad S1\quad \{Q\} \quad \quad \{P \wedge \neg B\}\quad S2\quad \{Q\}}{\{P\}\quad if\quad B\quad then\quad S1\quad else\quad S2 \quad \{Q\}}$$

Our two premises are therefore that we can both in the if-branch and the
else-branch reach the postcondition from the precondition.

The result of the weakest-precondition calculus for a conditional
statement reads as follows:
$$wp(if\ B\ then\ S1\ else\ S2 , Post) := (B \Rightarrow wp(S1, Post)) \wedge (\neg B \Rightarrow wp(S2, Post))$$
This means that the condition $B$ has to imply the weakest
precondition of $S1$ in order to safely arrive at the postcondition.
Analogously, the negation of $B$ must imply the weakest precondition
of $S2$.



\levelFourTitle{Empty \CodeInline{else}-branch}


Following this definition, we obtain for the case of an empty else-branch the
following rule by simply replacing the statement $S2$ by the empty
statement \texttt{skip}.
$$\dfrac{\{P \wedge B\}\quad S1\quad \{Q\} \quad \quad \{P \wedge \neg B\}\quad skip\quad \{Q\}}{\{P\}\quad if\quad B\quad then\quad S1\quad else\quad skip \quad \{Q\}}$$
The triple for \CodeInline{else} is:
$$\{P \wedge \neg B\}\quad skip\quad \{Q\}$$
which means that we need to ensure:
$$P \wedge \neg B \Rightarrow Q$$

In short, if the condition $B$ of \texttt{if} is false, this means
that the postcondition of the complete conditional statement is already
established before entering the else-branch (since it does not do
anything).



As an example, we consider the following code snippet where we reset a
variable $c$ to a default value in case it had not been properly
initialized by the user.



\begin{CodeBlock}{c}
int c;

// ... some code ...

if(c < 0 || c > 15){
  c = 0;
}
//@ assert 0 <= c <= 15;
\end{CodeBlock}



Let



$wp(if \neg (c \in [0;15])\ then\ c := 0, \{c \in [0;15]\})$



$:= (\neg (c \in [0;15])\Rightarrow wp(c := 0, \{c \in [0;15]\})) \wedge (c \in [0;15]\Rightarrow wp(skip, \{c \in [0;15]\}))$



$= (\neg (c \in [0;15]) \Rightarrow 0 \in [0;15]) \wedge (c \in [0;15] \Rightarrow c \in [0;15])$



$= (\neg (c \in [0;15]) \Rightarrow true) \wedge true$



The property can be verified: independent of the evaluation of
$\neg (c \in [0;15])$, the implication will hold.



\levelThreeTitle{Bonus Stage - Consequence rule}
\label{l3:statements-basic-consequence}


It can sometimes be useful to strengthen a postcondition or to weaken a
precondition. The former will often be established by us to
facilitate the work of the prover, the latter is more often verified by
the tool as the result of computing the weakest precondition.



The inference rule of Hoare logic is the following:
$$\dfrac{P \Rightarrow WP \quad \{WP\}\quad c\quad \{SQ\} \quad SQ \Rightarrow Q}{\{P\}\quad c \quad \{Q\}}$$

(We remark that the premises here are not only Hoare triples but also
formulas to verify.)

For example, if our postcondition is too complex, it may generate a
weaker precondition that is, however, too complicated, thus making the
work of provers more difficult. We can then create a simpler
intermediate postcondition $SQ$, that is, however, stricter and
implies the real postcondition. This is the part $SQ \Rightarrow Q$.

Conversely, the calculation of the precondition will usually generate a
complicated and often weaker formula than the precondition we want to
accept as input. In this case, it is our tool that will check the
implication between what we want and what is necessary for our code to
be valid. This is the part $P \Rightarrow WP$.

We can illustrate this with the following code. Note that here the code
could be proved by WP without the weakening and strengthening of
properties because the code is very simple, it is just to illustrate the
rule of consequence.



\begin{CodeBlock}{c}
/*@
  requires P: 2 <= a <= 8;
  ensures  Q: 0 <= \result <= 100 ;
  assigns  \nothing ;
*/
int constrained_times_10(int a){
  //@ assert P_imply_WP: 2 <= a <= 8 ==> 1 <= a <= 9 ;
  //@ assert WP:         1 <= a <= 9 ;

  int res = a * 10;

  //@ assert SQ:         10 <= res <= 90 ;
  //@ assert SQ_imply_Q: 10 <= res <= 90 ==> 0 <= res <= 100 ;

  return res;
}
\end{CodeBlock}



(Note: We have omitted here the control of integer overflow.)



Here we want to have a result between 0 and 100. But we know that the
code will not produce a result outside the bounds of 10 and 90. So we
strengthen the postcondition with an assertion that at the end
\CodeInline{res}, the result, is between 0 and 90. The calculation of the
weakest precondition of this property together with the assignment
\CodeInline{res = 10 * a} yields a weaker precondition
\CodeInline{1 <= a <= 9} and we know that
\CodeInline{2 <= a <= 8} gives us the desired
guarantee.



When there are difficulties to carry out a proof on more complex code,
then it is often helpful to write assertions that produce stronger, yet
easier to verify, postconditions. Note that in the previous code, the
lines \CodeInline{P\_imply\_WP} and\CodeInline{SQ\_imply\_Q} are never used
because this is the default reasoning of WP. They are just here for
illustrating the rule.


\levelThreeTitle{Bonus Stage - Constancy rule}
\label{l3:statements-basic-constancy}


Certain sequences of instructions may concern and involve different
variables. Thus, we may initialize and manipulate a certain number of
variables, begin to use some of them for a time, before using other
variables. When this happens, we want our tool to be concerned only with
variables that are susceptible to change in order to obtain the simplest
possible properties.



The rule of inference that defines this reasoning is the following:
$$\dfrac{\{P\}\quad c\quad \{Q\}}{\{P \wedge R\}\quad c\quad \{Q \wedge R\}}$$
where $c$ does not modify any variable in $R$. In other words:
``To check the triple, let's get rid of the parts of the formula that
involve variables that are not influenced by $c$ and prove the new
triple.'' However, we must be careful not to delete too much
information, since this could mean that we are not able to prove our
properties.


As an example, let us consider the following code (here gain, we ignore
potential integer overflows):



\begin{CodeBlock}{c}
/*@
  requires a > -99 ;
  requires b > 100 ;
  ensures  \result > 0 ;
  assigns  \nothing ;
*/
int foo(int a, int b){
  if(a >= 0){
    a++ ;
  } else {
    a += b ;
  }
  return a ;
}
\end{CodeBlock}


If we look at the code of the \CodeInline{if} block, we notice that it does
not use the variable \CodeInline{b}. Thus, we can completely omit the
properties about \CodeInline{b} in order to prove that \CodeInline{a} will be
strictly greater than 0 after the execution of the block:



\begin{CodeBlock}{c}
/*@
  requires a > -99 ;
  requires b > 100 ;
  ensures  \result > 0 ;
  assigns  \nothing ;
*/
int foo(int a, int b){
  if(a >= 0){
    //@ assert a >= 0; // and nothing about b
    a++ ;
  } else {
    a += b ;
  }
  return a ;
}
\end{CodeBlock}



On the other hand, in the \CodeInline{else} block, even if \CodeInline{b} is
not modified, formulating properties only about \CodeInline{a} would render
a proof impossible for humans. The code would be:



\begin{CodeBlock}{c}
/*@
  requires a > -99 ;
  requires b > 100 ;
  ensures  \result > 0 ;
  assigns  \nothing ;
*/
int foo(int a, int b){
  if(a >= 0){
    //@ assert a >= 0; // and nothing about b
    a++ ;
  } else {
    //@ assert a < 0 && a > -99 ; // and nothing about b
    a += b ;
  }
  return a ;
}
\end{CodeBlock}



In the \CodeInline{else} block, knowing that\CodeInline{a} lies between -99 and
0, but knowing nothing about \CodeInline{b}, we could hardly know if the
operation \CodeInline{a + = b} produces a result that is greater than 0.

The WP plug-in will, of course, prove the function without problems,
since it produces by itself the properties that are necessary for the
proof. In fact, the analysis which variables are necessary or not (and,
consequently, the application of the constancy rule) is conducted
directly by WP.

Let us finally remark that the constancy rule is an instance of the
consequence rule
$$\dfrac{P \wedge R \Rightarrow P \quad \{P\}\quad c\quad \{Q\} \quad Q \Rightarrow Q \wedge R}{\{P \wedge R\}\quad c\quad \{Q \wedge R\}}$$


If the variables of $R$ have not been modified by the operation
(which, on the other hand, may modify the variables of $P$ to produce
$Q$), then the properties $P \wedge R \Rightarrow P$ and
$Q \Rightarrow Q \wedge R$ hold.



\levelThreeTitle{Exercices}



\levelFourTitle{A serie of assignment}


Compute by hand the weakest precondition of the following program:


\begin{CodeBlock}{c}
/*@
  requires -10 <= x <= 0 ;
  requires 0 <= y <= 5 ;
  ensures -10 <= \result <= 10 ;
*/
int function(int x, int y){
  int res ;
  y += 10 ;
  x -= 5 ;
  res = x + y ;
  return res ;
}
\end{CodeBlock}


Deduce that the program is correct with respect to its contract using the
right rule.


\levelFourTitle{Empty ``then'' branch in conditional}


We previously shown that in a condition when the ``else'' branch is empty,
that the postcondition of the complete conditional is already verified if
with the conjunction of the negation of the condition and the precondition.
For both of the following question, we only need the inference rules and
no WP calculus.

Show that when, instead, the ``then'' branch is empty, the conjunction of
the condition and the precondition must implies the postcondition of the
``else'' branch.

Show that when both branches are empty, the overall condition is just a
skip operation.


\levelFourTitle{Short circuit}


C compilers implement short circuit for conditions. For example, that
means that a code like this one (\textbf{without ``else'' block}) :


\begin{CodeBlock}{c}
if(cond1 && cond2){
  // code
}
\end{CodeBlock}



can be written as:



\begin{CodeBlock}{c}
if(cond1){
  if(cond2){    
    // code
  }
}
\end{CodeBlock}



Show that on those two source code, the weakest precondition calculus
generates an equivalent weakest precondition for equivalent for any code
in the ``then'' block. Note that we assume the conditions to be pure
expressions (without side-effects).



\levelFourTitle{A larger program}


Compute by hand the weakest precondition of the following program:


\begin{CodeBlock}{c}
/*@ 
  requires -5 <= y <= 5 ; 
  requires -5 <= x <= 5 ; 
  ensures  -15 <= \result <= 25 ;
*/
int function(int x, int y){
  int res ;

  if(x < 0){
    x = 0 ;
  }
  
  if(y < 0){
    x += 5 ;
  } else {
    x -= 5 ;
  }
  
  res = x - y ;

  return res ;
}
\end{CodeBlock}


Deduce that the program is correct with respect to its contract using the
right rule.



\levelTwoTitle{Loops}

Loops need a particular treatment in deductive verification of
programs. These are the only control structures that will require
important work from us. We cannot avoid this because without loops, it
is difficult to write and prove interesting programs.



Before we look at the way we specify loop, we can answer to a rightful
question: why are loops so complex?



\levelThreeTitle{Induction and invariant}


The nature of loops makes their analysis complex. When we perform our
reasoning, we need a rule to determine the precondition from a given sequence
of instructions and a post-condition. Here, the problem is that we cannot
\emp{a priori} deduce how many times a loop will iterate, and consequently, we
cannot know how many times variables will be modified.



We will then proceed using an inductive reasoning. We have to find a
property that is true before we start to execute the loop and that, if
it is true at the beginning of an iteration, remains true at the end
(and that is consequently true at the beginning of the next iteration).



This type of property is called a loop invariant. A loop invariant is a
property that must be true before and after each loop iteration. For
example with the following loop:



\begin{CodeBlock}{c}
for(int i = 0 ; i < 10 ; ++i){ /* */ }
\end{CodeBlock}



The property $0 <= i <= 10$ is a loop invariant. The property
$-42 <= i <= 42$ is also an invariant (even if it is far less
precise). The property $0 < i <= 10$ is not an invariant because it is
not true at the beginning of the execution of the loop. The property
$0 <= i < 10$ \textbf{is not a loop invariant}, it is not true at the
end of the last iteration that sets the value of \texttt{i} to $10$.

To verify an invariant $I$, WP will then produce the following
``reasoning'':

\begin{itemize}
\item
  verify that $I$ is true at the beginning of the loop (establishment)
\item
  verify that if $I$ is true before an iteration, then $I$ is true
  after (preservation).
\end{itemize}

\levelFourTitle{Formal - Inference rule}

Let us note the invariant $I$, the inference rule corresponding to
loops is defined as follows:

\begin{center}
$\dfrac{\{I \wedge B \}\ c\ \{I\}}{\{I\}\ while(B)\{c\}\ \{I \wedge \neg B\}}$
\end{center}

And the weakest precondition calculus is the following:

\begin{center}
$wp(while (B) \{ c \}, Post) := I \wedge ((B \wedge I) \Rightarrow wp(c, I)) \wedge ((\neg B \wedge I) \Rightarrow Post)$
\end{center}

Let us detail this formula:

\begin{itemize}
\item
  \begin{enumerate}
  \def\labelenumi{(\arabic{enumi})}
  \item
    the first $I$ corresponds to the establishment of the invariant,
    in layman's terms, this is the ``precondition'' of the loop,
  \end{enumerate}
\item
  the second part of the conjunction
  ($(B \wedge I) \Rightarrow wp(c, I)$) corresponds to the
  verification of the operation performed by the body of the loop:

  \begin{itemize}
  \item
    the precondition that we know of the loop body (let us note $KWP$,
    ``Known WP'') is ($KWP = B \wedge I$). That is the fact we have
    entered the loop ($B$ is true), and that the invariant is verified
    at this moment ($I$, is true before we start the loop by (1), and
    we want to verify that it will be true at the end of the body of the
    loop in (2)),
  \item
    \begin{enumerate}
    \def\labelenumi{(\arabic{enumi})}
    \setcounter{enumi}{1}
    \item
      it remains to verify that $KWP$ implies the actual precondition*
      of the body of the loop ($KWP \Rightarrow wp(c, Post)$). What we
      want at the end of the loop is the preservation of the invariant
      $I$ ($B$ is maybe not true anymore however), formally
      $KWP \Rightarrow wp(c, I)$, that is to say
      $(B \wedge I) \Rightarrow wp(c, I)$,
    \end{enumerate}
  \item
    * it corresponds to the application of the consequence rule
    previously explained.
  \end{itemize}
\item
  finally, the last part ($(\neg B \wedge I) \Rightarrow Post$)
  expresses the fact that when the loop ends($\neg B$), and the
  invariant $I$ has been maintained, it must imply that the wanted
  postcondition of the loop is valid.
\end{itemize}

In this computation, we can notice that the $wp$ function does not
indicate any way to obtain the invariant $I$. We have to specify
ourselves this property about our loops.



\levelFourTitle{Back to the WP plugin}


There exist tools that can infer invariant properties (provided that
these properties are simple, automatic tools remain limited). This is not
the case for WP. We will have to manually annotate our programs to
specify the invariant of each loop. To find and write invariants for our
loops will always be the hardest part of our work when we want to prove
programs.



Indeed, when there are no loops, the weakest precondition calculus
function can automatically provide the verifiable properties of our
programs, this is not the case for loop invariant properties for which we
do not have computation procedures. We have to find and express them
correctly, and depending on the algorithm, they can be quite subtle and
complex.



In order to specify a loop invariant, we add the following annotations
before the loop:



\CodeBlockInput{c}{first_loop-1.c}



\begin{Warning}
  \textbf{REMINDER} : The invariant is: i \textbf{<=} 30 !
\end{Warning}


Why? Because along the loop, \texttt{i} will be comprised between 0 and
\textbf{included} 30. 30 is indeed the value that allows us to leave the
loop. Moreover, one of the properties required by the weakest
precondition calculus is that when the loop condition is invalidated, by
knowing the invariant, we can prove the postcondition (Formally
$(\neg B \wedge I) \Rightarrow Post$).

The postcondition of our loop is \texttt{i == 30} and must be implied
by $\neg$ \texttt{i < 30} $\wedge$
\texttt{0 <= i <= 30}. Here, it is true since:
\texttt{i >= 30 \&\& 0 <= i <= 30 ==> i == 30}.
On the opposite, if we exclude the equality to 30, the postcondition
would be unreachable.



Again, we can have a look at the list of proof obligations in ``WP
Goals'':



\image{i_30-1}[Proof obligations generated to verify our loop]


We note that WP produces two different proof obligations: the
establishment of the invariant and its preservation. WP produces exactly
the reasoning we previously described to prove the assertion. In recent
versions of Frama-C, Qed has become particularly aggressive and
powerful, and the generated proof obligation does not show these details
(showing directly ``True''). Using the option \texttt{-wp-no-simpl} at
start, we can however see these details:



\image{i_30-2}[Proof of the assertion, knowing the invariant and the
  invalidation of the loop condition]


But is our specification precise enough?



\CodeBlockInput{c}{first_loop-2.c}



And the result is:



\image{i_30-3}[Side effects, again]


It seems not.



\levelThreeTitle{The assigns clause \ldots{} for loops}


In fact, considering loops, WP \textbf{only} reasons about what is
provided by the user to perform its reasoning. And here, the invariant
does not specify anything about the way the value of \CodeInline{h} is
modified (or not). We could specify the invariant of all program
variables, but it would be a lot of work. ACSL simply allows to add
\CodeInline{assigns} annotations for loops. Any other variable is considered
to keep its old value. For example:



\CodeBlockInput{c}{first_loop-3.c}



This time, we can establish the proof that the loop correctly behaves.
However, we cannot prove that it terminates. The loop invariant is not
enough to perform such a proof. For example, in our program, we could
modify the loop, removing the loop body:



\begin{CodeBlock}{c}
/*@
  loop invariant 0 <= i <= 30;
  loop assigns i;
*/
while(i < 30){
   
}
\end{CodeBlock}



The invariant is still verified, but we cannot prove that the loop ends:
it is infinite.



\levelThreeTitle{Partial correctness and total correctness - Loop variant}


In deductive verification, we find two types of correctness, the partial
correctness and the total correctness. In the first case, the
formulation of the correctness property is ``if the precondition is
valid, and \textbf{if} the computation terminates, then the
postcondition is valid''. In the second case, ``if the precondition is
valid, \textbf{then} the computation terminates and the postcondition is
valid''. By default, WP considers only partial correctness:



\CodeBlockInput{c}{infinite.c}



If we try to verify this code activating the verification of absence of
RTE, we get this result:



\image{infinite}[Proof of false by non termination]


The assertion ``False'' is proved! For a very simple reason: since the
condition of the loop is ``True'' and no instruction of the loop body
allows to leave the loop, it will not terminate. As we are proving the
code with partial correctness, and as the execution does not terminate,
we can prove anything about the code that follows the non terminating
part of the code. However, if the termination is not trivially provable,
the assertion will probably not be proved.



\begin{Information}
  Note that a (provably) unreachable assertion is always proved to be true:
  \inlineImage{goto_end}
  And this is also the case when we trivially know that an instruction
  produces a runtime error (for example dereferencing \CodeInline{NULL}), or
  inserting ``False'' in post-condition as we have already seen with
  \CodeInline{abs} and the parameter \CodeInline{INT\_MIN}.
\end{Information}


In order to prove the termination of a loop, we use the notion of loop
variant. The loop variant is not a property but a value. It is an
expression that involves the element modified by the loop and that
provides an upper bound to the number of iterations that have to be
executed by the loop at each iteration. Thus, it is an expression
greater or equal to 0, and that strictly decreases at each loop
iteration (this will also be verified by induction by WP).


If we take our previous example, we add the loop variant with this
syntax:



\CodeBlockInput{c}{first_loop-4.c}



Again, we can have a look at the generated proof obligations:



\image{i_30-4}[Our loop entirely specified and proved]


The loop variant generates two proof obligations: verify that the value
specified in the variant is positive, and prove that it strictly
decreases during the execution of the loop. And if we delete the line of
code that increments \texttt{i}, WP cannot prove anymore that
\texttt{30\ -\ i} strictly decreases.

We can also note that being able to give a loop invariant does not
necessarily induce that we can give the exact number of remaining
iterations of the loop, as we do not always have a so precise knowledge
of the behavior of the program. We can for example build an example like
this one:



\CodeBlockInput{c}{random_loop.c}



Here, at each iteration, we decrease the value of the variable
\texttt{i} by a value comprised between 1 and \texttt{i}. Thus, we can
ensure that the value of \texttt{i} is positive and strictly decreases
during each loop iteration, but we cannot say how many loop iteration
will be executed.



The loop variant is then only an upper bound on the number of iteration,
not an expression of their exact number.



\levelThreeTitle{Create a link between post-condition and invariant}


Let us consider the following specified program. Our goal is to prove
that this function returns the old value of \texttt{a} plus 10.



\CodeBlockInput{c}{add_ten-0.c}



The weakest precondition calculus does not allow to deduce information
that is not part of the loop invariant. In a code like:



\begin{CodeBlock}{c}
/*@
    ensures \result == \old(a) + 10;
*/
int add_ten(int a){
    ++a;
    ++a;
    ++a;
    //...
    return a;
}
\end{CodeBlock}


By reading the instructions backward
from the postcondition, we always keep all knowledge about \texttt{a}. On
the opposite, as we previously mentioned, outside the loop, WP only
considers the information provided by the invariant. Consequently, our
``add\_10'' function cannot be proved: the invariant does not say anything
about \texttt{a}. To create a link between the postcondition and the
invariant, we have to add this knowledge. See, for example:



\CodeBlockInput{c}{add_ten-1.c}



\begin{Information}
  This need can appear as a very strong constraint. This is not really the
  case. There exists strongly automated analysis that can compute loop
  invariant properties. For example, without a specification, an abstract
  interpretation would easily compute \CodeInline{0 <= i <= 10}
  and \CodeInline{old(a) <= a <= \textbackslash{}old(a)+10}.
  However, it is often more difficult to compute the relations
  that exist between the different variables of a program, for
  example the equality expressed by the invariant we have
  added.
\end{Information}



\levelTwoTitle{Loops - Examples}

\levelThreeTitle{Exemple avec un tableau read-only}


S'il y a une structure de données que nous traitons avec les boucles, c'est bien
le tableau. C'est une bonne base d'exemples pour les boucles, car ils permettent
rapidement de présenter des invariants intéressants et surtout, ils nous 
permettront d'introduire des constructions très importantes d'ACSL.



Prenons par exemple la fonction qui cherche une valeur dans un tableau :



\CodeBlockInput{c}{search.c}



Cet exemple est suffisamment fourni pour introduire des notations importantes.



D'abord, comme nous l'avons déjà mentionné, le prédicat \CodeInline{\textbackslash{}valid\_read} (de 
même que \CodeInline{\textbackslash{}valid}) nous permet de spécifier non seulement la validité d'une 
adresse en lecture mais également celle de tout un ensemble d'adresses 
contiguës. C'est la notation que nous avons utilisée dans cette expression :



\begin{CodeBlock}{c}
//@ requires \valid_read(a + (0 .. length-1));
\end{CodeBlock}



Cette précondition nous atteste que les adresses \CodeInline{a+0}, 
\CodeInline{a+1}, \ldots{}, \CodeInline{a+length-1} sont valides en lecture.



Nous avons également introduit deux notations qui vont nous être très utiles, à 
savoir \CodeInline{\textbackslash{}forall} ($\forall$) et \CodeInline{\textbackslash{}exists} ($\exists$), les 
quantificateurs de la logique. Le premier nous servant à annoncer que pour tout
élément, la propriété suivante est vraie. Le second pour annoncer qu'il existe
un élément tel que la propriété est vraie. Si nous commentons les deux lignes en 
questions, nous pouvons les lire de cette façon :



\begin{CodeBlock}{c}
/*@
//pour tout "off" de type "size_t", tel que SI "off" est compris entre 0 et "length"
//                                 ALORS la case "off" de "a" est différente de "element"
\forall size_t off ; 0 <= off < length ==> a[off] != element;

//il existe "off" de type "size_t", tel que "off" soit compris entre 0 et "length"
//                                 ET que la case "off" de "a" vaille "element"
\exists size_t off ; 0 <= off < length && a[off] == element;
*/
\end{CodeBlock}



Si nous devions résumer leur utilisation, nous pourrions dire que sur un certain
ensemble d'éléments, une propriété est vraie, soit à propos d'au moins l'un
d'eux, soit à propos de la totalité d'entre eux. Un schéma qui reviendra 
typiquement dans ce cas est que nous restreindrons cet ensemble à travers une
première propriété (ici : \CodeInline{0 <= off < length}) puis nous voudrons prouver la
propriété réelle qui nous intéresse à propos d'eux. \textbf{Mais il y a une 
différence fondamentale entre l'usage de \CodeInline{exists} et celui de \CodeInline{forall}}.



Avec \CodeInline{\textbackslash{}forall type a ; p(a) ==> q(a)}, la restriction (\CodeInline{p}) est suivie
par une implication. Pour tout élément, s'il respecte une première propriété 
(\CodeInline{p}), alors il doit vérifier la seconde propriété \CodeInline{q}. Si nous mettions un ET
comme pour le « il existe » (que nous expliquerons ensuite), cela voudrait dire que 
nous voulons que tout élément respecte à la fois les deux propriétés. Parfois, 
cela peut être ce que nous voulons exprimer, mais cela ne correspond alors plus 
à l'idée de restreindre un ensemble dont nous voulons montrer une propriété 
particulière.



Avec \CodeInline{\textbackslash{}exists type a ; p(a) \&\& q(a)}, la restriction (\CodeInline{p}) est suivie
par une conjonction, nous voulons qu'il existe un élément tel que cet élément 
est dans un certain état (défini par \CodeInline{p}), tout en respectant l'autre 
propriété \CodeInline{q}. Si nous mettions une implication comme pour le « pour tout », 
alors une telle expression devient toujours vraie à moins que \CodeInline{p} soit une 
tautologie ! Pourquoi ? Existe-t-il « a » tel que p(a) implique q(a) ? Prenons 
n'importe quel « a » tel que p(a) est faux, l'implication devient vraie.



Cette partie de l'invariant mérite une attention particulière :



\begin{CodeBlock}{c}
//@ loop invariant \forall size_t j; 0 <= j < i ==> array[j] != element;
\end{CodeBlock}



En effet, c'est la partie qui définit l'action de notre boucle, elle indique à
WP ce que la boucle fera (ou apprendra dans le cas présent) tout au long de
son exécution. Ici en l'occurrence, cette formule nous dit qu'à chaque tour,
nous savons que pour toute case entre 0 et la prochaine que nous allons visiter
(\CodeInline{i} exclue), elle stocke une valeur différente de l'élément
recherché.



Le but de WP associé à la préservation de cet invariant est un peu compliqué, il
n'est pour nous pas très intéressant de se pencher dessus. En revanche, la 
preuve de l'établissement de cet invariant est intéressante :



\image{trivial}


Nous constatons que cette propriété, pourtant complexe, est prouvée par 
Qed sans aucun problème. Si nous regardons sur quelles parties du programme la 
preuve se base, nous voyons l'instruction \CodeInline{i = 0} surlignée, et c'est 
bien la dernière instruction que nous effectuons sur \CodeInline{i} avant de commencer
la boucle. Et donc effectivement, si nous faisons le remplacement dans la formule 
de l'invariant :



\begin{CodeBlock}{c}
//@ loop invariant \forall size_t j; 0 <= j < 0 ==> array[j] != element;
\end{CodeBlock}



« Pour tout j, supérieur ou égal à 0 et inférieur strict à 0 », cette partie est
nécessairement fausse. Notre implication est donc nécessairement vraie.



\levelThreeTitle{Exemples avec tableaux mutables}


Nous allons voir deux exemples avec la manipulation de tableaux en mutation. 
L'un avec une modification totale, l'autre en modification sélective.



\levelFourTitle{Remise à zéro}


Regardons la fonction effectuant la remise à zéro d'un tableau.



\CodeBlockInput{c}{reset.c}



Nous voyons que nous utilisons pour l'invariant une structure assez similaire
à ce que nous avons utilisé pour l'exemple précédent: nous indiquons un premier
invariant pour contraindre la valeur de \CodeInline{i}, et un autre qui exprime
à chaque itération ce que nous avons appris depuis le début de l'exécution de la
boucle (tous les éléments visités sont à $0$). Finalement, intéressons nous à la
clause \CodeInline{loop assigns}: à nouveau, nous utilisons la notation 
\CodeInline{n .. m} pour indiquer quelle partie du tableau a été modifiée.



\levelFourTitle{Chercher et remplacer}
\label{l4:statements-loops-ex-search-and-replace}


Le dernier exemple qui nous intéresse est l'algorithme « chercher et remplacer ». 
C'est un algorithme qui modifie sélectivement des valeurs dans une 
certaine plage d'adresses. Il est toujours un peu difficile de guider l'outil 
dans ce genre de cas car, d'une part, nous devons garder « en mémoire » ce qui est modifié 
et ce qui ne l'est pas et, d'autre part, parce que l'induction repose sur ce fait.



À titre d'exemple, la première spécification que nous pouvons réaliser pour 
cette fonction ressemblerait à ceci :



\CodeBlockInput{c}{search_and_replace-0.c}



Nous utilisons la fonction logique \CodeInline{\textbackslash{}at(v, Label)} qui nous donne la valeur de
la variable \CodeInline{v} au point de programme \CodeInline{Label}. Si nous regardons l'utilisation qui
en est faite ici, nous voyons que dans l'invariant de boucle, nous cherchons à 
établir une relation entre les anciennes valeurs du tableau et leurs potentielles 
nouvelles valeurs :



\begin{CodeBlock}{c}
/*@
  loop invariant \forall size_t j; 0 <= j < i && \at(array[j], Pre) == old 
                   ==> array[j] == new;
  loop invariant \forall size_t j; 0 <= j < i && \at(array[j], Pre) != old 
                   ==> array[j] == \at(array[j], Pre);
*/
\end{CodeBlock}


Pour tout élément que nous avons visité, s'il valait la valeur qui doit être
remplacée, alors il vaut la nouvelle valeur, sinon il n'a pas changé. Alors que
nous nous reposions sur la clause \CodeInline{assigns} pour ce genre de propriété
dans les exemples précédents, ici nous ne pouvons pas le faire. Même si ACSL nous
permettrait d'exprimer cette propriété de manière très précise, WP ne pourrait pas
vraiment en tirer parti, dû à la manière dont cette clause est traitée. Nous devons
donc utiliser un invariant et conserver une approximation des positions mémoire
affectées.

 
En fait, si nous essayons de prouver l'invariant, WP n'y parvient pas. Dans ce genre de 
cas, le plus simple est encore d'ajouter diverses assertions exprimant les 
propriétés intermédiaires que nous nous attendons à voir facilement prouvées 
et impliquant l'invariant. En fait, nous nous apercevons rapidement que WP 
n'arrive pas à maintenir le fait que nous n'avons pas encore modifié la fin du 
tableau :



\begin{CodeBlock}{c}
for(size_t i = 0; i < length; ++i){
    //@assert array[i] == \at(array[i], Pre); // échec de preuve
    if(array[i] == old) array[i] = new;
}
\end{CodeBlock}



Nous pouvons donc ajouter cette information comme invariant :



\CodeBlockInput[13][26]{c}{search_and_replace-1.c}



Et cette fois, la preuve passera. 



\levelThreeTitle{Exercices}


Pour tous ces exercices, utiliser la commande suivante pour démarrer la vérification:

\begin{CodeBlock}{bash}
frama-c-gui -wp -wp-rte -warn-unsigned-overflow your-file.c
\end{CodeBlock}  


\levelFourTitle{Fonctions sans modification : Forall, Exists, ...}


Actuellement, les pointeurs de fonction ne sont pas directement supportés par WP.
Considérons que nous avons une fonction :



\CodeBlockInput[3][9]{c}{ex-1-forall-exists.c}


Écrire un corps (au choix) pour cette fonction et un contrat l'accompagnant.
Ensuite, écrire les fonctions suivantes avec leurs contrats pour prouver leur
correction. Notons qu'il n'est pas possible d'utiliser une fonction C dans un
contrat, la propriété que choisie pour la fonction \CodeInline{pred}
devra donc être inlinée dans la spécification des différentes fonctions.

\begin{itemize}
\item \CodeInline{forall\_pred} retourne vrai si et seulement si \CodeInline{pred}
  est vraie pour tous les éléments ;
\item \CodeInline{exists\_pred} retourne vrai si et seulement si \CodeInline{pred}
  est vraie pour au moins un élément ;
\item \CodeInline{none\_pred} retourne vrai si et seulement si \CodeInline{pred}
  est fausse pour tous les éléments ;
\item \CodeInline{some\_not\_pred} retourne vrai si et seulement si \CodeInline{pred}
  est fausse pour au moins un élément.
\end{itemize}

Les deux dernières fonctions devraient être écrites en appelant seulement les deux
premières.



\levelFourTitle{Fonction sans modification : Égalité de plages de valeurs}


Écrire, spécifier et prouver la fonction \CodeInline{equal} qui retourne vrai
si et seulement si deux plages de valeurs sont égales. Écrire, en utilisant la
fonction \CodeInline{equal}, le code de \CodeInline{different} qui retourne vrai
si et seulement si deux plages de valeurs sont différentes, votre postcondition
devrait contenir un quantificateur existentiel.


\CodeBlockInput[3][9]{c}{ex-2-equal.c}



\levelFourTitle{Fonction sans modification : recherche dichotomique}
\label{l4:statements-loops-ex-bsearch}


La fonction suivante cherche la position d'une valeur fournie en entrée dans
un tableau, en supposant que le tableau est trié. D'abord, considérons que la
longueur du tableau est fournie en tant qu'int et utilisons des valeurs de ce
même type pour gérer les indices. Nous pouvons noter qu'il y a deux comportements
à cette fonction : soit la valeur existe dans le tableau (et le résultat est 
l'indice de cette valeur) ou pas (et le résultat est -1).
 
 
\CodeBlockInput{c}{ex-3-binary-search.c}


Cette fonction est un petit peu complexe à prouver, voici quelques conseils
pour en venir à bout. D'abord, la longueur de la fonction est reçue en utilisant
un type int, donc nous devons poser une restriction sur cette longueur en 
précondition pour qu'elle soit cohérente. Ensuite, l'un des invariants de la
boucle devrait indiquer les bornes des valeurs \CodeInline{low} et 
\CodeInline{up}, mais nous pouvons noter que pour chacune d'elles, l'une des
bornes n'est pas nécessaire. Finalement, la seconde propriété invariante 
devrait indiquer que si l'un des indices du tableau correspond à la valeur
recherchée, alors cette indice devrait être correctement borné.

\textbf{Plus dur :} Modifier cette fonction de façon à recevoir \CodeInline{len}
comme un \CodeInline{size\_t}. Il faut modifier légèrement l'algorithme et
la spécification/les invariants. Conseil : s'arranger pour que \CodeInline{up}
soit une borne exclue de la recherche.


\levelFourTitle{Fonction avec modification : addition de vecteurs}


Écrire, spécifier et prouver la fonction qui ajoute deux vecteurs dans un
troisième. Fixer des contraintes arbitraires sur les valeurs d'entrée pour
gérer le débordement des entiers. Considérer que le vecteur est résultant est
spacialement séparé des vecteurs d'entrée. En revanche, le même vecteur devrait
pouvoir être utilisé pour les deux vecteurs d'entrée.

\CodeBlockInput[3][5]{c}{ex-4-add-vectors.c}


\levelFourTitle{Fonction avec modification : inverse}


Écrire, spécifier et prouver la fonction qui inverse un vecteur en place.
Prendre garde à la partie non-modifiée du vecteur à une itération donnée de la
boucle. Utiliser la fonction \CodeInline{swap} précédemment prouvée.


\CodeBlockInput[3][7]{c}{ex-5-reverse.c}


\levelFourTitle{Fonction avec modification : copie}


Écrire, spécifier et prouver la fonction \CodeInline{copy} qui copie une plage
de valeur dans un autre tableau, en commençant pas la première cellule du
tableau. Considérer (et spécifier) d'abord que les deux plages sont entièrement
séparées.



\CodeBlockInput[3][5]{c}{ex-6-copy.c}


\textbf{Plus dur :} Les vraies fonctions \CodeInline{copy} et
\CodeInline{copy\_backward}.


En fait, une séparation aussi forte n'est pas nécessaire. Pour faire une copie
en partant du début, la précondition réelle doit simplement garantir que si les
deux plages se chevauchent en mémoire, le début de la destination ne doit pas être
dans la plage source :


\begin{CodeBlock}{c}
//@ requires \separated(&src[0 .. len-1], dst) ;
\end{CodeBlock}
 

Essentiellement, en copiant des éléments dans cet ordre, nous pouvons les faire
glisser depuis la fin d'une plage vers le début. En revanche, cela signifie que
nous devons être plus précis dans notre contrat : nous ne garantissons plus une
égalité avec le tableau source mais avec les \emph{anciennes} valeurs du tableau
source. Nous devons également être plus précis dans notre invariant, d'abord en
spécifiant aussi la relation avec l'état précédent de la mémoire, et ensuite en
ajoutant un invariant qui nous dit que le tableau source n'est pas modifié entre
le \CodeInline{i}$^{ème}$ élément visité et le dernier.


Finalement, il est aussi possible d'écrire une fonction qui copie les éléments de
la fin vers le début. Dans ce cas, à nouveau, les plages de valeurs peuvent se
chevaucher, mais la condition n'est pas exactement la même. Écrire, spécifier et
prouver la fonction \CodeInline{copy\_backward} qui copie les éléments dans le
sens inverse.



\horizontalLine



In this part, we have seen how assignment and control structure are translated
to a logic view of our program. We have spent quite a lot of time on loops
because they represent the main difficulty we have to face when we want to
specify and prove a program by deductive verification. The loop annotations
allow us to express as precisely as possible their behavior.



In the next part of this tutorial, we will see more precisely the logic
constructs provided by ACSL. They are important because they give us a way to
express write more abstract specification, that are easier to understand and
to prove.
