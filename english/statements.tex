\begin{levelTwo}
  {}
  {introduction}
\end{levelTwo}

\begin{levelTwo}
  {Assignment, sequence and conditional}
  {basic}
\end{levelTwo}

\begin{levelTwo}
  {Loops}
  {loops}
\end{levelTwo}

\begin{levelTwo}
  {Loops - Examples}
  {loops-examples}
\end{levelTwo}


\horizontalLine



In this part, we have seen how assignment and control structure are translated
to a logic view of our program. We have spent quite a lot of time on loops
because they represent the main difficulty we have to face when we want to
specify and prove a program by deductive verification. The loop annotations
allow us to express as precisely as possible their behavior.



In the next part of this tutorial, we will see more precisely the logic
constructs provided by ACSL. They are important because they give us a way to
express write more abstract specification, that are easier to understand and
to prove.
