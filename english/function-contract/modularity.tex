The end of this part will be dedicated to function call composition,
where we will start to have a closer look to WP. We will also have a
look at the way we can split our programs in different files when we
want to prove them using WP.

Our goal will be to prove the \CodeInline{max\_abs} function, that return
the maximum absolute value of two values:



\CodeBlockInput[6][11]{c}{max_abs.c}



Let us start by (over-)splitting the function we already proved in pairs
header/source for \CodeInline{abs} and \CodeInline{max}. We will obtain, for
\CodeInline{abs}:



File abs.h :

\CodeBlockInput{c}{abs.h}



File abs.c

\CodeBlockInput{c}{abs.c}



We can notice that we put our function contract inside the header file.
The goal is to be able to import the specification at the same time as
the declaration when we need it in another file. Indeed, WP will need it
to be able to prove that the precondition of the function is verified
when we call it.

We can create a file using the same format for the \texttt{max}
function. In both cases, we can open the source file (we do not need to
specify header files in the command line) with Frama-C and notice that
the specification is indeed associated to the function and that we prove
it.


Now, we can prepare our files for the \texttt{max\_abs} function with
the header:



\CodeBlockInput{c}{max_abs_uns.h}



And its source file:



\CodeBlockInput{c}{max_abs.c}



We can open the source file in Frama-C. If we look at the side panel, we
can see that the header files we have included in \texttt{abs\_max}
correctly appear and if we look at the function contracts for them, we
can see some blue and green bullets:



\image{max_abs}[The contract of \CodeInline{max} is assumed to be valid]


These bullets indicate that, since we do not have the implementation,
they are assumed to be true. It is an important strength of the
deductive proof of programs compared to some other formal methods:
functions are verified in isolation from each other.

When we are not currently performing the proof of a function, its
specification is considered to be correct: we do not try to prove it
when we are proving another function, we will only verify that the
precondition is correctly established when we call it. It provides very
modular proofs and specifications that are therefore more reusable. Of
course, if our proof relies on the specification of another function, it
must be provable to ensure that the proof of the program is complete.
But, we can also consider that we trust a function that comes from an
external library that we do not want to prove (or for which we do not
even have the source code).

The careful reader could specify and prove the \CodeInline{max\_abs}
function.

A solution is provided there:



\CodeBlockInput[4][14]{c}{max_abs.h}



\levelThreeTitle{Exercices}


\levelFourTitle{Days of the month}


Specify the function leap year that returns true if the year received
as an input is leap. Use this functions to complete the functions
\CodeInline{days\_of} in order to return the number of days of the
month received as an input, including the right behavior when the year
is leap for february.



\CodeBlockInput{c}{ex-1-days-month.c}


\levelFourTitle{Alpha-numeric character}


Write and specify the different functions used by
\CodeInline{is\_alpha\_num} provide a contract for each of them and
provide the contract of \CodeInline{is\_alpha\_num}.



\CodeBlockInput{c}{ex-2-alphanum.c}


Declare an enum with values \CodeInline{LOWER}, \CodeInline{UPPER},
\CodeInline{DIGIT} and \CodeInline{OTHER}, and a function
\CodeInline{character\_kind} that returns, using the different
functions \CodeInline{is\_lower}, \CodeInline{is\_upper},
\CodeInline{is\_digit}, the kind of character received in input. Use
behaviors to specify the contracts of this function and be sure that
they are disjoint and complete.




\levelFourTitle{Order 3 values}



Taking back the function \CodeInline{max\_ptr} that orders two values,
puting the maximum at the first location and the minimum at second, 
write a function \CodeInline{min\_ptr} that uses this function and
produced the opposite operation. Use these functions to complete the
four functions that orders 3 values. For each variant (increasing and
decreasing), write it once using only \CodeInline{max\_ptr} and once
using only \CodeInline{min\_ptr}.



\CodeBlockInput{c}{ex-3-order-3.c}
