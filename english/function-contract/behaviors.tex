Sometimes, a function can have behaviors that can be quite different
depending on the input. Typically, a function can receive a pointer to
an optional resource: if the pointer is \texttt{NULL}, we will have a
certain behavior, which will be different of the behavior expected when
the pointer is not \texttt{NULL}.

We have already seen a function that have different behaviors: the
\texttt{abs} function. We will use it again to illustrate behaviors. We
have two behaviors for the \texttt{abs} function: either the input is
positive or it is negative.

Behaviors allow us to specify the different cases for postconditions. We
introduce them using the \texttt{behavior} keyword. Each behavior will
have a name, the assumptions we have for the given case introduced with
the clause \texttt{assumes}, that expresses properties about the values
before the call of the function (thus a keyword like
\CodeInline{\textbackslash{}old} cannot be used in \texttt{assumes}
clauses), and the associated postcondition. Finally, we can ask WP to
verify that behaviors are disjoint (to guarantee determinism) and
complete (to guarantee that we cover all possible input).

Behaviors are disjoint if for any (valid) input of the function, it
corresponds to the assumption (\texttt{assumes}) of a single behavior.
Behaviors are complete if any (valid) input of the function corresponds
to at least one behavior.

For example, for \texttt{abs} we can write the specification:



\CodeBlockInput{c}{abs.c}



Note that declaring behaviors do not forbids to specify global postconditions.
For example here, we have specified that for any behavior, the function must
return a positive value.



It can be useful to experiment with the two possibilities to understand
the exact meaning of \CodeInline{complete} and \CodeInline{disjoint}:

\begin{itemize}
\item
  replace the assumption of \CodeInline{pos} with
  \CodeInline{val \textgreater{} 0}, in this case, behaviors will be
  disjoint but incomplete (we will miss \CodeInline{val == 0}),
\item
  replace the assumption of \CodeInline{neg} with
  \CodeInline{val \textless{}= 0}, in this case, behaviors will be
  complete but not disjoint (we will have two assumptions corresponding
  to \CodeInline{val == 0}.
\end{itemize}


\begin{Warning}
  Even if \CodeInline{assigns} is a postcondition, indicating different assigns
  in each behavior is currently not well-handled by WP. If we need to specify
  this, we will:
  \begin{itemize}
  \item put our \CodeInline{assigns} before the behaviors (as we have done in our
    example) with all potentially modified non-local elements,
  \item add in postcondition of each behaviors the elements that are in fact
    not modified by indicating their new value to be equal to the
    \CodeInline{\textbackslash{}old} one.
  \end{itemize}
\end{Warning}


Behaviors are useful to simplify the writing of specifications when
functions can have very different behaviors depending on their input.
Without them, specification would be defined using implications
expressing the same idea but harder to write and read (which would be
error-prone).



On the other hand, the translation of completeness and disjointedness
would be necessarily written by hand which would be tedious and again
error-prone.


\levelThreeTitle{Exercises}



\levelFourTitle{Distance}



Take back the example about the computation of the distance between to
integers. Considering that the written contract was:


\CodeBlockInput{c}{ex-1-distance.c}


Re-write it using behaviors.



\levelFourTitle{Reset on condition}



Take back the example ``reset on condition'' from the previous section.
Considering that the written contract was:



\CodeBlockInput[1][13]{c}{ex-2-reset-on-cond.c}



Re-write it using behaviors.



\levelFourTitle{Days of the month}


Take back the example ``day of the month'' from the first section.
Considering that the written contract was:



\CodeBlockInput{c}{ex-3-day-month.c}


Re-write it using behaviors.


\levelFourTitle{Max of pointer values, ordering}



Take back the example ``Max of pointed values'' from the previous section,
this time with the version that orders the values. Considering that the
contract was:



\CodeBlockInput[1][13]{c}{ex-4-max-ptr.c}



Re-write it using behaviors.



\levelFourTitle{Max of pointed values, returning the result}



Take back the example ``Max of pointed values'' from the previous section,
and more precisely, the version that returns the result. Considering that
the contract was:



\CodeBlockInput[1][10]{c}{ex-5-max-ptr.c}



\begin{enumerate}
\item Rewrite it using behaviors
\item Modify the contract of 1. in order to make the behaviors non-disjoint,
  except this property, the contract should remain verified,
\item Modify the contract of 1. in order to make the behaviors incomplete,
  add a new behavior that makes the contract complete again,
\item Modify the function of 1. in order to accept \CodeInline{NULL} pointers
  for both \CodeInline{a} and \CodeInline{b}. If both of them are nul pointers,
  return \CodeInline{INT\_MIN}, if one is a nul pointer, return the value of
  the other, else, return the maximum of them. Modify the contract accordingly
  by adding new behaviors. Be sure that they are disjoint and complete.
\end{enumerate}


\levelFourTitle{Order 3}


Take back the example ``Order 3 values'' from the previous section. Considering
That the contract was:


\CodeBlockInput{c}{ex-6-order-3.c}


Rewrite it using behaviors. Note that you should have one general behaviors
and 3 specific behaviors. Are these behaviors complete? Are they disjoint?
