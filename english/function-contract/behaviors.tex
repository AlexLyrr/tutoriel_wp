Sometimes, a function can have behaviors that can be quite different
depending on the input. Typically, a function can receive a pointer to
an optional resource: if the pointer is \texttt{NULL}, we have a
certain behavior, which is different of the behavior expected when
the pointer is not \texttt{NULL}.

We have already seen a function that have different behaviors: the
\texttt{abs} function. Let us use it again to illustrate behaviors. We
have two behaviors for the \texttt{abs} function: either the input is
positive or it is negative.

Behaviors allow us to specify the different cases for postconditions. We
introduce them using the \texttt{behavior} keyword. Each behavior
is named. For a given behavior, we have different assumptions about the
input of the function, they are introduced with the clause
\texttt{assumes} (note that since they characterize the input, the
keyword \CodeInline{\textbackslash{}old} cannot be used there). However,
the properties expressed by this clause do not have to be verified before
the call, they can be verified and in this case, the postcondition
specified in our behavior applies. The postconditions of a particular
behavior are introduced using \texttt{ensures}. Finally, we can ask WP to
verify that behaviors are disjoint (to guarantee determinism) and
complete (to guarantee that we cover all possible input).

Behaviors are disjoint if for any (valid) input of the function, it
corresponds to the assumption (\texttt{assumes}) of a single behavior.
Behaviors are complete if any (valid) input of the function corresponds
to at least one behavior.

For example, for \texttt{abs} we can write the specification:



\CodeBlockInput{c}{abs.c}



Note that declaring behaviors does not forbid to specify global postconditions.
For example here, we have specified that for any behavior, the function must
return a positive value.



Let us now slighlty modify the assumptions of each behavior to illustrate
the meaning of \CodeInline{complete} and \CodeInline{disjoint}:

\begin{itemize}
\item
  replace the assumption of \CodeInline{pos} with
  \CodeInline{val \textgreater{} 0}, in this case, behaviors are
  disjoint but incomplete (we miss \CodeInline{val == 0}),
\item
  replace the assumption of \CodeInline{neg} with
  \CodeInline{val \textless{}= 0}, in this case, behaviors are
  complete but not disjoint (we have two assumptions corresponding
  to \CodeInline{val == 0}.
\end{itemize}


\begin{Warning}
  Even if \CodeInline{assigns} is a postcondition, indicating different assigns
  in each behavior is currently not well-handled by WP. If we need to specify
  this, we will:
  \begin{itemize}
  \item put our \CodeInline{assigns} before the behaviors (as we have done in our
    example) with all potentially modified non-local elements,
  \item add in postcondition of each behaviors the elements that are in fact
    not modified by indicating their new value to be equal to the
    \CodeInline{\textbackslash{}old} one.
  \end{itemize}
\end{Warning}


Behaviors are useful to simplify the writing of specifications when
functions can have very different behaviors depending on their input.
Without them, specification would be defined using implications
expressing the same idea but harder to write and read (which would be
error-prone). On the other hand, the translation of completeness and
disjointedness would be necessarily written by hand which would be
tedious and again error-prone.


\levelThreeTitle{Exercises}


\levelFourTitle{Previous exercices}


From previous sections, take back the examples:


\begin{itemize}
\item about the computation of the distance between to integers,
\item ``reset on condition'',
\item ``days of the month'',
\item ``Max of pointed values''.
\end{itemize}


Considering that the contracts were:


\CodeBlockInput{c}{ex-1-past.c}


Re-write them using behaviors.


\levelFourTitle{Two other simple exercices}


Produce the code and the specification of the two following functions
and prove them. The specification should make use of behaviors.


First, a function that returns if a character is a vowel or a consonant,
one should assume (and express) that the function receives a lowercase
letter.


\CodeBlockInput[1][5]{c}{ex-2-simple.c}


Then, a function that returns the quadrant of a given coordinate. When
the coordinate is on an axis, arbitrary choose one of the quadrants.


\CodeBlockInput[7][9]{c}{ex-2-simple.c}


\levelFourTitle{Triangle}


Complete the following functions that receive the lengths of the different
sides of a triangle, and respectively return:

\begin{itemize}
\item if the triangle is scalene, isoscele, or equilateral,
\item if the triangle is right, acute or obtuse.
\end{itemize}


Assuming (and it must be expressed) that:


\begin{itemize}
\item the received parameters indeed define a triangle,
\item \CodeInline{a} is the hypotenuse of the triangle,
\end{itemize}


specify and prove that the functions do the right job.


\levelFourTitle{Max of pointed values, returning the result}


Take back the example ``Max of pointed values'' from the previous section,
and more precisely, the version that returns the result. Considering that
the contract was:



\CodeBlockInput[1][10]{c}{ex-4-max-ptr.c}



\begin{enumerate}
\item Rewrite it using behaviors
\item Modify the contract of 1. in order to make the behaviors non-disjoint,
  except this property, the contract should remain verified,
\item Modify the contract of 1. in order to make the behaviors incomplete,
  add a new behavior that makes the contract complete again,
\item Modify the function of 1. in order to accept \CodeInline{NULL} pointers
  for both \CodeInline{a} and \CodeInline{b}. If both of them are nul pointers,
  return \CodeInline{INT\_MIN}, if one is a nul pointer, return the value of
  the other, else, return the maximum of them. Modify the contract accordingly
  by adding new behaviors. Be sure that they are disjoint and complete.
\end{enumerate}


\levelFourTitle{Order 3}


Take back the example ``Order 3 values'' from the previous section. Considering
that the contract was:


\CodeBlockInput{c}{ex-5-order-3.c}


Rewrite it using behaviors. Note that you should have one general behaviors
and 3 specific behaviors. Are these behaviors complete? Are they disjoint?
