
\levelThreeTitle{Correctly write what we expect}


This is certainly the hardest part of our work. Programming is already
an effort that consists in writing algorithms that correctly respond to
our need. Specifying request the same kind of work, except that we do
not try to express \emph{how} we respond to the need but \emph{what} is
exactly our need. To prove that our code implements what we need, we
must be able to describe exactly what we need.

From now, we will use an other example, the \texttt{max} function:



\CodeBlockInput{c}{max-0.c}



The reader could write and prove their own specification. We will start
using this one:



\CodeBlockInput[1][6]{c}{max-1.c}



If we ask WP to prove this code, it will succeed without any problem.
However, is our specification really correct? We can try to prove this
calling code:



\CodeBlockInput[8][14]{c}{max-1.c}



There, it will fail. In fact, we can go further by modifying the body of
the \CodeInline{max} function and notice that the following code is also
correct with respect to the specification:



\CodeBlockInput[1][8]{c}{max-2.c}



Our specification is too permissive. We have to be more precise. We do
not only expect the result to be greater or equal to both parameters,
but also that the result is one of them:



\CodeBlockInput[1][7]{c}{max-3.c}



\levelThreeTitle{Pointers}


If there is one notion that we permanently have to confront with in C
language, this is definitely the notion of pointer. Pointers are quite
hard to manipulate correctly, and they still are the main source of
critical bugs in programs, so they benefit of a preferential treatment
in ACSL.

We can illustrate with a swap function for C integers:



\CodeBlockInput{c}{swap-0.c}



\levelFourTitle{History of values in memory}


Here, we introduce a first b>uilt-in logic function of ACSL:
\CodeInline{old}, that allows us to get the old value of a given element.
So, our specification defines that the function must ensure that after
the call, the value of \CodeInline{*a} is the old (that is to say, before
the call) value of \CodeInline{*b} and conversely.

The \CodeInline{\textbackslash{}old} function can only be used in the
post-condition of a function. If we need this type of information
somewhere else, we will use \CodeInline{at} that allows us to express that
we want the value of a variable at a particular program point. This
function receives two parameters. The first one is the variable (or
memory location) for which we want to get its value and the second one
is the program point (as a C label) that we want to consider.

For example, we could write:



\CodeBlockInput[2][6]{c}{at.c}



Of course, we can use any C label in our code, but we also have 6
built-in labels defined by ACSL that can be used, however WP does not
support all of them currently:



\begin{itemize}
\item \CodeInline{Pre}/\CodeInline{Old}: value before function call,
\item \CodeInline{Post}: value after function call,
\item \CodeInline{LoopEntry}: value at loop entry (not supported yet),
\item \CodeInline{LoopCurrent}: value at the beginning of the current step of
  the loop (not supported yet),
\item \CodeInline{Here}: value at the current program point.
\end{itemize}


\begin{Information}
  The behavior of \CodeInline{Here} is in fact the default behavior when we
  consider a variable. Its use with \CodeInline{\textbackslash{}at} with
  generally let us ensure that what we write is not ambigous, and is more
  readable, when we express properties about values at different program
  points in the same expression.
\end{Information}


Whereas \CodeInline{\textbackslash{}old} can only be used in function
post-conditions, \CodeInline{\textbackslash{}at} can be used anywhere.
However, we cannot use any program point with respect to the type
annotation we are writing. \CodeInline{Old} and \CodeInline{Post} are only
available in function post-conditions, \CodeInline{Pre} and \CodeInline{Here}
are available everywhere. \CodeInline{LoopEntry} and \CodeInline{LoopCurrent}
are only available in the context of loops (which we will detail later
in this tutorial).



For the moment, we will not need \CodeInline{\textbackslash{}at} but it can
often be useful, if not essential, when we want to make our
specification precise.



\levelFourTitle{Pointers validity}


If we try to prove that the swap function is correct (comprising RTE
verification), our post-condition is indeed verified but WP failed to
prove some possibilities of runtime-error, since we perform access to
some pointers that we did not indicate to be valid pointers in the
precondition of the function.

We can express that the dereferencing of a pointer is valid using the
\CodeInline{\textbackslash{}valid} predicate of ACSL which receives the
pointer in input:



\CodeBlockInput[3][11]{c}{swap-1.c}



Once we have specified that the pointers we receive in input are valid,
dereferencing is assured to not produce undefined behaviors.



As we will see later in this tutorial, \CodeInline{\textbackslash{}valid}
can take more than one pointer in parameter. For example, we can give it
an expression such as: \CodeInline{valid(p + (s .. e))} which means
``for all \CodeInline{i} between included \CodeInline{s} and \CodeInline{e},
\CodeInline{p+i} is a valid pointer. This kind of expression will be
extremely useful when we will specify properties about arrays in
specifications.



If we have a closer look to the assertions that WP adds in the swap
function comprising RTE verification, we can notice that there exists
another version of the \CodeInline{\textbackslash{}valid} predicate, denoted
\CodeInline{\textbackslash{}valid\_read}. As opposed to \CodeInline{valid}, the
predicate \CodeInline{\textbackslash{}valid\_read} indicates that a pointer
can be dereferenced, but only to read the pointed memory. This subtlety
is due to the C language, where the downcast of a const pointer is easy
to write but is not necessarily legal.



Typically, in this code:



\CodeBlockInput{c}{unref.c}



Dereferencing \CodeInline{p} is valid, however the precondition of
\CodeInline{unref} will not be verified by WP since dereferencing
\CodeInline{value} is only legal for a read-access. A write access will
result in an undefined behavior. In such a case, we can specify that the
pointer \CodeInline{p} must be \CodeInline{\textbackslash{}valid\_read} and not
\CodeInline{\textbackslash{}valid}.



\levelFourTitle{Side Effects}


Our \CodeInline{swap} function is provable with regard to the specification
and potential runtime errors, but is it however specified enough? We can
slightly modify our code to check this (we use \CodeInline{assert} to verify
some properties at some particular points):



\CodeBlockInput{c}{swap-1.c}



The result is not exactly what we expect:



\image{2-swap-1}[Proof failure on the property of a global variable which is not
  modified by \texttt{swap}]


Indeed, we did not specify the allowed side effects for our function. In
order to specify side effects, we use an \CodeInline{assign} clause which is
part of the postcondition of a function. It allows us to specify which
\textbf{non local} elements (we verify side effects) can be modified
during the execution of the function.



By default, WP considers that a function can modify everything in the
memory. So, we have to specify what can be modified by a function. For
example, our \CodeInline{swap} function will be specified like this:



\CodeBlockInput[3][14]{c}{swap-2.c}



If we ask WP to prove the function with this specification, it will be
validated (including with the variable added in the previous source
code).



Finally, we sometimes want to specify that a function is side effect
free. We specify this by giving \CodeInline{\textbackslash{}nothing} to
\CodeInline{assigns}:



\CodeBlockInput{c}{max_ptr.c}



The careful reader will now be able to take back the examples we
presented until now to integrate the right \CodeInline{assigns} clause.



\levelFourTitle{Memory location separation}


Pointers bring the risk of aliasing (multiple pointers can have access
to the same memory location). For some functions, it will not cause any
problem, for example when we give two identical pointers to the
\texttt{swap} function, the specification is still verified. However,
sometimes it is not that simple:



\CodeBlockInput{c}{incr_a_by_b-0.c}



If we ask WP to prove this function, we get the following result:



\image{2-incr_a_by_b-1}[Proof failure: potential aliasing]


The reason is simply that we do not have any guarantee that the pointer
\CodeInline{a} is different of the pointer \CodeInline{b}. Now, if these
pointers are the same,



\begin{itemize}
\item   the property \CodeInline{*a == \textbackslash{}old(*a) + *b} in fact
  means \CodeInline{*a == \textbackslash{}old(*a) + *a} which can only
  be true if the old value pointed by \CodeInline{a} was $0$, and we do
  not have such a requirement,
\item
  the property \CodeInline{*b == \textbackslash{}old(*b)} is not validated
  because we potentially modify this memory location.
\end{itemize}


\begin{Question}
  Why is the \CodeInline{assign} clause validated ?

  The reason is simply that \CodeInline{a} is indeed the only modified memory
  location. If \CodeInline{a != b}, we only modify the location pointed by
  \CodeInline{a}, and if \CodeInline{a == b}, \textbar{} that is still the case:
  \CodeInline{b} is not another location.
\end{Question}


In order to ensure that pointers address separated memory locations,
ACSL gives use the predicate
\texttt{\textbackslash{}separated(p1,\ ...,pn)} that receives in
parameter a set of pointers and that ensures that these pointers are
non-overlapping. Here, we specify:



\CodeBlockInput{c}{incr_a_by_b-1.c}



And this time, the function is verified:



\image{2-incr_a_by_b-2}[Solved aliasing problems]


We can notice that we do not consider the arithmetic overflow here, as
we do not focus on this question in this section. However, if this
function was part of a complete program, it would be necessary to define
the context of use of this function and the precondition guaranteeing
the absence of overflow.
