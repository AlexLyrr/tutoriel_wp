From the beginning of this tutorial, we have used different predicates
and logic functions provided by ACSL: \CodeInline{\textbackslash{}valid},
\CodeInline{\textbackslash{}valid\_read}, \CodeInline{\textbackslash{}separated},
\CodeInline{\textbackslash{}old} et \CodeInline{\textbackslash{}at}. There are others
built-in predicates but we will not present them all, the reader can refer to
\externalLink{the documentation (ACSL implementation)}{http://frama-c.com/download.html}
(note that everything is not necessarily supported by WP).



ACSL allows us to do something more than ``just'' specify our code using
existing predicates and functions. We can define our own predicates,
functions, relations, etc. Doing this, we can have more abstract
specifications. It also allows us to factor specifications (for example
defining what is a valid array), which have two pleasant consequences:
our specifications are more readable and more understandable, and we can
reuse existing proofs to ease the proof of new programs.



\levelTwoTitle{Some logical types}

ACSL fournit différents types logiques qui permettent d'écrire des
propriétés dans un monde plus abstrait, plus mathématique. Parmi les types qui
peuvent être utiles, certains sont dédiés aux nombres et permettent d'exprimer
des propriétés ou des fonctions sans avoir à nous soucier des contraintes dues
à la taille en mémoire des types primitifs du C. Ces types sont \CodeInline{integer}
et \CodeInline{real}, qui représentent respectivement les entiers mathématiques et
les réels mathématiques (pour ces derniers, la modélisation est aussi proche que
possible de la réalité, mais la notion de réel ne peut pas être parfaitement
représentée).



Par la suite, nous utiliserons souvent des entiers à la place des classiques
\CodeInline{ìnt} du C. La raison est simplement que beaucoup de propriétés sont
vraies quelle que soit la taille de l'entier (au sens C, cette fois) en entrée.



En revanche, nous ne parlerons pas de \CodeInline{real} VS \CodeInline{float/double}, parce que
cela induirait que nous parlions de preuve de programmes avec du calcul en virgule
flottante et que nous n'en parlerons pas ici. Par contre, ce tutoriel en parle :
\externalLink{Introduction à l'arithmétique flottante}{https://zestedesavoir.com/tutoriels/570/introduction-a-larithmetique-flottante/}.



\levelTwoTitle{Predicates}

A predicate is a property about different objects that can be true or
false. To sum up, we are writing predicates from the beginning of this
tutorial in precondition, postcondition, assertion and loop invariant.
ACSL allows us to name these predicates, as we could do for a boolean
function in C, for example. An important difference, however, is that
predicates (as well as logic functions that we will see later) must be pure.
For example, they cannot produce side effects by modifying a pointed
value.

These predicates can receive some parameters. Moreover, they can also
receive some C labels that will allow us to establish relations between
different program points.



\levelThreeTitle{Syntax}


Predicates are introduced using ACSL annotations. The syntax is the
following:



\begin{CodeBlock}{c}
/*@
  predicate named_predicate { Lbl0, ..., LblN }(type0 arg0, ..., typeN argN) =
    //a logic relations between all these things
*/
\end{CodeBlock}



For example, we can define the predicate that checks whether an integer
in memory is changed between two particular program points:



\CodeBlockInput[1][4]{c}{unchanged-loc.c}



\begin{Warning}
  Keep in mind that passing a value to a predicate is done, as it is done in C,
  by value. We cannot write this predicate by directly passing \texttt{i} in
  parameter:

\begin{CodeBlock}{c}
/*@
  predicate unchanged{L0, L1}(int i) =
    \at(i, L0) == \at(i, L1);
 */
\end{CodeBlock}

  Since \texttt{i} is just a copy of the received variable.
\end{Warning}


We can verify this code using our predicate:



\CodeBlockInput[6][15]{c}{unchanged-loc.c}



We can also have a look at the verification conditions generated by WP and
notice that, even it is slightly (syntactically) modified, the predicate is not
unrolled by WP. The provers will determine themselves whether they need to use
the definition of the predicate to establish the proof.



\image{unchanged-loc}



As we said earlier, one important use of predicates (and logic
functions) is to make our specifications more readable and to factor it.
An example can be to write a predicate that expresses the validity of an
array in reading or writing. It allows us to avoid writing the complete
expression every time we need it and to make it readable quickly:



\CodeBlockInput[3]{c}{search.c}



In this specification, we do not give an explicit label to predicates
for their definition, nor for their use. For the definition, Frama-C
automatically creates an implicit label. At predicate use, the given
label is implicitly \texttt{Here}. The fact we do not explicitly define
the label in the definition of a predicate does not forbid to explicitly
give a label when we use it.

Of course, predicates can be defined in header files in order to produce
a utility library for specification for example.



\levelFourTitle{Predicate overloading}



It is possible to overload predicates as long as the types of the
parameters are different or the number of parameters changes. For
example, we can redefine the \CodeInline{valid\_range\_r} as a
predicate that takes in parameters both the beginning and the end
of the range to consider. Then, we can write a overloaded version that
uses the previous one for the particular case of ranges that starts
at 0:



\CodeBlockInput[3]{c}{search-overload.c}




\levelThreeTitle{Abstraction}


An other important use of predicates is to define the logical state of
our data structures when programs start to be more complex. Our data
structures must usually respect an invariant (again) that each
manipulation function must maintain in order to ensure that the data
structure will always remain coherent and usable through future calls.



It allows us to ease the reading of specifications. For example, we can
define the specification required to ensure the safety of a fixed size
stack. It could be done as illustrated here (note that we do not
provide the definition of the predicates as it is not the purpose of
our example, the careful reader could consider this as an exercise):



\CodeBlockInput{c}{stack.c}



Here, the specification does not express functional properties. For
example, we do not specify that when we perform the push of a value, and
then we ask for the top of the stack, we get the same value. But we
already have enough details to ensure that, even if we cannot prove that
we always get the right result (behaviors such as ``if I push \(v\), top
returns \(v\)''), we can still guarantee that we do not produce runtime
errors (if we provide correct predicates for the stack, and prove
that the implementation of our functions ensures that no runtime errors
can occur).


\levelThreeTitle{Exercises}


\levelFourTitle{Days of the month}


Taking back the solution of the
exercise~\ref{l4:contract-modularity-ex-days-of-month} about days of the month,
write a predicate to express that a year is leap and adapt the contracts using
it.


\levelFourTitle{Alpha-numeric character}


Taking back the solution of the
exercise~\ref{l4:contract-modularity-ex-alpha-num} about alpha numeric
characters, write predicates to express that a character is an upper letter,
lower letter, and a digit. Adapt the contracts of the different functions using
them.


\levelFourTitle{Max of 3 values}


The following function returns the max of 3 input values:


\CodeBlockInput{c}{ex-3-max-3.c}


Write a predicate to express that a value is one of three pointed values at a
given memory state:

\begin{CodeBlock}{c}
/*@
  predicate one_of{L}(int value, int *a, int *b, int *c) =
    // ...
*/
\end{CodeBlock}

Use the set notation. Write a contract to the function and prove that it is
verified.


\levelFourTitle{Binary Search}
\label{l4:acsl-properties-predicates-ex-bsearch}


Taking back the solution of the
exercise~\ref{l4:statements-loops-ex-bsearch} about the binary search function
with unsigned types, write a predicate that expresses that an array is sorted on
a range of values starting at \CodeInline{begin} and ending at \CodeInline{end}
(excluded). Overload this predicate in order to make \CodeInline{begin} optional
with a default value of $0$. Define a predicate that checks if an element is in
a range of values of an array starting at index \CodeInline{begin} and ending at
\CodeInline{end} (excluded), again overload this predicate to make the first
bound optional.

Use those two predicates to simplify the contract of the function. Note that
both behaviors \CodeInline{assumes} clause should be modified.


\levelFourTitle{Search and replace}



Taking back the example~\ref{l4:statements-loops-ex-search-and-replace}, about
the search and replace function, write predicates that express that in some range
of an array starting at index \CodeInline{begin} and ending at \CodeInline{end}
(excluded), values

\begin{itemize}
\item remain unchanged between two labels,
\item are replaced with some new value when it equals to some old value, then
  left unchanged
\end{itemize}

Overload both predicates to make the first bound optional. Use the obtained
predicates to simplify the contract and loop invariant of the function.



\levelTwoTitle{Logic functions}

Logic functions are meant to describe functions that can only be used in
specifications. It allows us, first, to factor those specifications and,
second, to define some operations on \texttt{integer} or \texttt{real}
with the guarantee that they cannot overflow since they are mathematical
types.

Like predicates, they can receive different labels and values in
parameter.



\levelThreeTitle{Syntax}


To define a logic function, the syntax is the following:



\begin{CodeBlock}{c}
/*@
  logic return_type my_function\{ Label0, ..., LabelN \}( type0 arg0, ..., typeN argN ) =
    formula using the arguments ;
*/
\end{CodeBlock}



We can for example define a mathematical \externalLink{linear function}{https://en.wikipedia.org/wiki/Linear_function_(calculus)} using a logic function:



\CodeBlockInput[1][4]{c}{linear-0.c}



And it can be used to prove the source code of the following function:



\CodeBlockInput[6][12]{c}{linear-0.c}



\image{linear-1}


This code is indeed proved but some runtime-errors seems to be possible.
We can again define some mathematical logic function that will provide the bounds of
the linear function according to the machine type we use (from a logic point of
view). It allows us to then add our bounds checking in the precondition of the
function.



\CodeBlockInput[8][27]{c}{linear-1.c}



\begin{Information}
  Note that, as in specifications, computations are done using mathematical
  integers. We then do not need to care about some overflow risk with the
  computation of \texttt{INT\_MIN-b} or \texttt{INT\_MAX-b}.
\end{Information}


Once this specification is provided, everything is fine. Of course, we
could manually provide these bounds every time we create a linear logic
function. But, by creating these bound computation functions, we directly
get a way to compute them automatically which is quite comfortable.



\levelThreeTitle{Recursive functions and limits of logic functions}


Logic functions (as well as predicates) can be recursively defined.
However, such an approach will rapidly show some limits in their use for
program proof. Indeed, when the automatic solver reasons on such logic
properties, if such a function is met, it will be necessary to evaluate it.
SMT solvers are not meant to be efficient for this task, which will generally
be costly, producing too long proof resolution and eventually timeouts.

We can have a concrete example with the factorial function, using logic
and using C language:



\CodeBlockInput{c}{facto-0.c}



Without checking overflows, this function is easy and fast to prove. If
we add runtime error checking, we see that there is a possibility of
overflow on the multiplication.



On \CodeInline{int}, the maximum value for which we can compute
factorial is 12. If we go further, it overflows. We can then add this
precondition:



\CodeBlockInput[5][10]{c}{facto-1.c}



If we ask for a proof on this input, Alt-ergo will probably fail,
whereas Z3 can compute the proof in less than a second. The reason is
that in this case, the heuristics that are used by Z3 consider that it
is a good idea to spend a bit more time on the evaluation of the
function. We can for example change the maximum value of \texttt{n} to
see how the different provers behave. With an \texttt{n} fixed to 9,
Alt-ergo produces a proof in less than 10 seconds, whereas with a value
of 10, even a minute is not enough.



Logic functions can then be defined recursively but without some more
help, we are rapidly limited by the fact that provers will need to
perform evaluation or to ``reason'' by induction, two tasks for which
they are not efficient. This will limit our possibilities for program
proofs.



\levelTwoTitle{Lemmas}


Les lemmes sont des propriétés générales à propos des prédicats ou encore des
fonctions. Une fois ces propriétés exprimées, la preuve peut être réalisée en
isolation du reste de la preuve du programme, en utilisant des prouveurs
automatiques ou (plus souvent) des prouveurs interactifs. Une fois la preuve
réalisée, la propriété énoncée peut être utilisée directement par les prouveurs
automatiques sans que cela ne nécessite d'en réaliser la preuve à nouveau. Par
exemple, si nous énonçons un lemme $L$ qui dit que $P \Rightarrow Q$, et dans
une autre preuve nous avons besoin de prouver $Q$ alors que nous savons déjà
que $P$ est vérifiée, nous pouvons utiliser directement le lemme $L$ pour
conclure sans avoir besoin de faire à nouveau le raisonnement complet qui
amène de $P$ à $Q$.



Dans la section précédent, nous avons dit que les fonctions récursives logiques
peuvent rendre les preuves plus difficile pour les solveurs SMT. Dans un tel cas,
les lemmes peuvent nous aider. Nous pouvons écrire nous même les preuves qui
nécessitent de raisonner par induction pour certaines propriétés que nous
énonçons comme des lemmes, et ces lemmes peuvent ensuite être utilisés
efficacement par les prouveurs pour effectuer les autres preuves à propos du
programme.


\levelThreeTitle{Syntaxe}


Une nouvelle fois, nous les introduisons à l'aide d'annotations ACSL. La syntaxe
utilisée est la suivante :



\begin{CodeBlock}{c}
/*@
  lemma name_of_the_lemma { Label0, ..., LabelN }:
    property ;
*/
\end{CodeBlock}



Cette fois les propriétés que nous voulons exprimer ne dépendent pas de
paramètres reçus (hors de nos \textit{labels} bien sûr). Ces propriétés seront donc
exprimées sur des variables quantifiées. Par exemple, nous pouvons poser ce
lemme qui est vrai, même s'il est trivial :



\begin{CodeBlock}{c}
/*@
  lemma lt_plus_lt:
    \forall integer i, j ; i < j ==> i+1 < j+1;
*/
\end{CodeBlock}



Cette preuve peut être effectuée en utilisant WP. La propriété est bien sûr
trivialement prouvée par Qed.



\levelThreeTitle{Exemple : propriété fonction affine}


Nous pouvons par exemple reprendre nos fonctions affines et exprimer quelques
propriétés intéressantes à leur sujet :



\CodeBlockInput[8][18]{c}{linear-0.c}



Pour ces preuves, il est fort possible qu'Alt-ergo ne parvienne pas à les
décharger. Dans ce cas, le prouveur Z3 devrait, lui, y arriver. Nous pouvons
ensuite construire cet exemple de code :



\CodeBlockInput[20][48]{c}{linear-0.c}



Si nous ne renseignons pas les lemmes mentionnés plus tôt, il y a peu de chances
qu'Alt-ergo réussisse à produire la preuve que \CodeInline{fmin} est inférieur à \CodeInline{fmax}.
Avec ces lemmes présents en revanche, il y parvient sans problème car cette
propriété est une simple instance du lemme \CodeInline{ax\_b\_monotonic\_pos}, la preuve
étant ainsi triviale car notre lemme nous énonce cette propriété comme étant vraie.
Notons que sur cette version généralisée, Z3 sera probablement plus efficace pour
prouver l'absence d'erreurs à l'exécution.



\levelThreeTitle{Exemple: tableaux et labels}


Plus tard dans ce tutoriel, nous verrons certains types de définitions à propos
desquels il est parfois difficile de raisonner pour les solveurs SMT quand des
modifications ont lieu en mémoire. Par conséquent, nous aurons souvent besoin de
lemmes pour indiquer les relations qui existent à propos du contenu de la mémoire
entre deux labels.


Pour le moment, illustrons cela avec un exemple simple. Considérons les deux
prédicats suivant :


\CodeBlockInput[1][8]{c}{unchanged-sorted.c}


Nous pourrions par exemple vouloir énoncer que lorsqu'un tableau est trié, et que
la mémoire est modifiée (créant donc un nouvel état mémoire), mais que le contenu du
tableau reste inchangé, alors le tableau est toujours trié. Cela peut être réalisé
avec le lemme suivant :


\CodeBlockInput[10][16]{c}{unchanged-sorted.c}


Nous énonçons ce lemme pour deux labels \CodeInline{L1} et \CodeInline{L2}, et
exprimons que pour toute plage de valeurs dans un tableau, si elle est triée au label
\CodeInline{L1}, et reste inchangée depuis \CodeInline{L1} vers \CodeInline{L2},
alors elle reste triée au label \CodeInline{L2}.


Notons qu'ici, cette propriété est facilement prouvée par les prouveurs SMT. Nous
verrons plus tard des exemples pour lesquels il n'est pas si simple d'obtenir une
preuve.



\levelThreeTitle{Exercices}


\levelFourTitle{Propriété de la multiplication}


Écrire un lemme qui énonce que pour trois entiers $x$, $y$ et $z$, si $x$ est
plus grand ou égal à $0$, si $z$ est plus grand ou égal à $y$, alors $x * z$
est  plus grand ou égal à $x * y$.


Ce lemme ne sera probablement pas prouvé par les solveurs SMT. En revanche, en
demandant une preuve avec Coq, la tactique par défaut devrait la décharger
automatiquement.


\levelFourTitle{Localement trié vers globalement trié}
\label{l4:acsl-properties-lemmas-lsorted-gsorted}


Le programme suivant contient une fonction qui demande à ce qu'un tableau soit
trié au sens que chaque élément soit plus petit ou égal à l'élément qui le suit
puis appelle la fonction de recherche dichotomique.


\CodeBlockInput[59][81]{c}{ex-2-sorted-link.c}


Pour cet exercice, reprendre la solution de
l'exercice~\ref{l4:acsl-properties-predicates-ex-bsearch} à propos de la recherche
dichotomique. La précondition de cette recherche peut sembler plus forte que celle
reçue par la précondition de de \CodeInline{bsearch\_callee}. La première demande
chaque paire d'éléments d'être ordonnée, la seconde simplement que chaque élément
soit inférieur à celui qui le suit. Cependant, la seconde implique la première.
Écrire un un lemme qui énonce que si \CodeInline{element\_level\_sorted} est vraie
pour un tableau, \CodeInline{sorted} est vraie aussi. Ce lemme ne sera probablement
pas prouvé par un solveur SMT, toutes les autres propriétés devraient être prouvées
automatiquement.


Une solution et la preuve Coq du lemme sont disponibles sur le GitHub de ce tutoriel.


\levelFourTitle{Somme des N premiers entiers}
\label{l4:acsl-properties-lemmas-n-first-ints}


Reprendre la solution de l'exercice~\ref{l4:acsl-properties-functions-n-first-ints}
à propos de la somme des N premiers entiers. Écrire un lemme qui énonce que la valeur
calculée par la fonction logique récursive qui permet la spécification de la somme des
N premiers entiers est $n*(n+1)/2$. Ce lemme ne sera pas prouvé par un solveur SMT.


Une solution et la preuve Coq du lemme sont disponibles sur le GitHub de ce tutoriel.


\levelFourTitle{Transitivité d'un glissement d'éléments}
\label{l4:acsl-properties-lemmas-shift-trans}


Le programme suivant est composé de deux fonctions. La première est
\CodeInline{shift\_array} et permet de faire glisser des éléments dans un tableau
d'un certain nombre de cellules (nommé \CodeInline{shift}). La seconde effectue deux
glissements successifs des éléments d'un même tableau.



\CodeBlockInput{c}{ex-4-shift-transitivity.c}


Compléter les prédicats \CodeInline{shifted} et \CodeInline{unchanged}.
Le prédicat \CodeInline{shifted} doit utiliser \CodeInline{shifted\_cell}.
Le prédicat \CodeInline{unchanged} doit utiliser \CodeInline{shifted}.
Compléter le contrat de la fonction \CodeInline{shift\_array} et prouver
sa correction avec WP.


Exprimer deux lemmes à propos de la propriété \CodeInline{shifted}.


Le premier, nommé \CodeInline{shift\_ptr}, doit énoncer que déplacer une plage de
valeur \CodeInline{fst+s1} à \CodeInline{last+s1} dans un tableau \CodeInline{array}
d'un décalage \CodeInline{s2} est équivalent à déplacer une plage de valeurs
\CodeInline{fst} à \CodeInline{last} pour la position mémoire \CodeInline{array+s1}
avec un décalage \CodeInline{s2}.


Le second doit énoncer que quand les éléments d'un tableau sont déplacés une première
fois avec un décalage \CodeInline{s1} puis une seconde fois avec un décalage
\CodeInline{s2}, alors le déplacement final correspond à un décalage avec un
déplacement \CodeInline{s1+s2}.


Le lemme \CodeInline{shift\_ptr} ne sera probablement pas prouvé par un solveur SMT.
Nous fournissons une solution et la preuve Coq de ce lemme sur le GitHub de ce livre.
Les propriétés restantes doivent être prouvées automatiquement.


\levelFourTitle{Déplacement d'une plage triée}


Le programme suivant est composé de deux fonctions. La fonction
\CodeInline{shift\_and\_search} déplace les éléments d'un tableau puis effectue
une recherche dichotomique.


\CodeBlockInput{c}{ex-5-shift-sorted.c}


Reprendre la solution de la recherche dichotomique de
l'exercice~\ref{l4:acsl-properties-predicates-ex-bsearch}. Modifier cette
recherche et sa spécification de façon à ce que la fonction permette de
chercher dans toute plage triée de valeurs. La preuve doit toujours fonctionner.


Reprendre également la fonction prouvée \CodeInline{shift\_array} de l'exercice
précédent.


Compléter le contrat de la fonction \CodeInline{shift\_and\_search}. La
précondition qui demande à ce que le tableau soit trié avant la recherche ne
sera pas validée, ni la postcondition de l'appelant. Compléter le lemme
\CodeInline{shifted\_still\_sorted} qui doit énoncer que si une plage de valeur
est triée à un label, puis déplacée, alors elle reste triée. La précondition
devrait maintenant être validée. Ensuite, compléter le lemme
\CodeInline{in\_array\_shifted} qui doit énoncer que si une valeur est dans
une plage de valeur, alors lorsque cette plage est déplacée, la valeur est
toujours dans la nouvelle plage obtenue. La postcondition de l'appelant devrait
maintenant être prouvée.


Ces lemmes ne seront probablement pas prouvés par un solveur SMT. Une solution et
les preuves Coq sont disponibles sur le GitHub de ce livre.


\horizontalLine

In this part of the tutorial, we have seen different ACSL construct that
allow us to factor our specifications and to express general properties
that can be used by our solver to easy their work.




All techniques we have talk about are safe, since they do not \emph{a
priori} allow us to write false or contradictory specifications. At
least if the specification only use such logic constructions and if
every lemma, precondition (at call site), every postcondition,
assertion, variant and invariant is correctly proved, the code is
correct.




However, sometimes, such constructions are not enough to express all
properties we want to express about our programs. The next constructions
we will see give us some new possibilities about it, but it will be
necessary to be really careful using them since an error would allow us
to introduction false assumptions or silently modify the program we are
verifying.
