Lemmas are general properties about predicates or functions. Once these
properties are expressed, their proof can be performed (one time) and
the provers will then be able to use this result to perform other proofs
without requiring to repeat all steps needed to perform the
original proof if it appears in a much longer proof about another
property.




For example, lemmas allow us to express properties about recursive
functions in order to get easier proofs when we are interested in
proving properties that use such functions.



\levelThreeTitle{Syntax}


Again, we introduce lemmas using ACSL annotations. The syntax is
following:



\begin{CodeBlock}{c}
/*@
  lemma name_of_the_lemma { Label0, ..., LabelN }:
    property ;
*/
\end{CodeBlock}



This time, the properties we want to express do not depend on received
parameters (except for labels). So we will express these properties on
universally quantified variables. For example, we can state this lemma,
which is true, even if it is trivial:



\begin{CodeBlock}{c}
/*@
  lemma lt_plus_lt:
    \forall integer i, j ; i < j ==> i+1 < j+1;
*/
\end{CodeBlock}



This proof can be performed using WP. The property is, of course, proved
using only Qed.



\levelThreeTitle{Example: properties of linear functions}


We can come back to our linear functions and express some interesting
properties about them:



\CodeBlockInput[20][30]{c}{affine-0.c}



For these proofs, Alt-ergo, will probably not be able to discharge
generated goals. In this case, Z3 will certainly perform it. We can then
write some code examples:



\CodeBlockInput[32][57]{c}{affine-0.c}



If we do not give the lemmas provided earlier, Alt-ergo will not be able
to prove the proof that \CodeInline{fmin} is lesser or equal to
\CodeInline{fmax}. With the lemmas it is however very easy for it since the
property is the simply an instance of the lemma
\CodeInline{ax\_monotonic\_pos}, the proof is then trivial as our lemma is
considered to be true when are not currently proving it.


\levelThreeTitle{Exercises}



\levelFourTitle{Multiplication property}



Write a lemma that state that for three integers $x$, $y$ and $z$, if $x$ is
greater or equals to $0$, if $z$ is greater or equals to $y$, then $x * z$ is
greater or equals to $y * z$.


This lemma will not be proved by SMT solvers, however if you ask a proof with
Coq, the default tactic will probably discharged this verification condition
automatically.


\levelFourTitle{Locally sorted to globally sorted}


The following program contains a function that requires an array to be sorted
in the sense that each element is lower or equals to the element to follows it
and calls the binary search function.


\CodeBlockInput[59][81]{c}{ex-2-sorted-link.c}


Take back your proved binary search function from the
exercise~\ref{l4:acsl-properties-predicates-ex-bsearch}. You might notice that
the precondition of the binary search function is stronger than what we know in
precondition of the \CodeInline{bsearch\_callee}. However, our precondition
implies that the array is globally sorted. Write a lemma that states that if
an array is \CodeInline{element\_level\_sorted} then it is \CodeInline{sorted}.
This lemma will probably not be proved by SMT solvers, all remaining properties
should be.

We provide a solution and the corresponding Coq proof on the GitHub repository
of this book.


\levelFourTitle{Sum of the N first ints}


Take back your solution to the
exercise~\ref{l4:acsl-properties-functions-n-first-ints} about the sum of the
N first elements. Write a lemma that states that the result of the call to the
logic function is $n*(n+1)/2$. This lemma will not be proved by SMT solvers.


We provide a solution and the corresponding Coq proof on the GitHub repository
of this book.


\levelFourTitle{Shift transitivity}


The following program is composed of two functions. The first one is the
\CodeInline{shift\_array} function that shifts the elements of an array with a
given offset (named \CodeInline{shift}). The second performs two successive
shifts on the same array.


\CodeBlockInput{c}{ex-4-shift-transitivity.c}


Complete the predicates \CodeInline{shifted} and \CodeInline{unchanged} (the
last should use the former). Complete the contract of the
\CodeInline{shift\_array} function and prove it using WP.


Express two lemmas about the \CodeInline{shifted} property.


The first one \CodeInline{shift\_ptr} should state that shifting a range
\CodeInline{fst+s1} to \CodeInline{last+s1} of an \CodeInline{array} with an
offset \CodeInline{s2} is equivalent to shifting the range \CodeInline{fst} to
\CodeInline{last} of the memory location \CodeInline{array+s1} with an offset
\CodeInline{s2}.



The second one should state that when the elements of an array are shifted a
first time with an offset \CodeInline{s1} and then a second time with an offset
\CodeInline{s2}, then the complete shifts corresponds to a single shift with an
offset \CodeInline{s1+s2}.



The lemma \CodeInline{shift\_ptr} will not be proved by SMT solvers, we provide
a solution and the corresponding Coq proof on the GitHub repository of this
book. All remaining properties should be proved automatically.



\levelFourTitle{Shift sorted range}


In the following program is composed of two functions. The function
\CodeInline{shift\_and\_search} shifts the element of an array and then
performs a binary search.


\CodeBlockInput{c}{ex-5-shift-sorted.c}


Take back your proved binary search function from the
exercise~\ref{l4:acsl-properties-predicates-ex-bsearch}, modify the binary
search function, its contract and its proof in order to be able to search in
any range.


Use the \CodeInline{shift\_array} function proved in the previous exercise.


Complete the contract of the function \CodeInline{shift\_and\_search}. You might
notice that the precondition that requires the array to be sorted is not
validated. Complete the lemma \CodeInline{shifted\_still\_sorted} that should
state that if a range is sorted at some label and then shifted, the resulting
range is still sorted.

The lemma does not have to be proved by WP. We provide a solution and the
corresponding Coq proof on the GitHub repository of this book.
