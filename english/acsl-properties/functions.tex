Logic functions are meant to describe functions that can only be used in
specifications. It allows us, first, to factor those specifications and,
second, to define some operations on \texttt{integer} or \texttt{real}
with the guarantee that they cannot overflow since they are mathematical
types.

Like predicates, they can receive different labels and values in
parameter.



\levelThreeTitle{Syntax}


To define a logic function, the syntax is the following:



\begin{CodeBlock}{c}
/*@
  logic return_type my_function\{ Label0, ..., LabelN \}( type0 arg0, ..., typeN argN ) =
    formula using the arguments ;
*/
\end{CodeBlock}



We can for example define a mathematical \externalLink{linear function}{https://en.wikipedia.org/wiki/Linear_function_(calculus)} using a logic function:



\CodeBlockInput[1][4]{c}{linear-0.c}



And it can be used to prove the source code of the following function:



\CodeBlockInput[6][12]{c}{linear-0.c}



\image{linear-1}


This code is indeed proved but some runtime-errors seems to be possible.
We can again define some mathematical logic function that will provide the bounds of
the linear function according to the machine type we use (from a logic point of
view). It allows us to then add our bounds checking in the precondition of the
function.



\CodeBlockInput[8][27]{c}{linear-1.c}



\begin{Information}
  Note that, as in specifications, computations are done using mathematical
  integers. We then do not need to care about some overflow risk with the
  computation of \texttt{INT\_MIN-b} or \texttt{INT\_MAX-b}.
\end{Information}


Once this specification is provided, everything is fine. Of course, we
could manually provide these bounds every time we create a linear logic
function. But, by creating these bound computation functions, we directly
get a way to compute them automatically which is quite comfortable.



\levelThreeTitle{Recursive functions and limits of logic functions}


Logic functions (as well as predicates) can be recursively defined.
However, such an approach will rapidly show some limits in their use for
program proof. Indeed, when the automatic solver reasons on such logic
properties, if such a function is met, it will be necessary to evaluate it.
SMT solvers are not meant to be efficient for this task, which will generally
be costly, producing too long proof resolution and eventually timeouts.

We can have a concrete example with the factorial function, using logic
and using C language:



\CodeBlockInput{c}{facto-0.c}



Without checking overflows, this function is easy and fast to prove. If
we add runtime error checking, we see that there is a possibility of
overflow on the multiplication.



On \CodeInline{int}, the maximum value for which we can compute
factorial is 12. If we go further, it overflows. We can then add this
precondition:



\CodeBlockInput[5][10]{c}{facto-1.c}



If we ask for a proof on this input, Alt-ergo will probably fail,
whereas Z3 can compute the proof in less than a second. The reason is
that in this case, the heuristics that are used by Z3 consider that it
is a good idea to spend a bit more time on the evaluation of the
function. We can for example change the maximum value of \texttt{n} to
see how the different provers behave. With an \texttt{n} fixed to 9,
Alt-ergo produces a proof in less than 10 seconds, whereas with a value
of 10, even a minute is not enough.



Logic functions can then be defined recursively but without some more
help, we are rapidly limited by the fact that provers will need to
perform evaluation or to ``reason'' by induction, two tasks for which
they are not efficient. This will limit our possibilities for program
proofs.


\levelThreeTitle{Exercises}



\levelFourTitle{Distance}


Specify and prove the following program:



\CodeBlockInput{c}{ex-1-distance.c}



For this, define two logic functions \CodeInline{abs} and \CodeInline{distance}.
Use these functions to write the specification of the function.



\levelFourTitle{Square}


Write the body of the \CodeInline{square} function. Specify and prove the
program. Use a \CodeInline{square} logic function.


\CodeBlockInput{c}{ex-2-square.c}


Take care of the types of the variables and do not over-constrain the input of
the function. Furthermore, when verifying the absence of runtime errors, do not
forget to provide the option \CodeInline{-warn-unsigned-overflow}.


\levelFourTitle{Iota}


Here is a possible implementation of the iota function:


\CodeBlockInput{c}{ex-3-iota.c}


Write a logic function that returns the input value increased by one. Prove
that after the execution of \CodeInline{iota}, the first value of the array is
the input value and that each value of the array corresponds to the value that
precedes it increased by one (using the previously defined logic function).



\levelFourTitle{Vector add}



In the following program, the \CodeInline{vec\_add} function adds the second
vector in input into the first one. Write a contrat for the function
\CodeInline{show\_the\_difference} that expresses, for each value of the vector
\CodeInline{v1} the difference between the pre and the postcondition. For this,
define a logic function \CodeInline{diff} that returns the difference between
the value of a memory location at a label \CodeInline{L1} and the value at a
label \CodeInline{L2}.


\CodeBlockInput{c}{ex-4-vec-inc.c}



Re-express the \CodeInline{unchanged} predicate using the logic function you
have defined.



\levelFourTitle{The sum of the N first ints}
\label{l4:acsl-properties-functions-n-first-ints}


The following function compute the sum of the N first integers. Write a
recursive logic functions that returns the sum of the N first integers and
write a specification for the C function that show that it computes the same
value as provided by the logic function.


\CodeBlockInput{c}{ex-5-n-first-ints.c}


Try to verify the absence of runtime errors. The integer overflow is not so
simple to get rid of. However, write a precondition that should be enough to
prove the function (remember that the sum of the N first integers can be
expressed with a really simple formula ...). It will certainly not be enough
to directly prove the absence overflow, but we will see how to provide such
an information in the next section.
