ACSL provides two types that will allow us to write properties or
functions without having to think about constraints due to the size of
the representation of primitive C types in memory. These types are
\CodeInline{integer} and \CodeInline{real}, which respectively represent
mathematical integers and reals (that are modeled to be as close the
reality we can, but this notion cannot be perfectly handled).

From now, we will often use integers instead of classical C
\CodeInline{int}s. The reason is simply that a lot of properties and
definitions are true regardless the size of the machine integer we have
as input.

On the other hand, we will not talk about the differences that exist
between \CodeInline{real} and \CodeInline{float/double}. It would require to
speak about precise numerical calculus, and about proofs of programs
that rely on such calculus which could deserve an entire dedicated
tutorial.
