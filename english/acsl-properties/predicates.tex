A predicate is a property about different objects that can be true or
false. To sum up, we are writing predicates from the beginning of this
tutorial in precondition, postcondition, assertion and loop invariant.
ACSL allows us to name these predicates, as we could do for a boolean
function in C, for example. An important difference, however, is that
predicates (as well as functions that we will see later) must be pure,
they cannot produce side effects by modifying a pointed value for
example.

These predicates can receive some parameters. Moreover, they can also
receive some C labels that will allow us to establish relations between
different program points.



\levelThreeTitle{Syntax}


Predicates are introduced using ACSL annotations. The syntax is the
following:



\begin{CodeBlock}{c}
/*@
  predicate named_predicate { Label0, Label1, ..., LabelN }(type0 arg0, type1 arg1, ..., typeN argN) =
    //a logic relations between all these things
*/
\end{CodeBlock}



For example, we can define the predicate that checks whether an integer
in memory is changed between to particular program points:



\begin{CodeBlock}{c}
/*@
  predicate unchanged{L0, L1}(int* i) =
    \at(*i, L0) == \at(*i, L1);
*/
\end{CodeBlock}



\begin{Warning}
  Keep in mind that passing a value to a predicate is done, as it is done in C,
  by value. We cannot write this predicate by directly passing \texttt{i} in
  parameter:

\begin{CodeBlock}{c}
/*@
  predicate unchanged{L0, L1}(int i) =
    \at(i, L0) == \at(i, L1);
 */
\end{CodeBlock}

  Since \texttt{i} is just a copy of the received variable.
\end{Warning}


We can verify this code using our predicate:



\begin{CodeBlock}{c}
int main(){
  int i = 13;
  int j = 37;

 Begin:
  i = 23;
 
  //@assert ! unchanged{Begin, Here}(&i);
  //@assert   unchanged{Begin, Here}(&j);
}
\end{CodeBlock}



We can also have a look to the goals generated by WP and notice that,
even it is slightly (syntactically) modified, the predicate is not
unrolled by WP. The provers will determine if they need to use the
definition of the predicate.

As we said earlier, one important use of predicates (and logic
functions) is to make our specifications more readable and to factor it.
An example can be to write a predicate that expresses the validity of an
array in read or write. It allows us to avoid writing the complete
expression every time we need it and to make it readable quickly:



\CodeBlockInput{c}{search.c}



In these specification, we do not give an explicit label to predicates
for their definition, nor for their use. For the definition, Frama-C
automatically creates an implicit label. At predicate use, the given
label is implicitly \texttt{Here}. The fact we do not explicitly define
the label in the definition of a predicate does not forbid to explicitly
give a label when we use it.

Of course, predicates can be defined in header files in order to produce
an utility library for specification for example.


\levelThreeTitle{Abstraction}


An other important use of predicates is to define the logical state of
our data structures when programs start to be more complex. Our data
structures must usually respect an invariant (again) that each
manipulation function must maintain in order to ensure that the data
structure will always remain coherent and usable through future calls.



It allows us to ease the reading of specifications. For example, we can
define the specification required to ensure the safety of a fixed size
stack. It could be done as illustrated there:



\CodeBlockInput{c}{stack.c}



Here, the specification does not express functional properties. For
example, we do not specify that when we perform the push of a value, and
then we ask for the top of the stack, we get the same value. But we
already have enough details to ensure that, even if we cannot prove that
we always get the right result (behaviors such as ``if I push \(v\), top
returns \(v\)''), we can still guarantee that we do not produce runtime
errors (if we provide correct predicates for the stack, and to prove
that the implementation of our functions ensures that no runtime error
can occur).
