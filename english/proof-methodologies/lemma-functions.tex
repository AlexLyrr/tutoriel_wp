Assertions provide a way to give clues to the verification condition generator
so the SMT solvers will get enough information to make the proofs we
need. However, it is sometimes hard to write the assertion that will
create exactly the property needed by the SMT solver to trigger the right lemma
(for example, since the generator makes some optimization on the verification
condition that might slightly modify it and the context of the proof).
Furthermore, we rely on lemmas that often need to be proved with the Coq proof
assistant, and that means that we need to learn Coq.



In this section, we will see some techniques that we can use to make all of this
more predictable and does not require from us to use the Coq proof assistant.
While these techniques cannot be used in any case (and we will explain what are
the cases when it is not applicable), they are quite efficient to get almost
full automatic proof. This relies on ghost code.



\levelThreeTitle{Proof by induction}



Previously, we mentioned that SMT solvers are bad at reasoning by induction
(most of them), and this is the reason why we often need to express lemmas that
we then prove using the Coq proof assistant that allows us to write our proof
by induction. However, in the section~\ref{l2:statements-loops} about loops, we
find a subsection~\ref{l3:statements-loops-invariant} named ``Induction and
invariant'' where we explain how to prove that a loop does the right job ... by
induction. What is this sorcery?!




In fact, it is quite simple. When we prove a loop invariant by induction using
SMT solvers, they do not have to perform the reasoning by induction themselves.
The job of splitting the proof into two subproofs, one for the establishment of
the invariant (the base case of the proof), and one for the preservation (the
induction case) is performed by the verification condition generator. So when
the verification conditions are transmitted to the SMT solvers, this work is not
needed anymore.




How can we exploit this idea? We explained before that ghost code can be used to
provide more information than what is explicitly provided by the source code.
For this, we add some code (and possibly annotations about this code) that
allows to deduce more properties. Let us illustrate this with a simple example.
In a previous exercise (\ref{l4:acsl-properties-lemmas-lsorted-gsorted}), we
wanted to prove the following function call (we have excluded the postconditions
to shorten the example):



\CodeBlockInput{c}{ghost-code-usage-1.c}



For this, the solution that was asked in the exercise was to provide a lemma
that states that if a range is ``locally sorted'', meaning that each element
is greater or equals to the one that precedes it, then we can say that it is
``globally sorted'', that is to say for each pair of indices $i$ and $j$, if
$i \leq j$ then the $j^{th}$ of the array is greater or equals to the $i^{th}$
element. Then, the precondition could be proved by SMT solvers, but not the
lemma itself that requires a Coq proof. Can we do something using a loop?



The answer is yes. Before calling the function, we can build a proof that
shows that because the array is locally sorted, we can deduce that it is
globally sorted (which is basically a proof of the lemma we would need).
We want to prove that the range is globally sorted. To write this proof by hand,
we would proceed by induction on the size of the range. We have two cases.
First, the range is empty and the property trivially true. Second, let us suppose
that some range of size $i$ with $i < length$ ($length$ being the size of the
complete range), is globally sorted and show that if it is the case, then the
range of size $i+1$ is sorted. This is easy because, by our precondition, we
know that the $i^{th}$ element is greater than the $(i-1)^{th}$ element, that is
itself greater than all the preceding elements.




Now, how can we translate this into ghost code? We write a loop that goes from
$0$ (our base case), to the end of the range \CodeInline{len} and provide as an
invariant that the array is globally sorted from $0$ to the current step of the
loop. We also add some assertions to help the provers (namely the fact that the
current element is greater than the one that precedes it):



\CodeBlockInput[15][31]{c}{ghost-code-usage-2.c}



And we can see that all verification conditions are easily verified by SMT
solvers, without requiring to write a Coq proof or a lemma. The verification
conditions that are created respectively for the establishment and the preservation
of the invariant correspond to the two cases we needed to prove in our proof
by induction.


\image{ghost-code-base}

\image{ghost-code-ind}

This kind of code is called a proof carrying code: we have written some code
and annotations that carries a proof that some property we want to verify holds.



Note that here, as we can write quite a lot of ghost code, we augment our risk
to modify the behavior of the program by mistake. Most of the time we do not
need to modify any memory location, thus taking care to check (few) assigned
memory locations (for example, loop iterators), that we do not create a runtime
error (through RTE plugin) and that the ghost code terminates is enough to be
confident about the verification.



In this example, we had to write the ghost code directly in annotation of the
program, and that mean that if we have another call somewhere else in the code
with some similar precondition, we would have to do it again. Let us make this
easier by using lemma functions.



\levelThreeTitle{Lemma function}



The principle of lemma functions is basically the same as lemmas: from some
premises, we want to prove some conclusion. And once it has been done, we want
to use it at some other place to directly deduce the conclusion from the
premises without having to do the proof again, by instancing it with actual
values.



The way to do this is to use a function, using the \CodeInline{requires} to
express the premises of the lemma, and the \CodeInline{ensures} to express the
conclusion of the lemma. The universally quantified variables can either still
be quantified, or correspond to a parameter of the function. Namely, if a
variable is only bounded to premises, or only to conclusions, it can be
translated into a universally quantified variable, provided that it is not
necessary to bind its value to a C variable in our proof carrying code (a
quantified variable is not visible from the C code). If it is bounded to both
premises and conclusion, it must be a parameter of the function (as we cannot
quantify a variable for all a function contract in ACSL).



Let us first consider an example where we do not use (directly) universally
quantified variables in the contract, with our previous example about sorted
values. From the property \CodeInline{element\_level\_sorted(arr, len)}, we
want to deduce \CodeInline{sorted(arr, len)}. The corresponding lemma could be:


\begin{CodeBlock}{c}
/*@
  lemma element_level_sorted_is_sorted:
    \forall int* arr, integer len ;
       element_level_sorted(arr, len) ==> sorted(arr, len) ;
*/
\end{CodeBlock}


So let us write a function that takes two parameters: \CodeInline{arr} and
\CodeInline{len}, requires that the array is locally sorted and ensures that it
is globally sorted:



\CodeBlockInput[16][21]{c}{lemma-function-1-1.c}



Note that this function must assign \CodeInline{\textbackslash{}nothing}, indeed
we use it to deduce some properties about the program, with ghost code, and
thus it should not modify the content of the array, else the ghost code would
modify the behavior of the program. Now let us provide a body to this function,
the proof carrying code that guarantees that the conclusion is verified, provided
that the precondition holds. It corresponds to the code we previously wrote to
prove the precondition of the call to \CodeInline{bsearch}:



\CodeBlockInput[16][31]{c}{lemma-function-1-2.c}



With this specified loop, we get an inductive proof that the lemma holds,
now we can use this lemma function by simply calling it when we need
to perform this deduction:



\CodeBlockInput[34][40]{c}{lemma-function-1-2.c}



Which asks us to establish the premises thanks to the precondition of the lemma
function (and which we trivially get from the precondition of the
\CodeInline{bsearch\_callee} function), and provides us the conclusion for free
thanks to the postcondition of the lemma function (and we can use it as a
precondition for the call to \CodeInline{bsearch}).



As we explained, when universally quantified variables are bounded to both
conclusion and premises, they must be parameters, and it is the case here for
the variables \CodeInline{arr} and \CodeInline{len}. Whereas the quantified
variable that are used in the predicates:



\CodeBlockInput[4][8]{c}{lemma-function-1-2.c}



as they are only bound to respectively the premise and the conclusion remain
universally quantified (even if it is hidden by the predicate). We could for
example have written the contract like this:



\CodeBlockInput[16][31]{c}{lemma-function-1-3.c}



where we perfectly see that variables are still universally quantified. However,
we are not forced to maintain them universally quantified, and we could
perfectly translate them into parameters (provided that the conclusion we want
to get from the premises still makes sense). Let us for example translate the
\CodeInline{i} and \CodeInline{j} of the conclusion into parameters:



\CodeBlockInput[34][41]{c}{lemma-function-1-3.c}



Which is also perfectly fine and we could for example use this function to
deduce some properties about the content of the array. Note that here, we use a
call to the previous lemma function to make the proof easier. We even can go
further by transferring the ``premise of our conclusion'' as another premise of
a new lemma:



\CodeBlockInput[44][52]{c}{lemma-function-1-3.c}



All these lemmas state the same global relation, the difference is related to
the amount of information that is required to instantiate them (and thus the
precision of the property that we get in return).



Finally, let us present a last usage of lemma functions. On all previous
examples, we have considered only universally quantified variable. In fact, what
we have said before is applicable to existentially quantified variables: if they
are bound to both premises and conclusions, they must be parameters, else they
can either be parameters or remain quantified. However, about existentially
quantified variables, we sometimes can go further by building a function that
directly provide a witness for an existentially quantified formula.



For example, let us consider the axiomatic definition for occurrence counting,
and imagine that at some point in a program, we want to prove the following
assertion from the precondition:



\CodeBlockInput[24][32]{c}{lemma-function-2-1.c}



Of course, there exists some index \CodeInline{n} such that \CodeInline{in[n]}
is \CodeInline{v}, else the number of occurrences of this value would be $0$.
But, instead of just proving that such an index exists, let us directly find
some index that respects the constraints on \CodeInline{n} by using a lemma
function that returns it:



\CodeBlockInput[24][42]{c}{lemma-function-2-2.c}



If we only look at the body of the function, it has two behaviors: either some
cell of the array contains \CodeInline{v} and the function returns its index, or
there is not, and then the function returns -1. The first behavior is easy to
show, the return statement is performed in a branch where we know that the
considered index corresponds to a cell that is in the range of the array and has
a value \CodeInline{v}.



We prove that the second behavior respects the postcondition by showing that it
leads to a contradiction. If there is no cell of value \CodeInline{v}, then
the number of occurrences of \CodeInline{v} is 0, this is expressed by the
second invariant that shows that as we have not met any \CodeInline{v} from
the beginning of the loop,
the number of occurrences is 0. However, the precondition of the
function states that the number of occurrences is more than $0$ which leads to
a contradiction that we model model by an assertion of false (note that this is
not necessary, we explicitly write it for our explanation) which means here that
this path is infeasible.



Finally, we can call this function to show that there exists some index that
allows our assertion to be validated:



\CodeBlockInput[44][53]{c}{lemma-function-2-2.c}



The use of lemma functions makes reasoning by induction feasible for lemmas
without the need of interactive proof. Furthermore, the triggerring of lemmas
becomes more predictable as we instantiate them by hand. However, while lemmas
can consider multiple labels:



\begin{CodeBlock}{c}
/*@
  lemma my_lemma{L1, L2}:  P{L1} ==> P{L2} ;
*/
\end{CodeBlock}



Lemma functions do not provide an equivalent mechanism as they are basically
normal C functions that cannot take labels in input. Let us show what we can
do when we need such a construct.



\levelThreeTitle{Lemma macro}



When we have to deal with multiple labels, the idea is to directly ``inject''
the proof carrying code at the place where it is needed exactly as we did at the
beginning of the section. However, we do not want to write this code by hand
everytime we need such a proof, so let us use macros to do it.



For now, let us translate our previous code into a macro instead of a function.
As we use this macro in ghost code (thus, in annotation) we have to take care to
use the ghost annotation syntax to write the invariant of the loop and the
assertions:



\CodeBlockInput[16][33]{c}{lemma-macro-1.c}



Instead of providing a pre and a postcondition, we state these properties using
assertions before and after the proof carrying code. The proof carrying code
itself is basically the same as before, and it is used exactly as it was used in
the case of functions. However, we can see that it makes an important difference
once it has been preprocessed by Frama-C as the block of code and annotations is
directly injected in the function \CodeInline{bsearch\_callee}.



\image{lemma-macro-1}



So in fact, we use the macro to generate the code we previously wrote. In this
case, it is not really interesting as a function call allows us to make things
more modular. So let us study a case where we do not have any other choice than
using a macro.



We illustrate using the following lemma:



\CodeBlockInput[4][11]{c}{lemma-macro-2-1.c}


In order to prove the following program:


\CodeBlockInput[13][29]{c}{lemma-macro-2-1.c}


Where the lemma \CodeInline{shift\_ptr} is necessary to prove the postcondition
of \CodeInline{callee} from the postcondition of \CodeInline{shift\_array}. Our
goal is of course to get rid of the lemma, replacing it by a lemma macro.



There is no precise guideline for designing a macro used from the injection of
proof carrying code. However, most lemmas stated about multiple labels are quite
similar in the way they relate labels. So let us illustrate with this example,
most of the time designing a macro in such a situation will more or less follow
the same scheme.



In order to build the macro, we need a context where we can work on it. We build
the context using a function, let us name this function
\CodeInline{context\_to\_prove\_shift\_ptr}. The idea is to use the function to
build the macro in isolation of the rest of the program to make the verification
of the property easier. However, while lemma functions are then called to deduce
some properties in some other function, this function will never be called, its
only role is to provide us a ``place'' where we can build our proof. In
particular, as we need multiple memory labels, our function \textbf{needs} to
modify the content of the memory (else, there is a single memory state for all
the function).



Let us illustrate with our current problem to make all of this clearer. First,
we create a macro \CodeInline{shift\_array} that will contain our proof carrying
code, for now let us just indicate that it is an empty statement. In the
parameters of this lemma, we take the labels that are considered. Note that the
rules we previously mentioned about quantified variables still apply to macros.


\begin{CodeBlock}{c}
#define shift_ptr(_L1, _L2, _arr, _fst, _last, _s1, _s2) ;
\end{CodeBlock}


Then we create our context function:


\CodeBlockInput[22][41]{c}{lemma-macro-2-2.c}


Let us decompose this code, starting from the context function. In input, we
receive all the variables of the lemma. We also state some properties about the
bounds of the integer values we consider, basically these should be requirements
that are not related to memory states, or related to the first one. Then, we
introduce the label
\CodeInline{L1} and we call the function \CodeInline{assign\_array} that leads
us to the label \CodeInline{L2}. The role of this call is to ensure that WP will
create a new memory label (thus, it will not consider that the memory is the
same), and to establish our premises. Indeed, if we have a look at the contract
of \CodeInline{assign\_array}, we see that it assigns the array (which
guarantees the creation of a new memory label) and in postcondition, it ensures
that the content of the array, between the pre and the postcondition (thus, when
we call it: \CodeInline{L1} and \CodeInline{L2}) respects the premise of our
lemma (which we repeat on line 36, by adding an assertion). Then we use our
\CodeInline{shift\_ptr} macro (that will later contain the proof carrying code),
and we then expect to be able to prove the postcondition of our lemma (line 40).



By doing this, we ensure that we built a context that only contains the
information we need to build the proof carrying code that allows us to deduce
the conclusion (line 40) from the premise (line 36). Now let us write the macro.



\CodeBlockInput[9][19]{c}{lemma-macro-2-2.c}



We will not detail this code which is quite similar to what we have written in
the beginning of this section. the only small subtlety is the assert that helps
the SMT solvers to relate the memory locations between \CodeInline{L1} and
\CodeInline{L2} together on lines 16--17. With this macro, we can see that the
assertion at the end of the function \CodeInline{context\_to\_prove\_shift\_ptr}
is correctly, proved. Thus, we expect the macro to help the provers to get a
similar conclusion in a similar context (that is to say, a context were we now
that \CodeInline{shifted} holds for some array between two memory labels).


Finally, we can complete the proof of our function \CodeInline{callee} by using
our lemma macro:



\CodeBlockInput[52][61]{c}{lemma-macro-2-2.c}



As one could notice, while this technique allows to inject the proof carrying
code with a single line of code, it can inject quite a lot of code and
annotations each time we use it. Furthermore, once we inject the code in the
location where we expect it to be actually useful, the corresponding context
can sometimes be already complex. Thus, we could need to slightly modify the
code of the macro in order to add more information that is unnecessary in a
clean context.


All of this can make the proof context bigger, and harder to use for SMT solvers.
There are other limitations to this technique and the careful reader might have
notice them. Let us now talk about it.



\levelThreeTitle{Limitations}



The main limitation of lemma functions and lemma macros is the fact that we are
limited to C types. For example, if we compare our lemma
\CodeInline{element\_level\_sorted\_is\_sorted} with its corresponding lemma
function, the original type for the variable \CodeInline{len} is a mathematical
integer while in the lemma function its type is \CodeInline{size\_t}. It means
that while the lemma is true for any integer, and so it could be used no matter
if in the program the type of the variable that represents the size is an
\CodeInline{int}, or an \CodeInline{unsigned} (or another integer type), on the
opposite, our lemma function can be used only if this type can be safely
converted to \CodeInline{size\_t}. However, this limitation is often not a
problem: we just have to express our specification for the biggest type we have
to consider in our program and most of the time, it will be enough. And if it is
not, we can for example duplicate the lemma for the types we are interested in.
Most of the time this limitation is not a big deal since during a
verification, we just tend to work with the same types as the program uses.



In some cases, however, it can constrain our modeling of some properties, and
it is mainly related to the logic types we can use to model some actual data
structures. For example, in order to model a linked list, one could use the
ACSL logic type \CodeInline{\textbackslash{}list<Type>}, and express an
inductive or axiomatic definition in order to define how an actual linked list
can be modeled by a logic list, thus we could have some lemmas about logic
lists. For example:

\begin{CodeBlock}{c}
/*@
  lemma in_list_in_sublist:
    \forall \list<int> l, l1, l2, int element ;
      l == (l1 ^ l2) ==>      // Here, ^ denotes lists concatenation
      (in_list(element, l) <==> (in_list(element, l1) || in_list(element, l2))) ;
*/
\end{CodeBlock}

We cannot write lemma functions with proof carrying code for this property as
we have no way to use this type in C code, and thus, no way to write a loop and
an invariant that would allow us to prove this property.


The other limitation is related to lemma macros and what we already mentioned
in the previous part about assertions. By adding too many assertions, the proof
context can become too big and complex, thus hard to manipulate for SMT
solvers. Using lemma macros that can generate quite a lot code and annotations
can lead to bigger proof contexts, thus it should be used with care.


Finally, depending on the property to prove, it can be hard to find a proof
carrying code. Indeed, proof assistants like Coq have been designed to be able
to express proofs even for complex properties, mainly relying on an high level
view of our problems, while C has been designed to write programs, and with
really detailed low level view of our problems. Thus, it can be sometimes
difficult to write a C program to handle some properties and even more to find
a suitable invariant for the loops it would involve.



\levelThreeTitle{Back to the insertion sort}


Now let us go back to our proof of the insertion sort algorithm and see how we
can get rid of all our interactive proofs for this function. Note however that
in this proof, we often need to deal with macros since the program has not been
particularly written with the idea to formally verify it later (for this, the
reader can refer to the version proposed in the book ACSL by Example which can
be adapted with a similar technique and is easier to prove). Thus, in
this example, we push the solvers to their limit because of big proof contexts.
With this example, depending on how powerful the machine is, we might need to
increase the proof timeout to 120 seconds (which is already quite long for a
SMT solver). In this example, we will illustrate three actual usage of ghost
code that we have seen so far:


\begin{itemize}
\item directly writing code to build a proof,
\item writing (and using) lemma functions,
\item writing (and using) lemma macros.
\end{itemize}


We also make use of assertions to make the proof context richer so SMT solvers
succeed in proving the properties we are interested in. Some parts of the
annotations are equivalent to what we have done previously. First, we use some
assertions that were also useful in our previous proof. We will recall their
purpose for each function. Second, we re-use the same axiomatic definition for
occurrences counting. Furthermore, we keep the following predicate definitions:


\CodeBlockInput[25][42]{c}{insert-sort-auto.c}


As we will need them, as well as the lemma about the transitivity of
occurrences counting as it is automatically proved by SMT solvers (thus we can
keep it since it does not require an interactive proof from us):


\CodeBlockInput[43][48]{c}{insert-sort-auto.c}


Let us start with the \CodeInline{insertion\_sort} function itself. In this
function, we made use of three assertions:


\CodeBlockInput[87][108]{c}{insert-sort-auto.c}


The first one makes sure that the part of the array where we just inserted a
value is a permutation of the same range of values before the call to
\CodeInline{insert}, as this is the postcondition of the function, it is not
necessary but let us keep it for illustration. The last assertion is the
property we want to prove in order to get enough knowledge to use the lemma that
states that permutation is transitive (and show that after the block of the loop
since our array is a permutation of the array at the beginning, which is itself
a permutation of the original one, then after the body of the loop we have
maintained that the array is a permutation of the original one).



The second assertion says that the second part of the array is unchanged, and we
want to use this knowledge to show that the number of occurrences of the values
is unchanged. Here we could use a combination of lemma functions and macros to
prove that the complete range is a permutation (as we will do for the other
function) however, directly writing the code is here a bit simpler (requires
less proofs, as we will see later) so let us directly write the code that will
create the proof of our property.



In order to show that the complete range is a permutation, we have to show that
the number of occurrences did not change. We know that the first part of the
array is a permutation of the same range at the beginning of the body of the loop.
Thus, we already know that the number of occurrences of any $v$ did not change
for a part of our array. Let us continue the occurrences counting for the rest
of our array, knowing that the second part is unchanged (when \CodeInline{i+1}
is lower than \CodeInline{end} as else, we do not have anything to count):


\CodeBlockInput[267][286]{c}{insert-sort-auto-proved.c}


which is enough to ensure that the \CodeInline{insertion\_sort} function
respects its specification as long as we finish the proof of the function
\CodeInline{insert}. This second function makes a more complex action. Let us
depart from this annotated version:


\CodeBlockInput[50][84]{c}{insert-sort-auto.c}


Again, the proof that this function maintains a permutation of the array is
the hardest part of the job. The fact that the function guarantees that the
value are sorted is already easily established. Using the same technique as
for \CodeInline{insertion\_sort} is not so easy here. Indeed, the second part
of the array has been rotated which makes the properties slightly more
complex. So, let us show that we can split the array at the position where
we insert into two parts, in which we respectively show that:

\begin{itemize}
\item for the first part, since it is unchanged, for any $v$, the number of
  occurrences did not change either,
\item for the second part, since it is rotated, for any $v$, the number of
  occurrences did not change.
\end{itemize}

First let us define a lemma function that allows to explicitly split a range of
values into two subparts in which we can count separately.


\CodeBlockInput[57][75]{c}{insert-sort-auto-proved.c}


We can note that this property is proved in a way that is quite similar to what
we wrote for the body of our loop in \CodeInline{insertion\_sort}, we start from
the point where we want to count and show that the property remains true until
the end of the array. 



We can use our function to split the array at the right place after the loop.
However, we can only do it for the new content of the array, indeed, in order to
establish it for the original array, we have to call the function on the
original array, when we still do not know the value of $i$. Thus, let us write
another version of the ``split'' property that shows that we can split the
array at any index, thus make the \CodeInline{split} variable a universally
quantified variable, and use the previous function to prove that it is true:


\CodeBlockInput[77][98]{c}{insert-sort-auto-proved.c}


And we can split our original array and the new one:


\begin{CodeBlock}{c}
void insert(int* a, size_t beg, size_t last){
  size_t i = last ;
  int value = a[i] ;

  // split before modifying
  //@ ghost l_occurrences_of_split(a, beg, last+1);

  /*@ LOOP ANNOT */
  while(i > beg && a[i - 1] > value){
    a[i] = a[i - 1] ;
    --i ;
  }
  a[i] = value ;
  // Assertions ...

  // split after modifying, now we know "i"
  //@ ghost l_occurrences_of_explicit_split(a, beg, i, last+1);
}
\end{CodeBlock}


Now, the only remaining parts of the proof are first to show that an unchanged
array is a permutation and second that the rotate operation also maintains a
permutation. Here, we need macros. Let us start with the easiest: the unchanged
property that we already almost exactly proved in the
\CodeInline{insertion\_sort} function. We start by building the context for our
proof:


\CodeBlockInput[138][154]{c}{insert-sort-auto-proved.c}


The function \CodeInline{unchanged\_permutation\_premise} ensures that we have
modified the array (thus created a new memory state) and that the array is
unchanged from the precondition to the postcondition. We can build our
lemma macro:


\CodeBlockInput[125][135]{c}{insert-sort-auto-proved.c}


Which almost corresponds to what we have written previously in the
\CodeInline{insert\_sort} function and we can use the macro where it is needed
in the \CodeInline{insert} function.


\CodeBlockInput[245][247]{c}{insert-sort-auto-proved.c}


The only remaining property to prove is the hardest one and is about the
\CodeInline{rotate\_left} predicate. Let us first write our context to prepare
the macro.


\CodeBlockInput[183][200]{c}{insert-sort-auto-proved.c}


How can we prove this property? Basically, one has to notice that since all the
elements from the beginning to the penultimate have been shifted of one index
to the right the number of occurrences in the shifted part did not change. Then
one has to show that the number of occurrences of any $v$ in respectively the
last cell in the original array, and the first cell in the new array is the same
(since the corresponding element is the same). Again, we rely on the split
function to count separately the elements that are shifted and the one which is
moved from the end to the beginning. However, the call corresponding to the
original array has again to be put before the code that modifies the memory (see
line 194) in the previous code, and we will have to take that in account when we
will insert our use of the macro in the \CodeInline{insert} function.


Let us now present the macro that we use to prove that the lemma holds:


\CodeBlockInput[156][181]{c}{insert-sort-auto-proved.c}


The loop invariant is pretty similar to what we have written so far, the only
difference is that it takes in account the shift of the elements. Furthermore,
to prove the invariant we had to add an assertion to help the solvers notice
that the last element of both ranges is the same (note however that depending
on the versions of the solvers or how powerful is the machine, this might be
unneeded sometimes). A more important difference compared to our previous
examples is that here, we need to provide more information to SMT solvers by
adding other ghost functions calls (line 170, in order to split the first
element of the array), as well as assertions to guide the last steps of the
proof:


\begin{itemize}
\item 171--175: we recall that in the original array we can split the last
  element,
\item 176--180: we show that as the first element of the array is
  the last element of the original array (176), the number of occurrences for
  any value in these ranges is the same (177--179).
\end{itemize}


We can use the macro in our program:


\begin{CodeBlock}{c}
//@ assert rotate_left{Pre, Here}(a, i, last+1) ;
//@ ghost rotate_left_permutation(Pre, Here, a, i, last+1) ;
//@ assert permutation{Pre, Here}(a, i, last+1) ;
\end{CodeBlock}


However, we have to show that the considered range at label \CodeInline{Pre} can
be split at \CodeInline{last}. For this, we use another variant of the split
function, that shows that any subrange can be split before the last element (if
it is non empty):



\CodeBlockInput[100][122]{c}{insert-sort-auto-proved.c}



That we have to call before the loop in the \CodeInline{insert} function:


\CodeBlockInput[211][216]{c}{insert-sort-auto-proved.c}


Note that depending on the version of the solvers, the assertion on lines 176 to
180 of the macro, about the element at the beginning/the end of the array might
fail due to the complexity of the proof context. Let us help the solvers a last
time by adding a last lemma (automatically proved by SMT solvers) that states
this relation for any array and position in the array:



\CodeBlockInput[49][54]{c}{insert-sort-auto-proved.c}


which guarantees that our resulting annotated insertion function is entirely
proved:


\CodeBlockInput[202][250]{c}{insert-sort-auto-proved.c}


We finally highlight how the proof context can make the proof harder for SMT
solvers. Basically, if we swap the proofs for each part of the array, that is,
starting with the ``unchanged'' part and after the ``rotate'' part, the proof has
more chance to fail, since it would make the proof context bigger for the
hardest proof. For the same reason, it is here almost needed (again depending
on how powerful is the machine and the solvers) to separate the proof of the
absence of runtime errors, else it pollutes the proof context and it may fail.


\levelThreeTitle{Exercises}


\levelFourTitle{Sum of N integers}


Using lemma functions, we can prove the lemma about the sum of the N first
integers that we previously expressed. You might have done this proof when you
were in high school, now it is time to do it in C and ACSL. Write a contract for
the following function that expresses in postcondition that the sum of the N
first integers is \CodeInline{N(N+1)/2}. Complete the body of the function with
a loop in order to prove this property. We advise to slightly transform the
invariant in order to be sure that the property does not contain any division
(division on integers have some properties that can make them hard to deal with
for SMT solvers depending on the constraint that exists on the used values).


\CodeBlockInput[1][12]{c}{ex-1-sum-of-n-integers.c}


Now, let us generalize to any bounds with the sum of all the integers between
\CodeInline{fst} and \CodeInline{lst}. We provide the logic function and the
contract. Write a body for the function such that the postcondition is proved.
Note that again, we advise to express the invariant without division.


\CodeBlockInput[14][26]{c}{ex-1-sum-of-n-integers.c}


Finally, let us prove this function:


\CodeBlockInput[28][35]{c}{ex-1-sum-of-n-integers.c}


This should not be too hard, and what we obtain is a proof that we have written
a correct optimization for the function that computes the sum of the N first
integers.


\levelFourTitle{Properties about occurrences counting}


In this exercise, we want to prove some interesting properties about the
axiomatically defined logic function \CodeInline{l\_occurrences\_of}:


\CodeBlockInput{c}{ex-2-l_occurrences_of-props.c}


The function \CodeInline{occ\_bounds} should state that the number of
occurrences of \CodeInline{v} in the array is comprised between 0 and
\CodeInline{len}.


The function \CodeInline{not\_in\_occ\_0} should state that if \CodeInline{v} is
not in the array then the number of occurrences of \CodeInline{v} in the array
is 0.


The function \CodeInline{occ\_monotonic} should state that the number of
occurrences of \CodeInline{v} in the array from 0 to \CodeInline{pos} is lower
or equals to the number of occurrences of \CodeInline{v} in the array from 0
to \CodeInline{more}, if \CodeInline{more} is greater or equals to
\CodeInline{pos}.


The function \CodeInline{occ\_0\_not\_in} should state that if the number of
occurrences of \CodeInline{v} in the array is 0 then \CodeInline{v} is not in
the array. Note that you will probably need to use \CodeInline{occ\_monotonic}.


The function \CodeInline{occ\_pos\_find} should find an index \CodeInline{i}
such that the value \CodeInline{arr[i]} is \CodeInline{v}, provided that the
number of occurrences of \CodeInline{v} is positive. Note that you will
probably need to use \CodeInline{occ\_monotonic}.


Finally the function \CodeInline{occ\_pos\_exists} should translate the contract
of the previous function using an existentially quantified variable, and use the
previous function to obtain the proof for free.


For all these functions, WP should be parameterized with the control of the
absence of runtime errors as well as the option
\CodeInline{-warn-unsigned-overflow}.


\levelFourTitle{An actual example with sum}


Take back the proof performed in the previous chapter for the
exercise~\ref{l4:proof-methodologies-triggering-lemmas-exercises-sum}. Modify
the annotations in order to ensure that no more classic lemmas are necessary.
The skeleton of the file follows:


\CodeBlockInput{c}{ex-3-sum-content.c}
