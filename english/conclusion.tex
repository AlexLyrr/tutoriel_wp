\begin{Quotation}[Jean-Louis Aubert, \textit{Bleu Blanc Vert}, 1989]
Voilà, c'est fini ...
\end{Quotation}



\ldots{} for this introduction to the proof of C programs using Frama-C
and WP.

Along this tutorial, we have seen how we can use these tools to specify
what we expect of our programs and verify that the source code we have
produced indeed corresponds to the specification we have provided. This
specification is provided using annotations of our functions that
includes the contract they must respect. These contracts are properties
required about the input to ensure that the function will correctly
work, which is specified by properties about the output of the function
and enforced by the tool that allow us to check specific problems related
to the use of C (namely, the absence of runtime errors).

Starting from specified programs, WP is able to produce the weakest
precondition of our functions, provided what we want in postcondition,
and to ask some provers if the specified precondition is compatible with
the computed one. The reasoning is completely modular, which allows to
prove functions in isolation from each other and to compose the results.


WP cannot currently work with dynamic allocation. A function that would use it
could not be proved. However, even without dynamic allocation, a lot of function
can be proved since they work with data-structures that are already allocated.
And these functions can then be called with the certainty that they
perform a correct job. If we cannot or do not want to prove the client code of a
function, we can still write something like this:



\begin{CodeBlock}{c}
/*@
  requires some_properties_on(a);
  requires some_other_on(b);

  assigns ...
  ensures ...
*/
void my_function(int* a, int b){
  //this is indeed the  "assert" defined in "assert.h"
  assert(/*properties on a*/ && "must respect properties on a");  
  assert(/*properties on b*/ && "must respect properties on b");
}
\end{CodeBlock}



Which allows us to benefit from the robustness of our function having
the possibility to debug an incorrect call in a source code that we
cannot or do not want to prove.

Writing specifications is sometimes long or tedious. Higher-level
features of ACSL (predicates, logic functions, axiomatizations) allow us
to lighten this work, as well as our programming languages allow us to
define types containing other types, functions calling functions, bringing
us to the final program. But, despite this, write specification in a
formal language, no matter which one, is generally a hard task.

However, this \textbf{formalization} of our need is \textbf{crucial}.
Concretely, such a formalization is a work every developer should do.
And not only in order to prove a program. Even the definition of tests
for a function requires to correctly understand its goal if we want to
test what is necessary and only what is necessary. And writing
specification in a formal language is incredibly useful (even if it can
be sometimes frustrating) to get a clear specification.

Formal languages, that are close to mathematics, are precise.
Mathematics have this: they are precise. What is more terrible than
reading a specification written in a natural language, with complex
sentences, using conditional forms, imprecise terms, ambiguities,
compiled in administrative documents composed of hundreds of pages, and
where we need to determine, ``so, what this function is supposed to do?
And what do I have to validate about it?''.

Formal methods are probably not used enough currently. Sometimes because of
mistrust, sometimes because of ignorance, sometimes because of prejudice
based on ideas born at the beginning of the tools, 30 years ago. Our tools
evolve, the ways we develop change, probably faster than in any other
technical domain. Saying that these tools could never be
used for real life programs would certainly be a too big shortcut. After
all, we see everyday how much development is different from what it were
10 years, 20 years, 40 years ago and can barely imagine how much it will
be different in 10 years, 20 years, 40 years.

During the past few years, safety and security questions have become
more and more visible and crucial. Formal methods also progressed a lot,
and the improvement they bring for these questions are greatly
appreciated. For example, this
\externalLink{this link}{http://sfm.seas.harvard.edu/report.html} brings to the
report of a conference about security that brought together people from
academic and industrial world, in which we can read:



\begin{Quotation}[Formal Methods for Security, 2016]
Adoption of formal methods in various areas (including verification of hardware
and embedded systems, and analysis and testing of software) has dramatically 
improved the quality of computer systems.  We anticipate that formal methods 
can provide similar improvement in the security of computer systems.

...

\textbf{Without broad use of formal methods, security will always remain fragile.}
\end{Quotation}



\levelTwoTitle{Going further}


\levelThreeTitle{With Frama-C}


Frama-C provides different ways to analyse programs. Among these tools, the
most commonly used and interesting to know from a static and dynamic
verification point of view are certainly those ones:

\begin{itemize}
\item abstract interpretation analysis using
\externalLink{EVA}{http://frama-c.com/value.html},
\item the transformation of annotation into runtime verification using
\externalLink{E-ACSL}{http://frama-c.com/eacsl.html}.
\end{itemize}

The goal of the first one is to compute the domain of the different
variables at each program point. When we precisely know these domains,
we can determine if these variables can produce errors when they are
used. However this analysis is executed on the whole program and not
modularly. It is also strongly dependent of the type of domain we use (we
will not enter into details here) and it is not so good at keeping the
relations between variables. On the other side, it is really completely
automatic, we do not even need to give loop invariant! The most manual
part of the work is to determine whether or not an alarm is a true error
or a false positive.

The second analysis allows to generate from an original program, a new
program where the assertions are transformed into runtime verifications.
If these assertions fail, the program fails. If they are valid, the
program has the same behavior it would have without the assertions. An
example of use is to generate the verification of absence of runtime
errors as assertions using \texttt{-rte} and then to use E-ACSL to
generate the program containing the runtime verification that these
assertions do not fail.

There exist a lot of different plugins for very different tasks in Frama-C.

Finally, a last possibility that will motivate the use of Frama-C is the
ability to develop their own analysis plugins using the API provided by
the Frama-C kernel. A lot of tasks can be realized by the analysis of
the source code and Frama-C allows to build them as easily as possible.


\levelThreeTitle{With deductive proof}


Along this tutorial we used WP to generate proof obligation starting
from programs with their specification. Next we have used automatic
solvers to assure that these properties were indeed verified.

When we use other solvers than Alt-Ergo and Coq, the communication with
this solver in provided by a translation to the Why3 language that will
next be used to bridge the gap to automatic solvers. But this is not the
only way to use Why3. It can also be used itself to write programs and
prove them. It especially provides a set of theories for some common
data structures.

There are some proofs that cannot be discharged by automatic solvers. In
such a case, we have to provide these proofs interactively. WP, like
Why3, can extract its verification conditions to Coq, and it is very
interesting to study
this language. In the context of Frama-C, we can produce reusable lemmas
libraries proved with Coq. But Coq can also be used for many
different tasks, including programming. Note that Why3 can also extract
its proof obligations to Isabelle or PVS that are also proof assistants.

Finally, there exists other program logics, for example separation logic
or concurrent program logics. Again these notions are interesting to know
in the context of Frama-C: if we cannot directly use them, they can
inspire the way we specify our program in Frama-C for the proof with WP.
They could also be implemented into new plugins to Frama-C.

A whole new world of methods to explore.
