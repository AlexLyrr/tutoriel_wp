\levelThreeTitle{Calling a function}

\levelFourTitle{Formal - Weakest precondition calculus}

When a function is called, the contract of this function is used to determine
the pre-condition of the call. But one has to consider two important facts to
express the weakest precondition calculus.



First, the post-condition of the called function $f$ is not necessarily
directly the pre-condition that was computed for the instructions that follow
the call to $f$. For example, if we have a program: \CodeInline{x = f() ; c }
and $wp(\texttt{c}, Q) = 0 \leq x \leq 10$, whereas the postcondition of the
function \CodeInline{f} is $1 \leq x \leq 9$, we have to express some
weakening between the actual precondition of \CodeInline{c} and the computed
one. For this, we refer to the section~\ref{l3:statements-basic-consequence},
the idea is simply to verify that the post-condition of the function implies
the computed pre-condition.



Second, in C, a function can have side-effects. Thus, the values of the
variables referenced in input is not necessarily the same as it is after the
call to the function, and the contract may express some property about those
values before and after the call. So, if we have labels in the post-condition,
we must correctly replace them.



In order to defined the weakest-precondition calculus of function calls, let
us introduce some notation to make things clearer. For this, consider this
example:


\begin{CodeBlock}{c}
/*@ requires \valid(x) && *x >= 0 ;
    assigns *x ;
    ensures *x == \old(*x)+1 ; */
void inc(int* x);

void foo(int* a){
  L1:
  inc(a) ;
  L2:
}
\end{CodeBlock}




The weakest precondition of the function call asks us to consider the contract
of the function that is called (here, in \CodeInline{foo}, when we call the
\CodeInline{inc} function). Of course, before the call to the function we have
to verify its precondition, so it is part of the weakest precondition. But, we
also have to consider the post-condition of the function, else that would mean
that we do not consider its effect.




Thus, it is important to notice that in the precondition, the considered memory
state is the one where we compute the weakest precondition, whereas for the
postcondition it is not the case, the considered memory state is the one that
follows the call, while we need to explicitely mention the old state to speak
about the values before the call. For example, considering the contract of
\CodeInline{inc} when we call it in \CodeInline{foo}, \CodeInline{*x} in the
precondition is \CodeInline{*a} at \CodeInline{L1}, while \CodeInline{*x} in the
postcondition is \CodeInline{*a} at \CodeInline{L2}. Consequently the pre and
the postcondition must be considered slighlty differently when it comes to
mutable memory location. Note that for the value of the parameter \CodeInline{x}
itself, there is no such consideration: this value cannot be modified by the
call.




Now, let us define the weakest precondition of a function call. For this,
we denote:

\begin{itemize}
\item $\vec{v}$ a vector of values $v_1, ..., v_n$ and $v_i$ the $i^{th}$ value,
\item $\vec{t}$ the arguments provided to the function when we call it,
\item $\vec{x}$ the parameters in the function definition,
\item $\vec{a}$ the assigned values (seen from outside, once instantiated),
\item $here(x)$ a value in postcondition
\item $old(x)$ a value in precondition
\end{itemize}

We name $\texttt{f:Pre}$ the precondition of the function, and $\texttt{f:Post}$
the postcondition:



\begin{center}
\begin{tabular}{rl}
  $wp( f(\vec{t}), Q ) :=$ & $\texttt{f:Pre}[x_i \leftarrow t_i]$ \\
  $\wedge$ & $\forall \vec{v}, \quad (
              \texttt{f:Post}[x_i \leftarrow t_i,
                              here(a_j) \leftarrow v_j,
                              old(a_j) \leftarrow a_j] \Rightarrow
              Q[here(a_j) \leftarrow v_j])$
\end{tabular}
\end{center}


We can detail a little bit the reasoning for each part of this formula.


First, note that in both pre and post-condition, each named parameter $x_i$ is
replaced with the corresponding argument ($[x_i \leftarrow t_i]$), as we said
before we do not have to consider memory states there because those values
cannot be changed by the function call. For example in the contract of
\CodeInline{inc}, each \CodeInline{x} would be replaced by the argument
\CodeInline{a}.



Then, in the part of the formula that corresponds to the postcondition, we can
see that we introduce a $\forall \vec{v}$. The goal is here to model the fact
that the function can write any value in each memory location that is assigned.
So, for each of the assigned location $a_j$ (that is for our call to
\CodeInline{inc}, \CodeInline{*(\&a)}, we generate a value $v_j$ that is its
value after the call. But, if we want to check that the postcondition gives us
the right result, we cannot accept \emph{any value} for each assigned location,
we just want the ones \emph{that allows to statisfy the postcondition}.



So these values are used to transform the postcondition of the function and
verify that it implies the postcondition in input of the weakest precondition.
This is done by replacing, for each assigned location $a_j$, its value $here$
with the value $v_j$ that it is supposed to get after the call
($here(a_j) \leftarrow v_j$). Finally, we have to replace each $old$ value by
its value before the call, and for each $old(a_j)$, it is simply $a_j$
($old(a_j) \leftarrow a_j$).



\levelFourTitle{Formal - Example}



Let us illustrate this on an example by applying the weakest precondition
calculus to this short code, assuming the contract we previously proposed for
the \CodeInline{swap} function.



\begin{CodeBlock}{c}
  int a = 4 ;
  int b = 2 ;

  swap(&a, &b) ;

  //@ assert a == 2 && b == 4 ;
\end{CodeBlock}



We now compute the weakest precondition:


\begin{tabular}{l}
  $wp(a = 4; b = 2; swap(\&a, \&b), a = 2 \wedge b = 4) = $\\
  $\quad wp(a = 4, wp(b = 2; swap(\&a, \&b), a = 2 \wedge b = 4)) = $\\
  $\quad wp(a = 4, wp(b = 2, wp(swap(\&a, \&b), a = 2 \wedge b = 4)))$
\end{tabular}


Let us first consider separately:


$$wp(swap(\&a, \&b), a = 2 \wedge b = 4)$$



From this \CodeInline{assigns} clause, we know that the assigned values are
$*(\&a) = a$ and $*(\&b) = b$. (Let us shorten $here$ with $H$ and $old$ with
$O$).


\begin{tabular}{rl}
  $\quad \quad \texttt{swap:Pre}[x \leftarrow \&a,\ y \leftarrow \&b]$ & \\
  $\quad \wedge \forall v_a, v_b,(\texttt{swap:Post}$ & $ [ x \leftarrow \&a,\ y \leftarrow \&b, $ \\
                               & $H(*(\&a)) \leftarrow v_a,\ H(*(\&b)) \leftarrow v_b,$ \\
                               & $O(*(\&a)) \leftarrow *(\&a),\ O(*(\&b)) \leftarrow *(\&b)])$\\
  \multicolumn{2}{l}{$\quad \quad \Rightarrow (H(a) = 2 \wedge H(b) = 4)[H(a)) \leftarrow v_a, H(b)) \leftarrow v_b])$}
\end{tabular}


For the precondition, we get :
$$valid(\&a) \wedge valid(\&b)$$
For the postcondition part, first we replace the pointers:
$$H(*(\&a)) = O(*(\&b)) \wedge H(*(\&b)) = O(*(\&a)) \Rightarrow H(a) = 2 \wedge H(b) = 4$$
Then, the $here$ values, with the quantified $v_i$s:
$$v_a = O(*(\&b)) \wedge v_b = O(*(\&a)) \Rightarrow v_a = 2 \wedge v_b = 4$$
And the $old$ values, with the value before call:
$$v_a = *(\&b) \wedge v_b = *(\&a) \Rightarrow v_a = 2 \wedge v_b = 4$$
We can now simplify this formula to:
$$v_a = b \wedge v_b = a \Rightarrow v_a = 2 \wedge v_b = 4$$


So, $wp(swap(\&a, \&b), a = 2 \wedge b = 4)$ is:
$$P: valid(\&a) \wedge valid(\&b) \wedge \forall v_a, v_b, \quad v_a = b \wedge v_b = a \Rightarrow v_a = 2 \wedge v_b = 4$$
Let us immediately simplify the formula by noticing that validity property is
trivially true here:
$$P: \forall v_a, v_b, \quad v_a = b \wedge v_b = a \Rightarrow v_a = 2 \wedge v_b = 4$$


Let us now compute $wp(a = 4, wp(b = 2, P)))$, by first replacing $b$ with
$2$ by the assignement rule:
$$\forall v_a, v_b, \quad v_a = 2 \wedge v_b = a \Rightarrow v_a = 2 \wedge v_b = 4$$
and then replacing $a$ with $4$ by the same rule:
$$\forall v_a, v_b, \quad v_a = 2 \wedge v_b = 4 \Rightarrow v_a = 2 \wedge v_b = 4$$


This last property is trivially true, thus the program is verified.



\levelFourTitle{What should we keep in mind?}



Functions are absolutely necessary to modular programming, and the weakest
precondition is fully compatible with this idea, allowing to reason about each
function locally and compose proofs just as we compose function calls.


So as a reminder, we should just keep in mind the followiing general scheme:



\begin{CodeBlock}{c}
/*@
  requires foo_R ;
  assigns ... ;
  ensures foo_E ;
*/
type foo(parameters...){
  // Here we suppose that foo_R holds


  // Here we must prove that bar_R holds
  bar(some parameters ...) ;
  // Dere we assume that bar_E holds


  // Here we must prove that foo_E holds
  return ... ;
}
\end{CodeBlock}


Note that for the last statement, with weakest precondition calculus, the idea
is more to show our precondition is strong enough to ensure that the code leads
to our post-condition. However, first, this version is simpler to understand,
and second the WP plugin does not actually perform a strict weakest precondition
but an highly optimized one that does not follows exactly the same rules.


\levelThreeTitle{Recursive functions}



For now, WP does not check function termination. Of course, if a function is
only composed of loops that terminates (that have a verified variant) and
calls to function that terminates, it terminates. However, one particular
case require more reasoning: recursive and mutually recursive functions.
Currently terminations of such functions is not supported with WP.



That basically means that using a function that does not terminate we can
prove anything. For example:



\CodeBlockInput{c}{trick.c}


\image{recursive-trick}


We can see that the function and the assertion are proved. And indeed the
proof is correct: we consider partial correctness and we face a function
that does not terminate: anything that follows a call to this function would
be true.



Thus, the question is: what could we do in such a case? Again, we could use
some kind of variant to bound the number of recursive calls in ACSL, this is
the role of the \CodeInline{decreases} clause:


\CodeBlockInput{c}{decreases.c}


This clause expresses exactly the same idea as \CodeInline{loop variant}. The
expression considered by a decrease clause is a positive integer that
strictly decreases when the function is called again. However, it is still
not supported by WP. Thus, for the moment, a recursive function cannot be
totally proved with WP.
