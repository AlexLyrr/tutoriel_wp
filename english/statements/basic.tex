
\levelThreeTitle{Assignment}


Assignment is the most basic operation one can have in language (leaving
aside the ``do nothing'' operation that is not particularly
interesting). The weakest precondition calculus associates the following
computation to an assignment operation:




\begin{center}
$wp(x = E , Post) := Post[x \leftarrow E]$
\end{center}


Here the notation $P[x \leftarrow E]$ means ``the property $P$ where
$x$ is replaced by $E$''. In our case this corresponds to ``the
postcondition $Post$ where $x$ is replaced by $E$''. The idea is
that the postcondition of an assignment of $E$ to $x$ can only be
true if replacing all occurrences of $x$ in the formula by $E$ is
true. For example:



\begin{CodeBlock}{c}
// { P }
x = 43 * c ;
// { x = 258 }
\end{CodeBlock}




\begin{center}
$P = wp(x = 43*c , \{x = 258\}) = \{43*c = 258\}$
\end{center}


The function $wp$ allows us to compute as weakest precondition of the
the assignment the formula $\{43*c = 258\}$, thus obtaining the
following Hoare triple:



\begin{CodeBlock}{c}
// { 43*c = 258 }
x = 43 * c ;
// { x = 258 }
\end{CodeBlock}



In order to compute the precondition of the assignment we have replaced
each occurrence of $x$ in the postcondition by the assigned value
$E = 43*c$. If our program were of the form:



\begin{CodeBlock}{c}
int c = 6 ;
// { 43*c = 258 }
x = 43 * c ;
// { x = 258 }
\end{CodeBlock}



we could submit the formula " $43*6 = 258$ " to our automatic prover
in order to determine whether it is really valid. The answer would of
course be ``yes'' because the property is easy to verify. If we had,
however, given the value 7 to the variable \texttt{c} the prover's reply
would be ``no'' since the formula $43*7 = 258$ is not true.



Taking into account the weakest precondition calculus, we can now write
the inference rule for the Hoare triple of an assignment as



\begin{center}
$\dfrac{}{\{Q[x \leftarrow E] \}\quad x = E \quad\{ Q \}}$
\end{center}


We note that there is no precondition to verify. Does this mean that the
triple is necessarily true? Yes. However, it does not mean that the
precondition is respected by the program to which the assignment belongs
or that the precondition is at all possible. Here the automatic provers
come into play.



For example, we can ask Frama-C to verify the following line:



\begin{CodeBlock}{c}
int a = 42;
//@ assert a == 42;
\end{CodeBlock}



which is, of course, directly proven by Qed, since it is a simple
applications of the assignment rule.



\begin{Information}
  We remark that according to the C standard, an assignment is in fact an
  expression. This allows us, for example, to write
  \CodeInline{if( (a = foo()) == 42)}.
  In Frama-C, an assignment will always be treated as a statement. Indeed,
  if an assignment occurs within a larger expression, then the Frama-C
  preprocessor, while building the abstract syntax tree, systematically
  performs a \emph{normalization step} that produces a separate assignment
  statement.
\end{Information}



\levelThreeTitle{Composition of statements}

For a statement to be valid, its precondition must allow us by means of
executing the said statement to reach the desired postcondition. Now we
would like to execute several statements one after another. Here the
idea is that the postcondition of the first statement is compatible with
the required precondition of the second statement and so on for the
third statement.



The inference rule that corresponds to this idea utilizes the following
Hoare triples:




\begin{center}
$\dfrac{\{P\}\quad S1 \quad \{R\} \ \ \ \{R\}\quad S2 \quad \{Q\}}{\{P\}\quad S1 ;\ S2 \quad \{Q\}}$
\end{center}


In order to verify the composed statement $S1;\ S2$ we rely on an
intermediate property $R$ that is at the same time the postcondition
of $S1$ and the precondition of $S2$. (Please note that $S1$ and
$S2$ are not necessarily simple statements; they themselves can be
composed statements.) The problem is, however, that nothing indicates us
how to determine the properties $P$ and $R$.

The weakest-precondition calculus now says us that the intermediate
property $R$ can be computed as the weakest precondition of the second
statement. The property $P$, on the other hand, then is computed as
the weakest precondition of the first statement. In other words, the
weakest precondition of the composed statement $S1; S2$ is determined
as follows:



\begin{center}
$wp(S1;\ S2 , Post) := wp(S1, wp(S2, Post) )$
\end{center}


The WP plugin of Frama-C performs all these computations for us. Thus,
we do not have to write the intermediate properties as ACSL assertions
between the lines of codes.



\begin{CodeBlock}{c}
int main(){
  int a = 42;
  int b = 37;

  int c = a+b; // i:1
  a -= c;      // i:2
  b += a;      // i:3

  //@assert b == 0 && c == 79;
}
\end{CodeBlock}



\levelFourTitle{Proof tree}


When we have more than two statements, we can consider the last
statement as second statement of our rule and all the preceding ones as
first statement. This way we traverse step by step backwards the
statements in our reasoning. With the previous program this looks like:


\begin{center}
\begin{tabular}{ccc}
  $\{P\}\quad i_1 ; \quad \{Q_{-2}\}$ & $\{Q_{-2}\}\quad i_2 ; \quad \{Q_{-1}\}$ & \\
  \cline{1-2}
  \multicolumn{2}{c}{$\{P\}\quad i\_1 ; \quad i\_2 ; \quad \{Q_{-1}\}$} & $\{Q_{-1}\} \quad i_3 ; \quad \{Q\}$\\
  \hline
  \multicolumn{3}{c}{$\{P\}\quad i\_1 ; \quad i\_2 ; \quad i\_3; \quad \{ Q \}$}
\end{tabular}
\end{center}

The weakest-precondition calculus allows us to construct the property
$Q_{-1}$ starting from the property $Q$ and statement $i_3$ which
in turn enables us to derive the property $Q_{-2}$ from the property
$Q_{-1}$ and statement $i_2$. Finally, $P$ can be determined from
$Q_{-2}$ and $i_1$.



Now that we can verify programs that consist of several statements it
is time to add some structure to them.



\levelThreeTitle{Conditional rule}


For a conditional statement to be true, one must be able to reach the
postcondition through both branches. Of course, for both branches, the
same precondition (of the conditional statement) must hold. In addition,
we have that in the if-branch the condition is true while in the
else-branch it is false.

We therefore have, as in the case of composed statements, two facts to
verify (in order to avoid confusion we are using here the syntax
$if\ B\ then\ S1\ else\ S2$):




\begin{center}
$\dfrac{\{P \wedge B\}\quad S1\quad \{Q\} \quad \quad \{P \wedge \neg B\}\quad S2\quad \{Q\}}{\{P\}\quad if\quad B\quad then\quad S1\quad else\quad S2 \quad \{Q\}}$
\end{center}


Our two premises are therefore that we can both in the if-branch and the
else-branch reach the postcondition from the precondition.

The result of the weakest-precondition calculus for a conditional
statement reads as follows:



\begin{center}
$wp(if\ B\ then\ S1\ else\ S2 , Post) := (B \Rightarrow wp(S1, Post)) \wedge (\neg B \Rightarrow wp(S2, Post))$
\end{center}


This means that the condition $B$ has to imply the weakest
precondition of $S1$ in order to safely arrive at the postcondition.
Analogously, the negation of $B$ must imply the weakest precondition
of $S2$.



\levelFourTitle{Empty \CodeInline{else}-branch}


Following this definition, we obtain for the case of an empty else-branch the
following rule by simply replacing the statement $S2$ by the empty
statement \texttt{skip}.




\begin{center}
$\dfrac{\{P \wedge B\}\quad S1\quad \{Q\} \quad \quad \{P \wedge \neg B\}\quad skip\quad \{Q\}}{\{P\}\quad if\quad B\quad then\quad S1\quad else\quad skip \quad \{Q\}}$
\end{center}


The triple for \CodeInline{else} is:




\begin{center}
$\{P \wedge \neg B\}\quad skip\quad \{Q\}$
\end{center}


which means that we need to ensure:




\begin{center}
$P \wedge \neg B \Rightarrow Q$
\end{center}


In short, if the condition $B$ of \texttt{if} is false, this means
that the postcondition of the complete conditional statement is already
established before entering the else-branch (since it does not do
anything).



As an example, we consider the following code snippet where we reset a
variable $c$ to a default value in case it had not been properly
initialized by the user.



\begin{CodeBlock}{c}
int c;

// ... du code ...

if(c < 0 || c > 15){
  c = 0;
}
//@ assert 0 <= c <= 15;
\end{CodeBlock}



Let



$wp(if \neg (c \in [0;15])\ then\ c := 0, \{c \in [0;15]\})$



$:= (\neg (c \in [0;15])\Rightarrow wp(c := 0, \{c \in [0;15]\})) \wedge (c \in [0;15]\Rightarrow wp(skip, \{c \in [0;15]\}))$



$= (\neg (c \in [0;15]) \Rightarrow 0 \in [0;15]) \wedge (c \in [0;15] \Rightarrow c \in [0;15])$



$= (\neg (c \in [0;15]) \Rightarrow true) \wedge true$



The property can be verified: independent of the evaluation of
$\neg (c \in [0;15])$, the implication will hold.



\levelThreeTitle{Bonus Stage - Consequence rule}
\label{l3:statements-basic-consequence}


It can sometimes be useful to strengthen a postcondition or to weaken a
precondition. The former will often be established by us to
facilitate the work of the prover, the latter is more often verified by
the tool as the result of computing the weakest pre condition.



The inference rule of Hoare logic is the following:




\begin{center}
$\dfrac{P \Rightarrow WP \quad \{WP\}\quad c\quad \{SQ\} \quad SQ \Rightarrow Q}{\{P\}\quad c \quad \{Q\}}$
\end{center}


(We remark that the premises here are not only Hoare triples but also
formulas to verify.)

For example, if our post-condition is too complex, it may generate a
weaker precondition that is, however, too complicated, thus making the
work of provers more difficult. We can then create a simpler
intermediate postcondition $SQ$, that is, however, stricter and
implies the real postcondition. This is the part $SQ \Rightarrow Q$.

Conversely, the calculation of the precondition will usually generate a
complicated and often weaker formula than the precondition we want to
accept as input. In this case, it is our tool that will check the
implication between what we want and what is necessary for our code to
be valid. This is the part $P \Rightarrow WP$.

We can illustrate this with the following code. Note that here the code
could be proved by WP without the weakening and strengthening of
properties because the code is very simple, it is just to illustrate the
rule of consequence.



\begin{CodeBlock}{c}
/*@
  requires P: 2 <= a <= 8;
  ensures  Q: 0 <= \result <= 100 ;
  assigns  \nothing ;
*/
int constrained_times_10(int a){
  //@ assert P_imply_WP: 2 <= a <= 8 ==> 1 <= a <= 9 ;
  //@ assert WP:         1 <= a <= 9 ;

  int res = a * 10;

  //@ assert SQ:         10 <= res <= 90 ;
  //@ assert SQ_imply_Q: 10 <= res <= 90 ==> 0 <= res <= 100 ;

  return res;
}
\end{CodeBlock}



(Note: We have omitted here the control of integer overflow.)



Here we want to have a result between 0 and 100. But we know that the
code will not produce a result outside the bounds of 10 and 90. So we
strengthen the postcondition with an assertion that at the end
\texttt{res}, the result, is between 0 and 90. The calculation of the
weakest precondition of this property together with the assignment
\texttt{res = 10 * a} yields a weaker precondition
\texttt{1 <= a <= 9} and we know that
\texttt{2 < = a <= 8} gives us the desired
guarantee.



When there are difficulties to carry out a proof on more complex code,
then it is often helpful to write assertions that produce stronger, yet
easier to verify, postconditions. Note that in the previous code, the
lines \texttt{P\_imply\_WP} and\texttt{SQ\_imply\_Q} are never used
because this is the default reasoning of WP. They are just here for
illustrating the rule.


\levelThreeTitle{Bonus Stage - Constancy rule}
\label{l3:statements-basic-constancy}


Certain sequences of instructions may concern and involve different
variables. Thus, we may initialize and manipulate a certain number of
variables, begin to use some of them for a time, before using other
variables. When this happens, we want our tool to be concerned only with
variables that are susceptible to change in order to obtain the simplest
possible properties.



The rule of inference that defines this reasoning is the following:




\begin{center}
$\dfrac{\{P\}\quad c\quad \{Q\}}{\{P \wedge R\}\quad c\quad \{Q \wedge R\}}$
\end{center}


where $c$ does not modify any input variable in $R$. In other words:
``To check the triple, let's get rid of the parts of the formula that
involve variables that are not influence by $c$ and prove the new
triple.'' However, we must be careful not to delete too much
information, since this could mean that we are not able to prove our
properties.


As an example, let us consider the following code (here gain, we ignore
potential integer overflows):



\begin{CodeBlock}{c}
/*@
  requires a > -99 ;
  requires b > 100 ;
  ensures  \result > 0 ;
  assigns  \nothing ;
*/
int foo(int a, int b){
  if(a >= 0){
    a++ ;
  } else {
    a += b ;
  }
  return a ;
}
\end{CodeBlock}


If we look at the code of the \CodeInline{if} block, we notice that it does
not use the variable \CodeInline{b}. Thus, we can completely omit the
properties about \CodeInline{b} in order to prove that \CodeInline{a} will be
strictly greater than 0 after the execution of the block:



\begin{CodeBlock}{c}
/*@
  requires a > -99 ;
  requires b > 100 ;
  ensures  \result > 0 ;
  assigns  \nothing ;
*/
int foo(int a, int b){
  if(a >= 0){
    //@ assert a >= 0; //et rien à propos de b
    a++ ;
  } else {
    a += b ;
  }
  return a ;
}
\end{CodeBlock}



On the other hand, in the \CodeInline{else} block, even if \CodeInline{b} were
not modified, formulating properties only about \CodeInline{a} would render
a proof impossible for humans. The code would be:



\begin{CodeBlock}{c}
/*@
  requires a > -99 ;
  requires b > 100 ;
  ensures  \result > 0 ;
  assigns  \nothing ;
*/
int foo(int a, int b){
  if(a >= 0){
    //@ assert a >= 0; // et rien à propos de b
    a++ ;
  } else {
    //@ assert a < 0 && a > -99 ; // et rien à propos de b
    a += b ;
  }
  return a ;
}
\end{CodeBlock}



In the \CodeInline{else} block, knowing that\CodeInline{a} lies between -99 and
0, but knowing nothing about \CodeInline{b}, we could hardly know if the
operation \CodeInline{a + = b} produces a result that is greater than 0.

The WP plug-in will, of course, prove the function without problems,
since it produces by itself the properties that are necessary for the
proof. In fact, the analysis which variables are necessary or not (and,
consequently, the application of the constancy rule) is conducted
directly by WP.

Let us finally remark that the constancy rule is an instance of the
consequence rule:




\begin{center}
$\dfrac{P \wedge R \Rightarrow P \quad \{P\}\quad c\quad \{Q\} \quad Q \Rightarrow Q \wedge R}{\{P \wedge R\}\quad c\quad \{Q \wedge R\}}$
\end{center}


If the variables of $R$ have not been modified by the operation
(which, on the other hand, may modify the variables of $P$ to produce
$Q$), then the properties $P \wedge R \Rightarrow P$ and
$Q \Rightarrow Q \wedge R$ hold.



\levelThreeTitle{Exercices}



\levelFourTitle{A serie of assignment}


Compute by hand the weakest precondition of the following program:


\begin{CodeBlock}{c}
/*@
  requires -10 <= x <= 0 ;
  requires 0 <= y <= 5 ;
  ensures -10 <= \result <= 10 ;
*/
int function(int x, int y){
  int res ;
  y += 10 ;
  x -= 5 ;
  res = x + y ;
  return res ;
}
\end{CodeBlock}


Deduce that the program is correct with respect to its contract using the
right rule.


\levelFourTitle{Empty ``then'' branch in conditional}


We previously shown that in a condition when the ``else'' branch is empty,
that the postcondition of the complete conditional is already verified if
with the conjunction of the negation of the condition and the precondition.
For both of the following question, we only need the inference rules and
no WP calculus.

Show that when, instead, the ``then'' branch is empty, the conjunction of
the condition and the precondition must implies the post-condition of the
``else'' branch.

Show that when both branches are empty, the overall condition is just a
skip operation.


\levelFourTitle{Short circuit}


C compilers implement short circuit for conditions. For example, that
means that a code like this one (\textbf{without ``else'' block}) :


\begin{CodeBlock}{c}
if(cond1 && cond2){
  // code
}
\end{CodeBlock}



can be written as:



\begin{CodeBlock}{c}
if(cond1){
  if(cond2){    
    // code
  }
}
\end{CodeBlock}



Show that on those two source code, the weakest precondition calculus
generates an equivalent weakest precondition for equivalent for any code
in the ``then'' block. Note that we assume the conditions to be pure
expressions (without side-effects).



\levelFourTitle{A larger program}


Compute by hand the weakest precondition of the following program:


\begin{CodeBlock}{c}
/*@ 
  requires -5 <= y <= 5 ; 
  requires -5 <= x <= 5 ; 
  ensures  -15 <= \result <= 25 ;
*/
int function(int x, int y){
  int res ;

  if(x < 0){
    x = 0 ;
  }
  
  if(y < 0){
    x += 5 ;
  } else {
    x -= 5 ;
  }
  
  res = x - y ;

  return res ;
}
\end{CodeBlock}


Deduce that the program is correct with respect to its contract using the
right rule.
