Loops need a particular treatment in deductive verification of
programs. These are the only control structures that will require
important work from us. We cannot avoid this because without loops, it
is difficult to write and prove interesting programs.



Before we look at the way we specify loop, we can answer to a rightful
question: why are loops so complex?



\levelThreeTitle{Induction and invariant}


The nature of loops makes their analysis complex. When we perform our
reasoning, we need a rule to determine the precondition from a given sequence
of instructions and a post-condition. Here, the problem is that we cannot
\emph{a priori} deduce how many times a loop will iterate, and consequently, we
cannot know how many times variables will be modified.



We will then proceed using an inductive reasoning. We have to find a
property that is true before we start to execute the loop and that, if
it is true at the beginning of an iteration, remains true at the end
(and that is consequently true at the beginning of the next iteration).



This type of property is called a loop invariant. A loop invariant is a
property that must be true before and after each loop iteration. And more
precisely, each time the condition of the loop is checked. For example with
the following loop:



\begin{CodeBlock}{c}
for(int i = 0 ; i < 10 ; ++i){ /* */ }
\end{CodeBlock}



The property $0 <= i <= 10$ is a loop invariant. The property
$-42 <= i <= 42$ is also an invariant (even if it is far less
precise). The property $0 < i <= 10$ is not an invariant because it is
not true at the beginning of the execution of the loop. The property
$0 <= i < 10$ \textbf{is not a loop invariant}, it is not true at the
end of the last iteration that sets the value of \texttt{i} to $10$.

To verify an invariant $I$, WP will then produce the following
``reasoning'':

\begin{itemize}
\item
  verify that $I$ is true at the beginning of the loop (establishment)
\item
  verify that if $I$ is true before an iteration, then $I$ is true
  after (preservation).
\end{itemize}

\levelFourTitle{Formal - Inference rule}

Let us note the invariant $I$, the inference rule corresponding to
loops is defined as follows:

\begin{center}
$\dfrac{\{I \wedge B \}\ c\ \{I\}}{\{I\}\ while(B)\{c\}\ \{I \wedge \neg B\}}$
\end{center}

And the weakest precondition calculus is the following:

\begin{center}
$wp(while (B) \{ c \}, Post) := I \wedge ((B \wedge I) \Rightarrow wp(c, I)) \wedge ((\neg B \wedge I) \Rightarrow Post)$
\end{center}

Let us detail this formula:

\begin{itemize}
\item
  \begin{enumerate}
  \def\labelenumi{(\arabic{enumi})}
  \item
    the first $I$ corresponds to the establishment of the invariant,
    in layman's terms, this is the ``precondition'' of the loop,
  \end{enumerate}
\item
  the second part of the conjunction
  ($(B \wedge I) \Rightarrow wp(c, I)$) corresponds to the
  verification of the operation performed by the body of the loop:

  \begin{itemize}
  \item
    the precondition that we know of the loop body (let us note $KWP$,
    ``Known WP'') is ($KWP = B \wedge I$). That is the fact we have
    entered the loop ($B$ is true), and that the invariant is verified
    at this moment ($I$, is true before we start the loop by (1), and
    we want to verify that it will be true at the end of the body of the
    loop in (2)),
  \item
    \begin{enumerate}
    \def\labelenumi{(\arabic{enumi})}
    \setcounter{enumi}{1}
    \item
      it remains to verify that $KWP$ implies the actual precondition*
      of the body of the loop ($KWP \Rightarrow wp(c, Post)$). What we
      want at the end of the loop is the preservation of the invariant
      $I$ ($B$ is maybe not true anymore however), formally
      $KWP \Rightarrow wp(c, I)$, that is to say
      $(B \wedge I) \Rightarrow wp(c, I)$,
    \end{enumerate}
  \item
    * it corresponds to the application of the consequence rule
    previously explained.
  \end{itemize}
\item
  finally, the last part ($(\neg B \wedge I) \Rightarrow Post$)
  expresses the fact that when the loop ends($\neg B$), and the
  invariant $I$ has been maintained, it must imply that the wanted
  postcondition of the loop is valid.
\end{itemize}

In this computation, we can notice that the $wp$ function does not
indicate any way to obtain the invariant $I$. We have to specify
ourselves this property about our loops.



\levelFourTitle{Back to the WP plugin}


There exist tools that can infer invariant properties (provided that
these properties are simple, automatic tools remain limited). This is not
the case for WP. We will have to manually annotate our programs to
specify the invariant of each loop. To find and write invariants for our
loops will always be the hardest part of our work when we want to prove
programs.



Indeed, when there are no loops, the weakest precondition calculus
function can automatically provide the verifiable properties of our
programs, this is not the case for loop invariant properties for which we
do not have computation procedures. We have to find and express them
correctly, and depending on the algorithm, they can be quite subtle and
complex.



In order to specify a loop invariant, we add the following annotations
before the loop:



\CodeBlockInput{c}{first_loop-1.c}



\begin{Warning}
  \textbf{REMINDER} : The invariant is: i \textbf{<=} 30 !
\end{Warning}


Why? Because along the loop, \texttt{i} will be comprised between 0 and
\textbf{included} 30. 30 is indeed the value that allows us to leave the
loop. Moreover, one of the properties required by the weakest
precondition calculus is that when the loop condition is invalidated, by
knowing the invariant, we can prove the postcondition (Formally
$(\neg B \wedge I) \Rightarrow Post$).

The postcondition of our loop is \texttt{i == 30} and must be implied
by $\neg$ \texttt{i < 30} $\wedge$
\texttt{0 <= i <= 30}. Here, it is true since:
\texttt{i >= 30 \&\& 0 <= i <= 30 ==> i == 30}.
On the opposite, if we exclude the equality to 30, the postcondition
would be unreachable.



Again, we can have a look at the list of proof obligations in ``WP
Goals'':



\image{i_30-1}[Proof obligations generated to verify our loop]


We note that WP produces two different proof obligations: the
establishment of the invariant and its preservation. WP produces exactly
the reasoning we previously described to prove the assertion. In recent
versions of Frama-C, Qed has become particularly aggressive and
powerful, and the generated proof obligation does not show these details
(showing directly ``True''). Using the option \texttt{-wp-no-simpl} at
start, we can however see these details:



\image{i_30-2}[Proof of the assertion, knowing the invariant and the
  invalidation of the loop condition]


But is our specification precise enough?



\CodeBlockInput{c}{first_loop-2.c}



And the result is:



\image{i_30-3}[Side effects, again]


It seems not.



\levelThreeTitle{The assigns clause \ldots{} for loops}


In fact, considering loops, WP \textbf{only} reasons about what is
provided by the user to perform its reasoning. And here, the invariant
does not specify anything about the way the value of \CodeInline{h} is
modified (or not). We could specify the invariant of all program
variables, but it would be a lot of work. ACSL simply allows to add
\CodeInline{assigns} annotations for loops. Any other variable is considered
to keep its old value. For example:



\CodeBlockInput{c}{first_loop-3.c}



This time, we can establish the proof that the loop correctly behaves.
However, we cannot prove that it terminates. The loop invariant is not
enough to perform such a proof. For example, in our program, we could
modify the loop, removing the loop body:



\begin{CodeBlock}{c}
/*@
  loop invariant 0 <= i <= 30;
  loop assigns i;
*/
while(i < 30){
   
}
\end{CodeBlock}



The invariant is still verified, but we cannot prove that the loop ends:
it is infinite.



\levelThreeTitle{Partial correctness and total correctness - Loop variant}


In deductive verification, we find two types of correctness, the partial
correctness and the total correctness. In the first case, the
formulation of the correctness property is ``if the precondition is
valid, and \textbf{if} the computation terminates, then the
postcondition is valid''. In the second case, ``if the precondition is
valid, \textbf{then} the computation terminates and the postcondition is
valid''. By default, WP considers only partial correctness:



\CodeBlockInput{c}{infinite.c}



If we try to verify this code activating the verification of absence of
RTE, we get this result:



\image{infinite}[Proof of false by non termination]


The assertion ``False'' is proved! For a very simple reason: since the
condition of the loop is ``True'' and no instruction of the loop body
allows to leave the loop, it will not terminate. As we are proving the
code with partial correctness, and as the execution does not terminate,
we can prove anything about the code that follows the non terminating
part of the code. However, if the termination is not trivially provable,
the assertion will probably not be proved.



\begin{Information}
  Note that a (provably) unreachable assertion is always proved to be true:
  \inlineImage{goto_end}
  And this is also the case when we trivially know that an instruction
  produces a runtime error (for example dereferencing \CodeInline{NULL}), or
  inserting ``False'' in post-condition as we have already seen with
  \CodeInline{abs} and the parameter \CodeInline{INT\_MIN}.
\end{Information}


In order to prove the termination of a loop, we use the notion of loop
variant. The loop variant is not a property but a value. It is an
expression that involves the element modified by the loop and that
provides an upper bound to the number of iterations that have to be
executed by the loop at each iteration. Thus, it is an expression
greater or equal to 0, and that strictly decreases at each loop
iteration (this will also be verified by induction by WP).


If we take our previous example, we add the loop variant with this
syntax:



\CodeBlockInput{c}{first_loop-4.c}



Again, we can have a look at the generated proof obligations:



\image{i_30-4}[Our loop entirely specified and proved]


The loop variant generates two proof obligations: verify that the value
specified in the variant is positive, and prove that it strictly
decreases during the execution of the loop. And if we delete the line of
code that increments \texttt{i}, WP cannot prove anymore that
\texttt{30\ -\ i} strictly decreases.

We can also note that being able to give a loop invariant does not
necessarily induce that we can give the exact number of remaining
iterations of the loop, as we do not always have a so precise knowledge
of the behavior of the program. We can for example build an example like
this one:



\CodeBlockInput{c}{random_loop.c}



Here, at each iteration, we decrease the value of the variable
\texttt{i} by a value comprised between 1 and \texttt{i}. Thus, we can
ensure that the value of \texttt{i} is positive and strictly decreases
during each loop iteration, but we cannot say how many loop iteration
will be executed.



The loop variant is then only an upper bound on the number of iteration,
not an expression of their exact number.



\levelThreeTitle{Create a link between post-condition and invariant}


Let us consider the following specified program. Our goal is to prove
that this function returns the old value of \texttt{a} plus 10.



\CodeBlockInput{c}{add_ten-0.c}



The weakest precondition calculus does not allow to deduce information
that is not part of the loop invariant. In a code like:



\begin{CodeBlock}{c}
/*@
    ensures \result == \old(a) + 10;
*/
int add_ten(int a){
    ++a;
    ++a;
    ++a;
    //...
    return a;
}
\end{CodeBlock}


By reading the instructions backward
from the postcondition, we always keep all knowledge about \texttt{a}. On
the opposite, as we previously mentioned, outside the loop, WP only
considers the information provided by the invariant. Consequently, our
``add\_10'' function cannot be proved: the invariant does not say anything
about \texttt{a}. To create a link between the postcondition and the
invariant, we have to add this knowledge. See, for example:



\CodeBlockInput{c}{add_ten-1.c}



\begin{Information}
  This need can appear as a very strong constraint. This is not really the
  case. There exists strongly automated analysis that can compute loop
  invariant properties. For example, without a specification, an abstract
  interpretation would easily compute \CodeInline{0 <= i <= 10}
  and \CodeInline{old(a) <= a <= \textbackslash{}old(a)+10}.
  However, it is often more difficult to compute the relations
  that exist between the different variables of a program, for
  example the equality expressed by the invariant we have
  added.
\end{Information}


\levelThreeTitle{Early termination of loop}



A loop invariant must be true each the condition of the loop is checked. In
fact, that also means that it must be true before an iteration, and after
each \textbf{complete} iteration. Let us illustrate on an example this
important idea.



\CodeBlockInput{c}{first_loop-non-term-1.c}



In this function, when the loop reach the index 30, we break the loop before
checking the condition again. While the invariant is verified, let us show
that we can now further constrain it.



\CodeBlockInput{c}{first_loop-non-term-2.c}



Here we can see that we have excluded 30 from the range of values of
\CodeInline{i} and everything is still verified by WP. This is particularly
interesting because it does not apply only to the invariant, none of the loop
properties need to be verified in this last iteration. For example, we can
write this code that is still verified:



\CodeBlockInput{c}{first_loop-non-term-3.c}



We can see that we can write the variable \CodeInline{h} even if it is not
listed in the \CodeInline{loop assigns} clause, and that we can give the
value 42 to \CodeInline{i} which does not respect the invariant, and also
makes the expression of the variant negative. In fact, everything happens
as if we had written:



\CodeBlockInput{c}{first_loop-non-term-4.c}



This is an interesting scheme. It basically corresponds to any algorithm
that search, using a loop, a particular condition respected by an element
in a given data-structure and stops when it founds it to perform some
operations that are thus not really part of the loop. From a verification
point of view, it allows us to simplify the contract of the loop: we know
that the (potentially complex) operations performed just before we stop do
not need to be considered when designing the contract.
