It is time to enter the heart of the matter. Rather than starting with
basic notions of the C language, as we would do for a tutorial about C,
we will start with functions. First because it is necessary to be able
to write functions before starting this tutorial (to be able to prove
that a code is correct, being able to write it correct is required), and
then because it will allow us to directly prove some programs.



After this part about functions, we will on the opposite focus on simple
notions like assignments or conditional structures, to understand how our
tool really works.



In order to be able to prove that a code is valid, we first need to
specify what we expect of it. Building the proof of our program consists
in ensuring that the code we wrote corresponds to the specification that
describes its job. As we previously said, Frama-C provides the ACSL
language to let the developer write contracts about each function (but
that is not its only purpose, as we will see later).



\begin{levelTwo}
  {Contract definition}
  {contract}
\end{levelTwo}

\begin{levelTwo}
  {Well specified function}
  {well-specified}
\end{levelTwo}

\begin{levelTwo}
  {Behaviors}
  {behaviors}
\end{levelTwo}

\begin{levelTwo}
  {WP Modularity}
  {modularity}
\end{levelTwo}


\horizontalLine
\newpage


During this part of the tutorial, we have studied how we can specify
functions using contracts, composed of a pre and a postcondition, as
well as some features ACSL provides to express those properties. We have
also seen why it is important to be precise when we specify and how the
introduction of behaviors can help us to write more understandable and
concise specification.

However, we do not have studied one important point: the specification
of loops. Before that, we should have a closer look at the way WP works.
