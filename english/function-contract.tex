It is time to enter the heart of the matter. Rather than starting with
basic notions of the C language, as we would do for a tutorial about C,
we will start with functions. First because it is necessary to be able
to write functions before starting this tutorial (to be able to prove
that a code is correct, being able to write it correct is required), and
then because it will allow us to directly prove some programs.



After this part about functions, we will on the opposite focus on simple
notions like assignments or conditonal structures, to understand how our
tool really works.



In order to be able to prove that a code is valid, we first need to
specify what we expect of it. Building the proof of our program consists
in ensuring that the code we wrote corresponds to the specification that
describes its job. As we previously said, Frama-C provides the ACSL
language to let the developer write contracts about each function (but
that is not its only purpose, as we will see later).



\levelTwoTitle{Contract definition}

The goal of a function contract is to state the conditions under which
the function will execute. That is to say, what the function expect from
the caller to ensure that it will correctly behave: the \textbf{precondition},
the notion of ``correctly behave'' being itself defined in the contract
by the \textbf{postcondition}.



These properties are expressed with ACSL, the syntax is relatively
simple if one have already developed in C language since it shares most
of the syntax of boolean expressions in C. However, it also provides:



\begin{itemize}
\item
  some logic constructs and connectors that do not exists in C, to ease
  the writing of specifications,
\item
  built-in predicates to express properties that are useful about C
  programs (for example: a valid pointer),
\item
  as well as some primitive types for the logic that are more general
  than primitive C types (for example: mathematical integer).
\end{itemize}



We will introduce along this tutorial a large part of the notations
available in ACSL.



ACSL specifications are introduced in our source code using annotations.
Syntactically, a function contract is integrated in the source code with
this syntax:



\begin{CodeBlock}{c}
/*@
  //contract
*/
void foo(int bar){

}
\end{CodeBlock}



Notice the \CodeInline{@} at the beginning of the comment block, this
indicates to Frama-C that what follows are annotations and not a comment
block that it should simply ignore.



Now, let us have a look to the way we express contracts, starting with
postconditions, since it is what we want our function to do (we will
later see how to express precondition).



\levelThreeTitle{Postcondition}


The postcondition of a function is introduced with the clause \CodeInline{ensures}. 
We will illustrate its use with the following function
that returns the absolute value of an input. One of its postconditions
is that the result (which is denoted with the keyword
\CodeInline{\textbackslash{}result}) is greater or equal to 0.



\begin{CodeBlock}{c}
/*@
  ensures \result >= 0;
*/
int abs(int val){
  if(val < 0) return -val;
  return val;
}
\end{CodeBlock}


(Notice the \CodeInline{;} at the end of the line, exactly as we do in C).

But that it is not the only property to verify, we also need to specify
the general behavior of a function returning the absolute value. That
is: if the value is positive or 0, the function returns the same value,
else it returns the opposite of the value.

We can specify multiple postconditions, first by combining them with a
\CodeInline{\&\&} as we do in C, or by introducing a new \CodeInline{ensures}
clause, as we illustrate here:



\begin{CodeBlock}{c}
/*@
  ensures \result >= 0;
  ensures (val >= 0 ==> \result == val ) && 
          (val <  0 ==> \result == -val);
*/
int abs(int val){
  if(val < 0) return -val;
  return val;
}
\end{CodeBlock}



This specification is the opportunity to present a very useful logic
connector provided by ACSL and that does not exist in C: the implication
$A \Rightarrow B$, that is written \CodeInline{A ==> B} in
ACSL. The truth table of the implication is the following:

Cette spécification est l'opportunité de présenter un connecteur logique 
très utile que propose ACSL mais qui n'est pas présent en C : 
l'implication $A \Rightarrow B$, que l'on écrit en ACSL \CodeInline{A ==> B}.
La table de vérité de l'implication est la suivante :



\begin{longtabu}{|c|c|c|} \hline
$A$ & $B$ & $A \Rightarrow B$ \\ \hline
$F$ & $F$ & $T$ \\ \hline
$F$ & $T$ & $T$ \\ \hline
$T$ & $F$ & $F$ \\ \hline
$T$ & $T$ & $T$ \\ \hline
\end{longtabu}



That means that an implication $A \Rightarrow B$ is true in two cases:
either $A$ is false (and in this case, we do not check the value of
$B$), or $A$ is true and then $B$ must also be true. The idea
finally being ``I want to know if when $A$ is true, $B$ also is. If
$A$ is false, I don't care, I consider that the complete formula is
true''.



Another available connector is the equivalence $A \Leftrightarrow B$
(written \CodeInline{A <==> B} in ACSL), and it is
stronger. It is conjunction of the implication in both ways
$(A \Rightarrow B) \wedge (B \Rightarrow A)$. This formula is true in
only two cases: $A$ and $B$ are both ture, or false (it can be seen
as the negation of the exclusive or).



\begin{Information}
  Let's give a quick reminder about all
  truth tables of usual logic connectors in first order logic
  ($\neg$ = \CodeInline{!}, $\wedge$ = \CodeInline{\&\&}, $\vee$ = \CodeInline{||}) :

\begin{longtabu}{|c|c|c|c|c|c|c|} \hline
$A$ & $B$ & $\neg A$ & $A \wedge B$ & $A \vee B$ & $A \Rightarrow B$ & $A \Leftrightarrow B$ \\ \hline
$F$ & $F$ & $T$ & $F$ & $F$ & $T$ & $T$ \\ \hline
$F$ & $T$ & $T$ & $F$ & $T$ & $T$ & $F$ \\ \hline
$T$ & $F$ & $F$ & $F$ & $T$ & $F$ & $F$ \\ \hline
$T$ & $T$ & $F$ & $T$ & $T$ & $T$ & $T$ \\ \hline
\end{longtabu}
\end{Information}


We can come back to our specification. As our files become longer and
contains a lot of specifications, if can be useful to name the
properties we want to verify. So, in ACSL, we can specify a name
(without spaces) followed by a \CodeInline{:}, before stating the property.
It is possible to put multiple levels of names to categorize our
properties. For example, we could write this:



\begin{CodeBlock}{c}
/*@
  ensures positive_value: function_result: \result >= 0;
  ensures (val >= 0 ==> \result == val) && 
          (val < 0 ==> \result == -val);
*/
int abs(int val){
  if(val < 0) return -val;
  return val;
}
\end{CodeBlock}



In most of this tutorial, we will not name the properties we want to
prove, since they will be generally quite simple and we will not have
too many of them, names would not give us much information.

We can copy and paste the function \CodeInline{abs} and its specification in
a file \CodeInline{abs.c} and use Frama-C to determine if the implementation
is correct against the specification. We can start the GUI of Frama-C
(it is also possible to use the command line interface of Frama-C but we
will not use it during this tutorial) by using this command line:



\begin{CodeBlock}{bash}
$ frama-c-gui
\end{CodeBlock}



Or by opening it from the graphical environment.



It is then possible to click on the button ``Create a new session from
existing C files'', files to analyze can be selected by double-clicking
it, the OK button ending the selection. Then, adding other files will be
done by clicking Files > Source Files.



Notice that it is also possible to directly open file(s) from the
terminal command line passing them to Frama-C as parameter(s):



\begin{CodeBlock}{bash}
$ frama-c-gui abs.c
\end{CodeBlock}



\image{2-1-1-abs-1.png}[The side panel gives the files and functions tree]


The window of Frama-C opens and in the panel dedicated to files and
functions, we can select the function \texttt{abs}. At each
\texttt{ensures} line, we can see a blue circle, it indicates that no
verification has been attempted for these properties.



We ask the verification of the code by right-clicking the name of the
function and ``Prove function annotations by WP'':



\image{2-1-1-abs-2.png}[Start the verification of \texttt{abs} with WP]


We can see that blue circles become green bullets, indicating that the
specification is indeed ensured by the program. We can also prove
properties one by one by right-clicking on them and not on the name of
the function.



But is our code really bug free ? WP gives us a way to ensure that a
code respects a specification, but it does not check for runtime errors
(RTE). This is provided by another plugin that we will use here and that
is called RTE. Its goal is to add, in the program, some controls to
ensure that the program cannot create runtime errors (integer overflow,
invalid pointer dereferencing, 0 division, etc).



To active these controls, we check the box pointed by the screenshot (in
the WP panel). We can also ask Frama-C to add them in a function by
right-clicking on its name and then click ``Insert RTE guards''.



\image{2-1-1-abs-3.png}[Activate runtime error absence verification]


Finally, we execute the verification again (we can also click on the
``Reparse'' button of the toolbar, it will deletes existing proofs).

We can then see that WP fails to prove the absence of arithmetic
underflow for the computation of \CodeInline{-val}. And, indeed, on our
architectures, -\CodeInline{INT\_MIN} ($-2^{31}$) > \CodeInline{INT\_MAX} ($2^{31}-1$).



\image{2-1-1-abs-4.png}[Incomplete proof of \texttt{abs}]


\begin{Information}
We can notice that the underflow risk
is real for us, since our computers (for which the
configuration is detected by Frama-C) use the 
\externalLink{Two's complement}{https://en.wikipedia.org/wiki/Two\%27s_complement}
implementation of integers, which do not define
the behavior of under and overflows.
\end{Information}


Here, we can see another type of ACSL annotation. By the line
\CodeInline{//@ assert propriete ;} , we can ask the verification of
property at a particular program point. Here, RTE inserted for us, since
we have to verify that \CodeInline{-val} does not produce an underflow, but
we can also add such an assertion manually in the source code.



In this screenshot, we can see two new colors for our bullets:
green+brown and orange.



If the proof has not been entirely redone after adding the runtime error
checks, these bullets must still be green. Indeed, the corresponding
proofs have been realized without the knowledge of the property in the
assertion, so they cannot rely on this unproved property.



When WP transmits a proof obligation to an automatic prover, it
basically transmits two types of properties : $G$, the goal, the
property that we want to prove, and $A_1$ \ldots{} $A_n$, the
different assumptions we can have about the state of the memory at the
program point where we want to verify $G$. However, it does not
receive (in return) the properties that have been used by the prover to
validate $G$. So, if $A_3$ is an assumption, and if WP did not
succeed in getting a proof of $A_3$, it indicates that $G$ is true,
but only if we succeed in proving $A_3$.



The orange color indicates that no prover could determine if the
property is verified. There are two possibles reasons:



\begin{itemize}
\item the prover did not have enough information,
\item the prover did not have enough time to compute the proof and
  encountered a timeout (which can be configured in the WP panel).
\end{itemize}


In the bottom panel, we can select the ``WP Goals'' tab, it shows the
list of proof obligations, and for each prover the result is symbolized
by a logo that indicates if the proof has been tried and if it
succeeded, failed or encountered a timeout (here we can see a try with
Z3 where we had a timeout on the proof of absence of RTE).


\image{2-1-1-abs-5.png}[Proof obligations panel of WP for \CodeInline{abs}]


In the first column, we have the name of the function the proof
obligation belongs to. The second column indicates the name of proof
obligation. For example here, our postcondition is named
\CodeInline{Post-condition 'positive\_value,function\_result'}, we can notice
that if we select a property in this list, it is also highlighted in the
source code. Unnamed properties are automatically named by WP with the
kind of wanted property. In the third column, we see the memory model
that is used for the proof, we will not talk about it in this tutorial.
Finally, the last columns represent the different provers available
through WP.



In these provers, the first element is Qed. It is not really a prover.
In fact, if we double-click on the property ``absence of underflow''
(highlight in blue in the last screenshot), we can see corresponding
proof obligation:



\image{2-1-1-abs-6.png}[Proof obligation associated to the verification of absence of
  underflow in \CodeInline{abs}]


This is the proof obligation generated by WP about our property and our
program, we do need to understand everything here, but we can get the
general idea. It contains (in the ``Assume'' part) the assumptions that
we have specified and those that have been deduced by WP from the
instructions of the program. It also contains (in the ``Prove'' part)
the property that we want to verify.



What does WP do using these properties ? In fact, it transforms them
into a logic formula and then asks to different provers if it is
possible to satisfy this formula (to find for each variable, a value
that can make the formula true), and it determines if the property can
be proved. But before sending the formula to provers, WP uses a module
called Qed, which is able to perform different simplifications about it.
Sometimes, as this is the case for the other properties about
\CodeInline{abs}, these simplifications are enough to determine that the
property is true, in such a case, WP do not need the help of the
automatic solvers.



When automatic solvers cannot ensure that our properties are verified,
it it sometimes hard to understand why. Indeed, provers are generally
not able to answer something other than ``yes'', ``no'' or ``unknown'',
they are not able to extract the reason of a ``no'' or an ``unknown''.
There exists tools that can explore a proof tree to extract this type of
information, currently Frama-C do not provide such a tool. Reading proof
obligations can sometimes be helpful, but it requires a bit of practice
to be efficient. Finally, one of the best way to understand the reason
why a proof fails is to try to do it interactively with Coq. However, it
requires to be quite comfortable with this language to not being lost
facing the proof obligations generated by WP, since these obligations
need to encode some elements of the C semantics that can make them quite
hard to read.



If we go back to our view of proof obligations (see the squared button
in the last screenshot), we can see that our hypotheses are not enough
to determine that the property ``absence of underflow'' is true (which
is indeed currently impossible), so we need to add some hypotheses to
guarantee that our function will well-behave: a call precondition.



\levelThreeTitle{Precondition}



Preconditions are introduced using \CodeInline{requires} clauses. As we
could do with \CodeInline{ensures} clauses, we can compose logic expressions
and specify multiple preconditions:



\begin{CodeBlock}{c}
/*@
  requires 0 <= a < 100;
  requires b < a;
*/
void foo(int a, int b){
  
}
\end{CodeBlock}



Preconditions are properties about the input (and eventually about
global variables) that we assume to be true when we analyze the
function. We will verify that they are indeed true only at program
points where the function is called.



In this small example, we can also notice a difference with C in the
writing of boolean expressions. If we want to specify that \CodeInline{a} is
between 0 and 100, we do not have to write \CodeInline{0 <= a \&\& a < 100},
we can directly write \CodeInline{0 <= a < 100} and Frama-C will
perform necessary translations.



If we come back to our example about the absolute value, to avoid the
arithmetic underflow, it is sufficient to state that \CodeInline{val} must
be strictly greater than \CodeInline{INT\_MIN} to guarantee that the
underflow will never happen. We then add it as a precondition of the
function (notice that it is also necessary to include the header where
\CodeInline{INT\_MIN} is defined):



\begin{CodeBlock}{c}
#include <limits.h>

/*@
  requires INT_MIN < val;

  ensures \result >= 0;
  ensures (val >= 0 ==> \result == val) && 
          (val < 0 ==> \result == -val);
*/
int abs(int val){
  if(val < 0) return -val;
  return val;
}
\end{CodeBlock}



\begin{Warning}
  Reminder: The Frama-C GUI does not allow code source modification.
\end{Warning}


\begin{Information}
  For Frama-C NEON and older, the
  pre-processing of annotations is not activated by default. We
  have to start Frama-C with the option \CodeInline{-pp-annot}:

\begin{CodeBlock}{bash}
$ frama-c-gui -pp-annot file.c
\end{CodeBlock}
\end{Information}


Once we have modified the source code with our precondition, we click on
``Reparse'' and we can ask again to prove our program. This time,
everything is validated by WP, our implementation is proved:



\image{2-1-2-abs-1.png}[Proof of \texttt{abs} performed]


We can also verify that a function that would call \CodeInline{abs}
correctly respects the required precondition:



\begin{CodeBlock}{c}
void foo(int a){
   int b = abs(42);
   int c = abs(-42);
   int d = abs(a);       // False : "a" can be INT_MIN
   int e = abs(INT_MIN); // False : the parameter must be strictly greater than INT_MIN
}
\end{CodeBlock}



\image{2-1-2-foo-1.png}[Precondition checking when calling \CodeInline{abs}]


We can modify this example by revering the last two instructions. If we
do this, we can see that the call \CodeInline{abs(a)} is validated by WP if
it is placed after the call \CodeInline{abs(INT\_MIN)}! Why?



We must keep in mind that the idea of the deductive proof is to ensure
that if preconditions are verified, and if our computation terminates,
then the post-conditon is verified.



If we give a function that surely breaks the precondition, we can deduce
that the postconditon is false. Knowing this, we can prove absolutely
everything because this ``false'' becomes an assumption of every call
that follows. Knowing false, we can prove everything, because if we have
a proof of false, then false is true, as well as true is true. So
everything is true.



Taking our modified program, we can convince ourselves of this fact by
looking at proof obligations generated by WP for the bad call and the
subsequent call that becomes verified:



\image{2-1-2-foo-2.png}[Generated proof obligation for the bad call]


\image{2-1-2-foo-3.png}[Generated proof obligation for the call that follows]


We can notice that for function calls, the GUI highlights the execution
path that leads to the call for which we want to verify the preconditon.
Then, if we have a closer look to the call \CodeInline{abs(INT\_MIN)}, we
can notice that, simplifying, Qed deduced that we try to prove
``False''. Consequently, the next call \CodeInline{abs(a)} receives in its
assumptions the property ``False''. This is why Qed can immediately
deduce ``True''.



The second part of the question is then: why our first version of the
calling function (\CodeInline{abs(a)} and then \CodeInline{abs(INT\_MIN)}) did
not have the same behavior, indicating a proof failure on the second
call? The answer is simply that the call \CodeInline{abs(a)} can, or not,
produce an error, whereas \CodeInline{abs(INT\_MIN)} necessarily leads to an
error. So, while \CodeInline{abs(INT\_MIN)} necessarily gives us the
knowledge of ``false'', the call \CodeInline{abs(a)} does not, since it can
succeed.



Produce a correct specification is then crucial. Typically, by stating
false precondition, we can have the possibility to create a proof of
false:



\begin{CodeBlock}{c}
/*@
  requires a < 0 && a > 0;
  ensures  \false;
*/
void foo(int a){

}
\end{CodeBlock}


If we ask WP to prove this function, it will accept it without a problem
since the assumption we give in precondition is necessarily false.
However, we will not be able to give an input that respects the
precondition so we will be able to detect this problem by carefully
reading what we have specified.



Some notions we will see in this tutorial can expose us to the
possibility to introduce subtle incoherence. So, we must always be
careful specifying a program.



\levelFourTitle{Finding the right preconditions}


Finding the right preconditions for a function is sometimes hard. The
most important idea is to determine these preconditions without taking
in account the content of the function (at least, in a first step), in
order to avoid building a specification that would contain the same bugs
currently existing in the source code, for example taking in account an
erroneous conditional structure. In fact, it is generally a good
practice to work with someone else. One specifies the function and the
other implements it (even if they previously agreed on a common textual
specification).



Once these precondition has been stated, then we work on the
specifications that are due to the constraints of our language and our
hardware. For example, the absolute value do not really have a
precondition, this is our hardware that adds the condition we have given
in precondition due to the two's complement on which it relies.



\levelThreeTitle{Some elements about the use of WP and Frama-C}


In the two preceding sections, we have seen a lot of notions about the
use of the GUI to start proofs. In fact, we can ask WP to immediately
prove everything at Frama-C's startup with the option \texttt{-wp}:



\begin{CodeBlock}{bash}
$ frama-c-gui file.c -wp
\end{CodeBlock}



Which will collect all properties to be proved inside \texttt{file.c},
generate all proof obligations and try to discharge them.

About runtime-errors, it is generally advised to first verify the
program without generating RTE assertions, and then to generate them to
terminate the verification with WP. It allows WP to ``focus'' on the
functional properties in a first step without having in its knowledge
base purely technical properties, that are generally not useful for the
proof of functional properties. Again, it is possible to directly
produce this behavior using the command line:



\begin{CodeBlock}{bash}
$ frama-c-gui file.c -wp -then -rte -wp
\end{CodeBlock}



``Start Frama-C with WP, then create assertions to verify the absence of
RTE and start WP again''.



\levelTwoTitle{Well specified function}


\levelThreeTitle{Correctly write what we expect}


This is certainly the hardest part of our work. Programming is already
an effort that consists in writing algorithms that correctly respond to
our need. Specifying requests the same kind of work, except that we do
not try to express \emph{how} we respond to the need but \emph{what} is
exactly our need. To prove that our code implements what we need, we
must be able to describe exactly what we need.

From now, we will use an other example, the \texttt{max} function:



\CodeBlockInput{c}{max-0.c}



The reader could write and prove their own specification. We will start
using this one:



\CodeBlockInput[1][6]{c}{max-1.c}



If we ask WP to prove this code, it will succeed without any problem.
However, is our specification really correct? We can try to prove this
calling code:



\CodeBlockInput[8][14]{c}{max-1.c}



There, it will fail. In fact, we can go further by modifying the body of
the \CodeInline{max} function and notice that the following code is also
correct with respect to the specification:



\CodeBlockInput[1][8]{c}{max-2.c}



Our specification is too permissive. We have to be more precise. We do
not only expect the result to be greater or equal to both parameters,
but also that the result is one of them:



\CodeBlockInput[1][7]{c}{max-3.c}



\levelThreeTitle{Pointers}


If there is one notion that we permanently have to confront with in C
language, this is definitely the notion of pointer. Pointers are quite
hard to manipulate correctly, and they still are the main source of
critical bugs in programs, so they benefit of a preferential treatment
in ACSL.

We can illustrate with a swap function for C integers:



\CodeBlockInput{c}{swap-0.c}



\levelFourTitle{History of values in memory}


Here, we introduce a first built-in logic function of ACSL:
\CodeInline{\textbackslash{}old}, that allows us to get the old (that
is to say, before the call) value of a given element. So, our specification
defines that the function must ensure that after the call, the value of
\CodeInline{*a} is the old  value of \CodeInline{*b} and conversely.

The \CodeInline{\textbackslash{}old} function can only be used in the
postcondition of a function. If we need this type of information
somewhere else, we will use \CodeInline{\textbackslash{}at} that allows us
to express that we want the value of a variable at a particular program
point. This function receives two parameters. The first one is the variable
(or memory location) for which we want to get its value and the second one
is the program point (as a C label) that we want to consider.

For example, we could write:



\CodeBlockInput[2][6]{c}{at.c}



Of course, we can use any C label in our code, but we also have 6
built-in labels defined by ACSL that can be used:



\begin{itemize}
\item \CodeInline{Pre}/\CodeInline{Old}: value before function call,
\item \CodeInline{Post}: value after function call,
\item \CodeInline{LoopEntry}: value at loop entry
\item \CodeInline{LoopCurrent}: value at the beginning of the current step of
  the loop,
\item \CodeInline{Here}: value at the current program point.
\end{itemize}


\begin{Information}
  The behavior of \CodeInline{Here} is, in fact, the default behavior when we
  consider a variable. Its use with \CodeInline{\textbackslash{}at} will
  generally let us ensure that what we write is not ambiguous, and is more
  readable, when we express properties about values at different program
  points in the same expression.
\end{Information}


Whereas \CodeInline{\textbackslash{}old} can only be used in function
postconditions, \CodeInline{\textbackslash{}at} can be used anywhere.
However, we cannot use any program point with respect to the type
annotation we are writing. \CodeInline{Old} and \CodeInline{Post} are only
available in function postconditions, \CodeInline{Pre} and \CodeInline{Here}
are available everywhere. \CodeInline{LoopEntry} and \CodeInline{LoopCurrent}
are only available in the context of loops (which we will detail later
in this tutorial).



For the moment, we will not need \CodeInline{\textbackslash{}at} but it can
often be useful, if not essential, when we want to make our
specification precise.



\levelFourTitle{Pointers validity}


If we try to prove that the swap function is correct (comprising RTE
verification), our postcondition is indeed verified but WP failed to
prove some possibilities of runtime-error, since we perform access to
some pointers that we did not indicate to be valid pointers in the
precondition of the function.

We can express that the dereferencing of a pointer is valid using the
\CodeInline{\textbackslash{}valid} predicate of ACSL which receives the
pointer in input:



\CodeBlockInput[3][11]{c}{swap-1.c}



Once we have specified that the pointers we receive in input must be valid,
dereferencing is assured to not produce undefined behaviors.



As we will see later in this tutorial, \CodeInline{\textbackslash{}valid}
can take more than one pointer in parameter. For example, we can give it
an expression such as: \CodeInline{\textbackslash{}valid(p + (s .. e))} which means
``for all \CodeInline{i} between included \CodeInline{s} and \CodeInline{e},
\CodeInline{p+i} is a valid pointer''. This kind of expression will be
extremely useful when we will specify properties about arrays in
specifications.



If we have a closer look to the assertions that WP adds in the swap
function comprising RTE verification, we can notice that there exists
another version of the \CodeInline{\textbackslash{}valid} predicate, denoted
\CodeInline{\textbackslash{}valid\_read}. As opposed to \CodeInline{\textbackslash{}valid}, the
predicate \CodeInline{\textbackslash{}valid\_read} indicates that a pointer
can be dereferenced, but only to read the pointed memory. This subtlety
is due to the C language, where the downcast of a const pointer is easy
to write but is not necessarily legal.



Typically, in this code:



\CodeBlockInput{c}{unref.c}



Dereferencing \CodeInline{p} is valid, however the precondition of
\CodeInline{unref} will not be verified by WP since dereferencing
\CodeInline{value} is only legal for a read-access. A write access will
result in an undefined behavior. In such a case, we can specify that the
pointer \CodeInline{p} must be \CodeInline{\textbackslash{}valid\_read} and not
\CodeInline{\textbackslash{}valid}.



\levelFourTitle{Side Effects}


Our \CodeInline{swap} function is provable with regard to the specification
and potential runtime errors, but is however specification precise enough? We
can slightly modify our code to check this (we use \CodeInline{assert} to
verify some properties at some particular points):



\CodeBlockInput{c}{swap-1.c}



The result is not exactly what we expect:



\image{2-swap-1}


Indeed, we did not specify the allowed side effects for our function. In
order to specify side effects, we use an \CodeInline{assigns} clause which is
part of the postcondition of a function. It allows us to specify which
\textbf{non-local} elements (we verify side effects) can be modified
during the execution of the function.



By default, WP considers that a function can modify everything in the
memory. So, we have to specify what can be modified by a function. For
example, our \CodeInline{swap} function will be specified like this:



\CodeBlockInput[3][14]{c}{swap-2.c}



If we ask WP to prove the function with this specification, it will be
validated (including with the variable added in the previous source
code).



Finally, we sometimes want to specify that a function is side effect
free. We specify this by giving \CodeInline{\textbackslash{}nothing} to
\CodeInline{assigns}:



\CodeBlockInput{c}{max_ptr.c}



The careful reader will now be able to take back the examples we
presented until now to integrate the right \CodeInline{assigns} clause.



\levelFourTitle{Memory location separation}


Pointers bring the risk of aliasing (multiple pointers can have access
to the same memory location). For some functions, it will not cause any
problem, for example when we give two identical pointers to the
\texttt{swap} function, the specification is still verified. However,
sometimes it is not that simple:



\CodeBlockInput{c}{incr_a_by_b-0.c}



If we ask WP to prove this function, we get the following result:



\image{2-incr_a_by_b-1}


The reason is simply that we do not have any guarantee that the pointer
\CodeInline{a} is different of the pointer \CodeInline{b}. Now, if these
pointers are the same,



\begin{itemize}
\item   the property \CodeInline{*a == \textbackslash{}old(*a) + *b} in fact
  means \CodeInline{*a == \textbackslash{}old(*a) + *a} which can only
  be true if the old value pointed by \CodeInline{a} was $0$, and we do
  not have such a requirement,
\item
  the property \CodeInline{*b == \textbackslash{}old(*b)} is not validated
  because we potentially modify this memory location.
\end{itemize}


\begin{Question}
  Why is the \CodeInline{assigns} clause validated?

  The reason is simply that \CodeInline{a} is indeed the only modified memory
  location. If \CodeInline{a != b}, we only modify the location pointed by
  \CodeInline{a}, and if \CodeInline{a == b}, \textbar{} that is still the case:
  \CodeInline{b} is not another location.
\end{Question}


In order to ensure that pointers refer to separated memory locations,
ACSL provides the predicate
\texttt{\textbackslash{}separated(p1,\ ...,pn)} that receives in
parameter a set of pointers and is true if and only if these pointers are
non-overlapping. Here, we specify:



\CodeBlockInput{c}{incr_a_by_b-1.c}



And this time, the function is verified:



\image{2-incr_a_by_b-2}


We can notice that we do not consider the arithmetic overflow here, as
we do not focus on this question in this section. However, if this
function was part of a complete program, it would be necessary to define
the context of use of this function and the precondition guaranteeing
the absence of overflow.



\levelThreeTitle{Exercises}


\levelFourTitle{Division and remaining}



Specify the postcondition of the following function, that computes the
results of the division of \CodeInline{a} by \CodeInline{b} and its
remaining and stores it in two memory locations \CodeInline{p} and
\CodeInline{q}:


\CodeBlockInput{c}{ex-1-div-rem.c}

Run the command:

\begin{CodeBlock}{bash}
frama-c-gui your-file.c -wp 
\end{CodeBlock}


Once the function is successfully proved to verify your contract, run:


\begin{CodeBlock}{bash}
frama-c-gui your-file.c -wp -wp-rte
\end{CodeBlock}


If it fails, complete your condition by adding the right precondition.



\levelFourTitle{Reset on condition}



Provide a contract for the following function that reset its first parameter
if the second is true. Be sure to express that the second parameter remains
unmodified:


\CodeBlockInput{c}{ex-2-reset-on-cond.c}


Run the command:


\begin{CodeBlock}{bash}
frama-c-gui your-file.c -wp -wp-rte
\end{CodeBlock}



\levelFourTitle{Addition of pointed values}


The following function receives two pointers as an input and returns the
sum the pointed values. Write the contract of this function:



\CodeBlockInput{c}{ex-3-add-ptr.c}



Run the command:



\begin{CodeBlock}{bash}
frama-c-gui your-file.c -wp -wp-rte
\end{CodeBlock}



Once the function and calling code are successfully proved, modify the
signature of the function add as follows:



\begin{CodeBlock}{c}
void add(int* a, int* b, int* r);
\end{CodeBlock}


Now the result of the sum should be stored at \CodeInline{r}. Accordingly
modify the calls in the main function as well as the code and the contract
of \CodeInline{add}.



\levelFourTitle{Maximum of pointed values}



The following functions computes the maximum of the values pointed by
\CodeInline{a} and \CodeInline{b}. Write the contract of the function:



\CodeBlockInput{c}{ex-4-max-ptr.c}



Run the command:



\begin{CodeBlock}{bash}
frama-c-gui your-file.c -wp -wp-rte
\end{CodeBlock}



Once it is proved, modify the signature of the function as follows:



\begin{CodeBlock}{c}
void max_ptr(int* a, int* b);
\end{CodeBlock}


Now the function should ensure that after its execution \CodeInline{*a}
contains the maximum of the input value, and \CodeInline{*b} contains the
other value. Modify the code accordingly as well as the contract. Note that
the variable \CodeInline{x} is not necessary anymore in the \CodeInline{main}
function and that you can change the assertion on line 15 to reflect the
new behavior of the function.



\levelFourTitle{Order 3 values}



The following function should order the 3 input values in increasing order.
Write the corresponding code and specification of the function:


\CodeBlockInput{c}{ex-5-order-3.c}


And run the command:


\begin{CodeBlock}{bash}
frama-c-gui your-file.c -wp -wp-rte
\end{CodeBlock}


Remember that ordering values is not just ensuring that resulting values
are increasing order and one of each original value, all original values
should still be there after the sorting operation: new values are a
permutation of the original one. To express this idea, you can rely on the
set datatype. For example, this property is true:

\begin{CodeBlock}{c}
//@ assert { 1, 2, 3 } == { 2, 3, 1 };
\end{CodeBlock}



\levelTwoTitle{Behaviors}

Sometimes, a function can have behaviors that can be quite different
depending on the input. Typically, a function can receive a pointer to
an optional resource: if the pointer is \texttt{NULL}, we have a
certain behavior, which is different of the behavior expected when
the pointer is not \texttt{NULL}.

We have already seen a function that have different behaviors: the
\texttt{abs} function. Let us use it again to illustrate behaviors. We
have two behaviors for the \texttt{abs} function: either the input is
positive or it is negative.

Behaviors allow us to specify the different cases for postconditions. We
introduce them using the \texttt{behavior} keyword. Each behavior 
is named. For a given behavior, we have different assumptions about the
input of the function, they are introduced with the clause
\texttt{assumes} (note that since they characterize the input, the
keyword \CodeInline{\textbackslash{}old} cannot be used there). However,
the properties expressed by this clause do not have to be verified before
the call, they can be verified and in this case, the postcondition
specified in our behavior applies. The postconditions of a particular
behavior are introduced using \texttt{ensures}. Finally, we can ask WP to
verify that behaviors are disjoint (to guarantee determinism) and
complete (to guarantee that we cover all possible input).

Behaviors are disjoint if for any (valid) input of the function, it
corresponds to the assumption (\texttt{assumes}) of a single behavior.
Behaviors are complete if any (valid) input of the function corresponds
to at least one behavior.

For example, for \texttt{abs} we can write the specification:



\CodeBlockInput{c}{abs.c}



Note that declaring behaviors does not forbid to specify global postconditions.
For example here, we have specified that for any behavior, the function must
return a positive value.



Let us now slighlty modify the assumptions of each behavior to illustrate
the meaning of \CodeInline{complete} and \CodeInline{disjoint}:

\begin{itemize}
\item
  replace the assumption of \CodeInline{pos} with
  \CodeInline{val \textgreater{} 0}, in this case, behaviors are
  disjoint but incomplete (we miss \CodeInline{val == 0}),
\item
  replace the assumption of \CodeInline{neg} with
  \CodeInline{val \textless{}= 0}, in this case, behaviors are
  complete but not disjoint (we have two assumptions corresponding
  to \CodeInline{val == 0}.
\end{itemize}


\begin{Warning}
  Even if \CodeInline{assigns} is a postcondition, indicating different assigns
  in each behavior is currently not well-handled by WP. If we need to specify
  this, we will:
  \begin{itemize}
  \item put our \CodeInline{assigns} before the behaviors (as we have done in our
    example) with all potentially modified non-local elements,
  \item add in postcondition of each behaviors the elements that are in fact
    not modified by indicating their new value to be equal to the
    \CodeInline{\textbackslash{}old} one.
  \end{itemize}
\end{Warning}


Behaviors are useful to simplify the writing of specifications when
functions can have very different behaviors depending on their input.
Without them, specification would be defined using implications
expressing the same idea but harder to write and read (which would be
error-prone). On the other hand, the translation of completeness and
disjointedness would be necessarily written by hand which would be
tedious and again error-prone.


\levelThreeTitle{Exercises}



\levelFourTitle{Distance}



Take back the example about the computation of the distance between to
integers. Considering that the written contract was:


\CodeBlockInput{c}{ex-1-distance.c}


Re-write it using behaviors.



\levelFourTitle{Reset on condition}



Take back the example ``reset on condition'' from the previous section.
Considering that the written contract was:



\CodeBlockInput[1][13]{c}{ex-2-reset-on-cond.c}



Re-write it using behaviors.



\levelFourTitle{Days of the month}


Take back the example ``days of the month'' from the first section.
Considering that the written contract was:



\CodeBlockInput{c}{ex-3-day-month.c}


Re-write it using behaviors.


\levelFourTitle{Max of pointer values, ordering}



Take back the example ``Max of pointed values'' from the previous section,
this time with the version that orders the values. Considering that the
contract was:



\CodeBlockInput[1][13]{c}{ex-4-max-ptr.c}



Re-write it using behaviors.



\levelFourTitle{Max of pointed values, returning the result}



Take back the example ``Max of pointed values'' from the previous section,
and more precisely, the version that returns the result. Considering that
the contract was:



\CodeBlockInput[1][10]{c}{ex-5-max-ptr.c}



\begin{enumerate}
\item Rewrite it using behaviors
\item Modify the contract of 1. in order to make the behaviors non-disjoint,
  except this property, the contract should remain verified,
\item Modify the contract of 1. in order to make the behaviors incomplete,
  add a new behavior that makes the contract complete again,
\item Modify the function of 1. in order to accept \CodeInline{NULL} pointers
  for both \CodeInline{a} and \CodeInline{b}. If both of them are nul pointers,
  return \CodeInline{INT\_MIN}, if one is a nul pointer, return the value of
  the other, else, return the maximum of them. Modify the contract accordingly
  by adding new behaviors. Be sure that they are disjoint and complete.
\end{enumerate}


\levelFourTitle{Order 3}


Take back the example ``Order 3 values'' from the previous section. Considering
that the contract was:


\CodeBlockInput{c}{ex-6-order-3.c}


Rewrite it using behaviors. Note that you should have one general behaviors
and 3 specific behaviors. Are these behaviors complete? Are they disjoint?



\levelTwoTitle{WP Modularity}

The end of this part will be dedicated to function call composition,
where we will start to have a closer look to WP. We will also have a
look at the way we can split our programs in different files when we
want to prove them using WP.

Our goal will be to prove the \CodeInline{max\_abs} function, that return
the maximum absolute value of two values:



\CodeBlockInput[6][11]{c}{max_abs.c}



Let us start by (over-)splitting the function we already proved in pairs
header/source for \CodeInline{abs} and \CodeInline{max}. We will obtain, for
\CodeInline{abs}:



File abs.h :

\CodeBlockInput{c}{abs.h}



File abs.c

\CodeBlockInput{c}{abs.c}



We can notice that we put our function contract inside the header file.
The goal is to be able to import the specification at the same time as
the declaration when we need it in another file. Indeed, WP will need it
to be able to prove that the precondition of the function is verified
when we call it.

We can create a file using the same format for the \texttt{max}
function. In both cases, we can open the source file (we do not need to
specify header files in the command line) with Frama-C and notice that
the specification is indeed associated to the function and that we prove
it.


Now, we can prepare our files for the \texttt{max\_abs} function with
the header:



\CodeBlockInput{c}{max_abs_uns.h}



And its source file:



\CodeBlockInput{c}{max_abs.c}



We can open the source file in Frama-C. If we look at the side panel, we
can see that the header files we have included in \texttt{abs\_max}
correctly appear and if we look at the function contracts for them, we
can see some blue and green bullets:



\image{max_abs}[The contract of \CodeInline{max} is assumed to be valid]


These bullets indicate that, since we do not have the implementation,
they are assumed to be true. It is an important strength of the
deductive proof of programs compared to some other formal methods:
functions are verified in isolation from each other.

When we are not currently performing the proof of a function, its
specification is considered to be correct: we do not try to prove it
when we are proving another function, we will only verify that the
precondition is correctly established when we call it. It provides very
modular proofs and specifications that are therefore more reusable. Of
course, if our proof relies on the specification of another function, it
must be provable to ensure that the proof of the program is complete.
But, we can also consider that we trust a function that comes from an
external library that we do not want to prove (or for which we do not
even have the source code).

The careful reader could specify and prove the \CodeInline{max\_abs}
function.

A solution is provided there:



\CodeBlockInput[4][14]{c}{max_abs.h}



\levelThreeTitle{Exercices}


\levelFourTitle{Days of the month}


Specify the function leap year that returns true if the year received
as an input is leap. Use this functions to complete the functions
\CodeInline{days\_of} in order to return the number of days of the
month received as an input, including the right behavior when the year
is leap for february.



\CodeBlockInput{c}{ex-1-days-month.c}


\levelFourTitle{Alpha-numeric character}


Write and specify the different functions used by
\CodeInline{is\_alpha\_num} provide a contract for each of them and
provide the contract of \CodeInline{is\_alpha\_num}.



\CodeBlockInput{c}{ex-2-alphanum.c}


Declare an enum with values \CodeInline{LOWER}, \CodeInline{UPPER},
\CodeInline{DIGIT} and \CodeInline{OTHER}, and a function
\CodeInline{character\_kind} that returns, using the different
functions \CodeInline{is\_lower}, \CodeInline{is\_upper},
\CodeInline{is\_digit}, the kind of character received in input. Use
behaviors to specify the contracts of this function and be sure that
they are disjoint and complete.




\levelFourTitle{Order 3 values}



Taking back the function \CodeInline{max\_ptr} that orders two values,
puting the maximum at the first location and the minimum at second, 
write a function \CodeInline{min\_ptr} that uses this function and
produced the opposite operation. Use these functions to complete the
four functions that orders 3 values. For each variant (increasing and
decreasing), write it once using only \CodeInline{max\_ptr} and once
using only \CodeInline{min\_ptr}.



\CodeBlockInput{c}{ex-3-order-3.c}




\horizontalLine



During this part of the tutorial, we have studied how we can specify
functions using contracts, composed of a pre and a post-condition, as
well as some features ACSL provides to express those properties. We have
also seen why it is important to be precise when we specify and how the
introduction of behaviors can help us to write more understandable and
concise specification.

However, we do not have studied one important point: the specification
of loops. Before that, we should have a closer look to the way WP works.
