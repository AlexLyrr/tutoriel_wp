
\image{frama-c}


\levelThreeTitle{Frama-C? WP?}



Frama-C (FRAmework for Modular Analysis of C code) is a platform
dedicated to the analysis of C programs created by the CEA LIST and
Inria. It is based on a modular architecture allowing to use different
(collaborating or not) plugins. The default plugins comprises different
static analyses (that do not execute source code), dynamic analyses
(that requires code execution), or combining both.


\image{gallery}



Frama-C provides a specification language called ACSL (``Axel'') for
ANSI C Specification Language and that allows us to express the
properties we want to verify about our programs. These properties are
written using code annotations in comment sections. If one has
already used Doxygen, it is quite similar, except that we write logic
formulas and not text. During this tutorial, we will extensively write
ASCL code, so let us just skip this for now.



The analysis we will use in this tutorial is provided by the WP plugin
(for Weakest Precondition), a deductive verification plugin. It implements
the technique we mentioned earlier: from
ACSL annotations and the source code, the plugin generates what we call
verification conditions, that are logic formulas that must be verified to be
satisfiable or not. This verification can be performed manually or
automatically, here we use automatic tools.



We will use a SMT solver
(\externalLink{statisfiability modulo theory}{https://fr.wikipedia.org/wiki/Satisfiability\_modulo\_theories}, we do not detail how it works). This solver is
\externalLink{Alt-Ergo}{http://alt-ergo.lri.fr/}, that was initially developed
by the Laboratoire de Recherche en Informatique d'Orsay, and is today
maintained by OCamlPro.



\levelThreeTitle{Installation}



Frama-C is a software developed on Linux and OSX. Its support is thus better
on those operating system. Nevertheless, it is possible to install Frama-C
on Windows and in theory, its use will be identical to its use on Linux.
However:



\begin{Warning}
\begin{itemize}
  \item the tutorial presents the use of Frama-C on Linux (or OSX) and
    the author did not experiment the differences that could exists with Windows,
  \item in recent versions of Windows 10, a possibility is to use Windows
    Subsystem for Linux, in combination with a Xserver installed on Windows
    to get GUI,
  \item the ``Bonus'' section of this part could not be accessible on Windows.
  \end{itemize}
\end{Warning}


\levelFourTitle{Linux}


\levelFiveTitle{Using package managers}


On Debian, Ubuntu and Fedora, there exist packages for Frama-C. In
such a case, it is enough to type a command like:



\begin{CodeBlock}{bash}
apt-get/yum install frama-c
\end{CodeBlock}



However, these repositories are not necessarily up to date with the latest
version of Frama-C. Most of the tutorial should be usable for the version
available on your Linux distribution if it is recent enough. However, some
specific features could be missing.



Go to the section ``Verify installation'' to perform some tests about
the installation.



\levelFiveTitle{Via opam}



A second option is to use Opam, a package manager for Ocaml libraries
and applications.



First of all, Opam must be installed (see its documentation). Then, some
packages from the Linux distribution must be installed before
installing Frama-C. On most systems, it is possible to ask Opam to install
dependencies of the packages we want to install. For this, first install
the depext tool of Opam:


\begin{CodeBlock}{bash}
opam install depext
\end{CodeBlock}


And then use the depext tool to install the dependencies of Frama-C:


\begin{CodeBlock}{bash}
opam depext frama-c
\end{CodeBlock}


If the depext tool does not support your distribution, the following
libraries and software should be installed:

\begin{itemize}
\item GTK2 (development library)
\item GTKSourceview 2 (development library)
\item GnomeCanvas 2 (development library)
\item autoconf
\end{itemize}


Once these packages are installed, we can install Frama-C and Alt-Ergo.


\begin{CodeBlock}{bash}
opam install frama-c
opam install alt-ergo-free
\end{CodeBlock}


Note that for versions older than Potassium (Argon and before), if you
want to use the Why3 platform as a backend (introduced later), the version
0.88.3 should be used for Why3.


Go to the section ``Verify installation'' to perform some tests about
the installation.



\levelFiveTitle{Via ``manual'' compilation}


The packages we have listed in the Opam section are required (of course,
Opam itself is not). It requires a recent version of Ocaml and its
compiler (including a compiler to native code). Is is also necessary to
install Why3 (at least version 1.2.0), which is available either on Opam
or on their website (\externalLink{Why3}{http://why3.lri.fr/}).



After having extracted the folder available here :
\externalLink{http://frama-c.com/download.html}{http://frama-c.com/download.html} (Source distribution).
Navigate to the folder and then execute the command line:



\begin{CodeBlock}{bash}
autoconf && ./configure && make && sudo make install
\end{CodeBlock}



Go to the section ``Verify installation'' to perform some tests about
the installation.



\levelFourTitle{OSX}


On OSX, the use of Homebrew and Opam is recommended to install Frama-C.
The author does not use OSX, so here is a shameful copy and paste of the
installation guide of Frama-C for OSX.




General Mac OS tools for OCaml:



\begin{CodeBlock}{bash}
> xcode-select --install
> open http://brew.sh
> brew install autoconf opam
\end{CodeBlock}



Graphical User Interface:



\begin{CodeBlock}{bash}
> brew install gtk+ --with-jasper
> brew install gtksourceview libgnomecanvas graphviz
> opam install lablgtk ocamlgraph
\end{CodeBlock}



Recommended for Frama-C:



\begin{CodeBlock}{bash}
> brew install gmp
> opam install zarith
\end{CodeBlock}



Necessary for Frama-C/WP:



\begin{CodeBlock}{bash}
> opam install alt-ergo-free
> opam install frama-c
\end{CodeBlock}



\levelFourTitle{Windows}


Currently, the best way to install Frama-C on Windows is to use the
Windows Subsystem for Linux. Basically, once the Linux system is installed,
one should install Opam and follow the instructions provided in the Linux
section. To get a usable graphical user interface, a X server should be
installed on Windows.


\levelThreeTitle{Verify installation}


In order to verify that the installation has been correctly performed,
we will use the following code:



\CodeBlockInput{c}{verify.c}


Copy and paste this code in a file named \CodeInline{main.c}.
Then, from a terminal, in the folder where the file has been created, we
start Frama-C with the following command line:



\begin{CodeBlock}{bash}
frama-c-gui -wp -rte main.c
\end{CodeBlock}



The following window should appear:



\image{verif_install-1}


Clicking \CodeInline{main.c} in the left side panel to select it, we can see
its content (slightly) modified, and some green bullets on different
lines as illustrated here:



\image{verif_install-2}




\begin{Warning}
  The graphical user interface of Frama-C does not allow source code edition.
\end{Warning}


\begin{Information}
  For color blinds, it is possible to start Frama-C with another theme where
  color bullets are replaced:

\begin{CodeBlock}{bash}
frama-c-gui -gui-theme colorblind
\end{CodeBlock}
\end{Information}


\levelThreeTitle{(Bonus) Some more provers}


This part is optional, nothing in this section should be particularly
useful \emph{in the tutorial}. However, when we start to be interested
in proving more complex programs, it is often possible to reach the
limits of Alt-Ergo, which is available by default, and we would thus need
some other provers. For basic properties, almost all solvers have the
same capabilities, for more complex ones, each solvers has its predilection
domains.


\levelFourTitle{Why3}


\externalLink{Why3}{http://why3.lri.fr/} is a deductive proof platform
developed by the LRI in Orsay. It
provides a programming language and a specification language, as well as
a module that allows to interact with a wide variety of automatic and
interactive provers. This point is the one that interests us here. WP uses
Why3 as a backend to talk with external provers.


On their website, we can find the list of
\externalLink{supported provers}{http://why3.lri.fr/\#provers}.
We recommend to install
\externalLink{Z3}{https://github.com/Z3Prover/z3/wiki} which is
developed by Microsoft Research, and
\externalLink{CVC4}{http://cvc4.cs.stanford.edu/web/} which is developed by many
research teams (New York University, University of Iowa, Google, CEA
List). Those two provers are very efficient and somewhat complementary.


New provers can be installed any time after the installation of Frama-C.
However, Why3 provers list must be updated:

\begin{CodeBlock}{bash}
why3 config --detect
\end{CodeBlock}

And then activated in Frama-C. In the left side panel, in the WP part,
clic "Provers..." :


\image{provers}


And then "Detect" in the window that appears. Once it is done, provers
can be activated thanks to the button next to their name.


\image{detect-and-select}


\levelFourTitle{Coq}


Coq, which is developed by Inria, is a proof assistant. Basically, we
write the proofs ourselves in a dedicated language and the assistant
verifies (using typing) that the proof is actually a valid proof.



Why would we need such a tool? Sometimes, the properties we want to
prove can be too complex to be solved automatically by SMT solvers,
typically when they requires careful inductive reasoning with precise
choices at each step. In this situation, WP allows us to generate
verification conditions translated in Coq language, and to write the
proof ourselves.




To learn Coq, we would recommend
\externalLink{this tutorial}{http://www.cis.upenn.edu/~bcpierce/sf/current/index.html}.



\begin{Information}
  If Frama-C has been installed using the package manager of a Linux
  distribution, Coq could be automatically installed.
\end{Information}


If one needs more information about Coq and its installation, this page
can help: \externalLink{The Coq Proof Assistant}{https://coq.inria.fr/}.



The current support of Coq is deprecated in Frama-C but is still available.
As we provide Coq scripts that can be used, we provide these instruction
so that this way of proving programs can be tested by users. To run Frama-C
with Coq support, use:



\begin{CodeBlock}{bash}
  frama-c -wp-provers=native:coq
\end{CodeBlock}