
\begin{center}
\image{frama-c.png}
\end{center}


\levelThreeTitle{Frama-C? WP?}



Frama-C (FRAmework for Modular Analysis of C code) is a platform
dedicated to the analysis of C programs created by the CEA List and
Inria. It is based on a modular architectures allowing to use different
(collaborating or not) plugins. The default plugins comprises different
static analyses (that do not execute source code), dynamic analyses
(that requires code execution), or combining both.



Frama-C provides a specification language called ACSL (``Axel'') for
ANSI C Specification Language and that allows us to express the
properties we want to verify about our programs. These properties will
be written using code annotations in comment sections. If one has
already used Doxygen, it is quite similar, except that we write logic
formulas and not text. During this tutorial, we will extensively write
ASCL code, so let us just skip this for now.



The analysis we will use is provided by the WP plugin (for Weakest
Precondition), it implements the technique we presented earlier: from
ACSL annotations and the source code, the plugin generates what we call
proof goals, that are logic formulas that must be verified to be
satisfiable or not. This verification can be performed manually or
automatically, here we use automatic tools.



We will use a SMT solver
(\externalLink{statisfiability modulo theory}{https://fr.wikipedia.org/wiki/Satisfiability\_modulo\_theories}, we will not detailed how it works). This solver is
\externalLink{Alt-Ergo}{http://alt-ergo.lri.fr/}, that was initially developed
by the Laboratoire de Recherche en Informatique d'Orsay, and is today
maintained and updated by OCamlPro.



\levelThreeTitle{Installation}


Frama-C est un logiciel développé sous Linux et OSX. Son support est donc bien
meilleur sous ces derniers. Il existe quand même de quoi faire une installation 
sous Windows et en théorie l'utilisation sera sensiblement la même que sous 
Linux, mais :



\begin{Warning}
\begin{itemize}
  \item the tutorial presents the use of Frama-C on Linux (or OSX) and
    the author did not experiment the differences that could exists with Windows,
  \item the ``Bonus'' section of this part could not be accessible under Windows.
  \end{itemize}
\end{Warning}


\levelFourTitle{Linux}


\levelFiveTitle{Using package managers}


Under Debian, Ubuntu and Fedora, there exist packages for Frama-C. In
such a case, it is enough to type a command like:



\begin{CodeBlock}{bash}
apt-get/yum install frama-c
\end{CodeBlock}



However, these repositories are not necessarily up to date with the last
version of Frama-C. This is not a big problem since there is not new
versions of Frama-C every day, but it still important to know it.



Go to the section ``Verify installation'' to perform some tests about
your installation.



\levelFiveTitle{Via opam}


La deuxième solution consiste à passer par Opam, un gestionnaire de paquets 
pour les bibliothèques et applications OCaml.



D'abord Opam doit être installé et configuré sur votre distribution (voir 
leur documentation). Ensuite, il faut également que quelques paquets de votre
distribution soit présents préalablement à l'installation de Frama-C :



A second option is to use Opam, a package manager for Ocaml libraries
and applications.



First of all, Opam must be installed (see its documentation). Then, some
packages from your Linux distribution must be installed before
installing Frama-C:



\begin{itemize}
\item lib gtk2 dev
\item lib gtksourceview2 dev
\item lib gnomecanvas2 dev
\item (conseillé) lib zarith dev
\end{itemize}



Once it is done, we can install Frama-C and Alt-Ergo.




\begin{CodeBlock}{bash}
opam install frama-c
opam install alt-ergo
\end{CodeBlock}



Go to the section ``Verify installation'' to perform some tests about
your installation.



\levelFiveTitle{Via ``manual'' compilation}


The packages we have listed in the Opam section are required (of course,
Opam itself is not). It requires a recent version of Ocaml and its
compiler (including compiler to native code).



After having extracted the folder available here :
\externalLink{http://frama-c.com/download.html}{http://frama-c.com/download.html} (Source distribution).
Navigate to the folder and then execute the command line:



\begin{CodeBlock}{bash}
./configure && make && sudo make install
\end{CodeBlock}



Go to the section ``Verify installation'' to perform some tests about
your installation.



\levelFourTitle{OSX}


On OSX, the use of Homebrew and Opam is recommended to install Frama-C.
The author do not use OSX, so here is a shameful copy and paste of the
installation guide of Frama-C for OSX.




General Mac OS tools for OCaml:



\begin{CodeBlock}{bash}
> xcode-select --install 
> open http://brew.sh
> brew install autoconf opam 
\end{CodeBlock}



Graphical User Interface:



\begin{CodeBlock}{bash}
> brew install gtk+ --with-jasper
> brew install gtksourceview libgnomecanvas graphviz
> opam install lablgtk ocamlgraph 
\end{CodeBlock}



Recommended for Frama-C:



\begin{CodeBlock}{bash}
> brew install gmp
> opam install zarith
\end{CodeBlock}



Necessary for Frama-C/WP:



\begin{CodeBlock}{bash}
> opam install alt-ergo
> opam install frama-c
\end{CodeBlock}



Also recommended for Frama-C/WP:



\begin{CodeBlock}{bash}
> opam install altgr-ergo coq coqide why3
\end{CodeBlock}

\levelFourTitle{Windows}


Currently, the installation of Frama-C for Windows requires Cygwin and
an experimental version of Opam for Cygwin. So we need to install both
as well as the MinGW Ocaml compiler.



Installation instructions can be found there:



\externalLink{Frama-C - Windows}{https://bts.frama-c.com/dokuwiki/doku.php?id=mantis:frama-c:compiling\_from\_source}



Frama-C is then started using Cygwin.



Go to the section ``Verify installation'' to perform some tests about
your installation.




\levelThreeTitle{Verify installation}


In order to verify that the installation has been correctly performed,
we will use this simple code:



\CodeBlockInput{c}{verify.c}



Then, from a terminal, in the folder where the file has been created, we
start Frama-C with the following command line:



\begin{CodeBlock}{bash}
frama-c-gui -wp -rte main.c
\end{CodeBlock}



This window should appear.



\image{verif_install-1.png}[Verify installation 1]


Clicking \CodeInline{main.c} in the left side panel to select it, we can see
its content (slightly) modified, and some green bullets on different
lines as illustrated as follows:



\image{verif_install-2.png}[Verify installation 2]




\begin{Warning}
  The graphical user interface of Frama-C does not allow source code edition
\end{Warning}


\begin{Information}
  For color blinds, it is possible to start Frama-C with another theme where
  color bullets are replaced:
  
\begin{CodeBlock}{bash}
$ frama-c-gui -gui-theme colorblind
\end{CodeBlock}
\end{Information}


\levelThreeTitle{(Bonus) Some more provers}


This part is optional, nothing in this section should be particularly
useful \emph{in the tutorial}. However, when we start to be interested
in proving more complex programs, it is often possible to reach the
limits of Alt-Ergo, which is basically available, and we would thus need
some other provers.



\levelFourTitle{Coq}


Coq, which is developed by Inria, is a proof assistant. Basically, we
write the proofs ourselves in a dedicated language and the platform
verify (using typing) that the proof is actually a valid proof.



Why would we need such a tool? Sometimes, the properties we want to
prove can be too complex to be solved automatically by SMT solvers,
typically when they requires careful inductive reasoning with precise
choices at each step. In this situation, WP allows us to generate proof
goals translated in Coq language, and to write the proof ourselves.




To learn Coq, we would recommend
\externalLink{this tutorial}{http://www.cis.upenn.edu/\textasciitilde{}bcpierce/sf/current/index.html}.



\begin{Information}
  If Frama-C has been installed using the package manager of a Linux
  distribution, Coq could be automatically installed.
\end{Information}


If one needs more information about Coq and its installation, this page
can help: \externalLink{The Coq Proof Assistant}{https://coq.inria.fr/}.



When we want to use Coq for a proof with Frama-C, we have to select it
using the left side panel, in the WP part:



\image{select_coq.png}[Select the Coq proof assistant]


\begin{Information}
  The author does not know if it works under Windows.
\end{Information}


\levelFourTitle{Why3}


\begin{Warning}
  To the author's knowledge, it is not possible (or, at least, not easy
  at all) to install Why3 under Windows. The author cannot be charged for
  injuries that could result of such an operation.
\end{Warning}


Why3 is deductive proof platform developed by the LRI in Orsay. It
provides a programming language and a specification language, as well as
a module that allows to interact with a wide variety of automatic and
interactive provers. This point is the one that interest us here. WP can
translate its proof goals to the Why3 language and then use Why3 to
interact with solvers.



The \externalLink{Why3 website}{http://why3.lri.fr/} provides all information
about it. If Opam is installed, Why3 is available using it, else, there
is an another installation procedure.



On this website, we can find the list of
\externalLink{supported provers}{http://why3.lri.fr/\#provers}.
We recommend to install
\externalLink{Z3}{https://github.com/Z3Prover/z3/wiki} which is
developed by Microsoft Research, and
\externalLink{CVC4}{http://cvc4.cs.nyu.edu/web/} which is developed by many
research teams (New York University, University of Iowa, Google, CEA
List). Those two provers are very efficient and somewhat complementary.



To use these provers, the procedure is explained in the Coq part that
describes the selection of a prover for the proof. Notice that it could
be necessary to ask the detection of freshly installed provers using the
``Provers'' button and then ``Detect Provers'' in the window that should
pop.
