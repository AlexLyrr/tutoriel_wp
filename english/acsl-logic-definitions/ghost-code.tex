
The previous techniques we have seen in this chapter are meant to make the
specification more abstract. The role of ghost code is the opposite, here, we
find in fact a way to enrich our specification with information expressed as
concrete C code. The idea is to add variables and source code that is not part
of the actual program to verify and is thus only visible for the verification
tool. It is used to make explicit some logic properties that, else, would only
be known implicitly.


\levelThreeTitle{Syntax}



Ghost code is added using annotations that contain C code
introduced using the \texttt{ghost} keyword:



\begin{CodeBlock}{c}
/*@
  ghost
  // C code
*/
\end{CodeBlock}



The only rules we have to respect in such a code, is that we cannot
write a memory block that is not itself defined in ghost code, and that
the code must close any block it would open. Apart from this, any
computation can be inserted provided that it modifies \emph{only} ghost
variables.




Here are some examples of ghost code:



\begin{CodeBlock}{c}
//@ ghost int ghost_glob_var = 0;

void foo(int a){
  //@ ghost int ghost_loc_var = a;

  //@ ghost Ghost_label: ;
  a = 28 ;

  //@ ghost if(a < 0){ ghost_loc_var = 0; }

  //@ assert ghost_loc_var == \at(a, Ghost_label) == \at(a, Pre);
}
\end{CodeBlock}



While for this chapter, it will not be necessary, as we will see later,
it is sometimes useful to write some contracts or assertions in ghost
code. As it must be specified in code that is already in C comments, we
can use a specific syntax to provide ghost contracts or assertions.
We open ghost annotations with the syntax \CodeInline{/@} and close
them with \CodeInline{@/}.



For example :



\begin{CodeBlock}{c}
void foo(unsigned n){
 /*@ ghost
   unsigned i ;

   /@
     loop invariant 0 <= i <= n ;
     loop assigns i ;
     loop variant n - i ;
   @/
   for(i = 0 ; i < n ; ++i){

   }
   /@ assert i == n ; @/
 */
}
\end{CodeBlock}



\levelThreeTitle{Ghost code validity}



We must again be careful using ghost code. Indeed, the tool does not
perform any verification to ensure that we do not write in the memory of
the program by mistake. This problem being, in fact, an undecidable
problem, this analysis would require a proof by itself. For example,
this code is allowed as input of Frama-C even if it explicitly modifies
the state of the program we want to verify:



\begin{CodeBlock}{c}
int a;

void foo(){
  //@ ghost a = 42;
}
\end{CodeBlock}



Thus, it is easy to modify the behavior of the code we want to verify by
adding ghost code and in this case, if we prove a property about the
program, we would prove it about the modified program, not about the
actual one. One must thus take care of modifying only ghost memory locations
when writing ghost code.


Furthermore, as we do not have restriction on the kind of C code we can write,
we can write loops. As we said in section~\ref{l3:statements-loops-variant},
there are two kinds of correctness: partial and total correctness, the second
one allowing to prove that a program terminates. In the case of ghost code, it
is absolutely necessary to prove total correctness as an infinite loop in the
ghost code would allow us to prove anything about the program.


For example, the following program is proved because of the non terminating
ghost code:



\begin{CodeBlock}{c}
/*@ ensures \false ; */
void foo(void){
  /*@ ghost
    while(1){}
  */
}
\end{CodeBlock}



\levelThreeTitle{Make a logical state explicit}


The goal of ghost code is to make explicit some information that would else
be implicit. For example, in the verification of the sort algorithm, we used it
to create a label in the program that is not visible by the compiler but that
we can use in the verification. The fact that the value was swapped between the
two labels was implicitly provided by the contract of the function, adding this
ghost label allows us to write an explicit assertion of this fact.



Let us take a more complex example where we more clearly create explicit
knowledge about the program. Here, we want to prove that the
following function returns the value of the maximal sum of subarrays of
a given array. A subarray of an array \texttt{a} is a contiguous subset
of values of \texttt{a}. For example, for an array \CodeInline{\{ 0 , 3 , -1 , 4 \}},
some subarrays can be
\CodeInline{\{\}}, \CodeInline{\{ 0 \}}, \CodeInline{\{ 3 , -1 \}}
, \CodeInline{\{ 0 , 3 , -1 , 4 \}}, ... Note that as we allow
empty arrays for subarrays, the sum is at least 0. In the previous
array, the subarray with the maximal sum is \CodeInline{\{ 3 , -1 , 4 \}},
the function would then return 6.



\CodeBlockInput[3]{c}{max_subarray-0.c}



In order to specify the previous function, we need an axiomatic
definition about sum. This is not too complex, the careful reader
can express the needed axioms as an exercise:



\CodeBlockInput[3][5]{c}{max_subarray-1.c}



Some correct axioms are available at: \ref{l2:acsl-logic-definitions-answers}



The specification of the function is the following:



\CodeBlockInput[14][19]{c}{max_subarray-1.c}



For any bounds, the value returned by the function must be greater or
equal to the sum of the elements between these bounds, and there must
exist some bounds such that the returned value is exactly the sum of the
elements between these bounds. About this specification, when we want to
add the loop invariant, we realize that we miss some information.
We want to express what are the values \CodeInline{max} and \CodeInline{cur} and
what are the relations between them, but we cannot do it!

Basically, in order to prove our postcondition, we need to know that there
exists some bounds
\CodeInline{low} and \CodeInline{high} such that the computed sum corresponds to
these bounds. However, in our code, we do not have anything that expresses
it from a logic point of view, and we cannot \emph{a priori} make the
link between this logic formalization. We then use ghost code to
record these bounds and express the loop invariant.

We first need two variables to record the bounds
of the maximum sum range, let us call them \CodeInline{low} and
\CodeInline{high}. Every time we find a range where the sum is greater
than the current one, we update our ghost variables. These bounds
then correspond to the sum currently stored by \CodeInline{max}. That
induces that we need other bounds: the ones that corresponds to the sum
stored by the variable \CodeInline{cur} from which we build the bounds
corresponding to \CodeInline{max}. For these bounds, we only add a
single ghost variable: the current low bound \CodeInline{cur\_low}, the high
bound being the variable \CodeInline{i} of the loop.



\CodeBlockInput[14][50]{c}{max_subarray-1.c}



The invariant \CodeInline{BOUNDS} expresses how the different bounds are
ordered during the computation. The invariant \CodeInline{REL} expresses
what the variables \CodeInline{cur} and \CodeInline{max} mean depending on the
bounds. Finally, the invariant \CodeInline{POST} allows us to create a link
between the invariant and the postcondition of the function.



The reader can verify that this function is indeed correctly proved
without RTE verification. If we add RTE verification, the overflow on
the variable \CodeInline{cur}, that is the sum, seems to be possible (and it
is indeed the case).




Here, we do not try to fix this because it is not the topic of this
example. The way we can prove the absence of RTE here strongly depends
on the context where we use this function. A possibility is to strongly
restrict the contract, forcing some properties about values and the size
of the array. For example, we could limit the maximal size of the array
and bound each value of the different cells. Another possibility would be to
add an error value in case of overflow
(\(-1\) for example), and to specify that when an overflow is produced,
this value is returned.



\levelThreeTitle{Exercises}



\levelFourTitle{Two times}


This program computes \CodeInline{2 * x} using a loop. Use a ghost variable
\CodeInline{i} to express as an invariant that the value of \CodeInline{r}
is \CodeInline{i * 2} and complete the proof.


\CodeBlockInput{c}{ex-1-times-2.c}



\levelFourTitle{Playing with arrays}


In this function, we receive an array, and we have a loop where we do nothing
except that we have indicated that it assigns the content of the array. However,
we would like to prove in postcondition that the array has not been modified.



\CodeBlockInput{c}{ex-2-array.c}


Without modifying the \CodeInline{assigns} clause of the loop and without using
the keyword \CodeInline{\textbackslash{}at}, prove that the function does not
modify the array. For this, complete the ghost code and the loop invariant, by
assuring that the array \CodeInline{g} represents the old value of
\CodeInline{a}.
