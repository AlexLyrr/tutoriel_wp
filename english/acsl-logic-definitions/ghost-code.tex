
Behind this title, that seems to be an action movie, we find in fact a
way to enrich our specification with information expressed as regular C
code. Here, the idea is to add variables and source code that will not
be part of the actual program but will model logic states that will only
be visible from a proof point of view. Using it, we can make explicit
some logic properties that were previously only known implicitly.


\levelThreeTitle{Syntax}


Ghost code is added using annotations that will contain C code
introduced using the \texttt{ghost} keyword:



\begin{CodeBlock}{c}
/*@
  ghost
  // code en langage C
*/
\end{CodeBlock}



The only rules we have to respect in such a code, is that we cannot
write a memory block that is not itself defined in ghost code, and that
the code must close any block it would open. Apart of this, any
computation can be inserted provided it \emph{only} modifies ghost
variables.




Here are some examples of ghost code:



\begin{CodeBlock}{c}
//@ ghost int ghost_glob_var = 0;

void foo(int a){
  //@ ghost int ghost_loc_var = a;

  //@ ghost Ghost_label: ;
  a = 28 ;

  //@ ghost if(a < 0){ ghost_loc_var = 0; }

  //@ assert ghost_loc_var == \at(a, Ghost_label) == \at(a, Pre);
}
\end{CodeBlock}



We must again be careful using ghost code. Indeed, the tool will not
perform any verification to ensure that we do not write in the memory of
the program by mistake. This problem being, in fact, an undecidable
problem, this analysis would require a proof by itself. For example,
this code is allowed as input of Frama-C even if it explicitly modifies
the state of the program we want to verify:



\begin{CodeBlock}{c}
int a;

void foo(){
  //@ ghost a = 42;
}
\end{CodeBlock}



We then need to be really careful about what we are doing using ghost
code (by making it simple).



\levelThreeTitle{Make a logical state explicit}


The goal of ghost code is to make explicit some information that are
without them implicit. For example, in the previous section, we used it
to get an explicit logic state known at a particular point of the
program.

Let us take a more complex example. Here, we want to prove that the
following function returns the value of the maximal sum of subarrays of
a given array. A subarray of an array \texttt{a} is a contiguous subset
of values of \texttt{a}. For example, for an array \CodeInline{\{ 0 , 3 , -1 , 4 \}},
some subarrays can be
\CodeInline{\{\}}, \CodeInline{\{ 0 \}}, \CodeInline{\{ 3 , -1 \}}
, \CodeInline{\{ 0 , 3 , -1 , 4 \}}, ... Note that as we allow
empty arrays for subarrays, the sum is at least 0. In the previous
array, the subarray with the maximal sum is \CodeInline{\{ 3 , -1 , 4 \}},
the function would then return 6.



\CodeBlockInput[3]{c}{max_subarray-0.c}



In order to specify the previous function, we will need an axiomatic
definition about sum. This is not too complex, the careful reader
can express the needed axioms as an exercise:



\CodeBlockInput[3][5]{c}{max_subarray-1.c}



Some correct axioms are available at: \ref{l2:acsl-logic-definitions-answers}



The specification of the function is the following:



\CodeBlockInput[14][19]{c}{max_subarray-1.c}



For any bounds, the value returned by the function must be greater or
equal to the sum of the elements between these bounds, and there must
exist some bounds such that the returned value is exactly the sum of the
elements between these bounds. About this specification, when we want to
add the loop invariant, we will realize that we miss some information.
We want to express what are the values \CodeInline{max} and \CodeInline{cur} and
what are the relations between them, but we cannot do it!

Basically, our postcondition needs to know that there exists some bounds
\CodeInline{low} and \CodeInline{high} such that the computed sum corresponds to
these bounds. However, in our code, we do not have anything that express
it from a logic point of view, and we cannot \emph{a priori} make the
link between this logical formalization. We will then use ghost code to
record these bounds and express the loop invariant.

We will first need two variables that will allow us to record the bounds
of the maximum sum range, we will call them \CodeInline{low} and
\CodeInline{high}. Every time we will find a range where the sum is greater
than the current one, we will update our ghost variables. This bounds
will then corresponds to the sum currently stored by \CodeInline{max}. That
induces that we need other bounds: the ones that corresponds to the sum
store by the variable \CodeInline{cur} from which we will build the bounds
corresponding to \CodeInline{max}. For these bounds, we will only add a
single ghost variable: the current low bound \CodeInline{cur\_low}, the high
bound being the variable \CodeInline{i} of the loop.



\CodeBlockInput[14][50]{c}{max_subarray-1.c}



The invariant \CodeInline{BOUNDS} expresses how the different bounds are
ordered during the computation. The invariant \CodeInline{REL} expresses
what the variables \CodeInline{cur} and \CodeInline{max} mean depending on the
bounds. Finally, the invariant \CodeInline{POST} allows us to create a link
between the invariant and the postcondition of the function.



The reader can verify that this function is indeed correctly proved
without RTE verification. If we add RTE verification, the overflow on
the variable \CodeInline{cur}, that is the sum, seems to be possible (and it
is indeed the case).




Here, we will not try to fix this because it is not the topic of this
example. The way we can prove the absence of RTE here strongly depends
on the context where we use this function. A possibility is to strongly
restrict the contract, forcing some properties about values and the size
of the array. For example, we could strongly limit the maximal size of
the array and strong bounds on each value of the different cells. An
other possibility would be to add an error value in case of overflow
(\(-1\) for example), and to specify that when an overflow is produced,
this value is returned.
