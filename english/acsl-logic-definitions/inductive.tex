Inductive predicates gives a way to state properties whose verification requires
to reason by induction, that is to say: for a property $p(input)$, it is true
for some base cases (for example, $0$ is an even natural number), and knowing
that $p(input)$ is true, it is also true for a $bigger\ input$, provided that
some conditions relating $input$ and $bigger\ input$ are verified (for example,
knowing that $n$ is even, we define that $n+2$ is also even). Thus, inductive
predicates are generally useful to define properties recursively.

\levelThreeTitle{Syntax}

For now, let us introduce the syntax of inductive predicates:

\begin{CodeBlock}{c}
/*@
  inductive property{ Label0, ..., LabelN }(type0 a0, ..., typeN aN) {
  case c_1{Lq_0, ..., Lq_X}: p_1 ;
  ...
  case c_m{Lr_0, ..., Lr_Y}: p_km ;
  }
*/
\end{CodeBlock}

Where all \CodeInline{c\_i} are names and all \CodeInline{p\_i} are logic
formulas where \CodeInline{property} appears has a conclusion. Basically,
\CodeInline{property} is true for some parameters and some labels, if it
corresponds to one of the cases of the inductive definition.

We can have a look at the simple property we just talked about: how to define
that a natural number is even using induction. We can translate the sentence:
``0 is an even natural number, and if $n$ is an even natural number, $n+2$ is
an even natural number'':



\CodeBlockInput{c}{even-0.c}



Which perfectly describes the two cases:
\begin{itemize}
\item $0$ is even (base case),
\item if a natural $n$ is even, $n+2$ is also even.
\end{itemize}

However, this definition is not completely satisfying. For example, we cannot
deduce that an odd number is not even. If we try to prove that 1 is even, we
will have to check if -1 is even, and then -3, -5, etc, infinitely. Moreover,
we generally prefer to define the induction cases the opposite way: defining
the condition under which the wanted conclusion is true. For example, here,
how can we verify that some $n$ is natural and even? We first check whether it
is 0, if not, we check if $n$ is greater than $0$ and then we verify that $n-2$
is even:



\CodeBlockInput[1][9]{c}{even-1.c}



Here, we distinguish two cases:
\begin{itemize}
\item 0 is even,
\item a natural $n$ is even if it is greater than 0 and $n-2$ is an even
      natural.
\end{itemize}

Taking the second case, we recursively decrease the number until we reach
0, and then the number is even, since 0 is even, or -1, and then there
is no case in the inductive that corresponds to this value, so we could
deduce that the property is false (however, we will see later that in the
case of WP, we need the Coq proof assistant).


\CodeBlockInput[11][14]{c}{even-1.c}



Of course, defining that some natural number is even inductively is not a
good idea, since we can simply define it using modulo. We generally use them
to define complex recursive properties.

\levelFourTitle{Consistency}

Inductive definitions bring the risk to introduce inconsistencies. Indeed, the
property specified in the different cases are considered to be always true, so
if we introduce a property that allows to prove \CodeInline{false}, we will be
able to prove everything. While we will give more details about axioms in the
Section~\ref{l2:acsl-logic-definitions-axiomatic}, let us give two hints to
avoid building such a bad definition.

First, we can make sure that inductive predicates are well founded. This can
be done by syntactically restricting what we allow in an inductive definition,
by making sure that each case has the form:

\begin{CodeBlock}{c}
/*@
  \forall y1,...,ym ; h1 ==> ··· ==> hl ==> P(t1,...,tn) ;
*/
\end{CodeBlock}

where the predicate \CodeInline{P} can only appear positively (so not negated
with \CodeInline{!} - $\neg$) in the different hypotheses \CodeInline{hx}.
Basically, it ensures that we cannot build both positive and negative
occurrences of \CodeInline{P} for some parameters which would be incoherent.

This is for example verified by our previously defined predicate
\CodeInline{even\_natural}. While a definition like:

\CodeBlockInput[1][11]{c}{even-bad.c}

does not respect this constraint as the property \CodeInline{even\_natural}
appears negatively on line~8.

Second, we can simply write a function that have a contract that needs the
predicate \CodeInline{P}. For example, we can write a function:



\begin{CodeBlock}{c}
/*@
  requires P( params ... ) ;
  ensures  BAD: \false ;
*/ void function(params){

}
\end{CodeBlock}



For example for our definition of \CodeInline{even\_natural}, we would write:



\CodeBlockInput[13][18]{c}{even-bad.c}



During the generation of the proof obligation, we ask Why3 to create an
inductive definition from the one written in ACSL. As Why3 is stricter than
Frama-C and WP on this kind of definition, if the inductive predicate is too
strange (if it is not well founded), it will be rejected with an error. And
indeed, with the bad definition of \CodeInline{even\_natural} we just proposed,
Why3 complains when we try to prove the
\CodeInline{ensures \textbackslash{}false}, because there exists a non
positive occurrence of \CodeInline{P\_even\_natural}, that encodes the
\CodeInline{even\_natural} we wrote in ACSL.

\begin{CodeBlock}{bash}
frama-c-gui -wp -wp-prop=BAD file.c
\end{CodeBlock}


\image{why3-failed}

\image{why3-failed-error}


However, that does not mean that we cannot write an inconsistent inductive
definition. For example, the following definition is well founded, while it
allows us to prove false:


\begin{CodeBlock}{c}
/*@ inductive P(int* p){
      case c: \forall int* p ; \valid(p) && p == (void*)0 ==> P(p);
    }
*/

/*@ requires P(p);
    ensures \false ; */
void foo(int *p){}
\end{CodeBlock}


Here, we could detect the problem as \CodeInline{-wp-smoke-tests} can detect
that the precondition is unsatisfiable. But we must be careful when defining
inductive definitions to not introduce a definition that can lead to a proof
of false.


\begin{Warning}
  Before Frama-C 21 Scandium, the inductive definitions were translated, in
  Why3, as axioms. That means that this check was not performed. Thus, to
  get a similar behavior with older Frama-C versions, one has to use Coq
  and not a Why3 prover.
\end{Warning}


\levelThreeTitle{Recursive predicate definitions}

Inductive predicates are often useful to express recursive properties since it
prevents the provers to unroll the recursion when it is possible.

For example, if we want to define that an array only contains 0s, we could
write it as follows:


\CodeBlockInput[3][11]{c}{reset-0.c}


And we can again prove that our reset to 0 is correct with this new
definition:

\CodeBlockInput[14][28]{c}{reset-0.c}

Depending on the Frama-C or automatic solvers versions, the proof of the
preservation of the invariant could fail. A reason for this fail is the fact that
the prover forgets that cells preceding the one we are currently processing
are actually still set to 0. We can add a lemma in our knowledge base, stating
that if a set of values of an array did not change between two program points,
the first one being a point where the property "zeroed" is verified, then the
property is still verified at the second program point.


\CodeBlockInput[13][20]{c}{reset-1.c}


Then we can add an assertion to specify what did not change between the
beginning of the loop block (pointed by the label \CodeInline{L} in the code)
and the end (which is \CodeInline{Here} since we state the property at the end):


\CodeBlockInput[35][39]{c}{reset-1.c}


Note that in this new version of the code, the property stated by our lemma is
not proved using automatic solver, that cannot reason by induction. If the
reader is curious, the (quite simple) Coq proof can be found
in~\ref{l2:acsl-logic-definitions-answers}.


In this case study, using an inductive definition is not efficient: our
property can be perfectly expressed using the basic constructs of the first
order logic as we did before. Inductive definitions are meant to be used to
write definitions that are not easy to express using the basic formalism
provided by ACSL. It is here used to illustrate their use with a simple
example.



\levelThreeTitle{Example: sort}
\label{l3:acsl-logic-definitions-inductive-sort}


Let us prove a simple selection sort:


\CodeBlockInput[3]{c}{sort-0.c}


The reader can exercise by specifying and proving the search of the minimum and
the swap function. We hide there a correct version of these specification and
code (Answers~\ref{l2:acsl-logic-definitions-answers}), we will focus on the
specification and the proof of the sort function that is an interesting use case
for inductive definitions.

Indeed, a common error we could do, trying to prove a sort function, would
be to write this specification (which is correct!):


\CodeBlockInput[6]{c}{sort-incomplete.c}


\textbf{This specification is correct}. But if we recall correctly the part of the
tutorial about specifications, we have to \emph{precisely} express what we expect
of the program. With this specification, we do not prove all required properties
expected for a sort function. For example, this function correctly answers to
the specification:


\CodeBlockInput[8]{c}{sort-false.c}


Our specification does not express the fact that all elements initially present
in the array must still be in the array after executing the sort function. That
is to say: the sort function produces a sorted permutation of the original
array.

Defining the notion of permutation can be done using an inductive definition.
While we will see later a version of this property that is more general, let us
for now limit us to a notion of permutation that is more specific to our current
need. We can limit us to a few cases. First, the array is a permutation of
itself, then swapping two values of the array produces a new permutation if we
do not change anything else. And finally, if we create the permutation $p_2$ of
$p_1$, and then the permutation $p_3$ of $p_2$, then by transitivity $p_3$ is a
permutation of $p_1$.

The corresponding inductive definition is the following:


\CodeBlockInput[37][56]{c}{sort-1.c}


We can then specify that our sort function produces the sorted permutation of
the original array and we can then prove it by providing the invariant of the
function:


\CodeBlockInput[64][84]{c}{sort-1.c}



This time, our property is precisely defined, the proof is relatively easy to
produce, only requiring to add an assertion in the block of the loop to state
that it only performs a swap of values in the array (and then giving
the transition to the next permutation). To define this swap notion, we use
a particular annotation (at line 16), introduced using the keyword
\CodeInline{ghost}. Here, the goal is to introduce a label in the code that in
fact does not exist from the program point of view, and is only visible from
a specification point of view. We present the ``ghost'' features in the final
section of this chapter, for now let us focus on axiomatic definitions.



\levelThreeTitle{Exercises}


\levelFourTitle{Sum of the N first integers}


Take back your solution to the
exercise~\ref{l4:acsl-properties-lemmas-n-first-ints} about the sum of the N
first integers. Rewrite the logic function using an inductive predicate that
states that some integer equals the sum of the N first integers.


\CodeBlockInput{c}{ex-1-sum-N-first-ints.c}


Adapt the contract of the function and the lemma(s). Note that lemma(s) could
certainly not be proved by SMT solvers. We provide a solution and corresponding
Coq proofs on the GitHub repository of this book.


\levelFourTitle{Greatest Common Divisor}


Write an inductive predicate that states that some integer is the greatest common
divisor of two others. The goal of the exercise is to prove that the function
\CodeInline{gcd} computes the greatest common divisor. Thus, we do not have to
specify all the cases for the predicate. Indeed, a close look at the loop shows
us that after the first iteration \CodeInline{a} is greater or equals to
\CodeInline{b} and it is maintained by the loop. Thus, we consider two cases for
the inductive predicate:


\begin{itemize}
\item \CodeInline{b} is 0,
\item if some \CodeInline{d} is the GCD of \CodeInline{b} and \CodeInline{a \% b},
  then it is the GCD of \CodeInline{a} and \CodeInline{b}
\end{itemize}


\CodeBlockInput{c}{ex-2-gcd.c}


Express the postcondition of the function, and complete the invariant to prove
that it is verified. Note that the invariant should make use of the inductive
case \CodeInline{gcd\_succ}.



\levelFourTitle{Power function}


In this exercise, we do not consider RTEs.


Write an inductive predicate that states that some integer \CodeInline{r} equals
to $x^n$. Consider the two cases: either $n$ is 0 or it is greater and then it
should be related to the value $x^{n-1}$.


\CodeBlockInput[1][6]{c}{ex-3-fpower.c}


First prove the naive version of the power function:


\CodeBlockInput[13][28]{c}{ex-3-fpower.c}


Now, let us prove a faster version of the power function:


\CodeBlockInput[30][49]{c}{ex-3-fpower.c}


In this version, we exploit two properties about the power operator:


\begin{itemize}
\item $(x^2)^n = x^{2n}$
\item $x * (x^2)^n = x^{2n+1}$
\end{itemize}


that allows us to divide $n$ by 2 at each step of the loop instead of decreasing
it by one (which makes the algorithm $O(log\ n)$ instead of $O(n)$). Express the
two previous properties in lemmas:


\CodeBlockInput[8][11]{c}{ex-3-fpower.c}


First express the lemma \CodeInline{power\_even}, the SMT solver might be able
to combine the use of this lemma and the inductive predicate to deduce
\CodeInline{power\_odd}. The Coq proof of the \CodeInline{power\_even} lemma is
provided on the GitHub repository of this book, as well as the proof of the
\CodeInline{power\_odd} in case the SMT solver does not prove it.


Finally, complete the contract and loop invariant of the \CodeInline{fast\_power}
function. For this notice that at the beginning of the loop $x^{old(n)} = p^n$,
and that each iteration uses the previous properties to update $r$, in the sense
that we have $x^{old(n)} = r * p^n$ during all the loop, until we have $n = 0$ and
thus $p^n = 1$.


\levelFourTitle{Permutation}


Take back the definitions of the \CodeInline{shifted} and \CodeInline{unchanged}
predicates from the exercise~\ref{l4:acsl-properties-lemmas-shift-trans}. Use the
shift predicate to express the rotate predicate that expresses that some elements
of an array are rotated to the left in the sense that all elements are shifted of
one element to the left except the last one that is put in the first cell of the
range. Use this predicate to prove the rotate function:


\CodeBlockInput[12][27]{c}{ex-4-rotate-permutation.c}


Express a new version of the notion of permutation with an inductive predicate
that considers four cases:

\begin{itemize}
\item permutation is reflexive,
\item if the left part of the range (until an index of the range) is rotated
  between two labels, we still have a permutation,
\item if the right part of the range (from an index of the range) is rotated
  between two labels, we still have a permutation,
\item permutation is transitive.
\end{itemize}


\CodeBlockInput[30][37]{c}{ex-4-rotate-permutation.c}


Complete the contract of \CodeInline{two\_rotates} that successively rotates the
first half of the array and then the second half and prove that it maintains the
permutation.


\CodeBlockInput[43][47]{c}{ex-4-rotate-permutation.c}
